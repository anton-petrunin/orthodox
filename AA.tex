%%%%%%%%%%%%%%%%%%%%%%%%%%%%%%%%%%%%%%%%%
\subsection*{Curvature hollow}
\label{Curvature hollow}

\begin{pr}
Construct a Riemannian metric on $\RR^3$ 
which is Euclidean outside of an open bounded set $\Omega$ 
and with negative scalar curvature in $\Omega$.
\end{pr}

%%%%%%%%%%%%%%%%%%%%%%%%%%%%%%%%%%%%%%%%%%%%%%%%%%
\parbf{Curvature hollow.}
An example can be found among the  metrics with isometric $\mathbb{S}^1$ action which fix a line.

\medskip

First note that it is easy to construct such metric on the connected sum
$\RR^3\#\mathbb{S}^2\times\mathbb{S}^1$.
Indded make two holes in $\RR^3$, deform the obtained space conformally so that scalar curvature drops and the two holes fit and glue the holes to each other.
Further, note that such metric can be constructed in such a way that it has a closed geodesic $\gamma$ with trivial holonomy and with constant negative curvature in its a tubular neighborhood.

Cut tubular neighborhood $V\simeq\mathbb{D}^2\times \mathbb{S}^1$ of $\gamma$ 
and glue in the product $W\simeq\mathbb{S}^1\times \mathbb{D}^2$ with the swapped factors. 
Note that after this surgery we get back $\RR^3$.

It remains to construct a metric $g$ on $W$ with negative scalar curvature which 
is identical to the original metric on $V$ near its boundary.
The needed space $(W,g)$ can be found among wrap products $\mathbb{S}^1\times_f \mathbb{D}^2$ [see page~\pageref{page:warped product}].
\qeds


This construction was given by Joachim Lohkamp \cite[see][]{lohkamp},
he describes there yet an other equally simple construction.
In fact,
his constructions produce 
$\mathbb{S}^1$-invariant hollows 
with negative Ricci curvature.

On the other hand,
there are no hollows with 
positive scalar curvature and negative sectional curvature.
The former is equivalent to the positive mass conjecture \cite[see][and the references therein]{witten}
and the latter is an easy exercise.












%%%%%%%%%%%%%%%%%%%%%%%%%%%%%%%%%%%%%%%%%
\subsection*{Almost flat manifold\easy}
\label{almost-flat}

\emph{Nil-manifolds} form the minimal class of manifolds which includes a point, and has the following property:  
the total space of any principle $\mathbb{S}^1$-bundle over a nil-manifold is a nil-manifold. 

The nil-manifolds can be also defined as the quotients of a connected nilpotent Lie group by a lattice.

A compact Riemannian manifold $M$ is called $\eps$-flat if its sectional curvature at all points in all directions lie in the interval $[-\eps,\eps]$. 

The main theorem of Gromov in \cite{gromov-almost-flat}, 
states that for any positive integer $n$ there is $\eps>0$ such that any $\eps$-flat compact $n$-dimensional manifold with diameter at most $1$ admits a finite cover by a nil-manifold.
A more detailed proof can be found in \cite{buser-karcher}
and a more precise statement can be found in \cite{ruh}.

\begin{pr}
Given $\eps>0$ construct a compact Riemannian manifold $M$ of sufficiently large dimension which admits a Riemannian metric with diameter $\le 1$ and sectional
curvature $|K|<\eps$,
but does not admit a finite covering by a nil-manifold.
\end{pr}


%%%%%%%%%%%%%%%%%%%%%%%%%%%%%%%%%%%%%%%%%%%%%%%%%%
\parbf{Almost flat manifold.}
An example can be found among solve manifolds;
that is, quotients of solvable lie group by a lattice.
In fact torus bundles over circle circles are sufficient.

\medskip

A torus bundle $\TT^m\to E\to\mathbb{S}^1$ is obtained by taking $\TT^m\times [0,1]$ and gluing $\TT^m\times 0$ to $\TT^m\times 1$ along the map given by a matrix $A\in \SL(\ZZ,n)$.

The matrix $A$ has to meet two conditions.

On one hand, we need to make sure that the fundamental group of $E$ does not contain a nilpotent group of finite index.
This can be achieved by making at least one of eigenvalues of $A$ different from 1 by absolute value.

On the other hand the space $E$ has to admit a metric with small curvature and diameter, this can be achieved by making all 
eigenvalues of $A$ close enough to 1 by absolute value.
So the $n{\times}n$ matrix 
\[A=
\left(
\begin{matrix}
0&0&\dots&0&1&1
\\
1&0&\dots&0&0&0
\\
\vdots&\vdots&\ddots&\vdots&\vdots&\vdots
\\
0&0&\dots&0&0&0
\\
0&0&\dots&1&0&0
\\
0&0&\dots&0&1&0
\end{matrix}
\right).
\]
for large enough $n$ does the job;
its characteristic polynomial is 
\[\lambda^n+\lambda+1.\qedsin\] 
\medskip

\label{page-sol:almost-flat}
This example was constructed 
by Galina Guzhvina \cite[see][]{guzhvina}.

It is expected that for small enuf $\eps>0$,
a Riemannian manifold $(M,g)$ of any dimension 
with  $\diam(M,g)\le 1$ and $|K_g|\le \eps$ cannot be simply connected,
here $K_g$ denotes the sectional curvature of $g$.

The latter does not hold with the condition $K_g\le \eps$ instead.
In fact, 
for any $\eps>0$,
there is a metric $g$ on $\mathbb{S}^3$ 
with $K_g\le \eps$ and $\diam(\mathbb{S}^3,g)\le 1$.
This example was originally constructed by Mikhael Gromov in \cite{gromov-almost-flat}; 
a simplified proof was given by 
Peter Buser
and Detlef Gromoll in \cite{buser-gromoll}.












The space $P$ can be found among the spherical polyhedral spaces which admit
an isometric circle action $\mathbb S^1\acts P$ with geodesic orbits.
One of these orbits is the singular set.

\medskip



\begin{wrapfigure}{r}{19 mm}
\begin{lpic}[t(-0 mm),b(-2 mm),r(0 mm),l(0 mm)]{pics/deltoid(1)}
\lbl[b]{9,36;$n$}
\lbl[t]{9,1;$s$}
\lbl[r]{1.5,25;$w$}
\lbl[l]{17.5,25;$e$}
\lbl[t]{9.5,29.5;{\small $\tfrac{2{\cdot}\pi}p$}}
\lbl[b]{9.5,6.5;{\small $\tfrac{2{\cdot}\pi}q$}}
\lbl{5,24;{\small $\theta$}}
\lbl{14,24;{\small $\theta$}}
\end{lpic}
\end{wrapfigure}

First let us construct the quotient space $P'=P/\mathbb S^1$.
Consider the Hopf action $\mathbb S^1\acts\mathbb S^3$.
Let 
\[h\:\mathbb S^3\to\CP^1=\mathbb S^3/\mathbb S^1\] 
be the quotient map.
The complex projective line $\CP^1$ is isometric to the 2-sphere of radius $\tfrac12$.

Fix large relatively prime integers $p$ and $q$. 
Consider a solid deltoid $\Delta=[nesw]$ in $\CP^1$ with angles 
\begin{align*}
\measuredangle n&=2 \cdot\tfrac\pi p,
\\
\measuredangle s&=2\cdot\tfrac\pi q,
\\
\measuredangle e=\measuredangle w&=\theta<\pi.
\end{align*}

The quotient space $P'$ is obtained from $\Delta$ by gluing its boundary to itself along the reflection symmetry.
Note that $P'$ is homeomorphic to $\mathbb S^2$ and
it has 3 singular points with total angles $2\cdot\tfrac\pi p$,
$2\cdot\tfrac\pi q$ and $2\cdot\theta$.

The required $\mathbb S^1_{p,q}$-action on $\mathbb{S}^3\subset\CC^2$ is given by complex diagonal matrices $\left(\begin{smallmatrix}z^p&0\\0&z^q\end{smallmatrix}\right)$.
The orbits with isotropy groups $\ZZ_p$ and $\ZZ_q$ correspond to $n$ and $s$ in $P'$.
There is a metric $\rho$ on  $\mathbb{S}^3$ which is invariant with respect to the $\mathbb S^1_{p,q}$-action, which has $P'$  as the quotient space and geodesic orbits. 
(In fact such metric $\rho$ is unique up to isometry.)

This space $P=(\mathbb{S}^3,\rho)$ can be constructed the following way.
Set $W=h^{-1}(\Delta)$;
let $\tilde W$ be the universal metric cover of $W$.
The $\mathbb{S}^1$-action on $W$ lifts to $\RR$-action on $\tilde W$,
so the subgroup $2\cdot\pi\cdot\ZZ<\RR$ is the group of deck transformations of the covering $\tilde W\z\to W$. 
Consider the quotient $W'=\tilde W/a\cdot\ZZ$;
note that $W'$ is a solid torus with $\mathbb{S}^1$-action and the quotient isometric to $\Delta$.
For the right choice of the constant $a$, the gluing map $\Delta\to P'$ can be lifted to a gluing map boundary of $W'\to P$.



In the constructed example 
the singular points with total angles $2\cdot\tfrac\pi p$ and
$2\cdot\tfrac\pi q$
correspond to the points with isotropy groups $\ZZ_p$ and $\ZZ_q$ of the action.
The points in $P=(\mathbb{S}^3,\rho)$ on the orbits over these points will be regular; 
that is, they admit a neighborhoods isometric to open sets in the unit 3-sphere.
The singular locus $P_s$
of $P$ corresponds to the remaining singular point of $P'$.
Note that by construction,
\begin{itemize}
\item $P_s$ is a closed geodesic with angle $2\cdot\theta$ around it and $\theta$ can take arbitrary value close to $\pi$.
\item $P_s$ forms a $(p,q)$-torus knot in the ambient $\mathbb{S}^3$.
\end{itemize}
\qedsf






















Fix large relatively prime integers $p>q$.
Choose a regular $p$-gon $K_p$ and $q$-gon $K_q$ in $\CP^1$ with equal sides.
Consider their inverse images $W_p=h^{-1}K_p$ and $W_q=h^{-1}K_q$;
let $\tilde W_p$ and $\tilde W_q$ be their universal metric covers.

The spaces $\tilde W_p$ and $\tilde W_q$ admit natural isometric actions by groups $\RR\times\ZZ_p$ and $\RR\times\ZZ_q$ correspondingly.
It remains to find $\ZZ$ subactions on $\tilde W_p$ and $\tilde W_q$ such that the solid toruses $\tilde W_p/\ZZ$ and $\tilde W_q/\ZZ$ can be glued together long an isometry of their boundaries which sends orbits to orbits.

Consider the quotient space $\CP^1=\mathbb S^3/\mathbb S^1$;
it is isometric to the 2-sphere of radius $\tfrac12$.
Fix large relatively prime integers $p>q$. 
Consider the solid triangle $\Delta=[nsw]$ with angles $\measuredangle n=\tfrac\pi p$, $\measuredangle s=\tfrac\pi q$ and $\measuredangle w=\pi\cdot(1-\tfrac1 p)$ in $\CP^1$.
Denote by $\hat \Delta$ the  doubling of $\Delta$ along  its boundary.
Note that $\hat \Delta$ is homeomorphic to $\mathbb S^2$ and
it has 3 singular points with total angles $2\cdot\tfrac\pi p$,
$2\cdot\tfrac\pi q$ and $2\cdot\pi\cdot(1-\tfrac1 p)$.

Consider the quotent map  $\mathbb S^3\to \mathbb S^3/\mathbb S^1$ for the $\mathbb S^1$-action on $\mathbb{S}^3\subset\CC^2$ by the diagonal matrices $\left(\begin{smallmatrix}z^p&0\\0&z^q\end{smallmatrix}\right)$, $z\in\mathbb S^1\subset\CC$.
Identify the quotient space $\mathbb S^3/\mathbb S^1$ with $\hat \Delta$ such that the singular orbits mapped to $v$ and $w$ correspondingly.
Let us equip $\mathbb S^3$ with a metric such that the 

Note that there is a spherical polyhedral metric $\rho$ on  $\mathbb S^3$
such that the $\mathbb S^1$-orbits become geodesics 
and the quotient space $(\mathbb S^3,\rho)/\mathbb S^1$
is isometric to $\hat \Delta$.

To construct such metric, take the inverse image $W$ of $\Delta$ under the Hopf fibration $\mathbb S^3 \to \CP^1$.
The universal cover $\tilde W$  of $W$ admits a natural $\RR$-action with geodesic orbits.
For the right choice of $\ZZ$ subaction of $\RR$,
the quotient space 
















%%%%%%%%%%%%%%%%%%%%%%%%%%%%%%%%%%%%%%%%%
\subsection*{Function with no critical points}

\begin{pr}{}{Function with no critical points}\label{Function with no critical points}
Given an integer $m\ge 2$, 
construct a smooth function $f\:\RR^m\to \RR$ 
with no critical points in the unit ball $B^m$ 
such that the restriction $f|_{B^m}$ does not factor thru a linear function;
that is, 
$f|_{B^m}$ cannot be presented as a composition
$\ell\circ\phi$,
where $\ell\:\RR^m\to\RR$ is a linear function 
and $\phi\:B^m\to\RR^m$ is a smooth embedding.
\end{pr}


%%%%%%%%%%%%%%%%%%%%%%%%%%%%%%%%%%%%%%%%%%%%%%%%%%
\begin{wrapfigure}{o}{24 mm}
\begin{lpic}[t(-0 mm),b(0 mm),r(0 mm),l(0 mm)]{pics/no-critical-points()}
\end{lpic}
\end{wrapfigure}


\parbf{Function with no critical points.}
Construct an immersion 
$\psi\:B^m\z\looparrowright\RR^m$ such that 
\[\ell\circ\phi\ne\ell\circ\psi\]
for any embedding  $\phi\:B^m\to\RR^m$. 
The two-dimensional case can be guessed from the picture.

It remains to note that the composition $f=\ell\circ\psi$ has no critical points.\qeds

The problem was suggested by Petr Pushkar.














\parbf{A family of sets with no section.}
Identify $\mathbb{S}^1$ with $[0,1]/(0\sim 1)$.
Given $t\in[0,\tfrac12]$,
Consider the set $K_t\subset \mathbb{S}^1$
formed by all possible sums $\sum_{n=1}^\infty a_n\cdot t^n$,
where $a_n$ is $0$ or $1$.

Note that $K_{t}$ is a Cantor set and $K_{\frac12}=\mathbb{S}^1$.

Denote by $\rho_t\:\mathbb{S}^1\to\mathbb{S}^1$ 
the counterclockwise rotation by angle $\frac1{1-2\cdot t}$.
Set 
\[Z_t=\left[\begin{aligned}
             Z_t&=\rho_t(K_t)&&\text{if}\ t\in[0,\tfrac12),
\\
Z_{t}&=\mathbb{S}^1&&\text{if}\ t=\tfrac12.
            \end{aligned}
\right.
\]

Note that the Hausdorff distance from $K_t$ to $K_{\frac12}=\SS^1$ converges to $0$ as $t\to\tfrac12$.
Therefore $Z_t$ is a continuous family.

Any continuous section $z(t)\in Z_t$ for $t\in[0,\tfrac12)$ can be described as
\[z(t)=\rho_t\left(\sum_{n=1}^\infty a_n\cdot t^n\right)\]
for a fixed sequence $(a_n)$.
None of these can be extended continuousely to $[0,\tfrac12]$.\qeds












Let us denote by $\mathrm{N} M$ and $\T M$ the normal and tangent bundle of $M$ in the Euclidean space $\RR^n$.
Assume $m=\dim M$.

Consider the normal exponential map $\exp_M\:\mathrm{N} M\to\RR^n$ [defined on page~\pageref{page:Normal exponential map}]. Denote by $J_v$ the Jacobian of $\exp_M$ at $v\in \mathrm{N}M$.
Note that for all small $r>0$, we have
\[\vol B(M,r)=\int\limits_M d_p\vol_m\cdot\int\limits_{B(0,r)_{\mathrm{N}_p}}J_v\cdot d_v\vol_{n-m},\]
where $B(0,r)_{\mathrm{N}_p}$ denotes the ball in the fiber $\mathrm{N}_p\subset \mathrm{N} M$ at $p\in M$.

Fix $p\in M$.
Note that it is sufficient to show that following.
\begin{itemize}
\item[$({*})$] The integral
\[V(r)=\int\limits_{B(0,r)_{\mathrm{N}_p}}J_v\cdot d_v\vol_{n-m}\]
can be written in terms of $m$, $n$, $r$ and the Riemannian curvature tensor $R$ of $M$ at $p$.
\end{itemize}
In fact $V(r)$ is a polynomial in $r$ and its coefficient are up to a coefficient depending on $m$ and $n$ equal to the so called Lipschitz--Killing curvatures at $p$.

Denote by $s\:\T_p\times \T_p\to \mathrm{N}_p$
the second fundamental form of $M$ [defined on page~\pageref{page:second fundamental form}].
Recall that the curvature tensor $R$ of $M$ at $p$ can be expressed the following way
\[\langle R(X, Y) V, W\rangle 
=\langle s(X,W), s(Y,V)\rangle-\langle s(X,V), s(Y,W)\rangle.\]

Fix a orthonormal basis $e_1,\dots, e_{n-m}$ in $\mathrm{N}_p$.
Consider the real-valued quadratic forms  $s_i(X,Y)=\<s_p(X,Y),e_i\>$.
Consider the tensors $R_i$ defined as 
\[\langle R_i(X, Y) V, W\rangle 
=s_i(X,W)\cdot s_i(Y,V)-s_i(X,V)\cdot  s_i(Y,W).\]
Note that 
\[R=R_1+\dots+R_{n-m}.\leqno{({*}{*})}\]

Assume $v=r\cdot e_i$ for some $r\in \RR$.
Denote by $k_{1},\dots, k_{m}$ the eigenvalues of the real-valued quadratic form  $s_i(X,Y)$.
By straightforward calculations we get 
\begin{align*}
J_{v}&=r^{n-m-1}\cdot(1+k_1\cdot r)\cdots(1+k_n\cdot r),
\\
J_{-v}&=r^{n-m-1}\cdot(1-k_1\cdot r)\cdots(1-k_n\cdot r).
\end{align*}
It follows that the sum $J_{v}+J_{-v}$ is a polynomial in $r$ and its coefficient 
symmetric polynomials of $Q_j(k_1,\dots, k_n)$; more over each monomial of $Q_j$ is a product of even number of distinct $k_i$'s.
Since each product $k_a\cdot k_b$ is the sectional curvature in the tensor $R_i$,
we get that the coefficient $Q_j(k_1,\dots, k_n)$ can be expressed in terms of the tensor $R_i$.

Using $({*}{*})$, we can sum up by $i$, we get that
\[\sum_i(J_{r\cdot e_i}+J_{-r\cdot e_i})\]
can be expressed in terms of $r$, $m$, $n$ and $R$.
Taking average for all orthonormal frames $e_1,\dots, e_{n-m}$ in $\mathrm{N}_p$
we get the needed claim.





Show that for small $r$ the integral
\[v(r)=\int\limits_{B(0,r)_{\mathrm{N}_pM}}J_V\cdot d_V\vol_{n-m}\]
is a polynomial 
of $r$ and its coefficients can be expressed in terms of the curvature tensor $R_p$.

It follows that the right hand side in $({*})$ can be expressed in terms of curvature tensor of $M$.
The problem follows since the curvature tensor can be expressed in terms of metric tensor of $M$.\qeds


The formula for volume of tubular neighborhood 
was given by Hermann Weyl in \cite{weyl}.
























%%%%%%%%%%%%%%%%%%%%%%%%%%%%%%%%%%%%%%%%%
\subsection*{Bounded geometry}

Denote by $\mathcal{R}$ the space of 
all Riemannian metrics on $\mathbb S^5$
with absolute value of sectional curvature $\le1$,
and injectivity radius $\ge1$.

It is easy to see that any metric $g_1\in \mathcal{R}$ 
can be connected to the canonical metric $g_0$ on $\mathbb S^5$
by a continuous family of metrics $g_t\in \mathcal{R}$ where $t\in[0,1]$.
In fact, the one parameter family of metrics $g_t$
can be found among the metrics of the type 
\[g_t=a(t)\cdot g_0+b(t)\cdot g_1,\]
where $a,b\:[0,1]\to\RR$
are smooth functions such that $a(0)=1=b(1)$ and $a(1-s)=0=b(s)$ for $s\le \tfrac13$.
In order to keep the bounds on the curvature and injectivity radius,
the functions $a$ and $b$
have to take huge values in the middle of interval.

\begin{pr}{\thm}{Bounded curvature}\label{Bounded curvature}
Fix a fast growing function, say
\[\Phi(x)=1000^{1000\cdot (x+1000)}.\]

Show that there is a metric $g_1$ 
such that 
for any family $g_t$ as above
\[\max_{t\in[0,1]}\{\vol(g_t)\}
>
\Phi(\vol(g_1)).\]
\end{pr}

The expected solution requires Novikov theorem on the algorithmic undecidability of the problem of recognition of the sphere $\mathbb S^m$  for
$m\ge 5$. 
A detailed proof of this theorem can be found in \cite{nabutovsky-NovThm}.


%%%%%%%%%%%%%%%%%%%%%%%%%%%%%%%%%%%%%%%%%%%%%%%%%%
\parbf{Bounded geometry.}
Show that there is an algorithm to estimate the Gromov--Hausdorff distance between two Riemannian manifolds given in any reasonable way.

Show that if two manifolds with bounded curvature are sufficiently close to each other then they are diffeomorphic.

Now assume contrary;
that is, for any metric $g_1\in \mathcal{R}$ there is a path $g_t$ in $\mathcal{R}$
connecting the canonical metric $g_0$ to $g_1$ such that 
\[\vol g_t\le \Phi(\vol g_1).\]

If a 5-dimensional Riemannian manifold $M$ with curvature between $\pm1$ is diffeomorphic to $\mathbb{S}^5$ 
then it can be described by a metric $g_1$ in $\mathcal{R}$.
Let $g_t$ be the path as above.

Construct a finite set $F\subset \mathcal{R}$ 
which is sufficiently dense in the set of metrics in $\mathcal{R}$ with volume at most $\Phi(\vol g_1)$.
The path $g_t$ as above can be approximated by a sequence of metrics from $F$.

Using the algorithms above one can list all such sequences.
It implies existence of algorithm 
which recognize $\mathbb S^5$ among $5$-dimensional manifolds.
The later contradicts Novikov theorem.
\qeds

Instead of function $\Phi$ one can take any Turing computable function.
The problem and number of its generalizations 
are due to Alexander Nabutovsky \cite[see][]{nabutovsky-Disconnectedness}.



%%%%%%%%%%%%%%%%%%%%%%%%%%%%%%%%%%%%%%%%%%%%%%
{

\begin{wrapfigure}{r}{21 mm}
\begin{lpic}[t(-0 mm),b(-4 mm),r(0 mm),l(0 mm)]{pics/long-geodesic(1)}
\end{lpic}
\end{wrapfigure}

\subsection*{Long geodesic}

Recall that a closed curve 
 called \index{simple closed curve}\emph{simple} if it has no self-intersections.

\begin{pr}{}{Long geodesic}\label{Long geodesic}
Assume that the surface $\Sigma$ of a convex body $B$ in $\RR^3$
admits an arbitrary long simple closed geodesic.
Show that $B$ is a tetrahedron with equal opposite sides.
\end{pr}

}

A tetrahedron with equal opposite edges is called \emph{isosceles} it has many interesting properties.

Let us mention couple of theorems about intrinsic metric on convex surfaces which should help to solve this problem.
These theorems can be proved easily for smooth or polyhedral surfaces
and then the general case can be done by approximation.
A very short but comprehensive introduction to the subject was written by Alexander Alexandrov in \cite{alexandrov1941}.

On a convex surface (not necessary smooth) one could define 
so
called \index{curvature measure}\emph{curvature measure} which we denote further by $\kappa$.
It is the (necessary unique) non-negative measure such that for any triangle $\triangle$, we have
\[\kappa(\triangle)=\alpha+\beta+\gamma-\pi,\] 
where $\alpha$, $\beta$ and $\gamma$ are the angles of $\triangle$, measured in the intrinsic metric of the surface.

For curvature measure, an analog of Gauss--Bonnet formula holds;
in particular for any closed convex surface $\Sigma$ in $\RR^3$
\[\kappa(\Sigma)=4\cdot\pi\]
and 
if a closed geodesic cuts a disc $D$ from $\Sigma$ then
\[\kappa(D)=2\cdot\pi.\]


%???+PIC

Further, given a triangle $\triangle$ in a metric space,
its model triangle $\tilde \triangle$ is defined as a triangle in the plane with the same side lengths.
The angles $\tilde\alpha$, $\tilde\beta$ and $\tilde\gamma$ of the model triangle are called \index{model angle}\emph{model angle}
of triangle.
The comparison theorem states that for any triangle in a surface with non-negative curvature measure its model angles do not exceed the actual angles; that is,
\[\tilde\alpha\le \alpha,\quad
\tilde\beta\le\beta,\quad
\tilde\gamma\le\gamma.\]
The same holds for the area; that is,
\[\area\tilde\triangle\le \area\triangle.\]

%%%%%%%%%%%%%%%%%%%%%%%%%%%%%%%%%%%%%%%%%%%%%%%%%%
\parbf{Long geodesic.}
By cutting the surface $\Sigma$ along a sufficiently long closed simple geodesic,
we get two discs.
The key step is to show that each of these discs 
is long and thin.

\medskip

Let $D$ be one of these discs.
Equip it with the intrinsic metric further denoted by $|{*}-{*}|_D$.


Since $\Sigma$ has non-negative curvature in the sense of Alexandrov,
so does $D$.
Choose a pair of points $p,q\in\partial D$ which maximize the distance $|p-q|_D$.
Clearly,
\[|x-p|_D,|x-q|_D\le |p-q|_D\] 
for any other point $x\in\partial D$.
By comparison, 
\[\measuredangle[x\,^p_q]\ge \tfrac\pi3.\leqno({*})\]



\begin{wrapfigure}{r}{28 mm}
\begin{lpic}[t(-0 mm),b(-0 mm),r(0 mm),l(0 mm)]{pics/long-geodesic-diam(1)}
\lbl[r]{0,8;$p$}
\lbl[l]{28,8;$q$}
\lbl[b]{10,16;$\gamma_1(t)$}
\lbl[l]{12,9,-60;$\ge\tfrac\pi3$}
\end{lpic}
\end{wrapfigure}

The points $p$ and $q$ divide $\partial D$ into two arcs,
say $\gamma_1$ and $\gamma_2$;
let us parametrize them by arclength from $p$ to $q$. 
Then by $({*})$
\[\tfrac{d}{dt}\left(|x-\gamma_i(t)|_D-|x-\gamma_i(t)|_D\right)
\ge
\tfrac12.\]
In particular
\[|p-q|_D\ge \tfrac18{\cdot}\length[\partial D].\]
In particular, if $\partial D$ is long 
then $D$ has large diameter.

Choose two points $x\in \gamma_1$ and $y\in\gamma_2$ sufficiently close to $p$ such that $|x-q|_D=|y-q|_D$.
By comparison 
\[\area \tilde \triangle qxy\le \area \triangle qxy\le \area \Sigma.\]
It follows that 
\[\begin{aligned}|x-y|_D&\le2\cdot\frac{ \area[\tilde\triangle(xyq)]}{|q-x|_D}
\le 
\\
&\le 
100\cdot\frac{ \area\Sigma}{\length[\partial D]}.
\end{aligned}
\leqno({*}{*})\]

Cut $D$ along a minimizing geodesic $[xy]$ into two lenses.
Denote by $L_p$ the lens with the point $p$ in it.
Note that the curvature of $L_p$ is $\alpha+\beta$, where $\alpha$ and $\beta$ the angles as on the diagram.
By comparison $\alpha\ge \tilde\measuredangle(x\,^p_y)$ 
and $\beta\ge \tilde\measuredangle(y\,^p_x)$.
Therefore curvature of $L_p$ is at least $\pi-\tilde\measuredangle(p\,^x_y)$.
In particular, if $|x-y|_D$ much less then $|p-x|_D+|p-y|_D$ then the curvature of $L_p$ is $\pi$ or just a little less.

Fix $\eps>0$.
If $\length[\partial D]$ is long enuf,
by $({*})$, 
we can find a lens $L_p$ with diameter at most $\eps$,
such that curvature $L_p$ is at least $\pi-\eps$.

Using the same construction for $p$ and $q$ in the disc $D$,
and for the two points in the other disc,
we get four lenses in $\Sigma$ each of diameter at most $\eps$ and each with curvature at least $\pi-\eps$.

\begin{center}
\begin{lpic}[t(-0 mm),b(-0 mm),r(0 mm),l(0 mm)]{pics/long-geodesic-D(1)}
\lbl{46,6;$D$}
\lbl{6,6;$L_p$}
\lbl[r]{0,6;$p$}
\lbl[l]{92.5,6;$q$}
\lbl[b]{19,11.5;$x$}
\lbl[t]{12.5,.5;$y$}
\lbl[b]{46,12;$\gamma_1$}
\lbl[t]{46,-.5;$\gamma_2$}
\lbl[tr]{16.5,9;{\small $\alpha$}}
\lbl[br]{12.5,4.5;{\small $\beta$}}
\end{lpic}
\end{center}

By Gauss--Bonnet formula, the total curvature of $\Sigma$ is $4\cdot\pi$.
Since $\eps>0$ is arbitrary, we get that there are four singular points in $\Sigma$, each with curvature $\pi$
and the remaining part of $\Sigma$ is flat.

\begin{wrapfigure}{o}{21 mm}
\begin{lpic}[t(-4 mm),b(-3 mm),r(0 mm),l(0 mm)]{pics/akopyan(1)}
\end{lpic}
\end{wrapfigure}

It remains to show that any surface with this property 
is isometric to the surface of a tetrahedron with equal opposite edges.
To do this cut $\Sigma$ along three geodesics which connect one singular point to the remaining three,
develop the obtained flat surface on the plane and think;
also look at the diagram.\qeds

I learned the problem from Arseniy Akopyan,
the case of polyhedra was solved by Vladimir Protasov in \cite{protasov}.















%%%%%%%%%%%%%%%%%%%%%%%%%%%%%%%%%%%%%%%%%%%%%%
\subsection*{The last problem of Poincar\'e\hard}

\begin{pr}{\hard}{The last problem of Poincar\'e}\label{The last problem of Poincare}
Let $f\:\CC\to \CC$ be an area preserving homeomorphism
such that 
\[f(z)
=
\left[
\begin{aligned}
&z-i&&\text{if}&&\Re(z)\le -1,
\\
&z+i&&\text{if}&&\Re(z)\ge 1.
\end{aligned}
\right.
\] 
and $f(z+i)=f(z)+i$ for any $z\in\CC$.

Show that $f$ has a fixed point.
\end{pr}

\parbf{The last problem of Poincar\'e.}
Set 
\begin{align*}
H_+&=\set{z\in\CC}{\Re(z)\ge 1},
\\
H_-&=\set{z\in\CC}{\Re(z)\le -1}.
\end{align*}

Assume $f$ has no fixed points;
in other words the image of the map 
\[\phi\:z\mapsto f(z)-z\] 
lies in $\CC^*=\CC\backslash\{0\}$.

\begin{wrapfigure}{r}{44 mm}
\begin{lpic}[t(-0 mm),b(0 mm),r(0 mm),l(0 mm)]{pics/birkhoff(1)}
\lbl[r]{33,44.5;{\small $\hat\gamma(0)$}}
\lbl[r]{31,57.5;{\small $\hat\gamma(1)$}}
\lbl[rb]{6.5,63;{\small $\hat\gamma(n)$}}
\lbl[l]{10.3,28.3;{\small $\check\gamma(0)$}}
\lbl[l]{12.4,18.6;{\small $\check\gamma(1)$}}
\lbl[l]{37,19;{\small $\check\gamma(n)$}}
\lbl[b]{21.5,35.5;$0$}
\lbl[br]{33.5,35.5;$1$}
\lbl{40,3;$H_+$}
\lbl{3,3;$H_-$}
\end{lpic}
\end{wrapfigure}


Fix $\eps>0$ such that $|f(z)-z|>\eps$ for any $z\in\CC$.
Note that the map 
\[\check f\:z\mapsto f(z)+\eps\]
is area preserving and has no fixed points.


Prove that for some positive integer $n$,
there is a curve 
\[\check \gamma\:[0,n]\to \CC\]
which starts in $H_-$, ends in $H_+$
and 
$\check f\circ\check\gamma(t)=\check\gamma(t+1)$
for any $t\z\in [0,n-1]$.

Repeat the same construction for the function 
\[\hat f(z)=f(z)-\eps\] 
and obtain a curve 
\[\hat \gamma\:[0,m]\to \CC\] starting in $H_+$ and ending in $H_-$.

Connect $\check\gamma(n)$ to $\hat \gamma(0)$ by a curve in $H_+$ 
and 
$\hat\gamma(m)$ to  $\check\gamma(0)$ by a curve in $H_-$.
Denote by $\sigma$ the obtained loop.

Prove that
\begin{itemize}
\item The loop $\phi\circ\sigma$ has to be null-homotopic in $\CC^*$.
\item The loop $\phi\circ\sigma$ is a generator of $\pi_1\CC^*$.
\end{itemize}

\noindent
These two statements contradict each other. \qeds


The question was asked by Henri Poincar\'e \cite[see][]{poincare}
and answered by George Birkhoff in \cite{birkhoff}.



%%%%%%%%%%%%%%%%%%%%%%%%%%%%%%%%%%%%%%%%%%%%%%%%%%
%%%%%%%%%%%%%%%%%%%%%%%%%%%%%%%%%%%%%%%%%%%%%%%%%
%%%%%%%%%%%%%%%%%%%%%%%%%%%%%%%%%%%%%%%%%%%%%%%%%%
\begin{wrapfigure}{r}{40 mm}
\begin{lpic}[t(-0 mm),b(-0 mm),r(0 mm),l(0 mm)]{pics/serpinski-cemetery(1)}
\end{lpic}
\end{wrapfigure}


\subsection*{Fat curve}

\label{Fat curve}

\begin{pr}{\easy}{Fat curve}
Construct a simple plane curve with positive Lebesgue measure.
\end{pr}



\parbf{Fat curve.}
Modify your favorite space filling curve 
keeping its area nearly the same 
and removing the self-intersections.

It works for the \index{Sierpi\'nski curve}\emph{Sierpi\'nski curve} 
which can be constructed as a limit of 
recursively defined sequence of curves;
the 8th iteration is on the diagram.\qeds 

This example was constructed 
by William Osgood \cite[see][]{osgood}.

%Knopp, K. (1917), "Einheitliche Erzeugung und Darstellung der Kurven von Peano, Osgood und von Koch", Archiv der Mathematik und Physik, 26: 103--115.
%Lebesgue, H. (1903), "Sur le probl\`eme des aires", Bulletin de la Soci\'et\'e Math\'ematique de France , 31: 197--203








%%%%%%%%%%%%%%%%%%%%%%%%%%%%%%%%%%%%%%%%%
\subsection*{Characterization of polytope}

\begin{pr}{}{Characterization of polytope}
\label{conic neighborhoods}
Let $P$ be a compact subset of the Euclidean space.
Assume for every point $x\in P$
there is a cone $K_x$ with a tip at $x$ and $\eps>0$
such that 
$$B(x,\eps)\cap P
=
B(x,\eps)\cap K_x.$$
Show that $P$ is a polytope; 
that is, $P$ is a union of finite collection of simplices.
\end{pr}
%%%%%%%%%%%%%%%%%%%%%%%%%%%%%%%%%%%%%%%%%%%%%%%%%%
\parbf{Characterization of polytope.}
Arguing by contradiction, let us assume that $P\subset \RR^m$
is a counterexample and $m$ takes minimal possible value.

Choose a finite cover $B_1,B_2,\dots B_n$ of $K$,
where $B_i=B(z_i,\eps_i)$ 
and $B_i\cap P=B_i\cap K_i$, 
where $K_i$ is a cone with the tip at $z_i$.

For each $i$, consider function $f_i(x)=|z_i-x|^2-\eps_i^2$.
Note that
\[W_{i,j}=\set{x\in\RR^m}{f_i(x)=f_j(x)}\]
is a hyperplane for any pair $i\ne j$.

The subset $P_{i,j}=P\cap W_{i,j}$ satisfies the same assumptions as $P$, but lies in a hyperplane.
Since $m$ is minimal, we get that $P_{i,j}$ is a polytope for any pair $i,j$.

Consider Voronoi domains 
\[V_{i}=\set{x\in\RR^m}{f_i(x)\ge f_j(x) \ \text{for any}\ j}.\]
Note that $P\cap V_i$ is formed by the points which lie on the segments from $z_i$ to a point in  $P\cap \partial V_i$.

The statement follows since $\partial V_i$ is covered by the hyperplanes $W_{i,j}$.
\qeds

This problem is mentioned 
by Nina Lebedeva and me in \cite{lebedeva-petrunin}.














%%%%%%%%%%%%%%%%%%%%%%%%%%%%%%%%%%%%%%%%%
\subsection*{Thin triangles\easy}

A triangle in a metric space is defined as three minimizing  geodesics connecting three distinct points.
A model triangle is defined as a flat triangle with the same side lengths.
If the distance between any two points in the triangle is less or equal to the distance between corresponding points in 
A triangle is called thin if the distance between any pair of


\begin{pr}{\easy}{Unique geodesics imply $\mathrm{CAT}(0)$}\label{Unique geodesics imply CAT}
Let $P$ be a polyhedral space.
Assume that any two points in $P$ 
are connected by unique geodesic.
Show that $P$ is a $\mathrm{CAT}(0)$ space.
\end{pr}


%%%%%%%%%%%%%%%%%%%%%%%%%%%%%%%%%%%%%%%%%%%%%%%%%%
\parbf{Unique geodesics imply $\mathrm{CAT}(0)$.}
Uniqueness of geodesics implies that $P$ is contractible.
In particular, $P$ is simply connected.

It remains to prove that $P$ is locally $\mathrm{CAT}(0)$;
equivalently, the space of directions $\Sigma_p$
at any point $p\in P$ is  a $\mathrm{CAT}(1)$ space.

We can assume that the statement holds in all dimensions less than $\dim P$. 
In particular, $\Sigma_p$ is locally $\mathrm{CAT}(1)$.
If $\Sigma_p$ is not $\mathrm{CAT}(1)$,
then it contains a periodic geodesic $\gamma$ of length $\ell<2\cdot\pi$,
such that any arc of $\gamma$ of length $\tfrac\ell2$ is length minimizing.

Consider two points $x$ and $y$
in the tangent cone of $p$
in directions $\gamma(0)$ and $\gamma(\tfrac\ell2)$.
Show that there are two distinct minimizing geodesics between $x$ and $y$.
The latter leads to a contradiction.
\qeds

The existence of geodesic $\gamma$ was proved by Brian Bowditch in \cite{bowditch};
a simpler proof can be found in the book 
by Stephanie Alexander, Vitali Kapovitch and me \cite[see][]{akp}.















\begin{pr}{\thm}{Convex space-filling}\label{Convex space-filling}
Construct a curve $f\:[0,1]\to \RR^2$
such that 
the image $K_t=f([0,t])$ is convex set with nonempty interior
for any $t>0$. 
\end{pr}

%%%%%%%%%%%%%%%%%%%%%%%%%%%%%%%%%%%%%%%%%%%%%%%%%%
\parbf{\ref{Convex space-filling}.}
\textit{Convex space-filling.}
Let us write the boundary of $K_t$
in polar coordinates $(\theta,\rho_t(\theta)$.
Note that convexity of $K_t$ will follow if the function $\rho_t$ is almost constant and have small second derivative.

We will assume $\gamma(0)$ is the origin;
that is, $\rho_0(\theta)=0$ for any $\theta$.

Assume $t\le t'$.
Then $K_t\subset K_{t'}$ 
therefore 
$\rho_t(\theta)\le \rho_{t'}(\theta)$;
that is, $\rho_t$ is nondecresing in $t$.
Set $W(t,t')=\supp(\rho_t'-\rho_t)$.
and let $\ell(t,t')$ be the length of minimal inteval in $\SS^1$ containing $W(t,t')$.

Note that it is enuf to construct one parameter family of almost constant functions $\rho_t\:\mathbb{S}^1\to \RR_\ge$
with small second derivative,
nondecresing in $t$ and $\ell(t,t')\to 0$ as $|t'-t|\to 0$.








%\begin{pr}{\hard}{Besicovitch's set}\label{Besicovitch's set}
%Show that one can cut a unit plane disc along a finite number of radii into sectors and then move each sector by a parallel translation in such a way that their union have arbitrary small area.
%\sign{\cite{besicovitch}}
%\end{pr}







The following well known folklore problem is relevant
(it looks like a problem in calculus, but in fact it is geometry).

\begin{itemize}
\item \textit{ 
A cowboy is at the bottom of a mountain of frictionless ice in the shape of a perfect cone with a circular base. 
So, he throws up his lasso which slips neatly over the top of the cone, he pulls it tight and starts to climb. If the mountain is very steep, with a narrow angle at the top, there is no problem; the lasso grips tight and up he goes. On the other hand if the mountain is very flat, with a very shallow angle at the top, the lasso slips off as soon as the cowboy pulls on it. 
What is the critical angle at which the cowboy can no longer climb the ice-mountain?}
\end{itemize}


\begin{pr}{\easy}{Bodies with the same of shadows}\label{Bodies with the same of shadows}
Two convex bodies $K_1$ and $K_2$ in Euclidean 3-space are said to have the same shadows if any shape which can appear as an orthogonal projection of $K_1$ can also appear as an orthogonal projection of $K_2$ and the other way around.

Construct two non-congruent convex bodies $K_1$ and $K_2$ which have the same shadows.
\end{pr}

%%%%%%%%%%%%%%%%%%%%%%%%%%%%%%%%%%%%%%%%%%%%%%%%%%
\parbf{\ref{Bodies with the same of shadows}.} 
\textit{Bodies with the same of shadows.}
Let $B$ be the unit ball in $\RR^3$ centered at the origin.

Fix small $\eps>0$.
Consider two bodies 
\begin{align*}
B''&=\set{(x,y,z)\in B}{x\le 1-\eps,\  y\le 1-\eps},
\\ 
B'''&=\set{(x,y,z)\in B}{x\le 1-\eps,\  y\le 1-\eps,\  z\le 1-\eps}.
\end{align*}
Prove that $B''$ and $B'''$ have the same shadows.

\parit{Comments.} The question was asked by Joel Hamkins and answered by Sergei Ivanov, see \cite{hamkins}.









\begin{pr}{}{Disc and 2-sphere}\label{2-sphere is far from a ball}
Show that there is no sequence of Riemannian metrics on
$\mathbb{S}^2$ which converge to the unit disc in the sense of Gromov--Hausdorff.
\end{pr}
%%%%%%%%%%%%%%%%%%%%%%%%%%%%%%%%%%%%%%%%%%%%%%%%%%
\parbf{\ref{2-sphere is far from a ball}.} 
\textit{Disc and 2-sphere.}
Assume contrary, let $(\mathbb{S}^2,g)$ is sufficiently close to $B^2$.

Choose a closed simple curve $\gamma$ in $\mathbb{S}^2$ which is close to the boundary of $B^2$.
Choose two points $p_1$ and $p_2$ in $\mathbb{S}^2$ 
on the opposite sides of $\gamma$ which are sufficiently close to the center of $B^2$.

On one had $p_1$ and $p_2$ have to be close in $\mathbb{S}^2$.
On the other hang, to get from $p_1$ to $p_2$ in $\mathbb{S}^2$,
one has to cross $\gamma$.
Hence the distance from $p_1$ to $p_2$ in $\mathbb{S}^2$ has to be about $2$,
a contradiction.

\parit{Comments.}
In fact if $X$ is a Gromov--Hausdorff limit of $(\mathbb{S}^2,g_n)$
then any point $x_0\in X$ either admits a neighborhood homeomorphic to $\RR^2$ or it is a cut point;
that is, $X\backslash\{x_0\}$ is disconnected \cite[see][3.32]{gromov-MetStr}.





\comment{For the most of problems in this section 
it is enuf to know
definitions of curvature of curves,
second fundamental form,
Gauss curvature of surfaces,
Gauss--Bonnet theorem.
To solve Problem~\ref{Two discs}, 
it is better to know Hairy Ball theorem.
}

\comment{For most of the problems it is enuf to know second variation formula. 
Knowledge of  
O'Neil formula, 
Gauss formula, 
Gauss--Bonnet formula, 
Toponogov's comparison theorem, 
Soul theorem, 
Toponogov's splitting theorems 
and Synge's lemma 
also might help.
We suggest to solve problem~\ref{Immersed surface}
before solving problem~\ref{Immersed ball}.
To solve problem~\ref{kleiner-hopf}, 
it is better know that the quotient of positively curved Riemannian manifold 
by an isometry group is a positively curved Alexandrov space.
Problem \ref{scalar-curv} requires Liouville's theorem for geodesic flow.
}

\comment{A solution of problem \ref{gromomorphic-curves} relies on Gromov's pseudo-holomorphic curves. The problem \ref{Volume and convexity} uses Liouville's theorem for geodesic flow.
A solution of the problem \ref{short-homotopy} use a curve shortening process.
To solve problem \ref{Bounded curvature}, one needs to know Novikov theorem on  the algorithmic undecidability of the problem of recognition of the sphere $\mathbb S^m$ for
$m\ge 5$.
}

\comment{The necessary definitions can be found in \cite{bbi}. 
It is very hard to do \ref{compact} without using Kuratowski embedding.
To do problem  \ref{sub-Riemannian} you should be familiar with the proof of Nash--Kuiper theorem.
For the problem \ref{two2one} you have to know Rademacher's theorem on differentiability of Lipschitz maps.
To solve the problem \ref{hom-near-QI} one has to know theorems on deformation of homeomorphisms.}

\comment{One of the solutions of \ref{box-in-box} uses mixed volumes.
In order to solve problem \ref{Harnack}, it is better to know what is the genus of complex curve of degree $d$.
To solve problem \ref{2pts-on-line} one has to use axiom of choice.
In order to solve problem \ref{Right-angled polyhedron},
it is better to know \hyperref[Dehn--Sommerville equations]{\emph{Dehn--Sommerville equations}}, see an outline on page \pageref{Dehn--Sommerville equations}.}


\comment{Before solving problem \ref{conic neighborhoods}, it is better to learn what is Delaunay triangulation.}




















\begin{pr}{}{Deformation to a product}\label{Deformation to a product} 
Let $(M,g)$ be a compact Riemannian manifold with non-negative sectional curvature. 
Show that there is a continuous one parameter family of non-negatively curved metrics $g_t$ on $M$, $t\in[0,1]$, 
such that $g_0=g$ and a finite Riemannian cover of $(M,g_1)$ is isometric to a product of a flat torus and a simply connected manifold.
\end{pr}

%%%%%%%%%%%%%%%%%%%%%%%%%%%%%%%%%%%%%%%%%%%%%%%%%%
\parbf{\wrenches\ref{Deformation to a product}.} 
\textit{Deformation to a product.} 
Denote by $\Gamma$ the fundamental group of $M$.

Let $(\tilde M,\tilde g)$ be universal cover of $(M,g)$ with induced Riemannian metric.
The space $(\tilde M,\tilde g)$ is isometric to a product $\RR^k\times K$, 
where $k$ is nonnegative integer and $K$ is a compact Riemannian manifold.

Denote by $G$ the isometry group of $K$.
Given a continuous one parameter family of homomorphisms $\phi_t\:\RR^k\to G$,
consider the one parameter family of diffeomorphisms of $\RR^k\times K$ to itself defined as
\[\Phi_t\:(x,k)\mapsto (x,\phi_t(x)\cdot k).\]
Denote by 
 $\tilde g_t$ pullback 
of $\tilde g$ via $\Phi_t$,
so 
\[\Phi_t\:(\RR^k\times K,\tilde g_t)\to (\RR^k\times K,\tilde g)\]
is an isometry.

It remains to find the one parameter family $\phi_t$ such that 
(1) $\tilde g_t$ is $\Gamma$-invariant for all $t$
and (2) $(M,g_1)=(\tilde M,\tilde g)/\Gamma$ admits a finite Riemannian cover by the product of a flat torus and $K$.

First assume that $\Gamma$ is acting on $\RR^k$ by parallel translations.
 




In terms of $\phi_t$, it can be formulated the following way.
There is a normal subgroup of finite index $\Gamma_0\vartriangleleft\Gamma$ such that 
\begin{itemize}
\item $\Gamma_0$ acts on $\RR^k$ by parallel translations; 
in particular $\Gamma_0$ can be identified with a lattice in $\RR^k$.
\item the action of $\Gamma_0$ on $K$ is given by $\phi_1(\Gamma_0)$.
\end{itemize}


\parit{Comment.}
The problem is due to Burkhard Wilking, see \cite{wilking-2000}.














\begin{pr}{\thm}{If hemisphere then sphere}\label{hS=>S} 
Let $M$ be an $m$-dimensional Riemannian manifold with Ricci curvature at least $m-1$;
moreover there is a point $p\in M$ such that sectional curvature is exactly $1$ at all points on distance $\le \tfrac\pi2$ from $p$.
Show that $M$ has constant sectional curvature.
\end{pr}

%%%%%%%%%%%%%%%%%%%%%%%%%%%%%%%%%%%%%%%%%%%%%%%%%%
\parbf{\wrenches\ref{hS=>S}.} 
\textit{If hemisphere then sphere.}
Denote by $q$ a point in $M$ which lies on the maximal distance from $p$.

Consider the function $f=\cos \dist_p+\cos\dist_q$.
Note that 
\[\triangle f+m\cdot f\le 0\] 
in the sense of distributions.
If follows that $f\ge 0$, in particular 
\[B(p,\tfrac\pi2)\cup B(q,\tfrac\pi2)=M.\]

Set \[a(r)=\area(\partial [B(q,r)\backslash B(p,\tfrac\pi2)]).\]
Prove that the function
\[r \mapsto \tfrac{a(r)}{(\sin r)^{m-1}}\]
is nonincreasing 
and 
\[\tfrac{a(r)}{(\sin r)^{m-1}}\le\area\mathbb{S}^{m-1}.\]
Moreover if equality holds for some $r$ then $B(q,r)\backslash B(p,\tfrac\pi2)$ is isometric to an $r$-ball in the unit sphere.
This statement is analogous to the Bishop--Gromov inequality and can be proved the same way.

Finally note that $a(\tfrac\pi2)=\area\mathbb{S}^{m-1}$,
hence the result follows.
 

\parit{Comments.}
The problem is due to Fengbo Hang %Fengbo Hang fengbo@cims.nyu.edu
and Xiaodong Wang %Xiaodong Wang xwang@math.msu.edu 
\cite{hang-wang};
their proof is different.

The problem is still nontrivial 
even if instead of the first condition one has that sectional curvature $\ge 1$;
In dimension 2, 
it was proved by Victor Toponogov in \cite{toponogov},
but it also follow easily from the proof given by Victor Zalgaller in \cite{zalgaller-shperical-polygon}.


If instead of first condition one only has that scalar curvature $\ge m\cdot(m-1)$, then the conclusion does not hold; 
it was conjectured by Maung Min-Oo in 1995 
and disproved by
Simon Brendle,
Fernando Marques
and Andre Neves in \cite{brendle-marques-neve}.











\begin{pr}{}{Closed surface}\label{3D-moon-in-puddle}
Construct a smooth embedding $\mathbb{S}^2\hookrightarrow \RR^3$ 
with all the principle curvatures between $-1$ and $1$
such that it does not surround a ball of radius 1.
\end{pr}



%%%%%%%%%%%%%%%%%%%%%%%%%%%%%%%%%%%%%%%%%%%%%%%%%%
\parbf{\ref{3D-moon-in-puddle}.} 
\textit{Closed surface.}
The solution should be guessed from the picture.

%???CHANGE PIC
\begin{lpic}[t(-0mm),b(0mm),r(0mm),l(-5mm)]{pics/bing-1(.2)}
\end{lpic}
\begin{lpic}[t(-0mm),b(0mm),r(0mm),l(-5mm)]{pics/bing-2(.2)}
\end{lpic}
\begin{lpic}[t(-0mm),b(0mm),r(0mm),l(-5mm)]{pics/bing-3(.2)}
\end{lpic}

\begin{lpic}[t(-0mm),b(0mm),r(0mm),l(-5mm)]{pics/bing-6(.2)}
\end{lpic}
\begin{lpic}[t(-0mm),b(0mm),r(0mm),l(-5mm)]{pics/bing-5(.2)}
\end{lpic}
\begin{lpic}[t(-0mm),b(0mm),r(0mm),l(-5mm)]{pics/bing-4(.2)}
\end{lpic}

\parit{Comments.}
This solution is a fattening of \emph{Bing's House}, 


















\begin{pr}{}{Bounded curvature}\label{Bounded curvature}
Denote by $\mathcal{R}_{\mathbb S^5}$ the space of all Riemannian metrics with absolute value of sectional curvature at most $1$.
Show that there are two metrics $g_0,g_1\in \mathcal{R}_{\mathbb S^5}$ such that if $g_t\in\mathcal{R}_{\mathbb S^5}$,
$t\in [0,1]$ is a continuous one parameter famitly of metrirics connecting $g_0$ to $g_1$ then
\[\vol(g_t)>\exp(\exp(\exp(\exp(\exp(\max\{\vol(g_0),\vol(g_1)\})))))\]
for some $t\in [0,1]$.
\end{pr}

%%%%%%%%%%%%%%%%%%%%%%%%%%%%%%%%%%%%%%%%%%%%%%%%%%
\parbf{\ref{Bounded curvature}.} 
\textit{Bounded curvature.}
Assume there is no such pair of metrics.
Note that it implies that there is an algorithm which 
tells whether a given manifold diffeomorphic to $\mathbb{S}^5$.
Existance of such algorithm contradicts ???



























\begin{pr}{}{Anti-collapse}\label{anti-collaps} Construct
a sequence of Riemannian metric $g_n$ on a 2-sphere such that 
$\vol (\mathbb S^2,g_n)<1$ for any $n$ 
and the induced distance functions $d_n\:\mathbb{S}^2\times \mathbb{S}^2\to \RR_+$
converge to a metric $d\:\mathbb{S}^2\times \mathbb{S}^2\to \RR_+$ 
with arbitrary large Hausdorff dimension.
\end{pr}

%%%%%%%%%%%%%%%%%%%%%%%%%%%%%%%%%%%%%%%%%%%%%%%%%%
\parbf{\wrenches\ref{anti-collaps}.} 
\textit{Anti-collapse.}
Fix a decreasing sequence $\eps_0,\eps_1,\dots$ of positive numbers converging to $0$ as $n\to \infty$.

Let $T$ be the infinite binary tree
and $T_n\subset T$ be the subtree up to level $n$.
Let us equip $T$ with the length-metric such that the edges coming from $(n-1)$-th level to $n$-th level have length $\eps_{n-1}-\eps_n$.

Denote by $\bar T$ the completion of $T$.
Note that $C=\bar T\backslash T$ is a Cantor set;
The set $C$ can be identified with the set of $\{0,1\}$-sequences 
with the distance between two sequences $\bm{x}=(x_0,x_1,\dots)$ and $\bm{y}=(y_0,y_1,\dots)$ defined as $\eps_n$, where $n$ is the least number such that $x_n\ne y_n$.

Choosing $\eps_n$ one can make $C$ to have arbitrary large Hausdorff dimension.

Now choose $\delta_n\ll\eps_n$ and prepare for each edge of $T$ a cilinder with hight ... and radius of the base $\delta_n$.
 

Note that there is a natural embedding $T_{n}\to T_{n+1}$ for all $n$.


Assume $S$ be a surface with a flat disc $D_0\subset S$ of radius $r_0$.
Let us cut from $D_0$ two discs $D_1$ and $D_1'$ 
of radii $r_1=r_0/10$ and glue instead a cylinder with high $\eps_0$ 
with discs on the top.
Now repeat the operation for each $D_1$ and $D_1'$ cutting 
from each two discs of radius $r_2=r_1/10$
and glue instead cylinder with high $\eps_0$ 
with discs on the top.
Continue the process, we get an increasing sequence of Riemannian metrics on $S$.
%???PIC

\parit{Comments.}
The problem is due
Dmitri Burago, 
Sergei Ivanov 
and David Shoenthal,
see \cite{BIS}.














\begin{pr}{}{Loewner's theorem}
\label{Loewner's theorem}
Let $g$ be Riemannian metric
on the real projective space, $\RP^m$,
which is
conformally equivalent to the canonical metric $g_{\text{can}}$.
Denote by $\ell$ the minimal length of curves in $(\RP^m,g)$ which are not null-homotopic. 
Prove that 
$$\pi^m\cdot\vol(\RP^m,g)
\ge 
\ell^m
\cdot
\vol(\RP^m,g_{\text{can}})$$ 
and in case of equality $g=c\cdot g_{\text{can}}$ for some positive constant $c$.
\end{pr}

%%%%%%%%%%%%%%%%%%%%%%%%%%%%%%%%%%%%%%%%%%%%%%%%%%
\parbf{\ref{Loewner's theorem}.} 
\textit{Loewner's theorem.}
Denote by $\lambda$ the conformal factor of $g$;
i.e, $g=\lambda^2\cdot g_{\mathrm{can}}$.

Denote by $s$ the average of $g$-lengths of the lines in $\RP^m$.
Prove that 
\[\ell \le s=\pi\cdot\oint\limits_{\RP^m}\lambda\cdot d\vol_{\mathrm{can}},\]
where $\vol_{\mathrm{can}}$ denotes the volume for $g_{\mathrm{can}}$ and $\oint$ denoted the avarage value.

Note that
\[\vol(\RP^m,g)
=
\vol(\RP^m,g_{\mathrm{can}})
\cdot
\oint\limits_{\RP^m}\lambda^m\cdot d\vol_{\mathrm{can}}.\]
By H\"older's inequality, we have
\[\left(\,\,\oint\limits_{\RP^m}\lambda\cdot d\vol_{\mathrm{can}}\right)^m
\le \oint\limits_{\RP^m}\lambda^m\cdot d\vol_{\mathrm{can}}.\]
Hence the result follows.

\parit{Comments.}
An analogous inequality,
with worse constant holds for any Riemannina metric, 
not necessary conformal to the canonical one.
This is so called \emph{systolic inequality} proved by Mikhael Gromov in \cite{gromov-filling}.



















\begin{pr}{\thm}{Constant curvature is everything}\label{figure-eight-2} Given integer $m\ge 2$, prove that any compact \hyperref[Length-metric space]{\emph{length-metric spaces}} $K$ is a Gromov--Hausdorff limit of a sequence of $m$-dimensional
manifolds $M_n$ with curvature $-n^2$.
\end{pr}

\parbf{\wrenches\ref{figure-eight-2}.} 
\textit{Constant curvature is everything.}
Assume a fundamental group of a compact manifold $M$ admits an epimorphism to the free group with two generators.
In other words, there are two disjoint closed hypersurfaces $S_1$ and $S_2$ in $M$ 
which do not separate points in $M$
but each $S_i$ separates points of any of its small neighborhoods.

Prepare many copies of $M$, cut one copy along $S_1$ and $S_2$,
$n_1$ copies along $S_1$ and $n_2$ copies along $S_2$
then glue them as it shown on the picture. 
It produce a finite cover of $M$ which is after rescaling is close to figure eight.

Finally applying Problem~\ref{figure-eight-1}.

\parit{Comments} The problem appears in thesis of Sahovic \cite{sahovic}.



















\begin{pr}{}{Wild Cantor set}\label{Wild Cantor set} 
Contract a subset $C$ of $\RR^3$ 
such that $C$ is homeomorphic to the Cantor set and the complement $\RR^3\backslash C$ is not simply connected.
\end{pr}



%%%%%%%%%%%%%%%%%%%%%%%%%%%%%%%%%%%%%%%%%%%%%%%%%%
\parbf{\ref{Wild Cantor set}.}
\textit{Wild Cantor set.}
We will construct a compact totally discontinous subset $K\subset\RR^3$ without isolated points  
such that the complement $\RR^3\backslash K$ is not simply connected.
Then the problem will follow
since any compact totally discontinous set and without isolated points
is homeomorphic to the Cantor set.

The needed set can be obtained as intersection $K=\bigcap_n K_n$
of a nested sequence 
of sets $K_1\supset K_2\supset\dots$, where each $K_n$ is a union of finite collection of solid tori. 
The construction can be guessed from the first two itarations shown on the picture.

\parit{Comments.}
This example was constructed by Louis Antoine in \cite{antoine}.
The pictures above were made by \href{http://en.wikipedia.org/wiki/User:Blacklemon67}{blackle}.













%\begin{pr}{\hard}{Besicovitch's set}\label{Besicovitch's set}
%Show that one can cut a unit plane disc along a finite number of radii into sectors and then move each sector by a parallel translation in such a way that their union have arbitrary small area.
%\sign{\cite{besicovitch}}
%\end{pr}













\begin{pr}{}{Closed magnetic geodesics}\label{twisted-geodesic} %???
Give an example of a closed Riemannian 
2-manifold which has no closed smooth curve with constant geodesic curvature $=1$.
\end{pr}

%%%%%%%%%%%%%%%%%%%%%%%%%%%%%%%%%%%%%%%%%%%%%%%%%%
\parbf{\ref{twisted-geodesic}.} 
\textit{Closed magnetic geodesics.}
Take $\Sigma$ to be a compact surface with constant curvature $-1$;
that is, 
 $\Sigma=\HH^2/\Gamma$,
where $\HH^2$
is the hyperbolic plane 
and $\Gamma$ is a descrete subgroup of isometries of $\HH^2$ which acts freely.

Prove that $\Gamma$ contains only hyperbolic elements; that is, any transformation $\gamma\in\Gamma$ fix two distinct points on the absolute.

Assume $\Sigma$ contains a closed curve with curvature $1$.
Note that this curve lifts to a horo-cycle in $\HH^2$.
Therefore $\Gamma$ has to contain an element which preserves this horo-cycle.
Such an element has to be parabolic;
that is, it fixes a single point on the absolute of $\HH^2$,
 a contradiction.

\parit{Comments.}
The problem appears in the Ginzburg's paper \cite{ginzburg}.












\begin{pr}{}{Divergence}\label{Divergence} %???
Let $F$ be a convex figure in the $\RR^2$ with smooth boundary
and $\area F=\pi$.
Show that there is a vector field $\bm{v}$ on $F$
such that $|\bm{v}|\le 1$ and $\div \bm{v}\ge 2$. 
\end{pr}

%%%%%%%%%%%%%%%%%%%%%%%%%%%%%%%%%%%%%%%%%%%%%%%%%%
\parbf{\ref{Divergence}.}
\textit{Divergence.}
Fix a Cartesian coordinate system $(x,y)$ on the plane.
Denote by $D$ the unit disc centered at the origin;
so $\area F=\area D$.
 
Prove that there is a smooth area preserving map $F\to D$, 
of the form
$(x,y)\mapsto (v(x),w(x,y))$.

Show that the vector field $\bm{v}(x,y)=(v(x),w(x,y))$
is a solution for the problem.

\parit{Comments.}
This is the key ingredient in the famous proof of isoperimetric inequality --- to finish the proof apply the divergence formula for $\bm{v}$.

The same argument works in all dimensions.
See the Gromov's appendix in \cite{gromov-apendix} for more details.












\begin{pr}{}{Deformation of polygon}\label{Deformation of polygon}
Let $A_t$, $t\in [0,1]$ be a continuous one parameter family of not necessary convex polygons with fixed side lengths.
Assume that each angle of $A_t$ is either increasing or decreasing in $t$.
Show that the vertices with increasing angles of $A_0$
can not be separated from the vertices with decreasing angles by a line.
\end{pr}


Let us cut the polygon $A_0$ from the plane and glue instead the polygon $A_t$ for small enuf $t$.
Denote by $\Pi$ the obtained polyhedral space.

Note that 
$\Pi$ coincides with Euclidean plane outside of compact region.
The vertices with increasing/decreasing angles of $A_0$ 
correspond to the singular points of $\Pi$ with negative/positive curvature.

Assume we have at least two singular points with positive curvature in $\Pi$.
Choose two of them, 
say $p$ and $q$ which lie on the minimal distance from each other. 
Cut $\Pi$ along a geodesic $[pq]$
and patch the hole 
so that in the obtained polyhedral space, say $\Pi'$, 
the points $p$ and $q$ become regular and yet an other singular point apears in the patch. 
The patch is obtained by doubling a triangle along  two sides.
This way we can remove all singular points with positive curvature except one.

A similar procedure gives a way to 










\begin{pr}{}{Invisible island}
Let $g$ be a Riemannian metric on $\RR^n$ which coincides with the canonical Euclidean metric outside of a bounded set $K\subset \RR^n$.
Assume $\gamma$ and $\gamma'$ be geodesics in the canonical metric and in $g$
such that
\end{pr}





\begin{pr}{}{Bound on the area of sphere}
Let $M$ be a complete connected non-compact Riemannian manifold with non-negative Ricci curvature and $p\in M$.
Show that there is $\eps>0$ such that
$$\area [\partial B_r(p)]>\eps$$
for any $r>1$.
\sign{\cite{user16750}}
\end{pr}










\begin{pr}{}{Kirszbraun's theorem}\label{kirszbrun}
Let $X\subset\RR^2$ be an arbitrary subset and $f\:X\to\RR^2$ be a \hyperref[Short map]{\emph{short map}}.
Show that $f$ can be extended as a short map to whole $\RR^2$;
i.e., there is a short map $\bar f\:\RR^2\to\RR^2$ such that its restriction to $X$ coincides with~$f$.
\end{pr}
%%%%%%%%%%%%%%%%%%%%%%%%%%%%%%%%%%%%%%%%%%%%%%%%%%
\parbf{\wrenches\ref{kirszbrun}.} 
\textit{Kirszbraun's theorem.}
First observe that it is sufficient to prove the following weaker statement.

\textit{Assume $a_1,a_2,\dots,a_n$ and $b_1,b_2,\dots, b_n$ be the points in the plane such that 
$|a_i-a_j|\ge |b_i-b_j|$
for any $i$ and $j$.
Then for any point $x$ in the plane there is an other point $y$ such that
$|a_i-x|\ge |b_j-y|$
for each $i$.}

Take $y$ to be the minimum point  of the function ???
and show that it satisfies the inequalities.

\parit{Comments.}
The problem was discovered by Kirszbraun in his thesis, see \cite{kirszbraun}.
Later it was reproved by Valentine in \cite{valentine}.











\begin{pr}{}{Negative curvature vs. symmetry}\label{Negative curvature vs. symmetry} Let $(M,g)$ be a closed Riemannian manifold with negative Ricci curvature.
 
Prove that $(M,g)$ does not admit an isometric $\mathbb{S}^1$-action.
\end{pr}

%%%%%%%%%%%%%%%%%%%%%%%%%%%%%%%%%%%%%%%%%%%%%%%%%%
\parbf{\ref{Negative curvature vs. symmetry}.} 
\textit{Negative curvature vs. symmetry.}
Assume contrary.
Consider a longest orbit $\gamma\:\mathbb{S}^1\to M$ of the $\mathbb{S}^1$-action.
Note that $\gamma$ has to be a geodesic. 

Choose a point $p$ on $\gamma$
and an orthonormal basis $e_1,e_2,\dots ,e_{n-1}$ in the  subspace $N\subset \T_p$ which is normal to $\gamma$.
For each $i$ and $t\in \RR$,
consider the orbit of $\exp_p(t\cdot e_i)$,
denote by $\ell_i(t)$ its length.

Prove that
\[\sum_i\ell_i''(0)>0;\leqno{({*})}\]
here you need to use that $\mathop{\rm Ricc}(\dot\gamma(t),\gamma'(t))<0$.

Finally note that inequality $({*})$ implies that 
length of $\gamma$ is not maximal, a contradiction.

\parit{Comments.}
The problem discovered by Bochner, see \cite{bochner}.










\begin{pr}{\easy}{Cut which does not cut}\label{Cut which does not cut}
Construct an injective 
unit-speed curve 
$\gamma\:\RR\z\to\RR^2$ with bounded curvature
such that the complement $\RR^2\backslash\gamma(\RR)$ 
has connected interior.
\end{pr}

%???

%%%%%%%%%%%%%%%%%%%%%%%%%%%%%%%%%%%%%%%%%%%%%%%%%%
\parbf{\wrenches\ref{Cut which does not cut}.}
\textit{Cut which does not cut.}

\parit{Comments.} The problem is inspired by Panov's papper \cite{panov-torus}.
%???PIC







\begin{pr}{}{Area contracting map}\label{Area contracting map}
Let 
$$A=[0,1]\times[0,1]\times[0,1000]\ \ \text{and}\ \ B=[0,1]\times[0,10]\times[0,10].$$
Construct a Lipschitz map 
$f\:A\to B$ which is area contracting and such that the restriction $f|_{\partial A}$ 
is a degree one map $\partial A\to \partial B$.
\bigsign{\cite[\ref{guth-area-cnotractin}]{guth}}
\end{pr}

%%%%%%%%%%%%%%%%%%%%%%%%%%%%%%%%%%%%%%%%%%%%%%%%%%

\parbf{\ref{Area contracting map}.} 
\textit{Area contracting map.}
Smash point 
a small neighborhood of $\partial A$ into $\partial A$;
this can be done by a map $f_1\:A\to A$
with a Lipschitz constant arbitrary close to $1$, say $e^\eps$.

Now map $A$ into onto by a  $B$ by a $e^{-\eps}$-Lipshitz snake like map $f_2$ shown on the diagram.
The composition $f_2\circ f_1\:A\to B$ is 1-Lipshitz, in particular area nonexpanding. 


%???










\begin{pr}{\thm}{Flexible convex polyhedron}\label{Flexible convex polyhedron}
Show that no triangulation of the surface of a convex polyhedron in $\mathbb R^3$ 
can be flexible.
That is, given a convex polyhedral surface with a triangulation $\mathcal T$, one can not move the vertices of $\mathcal T$ continuously in such a way that one triangle in $\mathcal T$ is in a fixed
and 
all the edges of $\mathcal T$ have fixed length.
\sign{\cite{connelly}}
\comment{Note that by Alexandrov's theorem, if such banding exists then the surface immediately becomes non-convex.}
\end{pr}



\parbf{\ref{Flexible convex polyhedron}.} 
\textit{Flexible convex polyhedron.}
Assume contrary; i.e., there is a convex polyhedron $P$ which admits a flexible triangulation $\tau$. 
Denote by $P^t$ the continuous deformation of $P$ which moves each triangle $\triangle_1,\dots,\triangle_n$ of $\tau$ 
by motions $m_1^t,\dots,m_n^t$ of $\RR^3$.
Given triangle edge or vertex of $\tau$ we will mark by upper index $t$ the corresponding object in $P^t$; 
for example $\triangle_i^t=m^t_i(\triangle_i)$.

Denote by $v_1,v_2,\dots,v_k$ the vertices of $\tau$.
We can assume that one of the triangles, say $[v_1v_2v_3]$ of $\tau$ does not move in the deformation.

Let us choose a sequence $t_n\to 0$ and set 
\[\eps_n=\max_i\{|v_i^{t_n}-v_i|\}.\leqno{({*})}\]

Prove that if $v_i$ and $v_j$ lie in one face of $P$ then
\[|v_i^{t_n}-v_j^{t_n}|=|v_i-v_j|+O(\eps_n^2).\leqno{({*}{*})}\]

Passing to a subsequence if necessary,
we can assume that the limit
\[w_i=\lim_{n\to\infty} \frac{v_i^{t_n}-v_i}{\eps_n}\] 
is defined for each $i$.
Note that $w_1=w_2=w_3=0$,
$|w_i|\le 1$ for each $i$ 
and $|w_k|=1$ for some $k$.

Note that from the equation $({*}{*})$ it follows that $w_i$ restricted to the vertices of the original polyhedron gives its  infintesimal deformation;
i.e., 
\[\<v_i-v_j,w_i-w_j\>=0\]
for any edge $[v_iv_j]$ of the polyhedron.
The latter contradicts Dehn's theorem on the infinetesimal rigidity of simplicial polyhedron. 

\bibitem[R.~Connelly]{connelly}\textit{The rigidity of certain cabled frameworks and the second-order rigidity of arbitrarily triangulated convex surfaces.} Adv. in Math. 37 (1980), no. 3, 272--299.











\begin{pr}{}{Fixed point of conformal mappings}\label{Fixed point of conformal mappings}
Let $(M,g)$ be an oriented positively curved closed Riemannian manifold with even dimension
and $f\:M\to M$ be a conformal orientation preserving map. 
Prove that $f$ has a fixed point.
\sign{\cite[\ref{weinstein-Fixed-pnt}]{weinstein}}
\end{pr}








\begin{pr}{}{Minimal hypersurfaces}\label{Minimal hypersurfaces}
Show that any two compact \hyperref[Minimal surface]{\emph{minimal hypersurfaces}} in a Riemannian manifold with positive Ricci curvature must intersect.
\sign{\cite{frankel}}
\end{pr}

%%%%%%%%%%%%%%%%%%%%%%%%%%%%%%%%%%%%%%%%%%%%%%%%%%
\parbf{\ref{Minimal hypersurfaces}.} 
\textit{Minimal hypersurfaces.}
Assume contrary, let $V$ and $W$ be disjoint compact minimal hypersurfaces in the manifold $M$.
Consider a pair of points $v\in V$ and $w\in W$ which realize the minimum of the distance.

Note that the minimizing geodesic $[vw]$ is unique and it is normal to each of the hypersurface at the end points.
In particular the parallel translation of the tangent hyperplane $\T_vV$ along $[vw]$ gives the tangent hyperplane $\T_wW$.

Consider pairs of unit-speed geodesics $\alpha$ and $\beta$ 
in $V$ and $W$ correspondingly 
which start at $v$ and $W$ and go in the parallel directions. 
Set $\ell(t)=|\alpha(t)-\beta(t)|_M$.

Use the second variation formula to show that $\ell''(0)$ has negative average value. 
In particular $\ell''(0)<0$ for a pair $\alpha$ and $\beta$ as above
and therefore  $v$ and $w$ is not a minimal pair, a contradiction.



















\chapter{letter of S. Tabachnikov}

Dear AHTOH, here are my (sorry, much belated!) comments on your list
of geometry problems. I will be happy to provide more details if you
wish. Do you include open problems? Privet, Serezha

1. Spiral. A reference to Kneser's theorem is appropriate: as long as
the curvature is monotonic, the osculating circles are nested.???

2. A curve in a sphere. Isn't it an immediate consequence of integral
geometry (length=const \times average number of intersections with
great circles)?

3.

4. Convex polygon. This is known as Cauchy Lemma; this was the main
ingredient in his proof that convex polyhedra in 3-space are rigid.

5. Convex triangulations. I think, this is discussed, with
references, in Gelfand-Kapranov-Zelevinsky book.

6. Short curve in a sphere. Doesn't it follow from integral geometry
that every open interval of this curve doesn't intersect some grat
circle?

7. Transversal foliation. To this I'd add that a non-vanishing
divergence free vector field on a closed 3-mfld with finite
fundamental group is not tangent to any foliation. This is proved in
my 1989 Func. Anal. note.

I don't know whether the following results qualifies for your list.

A. Given partition of a square into triangles of equal areas, the
number of triangles is even. There are many generalizations. E.g.,
the same holds for a centrally symmetric polygon. There is literature
on this subject.

B. Let G be a closed convex plane curve and g a closed immersed curve
inside G. Then average absolute curvature (g) \geq that of G, with
equality iff g is G, possibly traversed more than once. Average
absolute curvature = integral of abs. value of curvature divided by
total length.
For G a circle, this is an old Fary theorem (I know a few proofs);
for arbitrary G this was proved recently (there are two papers).



\chapter{letter of P. Petersen}

Peter Petersen 	to Anton

I'll be happy to make a few corrections.
In the problem a la Abresch-Gromoll you use Minkowski space for finite
dimensional normed vector space. This is a bit confusing to those who
think of Minkowski spaces as space-times!
In the Immerses convex hypersurface problem the third and last question
should say nonnegative (sectional) curvature rather than nonpositive which
is the second question.
In Fixed points of conformal mappings the dimension should be even? The
antipodal map on S^3 has no fixed points. Actually you might want to say
that in even dimensions the map is orientation preserving and in odd
dimensions it is orientation reversing. 
Then the problem should be correct.

And here are two problems I have always liked.
1. Show that if a
simply connected
compact positively curved manifold admits a totally geodesic hypersurface,
then it is a sphere. Actually one just needs to know that the hypersurface
is two sided.
2. Show that two minimal hypersufaces in positive Ricci curvature must
intersect.

Best regards, Peter








\chapter{AA}

{\bf A.} Let $S^2$ be unit $2$-sphere and
$B_\alpha(p)$ be a ball of
radius $\alpha$ with center $p$ in $S^2$.
Assume $f\:B_\alpha(p)\to \RR^2$ is a bi-Lipschitz map;
that is, there are positive constants $c_1 < c_2$, such that for any two points
$x,y\in D_\alpha$ we have
$$c_1\cdot |x y|_{S^2}\le |f(x)f(y)|_{\RR^2} \le c_2\cdot|x y|_{S^2}.$$
Prove that
$$\frac{c_2}{c_1}\ge \frac{\alpha}{\sin\alpha}$$
and in case of equality
$f$ is similar to $\exp_p^{-1}\:B_\alpha(p)\to T_p(S^2)=\RR^2$.

\item{\bf Conformal factor/map.} Two Riemannian metrics $g$ and $h$ on a manifold $M$ are called {\it conformally equivalent} if for $g=u h$ for some positive function $u$on $M$
The function $u$ called {\it conformal factor}.
A diffeomorphism between two Riemannian manifolds is called conformal map if the pulled back metric is conformally equivalent to the original one.

\begin{pr}{}{Hilbert's problem} Let $\sigma$ be a centrally
symmetric measure on $S^2\subset \RR^3$, such that $\sigma(X)=0$ if $X$ is a
subset of big circle and $\sigma(X)>0$ if $X$ is open.
A continuous centrally symmetric metric $\rho$ on $S^2$
is called $\sigma$-metric
if there is a measure $\sigma$ as above such that
$$\rho(x,y)=\sigma\bigl(B_{\pi/2}(x)\backslash B_{\pi/2}(y)\bigr).$$

Prove that all geodesics of any $\sigma$-metric coincide with big circles
of $S^2$.
\end{pr}

\begin{pr}{}
{Transversal foliation.} (D.~Burago)

{\bf A.} Find a vector field on $S^3$ which does not admit a transversal foliation.

{\bf B.} Show that any volume preserving vector field on $S^3$ does not admit a transversal foliation.
Transversal foliation. To this I'd add that a non-vanishing divergence free vector field on a closed 3-mfld with finitefundamental group is not tangent to any foliation. This is proved in my 1989 Func. Anal. note.
\sign{???}
\end{pr}

\begin{pr}{}{Area of minimal surface.} Let $\Sigma$ be a minimal 2-disc in $\RR^3$ which passes thru the origin and its boundary $\partial\Sigma$ on the unit sphere is a curve in a unit sphere and center of the sphere belongs to $S$.
(i) Prove that area of $S$ is at least $\pi$.
(ii) Prove that length of $\partial S$ is at least $2\cdot\pi$
\sign{???}
\end{pr}

\begin{pr}
{\bf  Rescaling-invariant metrics} Let $(M,d)$ be a manifold with an intrinsic metric. 
Assume it is simply connected homogeneous; 
that is, there is an isometric transitive group action, and rescaling invariant, 
that is, for any $\lambda>0$ there is a homeomorphism $h_\lambda\:M\to M$ such that $$d(h_\lambda(x)h_\lambda(y))=\lambda d(x,y),$$ Prove that $M$ is isometric to a nilpotant Lie group with a left invariant sub-Riemannian metric.
\end{pr}

\begin{pr}
{\bf  Tennis ball} \cite{arnold}\cite{angenent} Any smooth curve in a sphere which divides its area into two equal parts has at least 4 {\it inflection points}.
%Tennis ball. A more general theorem was proved by B. Segre before Arnold.???B. Segre, Alcune proprieta differenziali in grande delle curve chiuse sghembe}, Rend. Mat. 1 (1968), 237--297.
\end{pr}

\begin{pr}{\easy}{Cauchy lemma} Let $a_1a_2a_3....a_n$ and $b_1b_2b_3...b_n$ be two convex non-isometric polygons such that correspondent sides have the same length.
Write next to each vertex $a_i$  a plus if $\measuredangle a_i>\measuredangle b_i$, zero if  $\measuredangle a_i=\measuredangle b_i$ and minus if $\measuredangle a_i<\measuredangle b_i$.
Prove that if one would go around $a_i$ it will be at least four changes of sign.
\end{pr}

\begin{pr}{}{Lipschitz approximation}
Prove that any compact Riemannian manifold admits a Lipschitz approximation by Euclidean polyhedra; 
that is, for any Riemannian compact manifold $M$ there is a sequence of polyhedra $P_n$ with $(1\pm\tfrac1n)$-bi-Lipschitz homeomorphisms $h_n\:P_n\to M$.
\end{pr}

\begin{pr}{}
{\bf  Convex surface}
Let $P$ be a \emph{polyhedral space} and $f_0,f_1\:P\to \RR^q$ be two piecewise linear isometries with the same asociated triangulation $T$.
Consider $\RR^q$ as a subspace of $\RR^{2q}$.
Prove that there is homotopy $F_t\:P\to\RR^{2q}$, $F_0=f_0$, $F_1=f_1$, such that for each $t\in[0,1]$ the map $F_t$ is a piecewise linear isometry with asociated triangulation $T$.
\end{pr}

\begin{pr}{}
{\bf  Convex surface}
Construct a closed convex surface $\Sigma\subset \RR^3$ such that for any two points $x,y\in \Sigma$ we have $|x y|_\Sigma< \frac{\pi}{2}|x y|$, where $|**|$ denotes distance in $\RR^3$ and $|**|_\Sigma$ the induced distance in $\Sigma$.
\end{pr}

\begin{pr}{}{Ocean of scalar curvature} Show that if a Riemannian metric on $\RR^3$ coincides with Euclidean outside of a ball and its scalar curvature is nonnegative then it is flat.
\end{pr}

\begin{pr}{}{Fixed set of group action} Let $M$ be a smooth manifold with a smooth action of a compact group.
Prove without using Riemannian geometry that any connected component of fixed point set forms a submanifold.
\sign{A.~Katok}
\end{pr}

\begin{pr}{\hard}{Katok's paradox}
Construct a set of full measure $E$ in $\RR^2$ and a one-parameter family of smooth curves $\gamma_t\:\RR\to\RR^2$ such that the map $(t,\tau)\mapsto\gamma_t(\tau)$ gives a homeomorphism of $\RR^2$ and for each $t$ there is at most one point of $A$ on $\gamma_t$.
\sign{\cite{milnor:katok}}
\end{pr}

\begin{pr}{\hard}{Triangulation of a square} 
Show that there is no triangulation of a square with an odd number of triangles with equal area.
\sign{\cite{monsky}}
\end{pr}

\begin{pr}{\hard}{Almost non-positively curved sphere}  Construct a Riemannian metric on $S^3$ with sectional curvature $\le 1$ and arbitrary small diameter.
\bigsign{M.~Gromov, \cite{buser-gromoll}}
\end{pr}

\begin{pr}{}{???Saddle graph}
Let $f\:\RR^2\to\RR$ be a smooth bounded function.
Assume that its graph $z=f(x,y)$ is a strictly \emph{saddle suface}.
Show that $f$ is constant. 
\sign{???}
\end{pr}

\begin{pr}{}{??? A curve in the plane}
Let $\gamma\:[0,1]\to \RR^2$ be a simple curve%
\footnote{that is, injective and continuous map}.
Prove that for any $\eps>0$ there is a sequence of numbers
$0=t_0<t_1<t_2<...<t_n=1$ such that
$\sum_{i=1}^n|\gamma(t_{i-1})\gamma(t_{i})|^2<\eps$.
\sign{???}
\end{pr}

%\parbf{Cartography.}

\begin{pr}{}{???bi-Lipschitz cartography} Let $\mathbb{S}^2$ be unit $2$-sphere.
Assume $\Omega\subset \mathbb{S}^2$ is a convex domain with perimeter%
\footnote{An analogous questin for area is open.}
$\le 2\cdot\pi\sin\alpha$, $0<\alpha<\pi/2$.
Show that there is a bi-Lipschitz mapping $f\:\Omega\to \RR^2$
with constants $c_1 < c_2$ such that
$$\frac{c_2}{c_1}\le \frac{\alpha}{\sin\alpha}.$$
\sign{\cite{milnor}$+$\cite{lawlor}}
\end{pr}


\begin{pr}{}{???Conformal cartography}
Let $U$ be a simply connected region
bounded by smooth curve in a positively curved \emph{surface}, $f\:U\to \RR^2$
be a conformal map and $\sigma\:U\to R_+$ be its conformal factor.
Prove that there is one, and up to similarity transformation of $\RR^2$,
only one conformal map $f^*\:U\to \RR^2$ which minimize ratio
$\sup\sigma/\inf\sigma$ and $f^*$ is characterized by property that $\sigma$
is constant along the boundary of $U$.

Prove that if $U$ is convex then $f^*$
is one-to-one and $f^*(U)$ is convex.
\sign{\cite{milnor}}
\end{pr}

\begin{pr}{}{???Chebyshev's net}
Assume a complete positively curved Riemannian metric on $\RR^2$ is invariant with respect to reflections $(x,y)\mapsto (\pm x,\mp y)$.

Show that there are new coordinates $(u,v)$ on $\RR^2$ with first fundamental form given by
\def\d{d}
$\d s^2=\d u^2+2\cdot\d u\cdot\d v\cdot \cos\alpha(u,v)+\d v^2$.
\sign{\cite{bakelman}}
\end{pr}

\begin{pr}{}{Convex/Strictly convex}\label{strict-convex} 
Show that there is a Riemannian metric $g$ on $\mathbb{S}^3$ for which any  $C^\infty$-close metric $g'$ has the following property: 
Every $C^1$-smooth convex surface in $(\mathbb{S}^3,g')$ is strictly convex; 
that is, it has no geodesic segments on~it.
\sign{}
\end{pr}

\begin{pr}{\thm}{Concave 3-ball}\label{hass}
Construct a negatively curved Riemannian metric on a closed 3-ball $\bar B$ with concave boundary; 
that is, such that  $\dist_{\partial \bar B}$ is convex function in a neighborhood of $\partial \bar B$.
\sign{\cite{hass}}
\end{pr}
for doing problem \ref{hass} you have to know basic examples of hyperbolic 3-manifolds with one end








\begin{pr}{}{Inscribed triangulation}
Let $Q$ be a unit square and $f\:Q\to\RR_>$ be an arbitrary (not nesessury continuous) function.
Prove that there is a triangulation $\mathcal T$ of $Q$ such that each triangle $\triangle xyz$ in $\mathcal T$ is covered by three discs $B_{f(x)}(x)$, $B_{f(y)}(y)$ and $B_{f(z)}(z)$.
\sign{S.~Buyalo}
\end{pr}




\begin{pr}{}{Telescopic group}
Given a group $\Gamma$, let us denote by $\langle\Tor\Gamma\rangle$ the subgroup of $\Gamma$ generated by all elements of finite order.

Construct a finitely presented hyperbolic group, $\Gamma$ such that for any finitely presented group $G$ there is a subgroup $\Gamma'<\Gamma$ of finite index such that 
$G$ is isomorphic to the factor group $\Gamma'/\langle\Tor\Gamma'\rangle$
\sign{\cite{panov-petrunin}}
\end{pr}



\begin{pr}{}{Smooth orbifold}\label{Smooth orbifold}
Let $M$ be a smooth Riemannian manifold with isometric 
$\mathbb{S}^1$-action.
Show that there is a smooth Riemannian orbifold $\mathcal{O}$
such that its underlying metric space 
is isometric to the orbit space $M/\mathbb{S}^1$.
\sign{\cite{lytchak}}
\end{pr}


\begin{pr}{}{Finite orbits}\label{Finite orbits}
Let $P$ be a compact polyhedral space and $T\:P\z\to P$ be a piecewise linear map.
Assume for any $x\in P$ there is a positive integer $n$ such that
$T^n(x)=x$.
Show that $T^n$ is the identity map for some $n$. 
\end{pr}

\begin{pr}{\easy}{Periodic homeomorphism}\label{Finite action}
Assume a homeomorphism $h$ of unit $m$-dimensional sphere $\mathbb S^m$ has  finite period $n>1$.
Show that there is an orbit which does not lie in any open hemisphere.
\sign{Newman???}
\end{pr}
