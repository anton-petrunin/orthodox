\csname @openrightfalse\endcsname
\chapter{Actions and coverings}

In this chapter we consider isometric actions and metric coverings.
Given a covering $f\:\tilde X\to X$  of the lenght-metric space $X$,
one can consider the induced length-metric on $\tilde X$
defining length of curve $\alpha$ in $X$ as the length of the composition $f\circ\alpha$.



%%%%%%%%%%%%%%%%%%%%%%%%%%%%%%%%%%%%%%%%%
\subsection*{Bounded orbit}

Recall that a metric space is called \index{proper metric space}\emph{proper} if all its bounded closed sets are compact.

\begin{pr}{}{Bounded orbit}\label{Bounded orbit} Let $X$ be a 
proper metric space 
and $\iota\:X\to X$ is an isometry.
Assume that for some $x\in X$, the sequence $x_n\z=\iota^n(x)$, $n\in\ZZ$ has a converging subsequence.
Prove that $x_n$ is bounded.
\end{pr}

%%%%%%%%%%%%%%%%%%%%%%%%%%%%%%%%%%%%%%%%%%%%%%%%%%
\parit{Semisolution.}
Note that we can assume that the orbit $x_n=\iota^n(x)$ is dense in $X$;
otherwise pass to the closure of this orbit.
In particular, we can choose a finite number of positive integers values $n_1$, $n_2,\dots,n_k$
such that the points $x_{n_1}$, $x_{n_2},\dots,x_{n_k}$ form a $\tfrac1{10}$-net in $B(x_0,10)$,
that is, for any $x\in B(x_0,10)$ there is $x_{n_i}$ such that
\[|x-x_{n_i}|<\tfrac1{10}.\]

Prove that 
if $x_m\in B(x_0,1)$, 
then $x_{m+n_i}\in B(x_0,1)$ for some $i\z\in\{1,\dots,k\}$.

Set $N=\max_i\{n_i\}$.
It follows 
that among any $N$ elements in a row $x_{i+1},\dots x_{i+N}$
there is at least one in $B(x_0,1)$.
In particular, $N$ isometric copies of $B(x_0,1)$ cover whole $X$.
Hence the result follows.\qeds

The problem is due to Aleksander Ca{\l}ka's \cite[see][]{calka}.

%%%%%%%%%%%%%%%%%%%%%%%%%%%%%%%%%%%%%%%%%
\subsection*{Finite action}

\begin{pr}{}{Finite action}\label{Finite action}
Show that for any nontrivial continuous action of a finite group on the unit sphere
there is an orbit which does not lie in the interior of a hemisphere.
\end{pr}

%%%%%%%%%%%%%%%%%%%%%%%%%%%%%%%%%%%%%%%%%
{

\begin{wrapfigure}[3]{r}{28 mm}
\begin{lpic}[t(-5 mm),b(-5 mm),r(0 mm),l(0 mm)]{pics/figure-eight(1)}
\end{lpic}
\end{wrapfigure}

\subsection*{Covers of figure eight}


Let us define \index{figure eight}\emph{figure eight} as the
length space which
is obtained by gluing together all four ends of two unit segments.

}

\begin{pr}{}{Covers of figure eight}\label{figure-eight-1}
Prove that any compact length spaces $K$ 
can be presented as a Gromov--Hausdorff limit of a sequence of
metric covers  
\[(\widetilde \Phi_n, \tilde d/n)\to(\Phi,d/n),\]
where $(\Phi,d)$ denotes the figure eight.
\end{pr}


%%%%%%%%%%%%%%%%%%%%%%%%%%%%%%%%%%%%%%%%%
\subsection*{Diameter of \textit{m}-fold cover\hard}

\begin{pr}{\hard}{Diameter of \textit{m}-fold cover}\label{m-fold-cover}
Let $X$ be a length space
and $\tilde X$ be a connected $m$-fold cover of $X$ 
equipped with the induced intrinsic metric.
Show that
$$\diam\tilde X\le m\cdot \diam X.$$
\end{pr}

From the diagam below you could guess an example of 5-fold cover with diameter of the total space exactly 5 times diameter of the target.

\begin{center}
\begin{lpic}[t(0mm),b(0 mm),r(0 mm),l(0 mm)]{pics/5-fold(1)}
\lbl[t]{49,4.5;$\to$}
\end{lpic}
\end{center}

%%%%%%%%%%%%%%%%%%%%%%%%%%%%%%%%%%%%%%%%%
\subsection*{Symmetric square\easy}

Let $X$ be a topological space.
Note that $X{\times} X$ admits a natural $\ZZ_2$-action by $(x,y)\mapsto (y,x)$.
The quotient  space $X{\times} X/\ZZ_2$ is called \index{symmetric square}\emph{symmetric square} of $X$.

\begin{pr}{\easy}{Symmetric square}\label{Symmetric square} 
Show that symmetric square 
of any path connected topological space 
has commutative the fundamental group.
\end{pr}

{

\begin{wrapfigure}[4]{r}{21 mm}
\begin{lpic}[t(2 mm),b(-0 mm),r(0 mm),l(0 mm)]{pics/sierpinski-triangle(1)}
\end{lpic}
\end{wrapfigure}

%%%%%%%%%%%%%%%%%%%%%%%%%%%%%%%%%%%%%%%%%
\subsection*{Sierpi\'nski gasket\easy}

To construct Sierpi\'nski gasket, start with a solid  equilateral triangle, subdivide it into four smaller congruent equilateral triangles and remove the iterior of the central one.
Repeat this procedure recuresevly for each of the remaining solid triangles.

}

\begin{pr}{\easy}{Sierpi\'nski triangle}\label{Sierpinski triangle} 
Find the homeomorphism group of the Sierpi\'nski gasket.
\end{pr}



%%%%%%%%%%%%%%%%%%%%%%%%%%%%%%%%%%%%%%%%%
\subsection*{Lattices in a Lie group}

\begin{pr}{}{Lattices in a Lie group}\label{Boys and girls in a Lie group}
Let $L$ and $M$ be two discrete subgroups
of a connected Lie group $G$ and $h$ be a left
invariant metric on $G$.
Equip the groups $L$ and $M$ 
with the induced left invariant metric from $G$.
Assume $L\backslash G$ and $M\backslash G$ are compact and moreover
$$\vol(L\backslash (G,h))
=
\vol(M\backslash (G,h)).$$
Prove that there is a bi-Lipschitz one-to-one mapping
(not necessarily a homomorphism)
\[f\:L
\to
M.\]
\end{pr}


%%%%%%%%%%%%%%%%%%%%%%%%%%%%%%%%%%%%%%%%%
\subsection*{Piecewise Euclidean quotient}

Note that the quaotent of Euclidean space by a finite subgroup is a \index{polyhedral space}\emph{polyhedral space} as it defiend on page \pageref{piecewise linear map};
on the same page you find the definition of piecewise linear homeomorphism.


\begin{pr}{}{Piecewise Euclidean quotient}\label{Piecewise Euclidean quotient}
Let $\Gamma$ be a finite subgroup of $\SO(m)$.
Denote by $P$ the quotient $\RR^m/\Gamma$ equipped with induced
polyhedral metric.
Assume $P$ admits a piecewise linear homeomorphism to $\RR^m$.
Show that $\Gamma$ is generated by rotations  around subspaces of codimension $2$.
\end{pr}

%%%%%%%%%%%%%%%%%%%%%%%%%%%%%%%%%%%%%%%%%
\subsection*{Subgroups of the free group}

\begin{pr}{}{Subgroups of the free group}\label{Subgroups of free group} 
Show that every finitely generated subgroup of the free group 
is an intersection of subgroups of finite index.
\end{pr}

%%%%%%%%%%%%%%%%%%%%%%%%%%%%%%%%%%%%%%%%%
\subsection*{Lengths of generators of the fundamental group\easy}

\begin{pr}{\easy}{Lengths of generators of the fundamental group}\label{Lengths of generators of the fundamental group}
Let $M$ be a compact Riemannian manifold and $p\in M$.
Show that the fundamental group $\pi_1(M,p)$
is generated by the homotopy classes of loops with length at most $2\cdot\diam M$.
\end{pr}

%%%%%%%%%%%%%%%%%%%%%%%%%%%%%%%%%%%%%%%%%
\subsection*{Number of generators}

\begin{pr}{}{Number of generators}\label{Number of generators}
Let $M$ be a complete connected Riemannian manifold with non-negative sectional curvature.
Show that the minimal number of generators of the fundamental group $\pi_1 M$
can be bounded above in terms of the dimension of $M$.
\end{pr}

%%%%%%%%%%%%%%%%%%%%%%%%%%%%%%%%%%%%%%%%%
\subsection*{Equations in the group\easy}

\begin{pr}{\easy}{Equations in the group}\label{Equations in the group}
Assume $G$ is a compact connected Lie group and $n$ is a positive integer.
Show that given a collection of elements $g_1,g_2\dots,g_n\in G$
the equation 
\[x\cdot g_1\cdot x\cdot g_2\cdots x\cdot g_n=1\]
has a solution $x\in G$.
\end{pr}

\section*{Semisolutions}
%%%%%%%%%%%%%%%%%%%%%%%%%%%%%%%%%%%%%%%%%%%%%%%%%%
\parbf{Finite action.}
Without loss of generality, we may assume that the action is generated by a nontrivial homeomorphism 
\[a\:\mathbb{S}^m\to\mathbb{S}^m\] 
and $a^p=\id_{\mathbb{S}^m}$ for some prime $p$.

Assume that any $a$-orbit lies in an open hemisphere.
Then 
\[h(x)=\sum_{n=1}^p a^n\cdot x\ne0\]
for any $x\in\mathbb{S}^m$.

Consider the map $f\:\mathbb{S}^m\to\mathbb{S}^m$ 
defined as  $f(x)=\tfrac{h(x)}{|h(x)|}$.
Note that 
\begin{enumerate}[(i)]
\item if $a(x)=x$, then $f(x)=x$;
\item\label{f(x)=f(a(x))} $f(x)=f\circ a(x)$ for any $x\in\mathbb{S}^m$.
\end{enumerate}

Prove that $f$ is homotopic to the identity; 
in particular 
\[\deg f=1.\leqno({*})\]

Fix $x\in \mathbb{S}^m$ such that $a(x)\ne x$.
Note that $a$ acts without fixed points 
on the inverse image $W=f^{-1}(V)$ 
of a small open neighborhood $V\ni x$.
Therefore the quotient map $\theta\:W\to W'=W/\ZZ_p$ is a $p$-fold covering.
From  (\ref{f(x)=f(a(x))}),
the restriction $f|_W$ factors through $\theta$;
that is
there is $f'\:W'\to V$ such that
$f|_W=f'\circ\theta$.

Assume $p\ne 2$.
Show that $f'$ and $\theta$ have well defined degrees and 
\[\deg f\equiv\deg \theta\cdot\deg f'\pmod p\]
Since $\theta$ is a $p$-fold covering, we have $\deg \theta\equiv0\pmod p$.
Therefore
\[\deg f\equiv 0\pmod p.\leqno({*}{*})\]

Finally observe that $({*})$ contradicts $({*}{*})$.

In the case $p=2$ the same proof works, 
but the degrees have to be defined only modulo $2$.\qeds

Along the same lines one can get a lower bound for the maximal diameter of orbit for any nontrivial actions of finite groups on a Riemannian manifold.

Applying the problem to a conjugate action, one gets that 
if a fixed point set of a finite group action $F\acts \mathbb{S}^m$
has nonempty interior, 
then the action is trivial.
The same holds for any connected manifold;
it was proved by Max Newman in \cite{newman}.

The Newman's theorem was used by Deane Montgomery in \cite{montgomery} 
to show that 
\emph{if $h$ is a homeomorphism of a connected manifold $M$ 
such that each $h$-orbit is finite,
then $h^n=\id_M$ for some positive integer $n$}.


%%%%%%%%%%%%%%%%%%%%%%%%%%%%%%%%%%%%%%%%%%%%%%%%%%
\parbf{Covers of figure eight.}
First show that any compact metric space can be presented as a limit of a sequence of finite metric graphs $\Gamma_n$.
Moreover, show that one can assume  each vertex of $\Gamma_n$ has degree 3 
and the length of each edge in $\Gamma_n$ is multiple of $\tfrac 1n$.

\begin{wrapfigure}{o}{27 mm}
\begin{lpic}[t(-0 mm),b(-3 mm),r(0 mm),l(0 mm)]{pics/fig8(1)}
\end{lpic}
\end{wrapfigure}

It remains to approximate $\Gamma_n$ by finite coverings of $(\Phi,d/n)$.
Guess this part  
from the following picture; it shows the needed approximation of the doted graph.\qeds


The same idea works if instead of figure eight, we have any compact length space $X$ which admits a map $X\to\Phi$
which is surjective on fundamental groups.
Such spaces $X$ can be found among compact hyperbolic manifolds of any dimension $\ge 2$.
All this due to Vedrin Sahovic \cite[see][]{sahovic}.

A similar idea was used later to show that any group can appear as a fundamental group of underlying space of 3-dimensional hyperbolic orbifold \cite[see][]{panov-petrunin-telescopic}.





%%%%%%%%%%%%%%%%%%%%%%%%%%%%%%%%%%%%%%%%%%%%%%%%%%
\parbf{Diameter of \textit{m}-fold cover.}
Fix points $\tilde p,\tilde q\in\tilde M$.
Let  
$\tilde\gamma\:[0,1]\to \tilde M$ be a minimizing geodesic from $\tilde p$ to $\tilde q$. 

We need to show that 
\[\length \tilde\gamma\le m\cdot diam(M).\]
Suppose the contrary.

Denote by $p,q$ and $\gamma$ the projections to $M$ of $\tilde p,\tilde q$ and $\tilde \gamma$. 
Represent $\gamma$
as the concatenation of $m$ paths of equal length,
\[\gamma=\gamma_1{*}\dots{*}\gamma_m,\] 
so
\[\length(\gamma_i)=\length(\gamma)/m>\diam(M).\] 

Let $\sigma_i$ be a minimizing geodesic in $M$ connecting the endpoints of $\gamma_i$. 
Note that 
\[\length\sigma_i\le \diam M< \length\gamma_i.\] 

Consider $m+1$ paths $\alpha_0,\dots,\alpha_m$ defined as 
\[\alpha_i=\sigma_1{*}\dots{*}\sigma_i{*}\gamma_{i+1}{*}\dots{*}\gamma_m.\]

Consider their liftings $\tilde\alpha_0,\dots,\tilde\alpha_m$ 
with $\tilde q$ as the endpoint.
Note that two curves, say $\alpha_i$ and $\alpha_j$ for $i<j$, 
have the same starting point in $\tilde M$.

Consider the path
\[\beta=\gamma_1{*}\dots{*}\gamma_i{*}\sigma_{i+1}{*}\dots{*}\sigma_j{*}\gamma_{j+1}{*}\dots{*}\gamma_m.\]
Prove that there is lift $\tilde\beta$ of $\beta$ 
which connects $\tilde p$ to $\tilde q$ in $\tilde M$.
Clearly $\length \beta<\length \gamma$, a contradiction.
\qeds

The question was asked by Alexander  Nabutovsky
and answered by Sergei Ivanov \cite[see][]{ivanov}.



%%%%%%%%%%%%%%%%%%%%%%%%%%%%%%%%%%%%%%%%%%%%%%%%%%
\parbf{Symmetric square.}
Let $\Gamma=\pi_1 X$ and $\Delta=\pi_1((X\times X)/\ZZ_2)$.
Consider the homomorphism $\phi\:\Gamma\times \Gamma\to \Delta$
induced by the projection $X\times X\to (X\times X)/\ZZ_2$.

Prove that the restrictions $\phi|_{\Gamma\times \{1\}}$ and $\phi|_{\{1\}\times\Gamma}$
are onto.

It remains to note that 
$$\phi(\alpha,1)\phi(1,\beta)=\phi(1,\beta)\phi(\alpha,1)$$
for any $\alpha$ and $\beta$ in $\Gamma$.
\qeds

 
The problem was suggested by Rostislav Matveyev.



%%%%%%%%%%%%%%%%%%%%%%%%%%%%%%%%%%%%%%%%%%%%%%%%%%
\parbf{Sierpi\'nski gasket.}
Denote the Sierpi\'nski gasket by $\triangle$.

Let us show that any homeomorphism of $\triangle$ is also its isometry.
Therefore the group homeomorphisms is the symmetric group $S_3$. 

Let $\{x,y,z\}$ be a 3-point set in $\triangle$ such that $\triangle \backslash\{x,y,z\}$ has 3 connected components.
Prove that there is unique choice for the set $\{x,y,z\}$ and 
it is formed by the midpoints of its big sides.

It follows that any homeomorphism of $\triangle$ permutes the set $\{x,y,z\}$.

A similar argument shows that this permutation  uniquely describes the homeomorphism.
\qeds

The problem was suggested by Bruce Kliener.
The homeomorphism group of Sierpi\'nski carpet is much bigger,
it is instructive to describe this group.



%%%%%%%%%%%%%%%%%%%%%%%%%%%%%%%%%%%%%%%%%%%%%%%%%%
\parbf{Latices in a Lie group.}
Denote by $V_\ell$ and $W_m$
the Voronoi domain of for each $\ell\in L$ and $m\in M$ correspondingly;
that is,
\[V_\ell=\set{g\in G}{|g-\ell|_G\le|g-\ell'|\ \text{for any}\ \ell'\in L}\]
\[W_m=\set{g\in G}{|g-m|_G\le|g-m'|\ \text{for any}\ m'\in M}\]

Note that for any $\ell\in L$ and $m \in M$ we have
\[\begin{aligned}
\vol V_\ell&=\vol(L\backslash (G,h))=
\\
&=\vol(M\backslash (G,h))=
\\
&=\vol W_m.
\end{aligned}
\leqno({*})
\]

Consider the bipartite graph $\Gamma$ with vertices formed by the elements of $L$ and $M$
such that $\ell\in L$ is adjacent  to $m \in M$ if and only if $V_\ell\cap W_m\ne\emptyset$.

By $({*})$ the graph $\Gamma$ satisfies the condition in the marriage theorem ---
any subset in $L$ has at least that many neighbors in $M$ and the other way around.
Therefore there is a bijection $f\: L\to M$ such that 
\[V_\ell\cap W_{f(\ell)}\ne\emptyset\] for any $\ell\in L$. 

It remains to notice that $f$ is bi-Lipschitz.
\qeds

The problem is due to 
Dmitri Burago 
and Bruce Kleiner \cite[see][]{burago-kleiner}. 
For a finitely generated group $G$  
it is not known if $G$ and $G\times \ZZ_2$ can fail to be bi-Lipschitz.
(The groups are assumed to be equipped with word metric.)
 



%%%%%%%%%%%%%%%%%%%%%%%%%%%%%%%%%%%%%%%%%%%%%%%%%%
\parbf{Piecewise Euclidean quotient.}
Note that the group $\Gamma$ serves as holonomy group of the quotient space $P=\RR^m/\Gamma$ with the induced polyhedral metric.
More precisely, one can identify $\RR^m$ with the tangent space of a regular point $x_0$ of $P$ in such a way that
for any $\gamma\in\Gamma$ there is a loop $\ell$ in $P$ which pass only through regular points and has the holonomy $\gamma$.

Fix $\gamma\in\Gamma$. 
Let $\ell$ be the corresponding loop.
Since $P$ is simply connected, we can shrink $\ell$ by a disc.
By general position argument we can assume that the disc 
only pass through simplices of codimension $0$, $1$ and $2$
and intersect the simplices of codimension $2$ transversely.

In other words, $\ell$ can be presented as a product of 
loops such that each loop goes around a single simplex of codimension $2$ and comes back.
The holonomy for each of these loops is a rotation around a hyperplane.
Hence the result follows.
\qeds

The converse to the problem also holds;
it was proved by Christian Lange in \cite{lange},
his proof based earlier results of 
Marina Mikhailova \cite[see][]{mikhailova}.

Note that the cone over spherical suspension over Poincar\'e sphere is homeomorphic to $\RR^5$ and it is quotient of $\RR^5$ by a finite subgroup of $\SO(5)$. 
Therefore, 
if one exchanges ``piecewise linear homeomorphism'' to ``homeomorphism'' in the formulation, 
then the answer is different; 
a complete classification of such actions was also obtained in \cite{lange}.

%%%%%%%%%%%%%%%%%%%%%%%%%%%%%%%%%%%%%%%%%%%%%%%%%%
\parbf{Subgroups of free group.}
Let $F$ be a free group;
note that we can assume that $F$ has finite number, say $m$, generators.

Fix a finitely generated subgroup $G$ of free group $F$.

Let $W$ be the wedge sum of $m$ circles, 
so  $\pi_1(W,p)=F$.
Equip $W$ with the length-metric 
such that each circle has unit length.

Pass to the metric cover $\tilde W$ of $W$ 
such that  $\pi_1(\tilde W,\tilde p)=G$ 
for a lift $\tilde p$ of $p$.

\begin{wrapfigure}{r}{18 mm}
\begin{lpic}[t(-4 mm),b(-0 mm),r(0 mm),l(0 mm)]{pics/ball-in-group(1)}
\lbl[t]{8.5,7;$\tilde p$}
\lbl[t]{8.5,-2;{\small $\bar B(\tilde p,2+\tfrac12)$}}
\end{lpic}
\end{wrapfigure}

Fix sufficiently large integer $n$ and consider doubling of the closed ball $\bar B(\tilde p,n+\frac12)$ in its boundary.
Let us denote the obtained doubling by $Z_n$ and set $G_n=\pi(Z_n,\tilde p)$.

Prove that $Z_n$ is a metric covering of $W$;
it makes possible to consider $G_n$ as a subgroup of $F$.
By construction, $Z_n$ is compact;
therefore $G_n$ has finite order in $F$.


It remains to show that 
\[G=\bigcap_{n>k} G_n,\]
where $k$ is the maximal length of word in the generating set of $G$.
\qeds

Originally the problem was solved by Marshall Hall in \cite{hall}.
The proof presented here is close to the solution of John Stalings in \cite{stallings};
see also \cite{wilton}.

The same idea can be used to solve many other problems; here are some examples.
\begin{itemize}
\item {\it Show that subgroups of free groups are free.}
\item {\it Show that two elements of the free groups $u$ and $v$ commute 
if and only if they are both powers of
the some element $w$.}
\end{itemize}



%%%%%%%%%%%%%%%%%%%%%%%%%%%%%%%%%%%%%%%%%%%%%%%%%%
\parbf{Lengths of generators of the fundamental group.}
Choose a length minimizing loop $\gamma$ 
which represents a given element $a\in\pi_1M$.

Fix $\eps>0$.
Represent $\gamma$ 
as a concatenation
\[\gamma=\gamma_1{*}\dots{*}\gamma_n\]
of paths with $\length\gamma_i<\eps$ for each $i$.
 
Denote by $p=p_0,p_1,\dots, p_n=p$ the endpoints of these arcs.
Connect $p$ to $p_i$ by a minimizing geodesic $\sigma_i$.
Note that $\gamma$ is homotopic to a product of loops
\[\alpha_i=\sigma_{i-1}{*}\gamma_i{*}\sigma_{i-1}\]
and $\length \alpha_i<2\cdot\diam M+\eps$ for each $i$.

It remains to show that for sufficiently small $\eps>0$
any loop with length less than $2\cdot\diam M+\eps$ 
is homotopic to a loop with length at most $2\cdot\diam M$.
\qeds

The statement is due to 
Mikhael Gromov \cite[Prop. 3.22 in][]{gromov-MetStr}.

%%%%%%%%%%%%%%%%%%%%%%%%%%%%%%%%%%%%%%%%%%%%%%%%%%
\parbf{Number of generators.}
Consider universal Riemannian cover $\tilde M$ of $M$.
Note that $\tilde M$ is non-negatively curved and
$\pi_1M$ acts by isometries on $\tilde M$.

Fix $p\in \tilde M$.
Given  $a\in \pi_1M$,
set 
\[|a|=|p- a\cdot p|_{\tilde M}.\]
Construct a sequence of elements $a_1,a_2,\dots\in \pi_1M$ the following way:
\begin{enumerate}[(i)]
\item Choose $a_1\in\pi_1M$ so that $|a_1|$ takes the minimal value.
\item Choose $a_2\in\pi_1M\backslash\langle a_1 \rangle$ so that $|a_2|$ takes the minimal value.
\item Choose $a_3\in\pi_1M\backslash\langle a_1,a_2 \rangle$ so that $|a_2|$ takes the minimal value.
\item and so on.
\end{enumerate}

Note that the sequence terminates at $n$-th step 
if 
$(a_1,\dots,a_n)$ forms a generating system.
By construction, we have
\begin{align*}
|a_j\cdot a_i^{-1}|&\ge |a_j|\ge |a_i|
\intertext{for any $j>i$. 
Set $p_i=a_i\cdot p$.
Note that}
|p_j-p_i|_{\tilde M}
&=|a_j\cdot a_i^{-1}|\ge
\\
&\ge |a_j|=
\\
&=|p_j-p|_{\tilde M}\ge
\\
&\ge|a_i|=
\\
&=|p_i-p|_{\tilde M}.
\intertext{By Toponogov comparison theorem we get}
\measuredangle \hinge p{p_i}{p_j}&\ge \tfrac\pi3.
\end{align*}
That is, the directions from $p$ to all $p_i$ lie on the angle at least $\tfrac\pi3$ from each other.

Therefore the number of $p_i$ can be bounded in terms of the dimension of $M$.
Hence the result follows.
\qeds

This construction introduced by Mikhael Gromov 
in his paper on almost flat manifolds \cite[see][]{gromov-almost-flat}.

%%%%%%%%%%%%%%%%%%%%%%%%%%%%%%%%%%%%%%%%%%%%%%%%%%
\parbf{Equations in the group.} 
Set 
\[f(x)=x\cdot g_1\cdot x\cdot g_2\cdots x\cdot g_n.\]
Note that $f$ is homotopic to the map 
\[h\:x\mapsto x^n.\]
It remains to note that degree of $h$ is not zero.
\qeds

The idea of solution is due to Murray Gerstenhaber and Oscar Rothaus 
\cite[see][]{gerstenhaber-rothaus}.
In fact $\deg h=n^k$, where $k$ is the rank of $G$;
it was proved by Heinz Hopf in \cite{hopf}.