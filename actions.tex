\csname @openrightfalse\endcsname
\chapter{Actions and coverings}

%%%%%%%%%%%%%%%%%%%%%%%%%%%%%%%%%%%%%%%%%
\subsection*{Bounded orbit}
\label{Bounded orbit}

Recall that a metric space is called \index{proper metric space}\emph{proper} if all its bounded closed sets are compact.

\begin{pr} Let $X$ be a 
proper metric space 
and $\iota\:X\to X$ is an isometry.
Assume that for some $x\in X$, the sequence $x_n\z=\iota^n(x)$, $n\in\ZZ$ has a converging subsequence.
Prove that $x_n$ is bounded.
\end{pr}

%%%%%%%%%%%%%%%%%%%%%%%%%%%%%%%%%%%%%%%%%%%%%%%%%%
\parit{Semisolution.}
Note that we can assume that the orbit $\{x_n\}$ is dense in $X$;
otherwise pass to the closure of the orbit.
In particular, we can choose a finite number of positive integer values $n_1,\dots,n_k$
such that the set of points $\{x_{n_1},\dots,x_{n_k}\}$ is a $\tfrac1{10}$-net for the ball $B(x_0,10)$;
that is, for any $x\in B(x_0,10)$ there is $x_{n_i}$ such that
\[|x-x_{n_i}|<\tfrac1{10}.\]

Assume $x_m\in B(x_0,1)$ for some $m$.
Then 
\[B(x_m,10)=f^m( B(x_0,10))\supset B(x_0,1).\] 
In particular, $\{x_{m+n_1},\dots,x_{m+n_k}\}$ is a $\tfrac1{10}$-net for the ball $B(x_0,1)$
Therefore $x_{m+n_i}\in B(x_0,1)$ for some $i\z\in\{1,\dots,k\}$.

Set $N=\max_i\{n_i\}$.
Applying the above observation inductively, we get that from any string $x_{i+1},\dots x_{i+N}$
at least one point lies in $B(x_0,1)$.
In particular, the $N$ balls
\[B(x_1,10),\dots,B(x_N,10)\]
cover whole $X$.
Hence the result follows.\qeds

The problem is due to Aleksander Ca{\l}ka's \cite[see][]{calka}.

%%%%%%%%%%%%%%%%%%%%%%%%%%%%%%%%%%%%%%%%%
\subsection*{Finite action}\label{Finite action}

\begin{pr}
Show that for any nontrivial continuous action of a finite group on the unit sphere
there is an orbit which does not lie in the interior of a hemisphere.
\end{pr}

%%%%%%%%%%%%%%%%%%%%%%%%%%%%%%%%%%%%%%%%%


\subsection*{Covers of figure eight}\label{figure-eight-1}

Given a covering 
\[f\:\tilde X \to X\]
of the length-metric space $X$,
one can consider the induced length-metric on $\tilde X$
defining length of curve $\alpha$ in $X$ as the length of the composition $f\circ\alpha$; the obtained metric space $\tilde X$ is called \index{metric covering}\emph{metric covering} of $X$.

{

\begin{wrapfigure}[3]{r}{28 mm}
\begin{lpic}[t(-7 mm),b(-5 mm),r(0 mm),l(0 mm)]{pics/figure-eight(1)}
\end{lpic}
\end{wrapfigure}

Let us define \index{figure eight}\emph{figure eight} as the
length-metric space which
is obtained by gluing together all four ends of two unit segments.

}

\begin{pr}
Show that any compact length-metric space 
is a Gromov--Hausdorff limit of a sequence of
metric covers  
\[(\widetilde \Phi_n, \tilde d/n)\to(\Phi,d/n),\]
where $(\Phi,d)$ denotes the figure eight.
\end{pr}


%%%%%%%%%%%%%%%%%%%%%%%%%%%%%%%%%%%%%%%%%
\subsection*{Diameter of \textit{m}-fold cover\hard}\label{m-fold-cover}

The metric covering is defined in the previous problem.

\begin{pr}
Let $X$ be a length-metric space
and $\tilde X$ be its $m$-fold metric covering of $X$.
Show that
$$\diam\tilde X\le m\cdot \diam X.$$
\end{pr}

From the diagram below you could guess an example of 5-fold cover with diameter of the total space exactly 5 times diameter of the target.

\begin{center}
\begin{lpic}[t(0mm),b(0 mm),r(0 mm),l(0 mm)]{pics/5-fold(1)}
\lbl[t]{49,4.5;$\to$}
\end{lpic}
\end{center}

%%%%%%%%%%%%%%%%%%%%%%%%%%%%%%%%%%%%%%%%%
\subsection*{Symmetric square\easy}\label{Symmetric square}

Let $X$ be a topological space.
Note that $X{\times} X$ admits a natural $\ZZ_2$-action generated by the involution $(x,y)\mapsto (y,x)$.
The quotient  space $X{\times} X/\ZZ_2$ is called \index{symmetric square}\emph{symmetric square} of $X$.

\begin{pr} 
Show that symmetric square 
of any path connected topological space 
has commutative the fundamental group.
\end{pr}

{

\begin{wrapfigure}[4]{r}{23 mm}
\begin{lpic}[t(2 mm),b(-0 mm),r(0 mm),l(0 mm)]{pics/serpinski-triangle(1)}
\end{lpic}
\end{wrapfigure}

%%%%%%%%%%%%%%%%%%%%%%%%%%%%%%%%%%%%%%%%%
\subsection*{Sierpi\'nski gasket\easy}\label{Sierpinski triangle}

To construct Sierpi\'nski gasket, start with a solid  equilateral triangle, subdivide it into four smaller congruent equilateral triangles and remove the interior of the central one.
Repeat this procedure recursively for each of the remaining solid triangles.

}

\begin{pr} 
Find the homeomorphism group of the Sierpi\'nski gasket.
\end{pr}



%%%%%%%%%%%%%%%%%%%%%%%%%%%%%%%%%%%%%%%%%
\subsection*{Lattices in a Lie group}\label{Boys and girls in a Lie group}

\begin{pr}
Let $L$ and $M$ be two discrete subgroups
of a connected Lie group $G$ and $h$ be a left
invariant metric on $G$.
Equip the groups $L$ and $M$ 
with the metrics induced from $G$.
Assume $L\backslash G$ and $M\backslash G$ are compact and
$$\vol(L\backslash (G,h))
=
\vol(M\backslash (G,h)).$$
Prove that there is a bi-Lipschitz one-to-one mapping $f\:L
\to
M$, not necessarily a homomorphism.
\end{pr}


%%%%%%%%%%%%%%%%%%%%%%%%%%%%%%%%%%%%%%%%%
\subsection*{Piecewise Euclidean quotient}\label{Piecewise Euclidean quotient}

Note that the quotient of Euclidean space by a finite subgroup of $\SO(m)$ is a {}\emph{polyhedral space} as it defined on page \pageref{piecewise linear map};
on the same page you find the definition of piecewise linear homeomorphism.


\begin{pr}
Let $\Gamma$ be a finite subgroup of $\SO(m)$.
Denote by $P$ the quotient $\RR^m/\Gamma$ equipped with induced
polyhedral metric.
Assume $P$ admits a piecewise linear homeomorphism to $\RR^m$.
Show that $\Gamma$ is generated by rotations  around subspaces of codimension $2$.
\end{pr}

%%%%%%%%%%%%%%%%%%%%%%%%%%%%%%%%%%%%%%%%%
\subsection*{Subgroups of a free group}\label{Subgroups of free group} 

\begin{pr}
Show that every finitely generated subgroup of the free group 
is an intersection of subgroups of finite index.
\end{pr}

%%%%%%%%%%%%%%%%%%%%%%%%%%%%%%%%%%%%%%%%%
\subsection*{Short generators\easy}\label{Lengths of generators of the fundamental group}

\begin{pr}
Let $M$ be a compact Riemannian manifold and $p\in M$.
Show that the fundamental group $\pi_1(M,p)$
is generated by the homotopy classes of loops with length at most $2\cdot\diam M$.
\end{pr}

%%%%%%%%%%%%%%%%%%%%%%%%%%%%%%%%%%%%%%%%%
\subsection*{Number of generators}\label{Number of generators}

\begin{pr}
Let $M$ be a complete connected Riemannian manifold with non-negative sectional curvature.
Show that the minimal number of generators of the fundamental group $\pi_1 M$
can be bounded above in terms of the dimension of $M$.
\end{pr}

%%%%%%%%%%%%%%%%%%%%%%%%%%%%%%%%%%%%%%%%%
\subsection*{Equation in a Lie group\easy}\label{Equations in the group}

\begin{pr}
Assume $G$ is a compact connected Lie group and $n$ is a positive integer.
Show that given a collection of elements $g_1,g_2\dots,g_n\in G$
the equation 
\[x\cdot g_1\cdot x\cdot g_2\cdots x\cdot g_n=1\]
has a solution $x\in G$.
\end{pr}

\section*{Semisolutions}
%%%%%%%%%%%%%%%%%%%%%%%%%%%%%%%%%%%%%%%%%%%%%%%%%%
\parbf{Finite action.}
Without loss of generality, we may assume that the action is generated by a nontrivial homeomorphism 
\[a\:\mathbb{S}^m\to\mathbb{S}^m\] 
with prime order $p$.

Assume contrary, that is, any $a$-orbit lies in an open hemisphere.
Then 
\[h(x)=\sum_{n=1}^p a^n\cdot x\ne0\]
for any $x\in\mathbb{S}^m$; here we consider $\mathbb{S}^m$ as the unit sphere in $\mathbb{R}^{m+1}$.

Consider the map $f\:\mathbb{S}^m\to\mathbb{S}^m$ 
defined as  $f(x)=\tfrac{h(x)}{|h(x)|}$.
Note that 
\begin{itemize}
\item if $a(x)=x$, then $f(x)=x$;
\item\label{f(x)=f(a(x))} $f(x)=f\circ a(x)$ for any $x\in\mathbb{S}^m$.
\end{itemize}

Note further that $f$ is homotopic to the identity; 
in particular 
\[\deg f=1.
\leqno({*})\]
The homotopy can be constructed as $(x,t)\mapsto \gamma_x(t)$,
where $\gamma_x$ is the minimizing geodesic path in $\mathbb{S}^m$ from $x$ to $f(x)$.
By construction, $|x-f(x)|_{\mathbb{S}^m}<\tfrac\pi2$; 
therefore $\gamma_x$ is uniquely defined.



Fix $x\in \mathbb{S}^m$ such that $a(x)\ne x$.
Note that the group acts without fixed points 
on the inverse image $W=f^{-1}(V)$ 
of a small open neighborhood $V\ni x$.
Therefore the quotient map $\theta\:W\to W'=W/\ZZ_p$ is a $p$-fold covering.
From  (\ref{f(x)=f(a(x))}),
the restriction $f|_W$ factors thru $\theta$;
that is,
there is $f'\:W'\to V$ such that
$f|_W=f'\circ\theta$.

Assume $p\ne 2$.
Note that $f'$ and $\theta$ have well defined degrees and 
\[\deg f\equiv\deg \theta\cdot\deg f'\pmod p\]
Since $\theta$ is a $p$-fold covering, we have $\deg \theta\equiv0\pmod p$.
Therefore
\[\deg f\equiv 0\pmod p.\leqno({*}{*})\]

Finally observe that $({*})$ contradicts $({*}{*})$.

In the case $p=2$ the same proof works, 
but the degrees have to be considered modulo $2$.\qeds

Along the same lines one can get a lower bound for the maximal diameter of orbit for any nontrivial actions of finite groups on a Riemannian manifold.

Applying the problem to the conjugate actions, 
one gets that if a fixed point set of a finite group acting on a sphere
has nonempty interior, 
then the action is trivial.
The same holds for any connected manifold.
All this was proved by Max Newman \cite[see][]{newman}.

The following problem from \cite{montgomery} can be solved using Newman's theorem. 
\begin{pr}
Assume $h$ is a homeomorphism of a connected manifold $M$ 
such that each $h$-orbit is finite.
Show that $h$ has finite order.
\end{pr}


%%%%%%%%%%%%%%%%%%%%%%%%%%%%%%%%%%%%%%%%%%%%%%%%%%
\parbf{Covers of figure eight.}
First note that any compact length-metric space $K$ can be approximated by finite metric graph.

Indeed, fix a finite $\eps$-net $F$ in $K$.
For each pair $x,y\in F$ choose a chain of points $x=x_0,x_1\dots x_n=y$ such that
$|x_i-x_{i-1}|_K<\eps$ for each $i$ and 
\[|x-y|_K=|x_0-x_1|_K+\dots+|x_{n-1}-x_n|_K.\]
Denote by $F'$ the union of all these chains with $F$;
Consider the metric graph with $F'$ as the set of vertexes
where every pair of vertexes $v$ and $w$ such that $|v-w|_K<\eps$ is connected by an edge of length $|v-w|_K$.
Note that the obtained metric graph is $\eps$ close to $K$ in the sense of Gromov--Hausdorff.

\begin{wrapfigure}{o}{27 mm}
\begin{lpic}[t(-0 mm),b(-3 mm),r(0 mm),l(0 mm)]{pics/fig8(1)}
\end{lpic}
\end{wrapfigure}

Further, any finite metric graph is a limit of metric graphs $\Gamma_n$ such that the length of each edge is a multiple of $\tfrac 1n$ and degree of each vertex is 3.

It remains to approximate $\Gamma_n$ by finite coverings of $(\Phi,d/n)$.
Guess this part from the picture; 
it shows the needed covering of figure eight for the doted graph.\qeds


The same idea works if instead of figure eight, we have any compact length-metric space $X$ which admits a map $X\to\Phi$
which is surjective on fundamental groups.
Such spaces $X$ can be found among compact hyperbolic manifolds of any dimension $\ge 2$.
All this due to Vedrin Sahovic \cite[see][]{sahovic}.

A similar idea was used later to show that any group can appear as a fundamental group of underlying space of 3-dimensional hyperbolic orbifold \cite[see][]{panov-petrunin-telescopic}.





%%%%%%%%%%%%%%%%%%%%%%%%%%%%%%%%%%%%%%%%%%%%%%%%%%
\parbf{Diameter of \textit{m}-fold cover.}
Fix points $\tilde p,\tilde q\in\tilde M$.
Let  
$\tilde\gamma\:[0,1]\to \tilde M$ be a minimizing geodesic path from $\tilde p$ to $\tilde q$. 

We need to show that 
\[\length \tilde\gamma\le m\cdot \diam M.\]
Suppose the contrary.

Denote by $p,q$ and $\gamma$ the projections to $M$ of $\tilde p,\tilde q$ and $\tilde \gamma$. 
Represent $\gamma$
as the concatenation of $m$ paths of equal length,
\[\gamma=\gamma_1{*}\dots{*}\gamma_m,\] 
so
\[\length\gamma_i=\tfrac{1}m\cdot\length\gamma>\diam M.\] 

Let $\sigma_i$ be a minimizing geodesic in $M$ connecting the endpoints of $\gamma_i$. 
Note that 
\[\length\sigma_i\le \diam M< \length\gamma_i.\] 

Consider $m+1$ paths $\alpha_0,\dots,\alpha_m$ defined as the concatenations 
\[\alpha_i=\sigma_1{*}\dots{*}\sigma_i{*}\gamma_{i+1}{*}\dots{*}\gamma_m.\]

Let $\tilde\alpha_0,\dots,\tilde\alpha_m$ be their liftings
with $\tilde q$ as the endpoint.

The staring points of $\tilde\alpha_i$ lies in one of $m$ inverse images of $p$. 
Therefore two curves, $\alpha_i$ and $\alpha_j$ for $i<j$, 
have the same starting point in $\tilde M$.

Note that the concatenation
\[\beta=\gamma_1{*}\dots{*}\gamma_i{*}\sigma_{i+1}{*}\dots{*}\sigma_j{*}\gamma_{j+1}{*}\dots{*}\gamma_m.\]
admits a lift $\tilde\beta$ 
which connects $\tilde p$ to $\tilde q$ in $\tilde M$.
Clearly $\length \tilde\beta<\length \gamma$, a contradiction.
\qeds

The question was asked by Alexander  Nabutovsky
and answered by Sergei Ivanov \cite[see][]{ivanov}.



%%%%%%%%%%%%%%%%%%%%%%%%%%%%%%%%%%%%%%%%%%%%%%%%%%
\parbf{Symmetric square.}
Let $\Gamma=\pi_1 X$ and $\Delta=\pi_1((X\times X)/\ZZ_2)$.
Consider the homomorphism $\phi\:\Gamma\times \Gamma\to \Delta$
induced by the projection $X\times X\to (X\times X)/\ZZ_2$.

Note that $\phi(\alpha,1)=\phi(1,\alpha)$ for any $\alpha\in \Gamma$ and the restrictions $\phi|_{\Gamma\times \{1\}}$ and $\phi|_{\{1\}\times\Gamma}$
are onto.

It remains to note that 
$$\phi(\alpha,1)\phi(1,\beta)=\phi(1,\beta)\phi(\alpha,1)$$
for any $\alpha$ and $\beta$ in $\Gamma$.
\qeds

 
The problem was suggested by Rostislav Matveyev.



%%%%%%%%%%%%%%%%%%%%%%%%%%%%%%%%%%%%%%%%%%%%%%%%%%
\parbf{Sierpi\'nski gasket.}
Denote the Sierpi\'nski gasket by $\triangle$.

Let us show that any homeomorphism of $\triangle$ is also its isometry.
Therefore the group homeomorphisms is the symmetric group $S_3$. 

Let $\{x,y,z\}$ be a 3-point set in $\triangle$ such that $\triangle \backslash\{x,y,z\}$ has 3 connected components.
Note that there is unique choice for the set $\{x,y,z\}$ and 
it is formed by the midpoints of its big sides.

It follows that any homeomorphism of $\triangle$ permutes the set $\{x,y,z\}$.

A similar argument shows that this permutation  uniquely describes the homeomorphism.
\qeds

The problem was suggested by Bruce Kleiner.
The homeomorphism group of Sierpi\'nski carpet is much more interesting .



%%%%%%%%%%%%%%%%%%%%%%%%%%%%%%%%%%%%%%%%%%%%%%%%%%
\parbf{Latices in a Lie group.}
Denote by $V_\ell$ and $W_m$
the Voronoi domain of for each $\ell\in L$ and $m\in M$ correspondingly;
that is,
\[V_\ell=\set{g\in G}{|g-\ell|_G\le|g-\ell'|_G\ \text{for any}\ \ell'\in L}\]
\[W_m=\set{g\in G}{|g-m|_G\le|g-m'|_G\ \text{for any}\ m'\in M}\]

Note that for any $\ell\in L$ and $m \in M$ we have
\[\begin{aligned}
\vol V_\ell&=\vol(L\backslash (G,h))=
\\
&=\vol(M\backslash (G,h))=
\\
&=\vol W_m.
\end{aligned}
\leqno({*})
\]

Consider the bipartite graph $\Gamma$ with the parts $L$ and $M$
such that $\ell\in L$ is adjacent  to $m \in M$ if and only if $V_\ell\cap W_m\ne\emptyset$.

By $({*})$ the graph $\Gamma$ satisfies the condition in the marriage theorem  ---
any subset in $L$ has at least that many neighbors in $M$ and the other way around \cite[see][]{hall}.
Therefore there is a bijection $f\: L\to M$ such that 
\[V_\ell\cap W_{f(\ell)}\ne\emptyset\] for any $\ell\in L$. 

It remains to observe that $f$ is bi-Lipschitz.
\qeds

The problem is due to 
Dmitri Burago 
and Bruce Kleiner \cite[see][]{burago-kleiner}. 
For a finitely generated group $G$  
it is not known if $G$ and $G\times \ZZ_2$ can fail to be bi-Lipschitz.
(The groups are assumed to be equipped with word metric.)
 



%%%%%%%%%%%%%%%%%%%%%%%%%%%%%%%%%%%%%%%%%%%%%%%%%%

\begin{wrapfigure}{r}{41 mm}
\begin{lpic}[t(-4 mm),b(-0 mm),r(0 mm),l(0 mm)]{pics/loop(1)}
\lbl[t]{10,-1;$x_0$}
\lbl[lb]{35,35;$\ell$}
\lbl{18,33;disc}
\end{lpic}
\end{wrapfigure}

\parbf{Piecewise Euclidean quotient.}
Note that the group $\Gamma$ serves as holonomy group of the quotient space $P=\RR^m/\Gamma$ with the induced polyhedral metric.
More precisely, one can identify $\RR^m$ with the tangent space of a regular point $x_0$ of $P$ in such a way that
for any $\gamma\in\Gamma$ there is a loop $\ell$ in $P$ which pass only thru regular points and has the holonomy $\gamma$.

Fix $\gamma$ and $\ell$ as above.
Since $P$ is simply connected, we can shrink $\ell$ by a disc.
By general position argument we can assume that the disc 
only pass thru simplices of codimension $0$, $1$ and $2$
and intersect the simplices of codimension $2$ transversely.

In other words, $\ell$ can be presented as a product of 
loops such that each loop goes around a single simplex of codimension $2$ and comes back.
The holonomy for each of these loops is a rotation around a hyperplane.
Hence the result follows.
\qeds

The converse to the problem also holds;
it was proved by Christian Lange \cite[see][]{lange};
his proof based earlier results of 
Marina Mikhailova \cite[see][]{mikhailova}.

Note that the cone over spherical suspension over Poincar\'e sphere is homeomorphic to $\RR^5$ and it is the quotient of $\RR^5$ by the binary icosahedral group, which is a subgroup of $\SO(5)$ of order 120. 
Therefore, 
if one exchanges ``piecewise linear homeomorphism'' to ``homeomorphism'' in the formulation, 
then the answer is different; 
a complete classification of such actions is given in \cite{lange}.

%%%%%%%%%%%%%%%%%%%%%%%%%%%%%%%%%%%%%%%%%%%%%%%%%%
\parbf{Subgroups of a free group.}
The proof exploits that free group is a fundamental group of graph.

\begin{wrapfigure}{r}{18 mm}
\begin{lpic}[t(-4 mm),b(-0 mm),r(0 mm),l(0 mm)]{pics/ball-in-group(1)}
\lbl[t]{8.5,7;$\tilde p$}
\lbl[t]{8.5,-2;{\small $\bar B(\tilde p,2+\tfrac12)$}}
\end{lpic}
\end{wrapfigure}

\medskip

Let $F$ be a free group and $G$ be a finitely generated subgroup in $F$.
We need to show that $G$ is an intersection of subgroups of finite index in $F$.
Without loss of generality we can assume that $F$ has finite number generators, denote it by $m$.

Let $W$ be the wedge sum of $m$ circles, 
so  $\pi_1(W,p)=F$.
Equip $W$ with the length-metric 
such that each circle has unit length.

Pass to the metric cover $\tilde W$ of $W$ 
such that  $\pi_1(\tilde W,\tilde p)=G$ 
for a lift $\tilde p$ of $p$.

Fix sufficiently large integer $n$ and consider doubling of the closed ball $\bar B(\tilde p,n+\frac12)$ along  its boundary.
Let us denote the obtained doubling by $Z_n$ and set $G_n=\pi(Z_n,\tilde p)$.

Note that $Z_n$ is a metric covering of $W$;
it makes possible to consider $G_n$ as a subgroup of $F$.
By construction, $Z_n$ is compact;
therefore $G_n$ has finite index in $F$.


It remains to show that 
\[G=\bigcap_{n>k} G_n,\]
where $k$ is the maximal length of word in the generating set of $G$.
\qeds

Originally the problem was solved by Marshall Hall \cite[see][]{hall}.
The proof presented here is close to the solution of John Stalings [see \ncite{stallings} and also \ncite{wilton}].

The same idea can be used to solve many other problems; here are some examples.
\begin{itemize}
\item {\it Show that subgroups of free groups are free.}
\item {\it Show that two elements of the free groups $u$ and $v$ commute 
if and only if they are both powers of
the some element $w$.}
\end{itemize}



%%%%%%%%%%%%%%%%%%%%%%%%%%%%%%%%%%%%%%%%%%%%%%%%%%
\parbf{Short generators.}
Choose a length minimizing loop $\gamma$ 
which represents a given element $a\in\pi_1M$.

Fix $\eps>0$.
Represent $\gamma$ 
as a concatenation
\[\gamma=\gamma_1{*}\dots{*}\gamma_n\]
of paths with $\length\gamma_i<\eps$ for each $i$.
 
Denote by $p=p_0,p_1,\dots, p_n=p$ the endpoints of these arcs.
Connect $p$ to $p_i$ by a minimizing geodesic $\sigma_i$.
Note that $\gamma$ is homotopic to a product of loops
\[\alpha_i=\sigma_{i-1}{*}\gamma_i{*}\sigma_{i-1}\]
and $\length \alpha_i<2\cdot\diam M+\eps$ for each $i$.

Given $\ell>0$, there are only finitely many elements of fundamental group which which can be realized by loops shorter than $\ell$ of at $p$.
It follows that for right choice of $\eps>0$, 
any loop $\sigma_i$ is homotopic to a loop of length at most $2\cdot\diam M$.
Hence the result follows.
\qeds

The statement is due to 
Mikhael Gromov \cite[see Proposition 3.22 in][]{gromov-MetStr}.

%%%%%%%%%%%%%%%%%%%%%%%%%%%%%%%%%%%%%%%%%%%%%%%%%%
\parbf{Number of generators.}
Consider the universal Riemannian cover $\tilde M$ of $M$.
Note that $\tilde M$ is non-negatively curved and
$\pi_1M$ acts by isometries on $\tilde M$.

Fix $p\in \tilde M$.
Given  $a\in \pi_1M$,
set 
\[|a|=|p- a\cdot p|_{\tilde M}.\]

Consider the so called \index{short basis}\emph{short basis} in $\pi_1M$;
that is, a sequence of elements $a_1,a_2,\dots\in \pi_1M$ defined the following way:
\begin{enumerate}[(i)]
\item Choose $a_1\in\pi_1M$ so that $|a_1|$ takes the minimal value.
\item Choose $a_2\in\pi_1M\backslash\langle a_1 \rangle$ so that $|a_2|$ takes the minimal value.
\item Choose $a_3\in\pi_1M\backslash\langle a_1,a_2 \rangle$ so that $|a_2|$ takes the minimal value.
\item and so on.
\end{enumerate}

Note that the sequence terminates at $n$-th step 
if 
$a_1,\dots,a_n$  generate $\pi_1M$.
By construction, we have
\begin{align*}
|a_j\cdot a_i^{-1}|&\ge |a_j|\ge |a_i|
\intertext{for any $j>i$. 
Set $p_i=a_i\cdot p$.
Note that}
|p_j-p_i|_{\tilde M}
&=|a_j\cdot a_i^{-1}|\ge
\\
&\ge |a_j|=
\\
&=|p_j-p|_{\tilde M}\ge
\\
&\ge|a_i|=
\\
&=|p_i-p|_{\tilde M}.
\intertext{By Toponogov comparison theorem we get}
\measuredangle \hinge p{p_i}{p_j}&\ge \tfrac\pi3.
\end{align*}
That is, the directions from $p$ to all $p_i$ lie on the angle at least $\tfrac\pi3$ from each other.

Therefore the number of points $p_i$ can be bounded in terms of the dimension of $M$.
Hence the result follows.
\qeds

The \emph{short basis construction} as well as the result above are due to Mikhael Gromov \cite[see][]{gromov-almost-flat}.

%%%%%%%%%%%%%%%%%%%%%%%%%%%%%%%%%%%%%%%%%%%%%%%%%%
\parbf{Equation in a Lie group.} 
We will assume that $G$ is equipped with bi-invariant metric.
In particular geodesics starting from $1\in G$ are given by homomorphisms $\RR\to G$.

Consider the map $f\:G\to G$ defined as
\[f(x)=x\cdot g_1\cdot x\cdot g_2\cdots x\cdot g_n.\]
We need to show that $f$ is onto.
Note that it is sufficient to show that $f$ has non zero degree.

The map $f$ is homotopic to the map $h\:x\mapsto x^n$.
Therefore it is sufficient to show that
\[\deg h\ne 0\leqno({*})\]

Note that the claim $({*})$ follows from $({*}{*})$.
\begin{cl}{$({*}{*})$} For any $x\in G$ the differential 
 \[d_xh\:\T_xG\to \T_{x^n}G\] 
does not revert orientation.
\end{cl}


Indeed, connect $1$ to a given point $y\in G$ by a geodesic path $\gamma$, so $\gamma(0)=1$ and $\gamma(1)=y$.
Since $\gamma\:\RR\to G$ is a homomorphism,
$h(x)=y$ for $x=\gamma(\tfrac1n)$.
In particular the inverse image $h^{-1}\{y\}$ is nonempty for any $y\in G$.

By $({*}{*})$, for a regular value $y$, each point in the  inverse image $h^{-1}\{y\}$ conributes 1 to the degree of $h$. Hence $({*})$ follows.

It remains to prove $({*}{*})$.
Given an element $g\in G$, denote by $L_g,R_g\:G\to G$ its left and right shifts;
that is, $L_g(x)=g\cdot x$ and $R_g(x)=x\cdot g$.
Identify the tangent spaces $\T_xG$ and $\T_{x^n}G$ with the Lie algebra $\mathfrak{g}=\T_eG$
using $d{R_x}\:\mathfrak{g}\to \T_xG$ and $d{R_x^n}\:\mathfrak{g}\to \T_{x^n}G$ correspondingly.
Then for any $V\in \mathfrak{g}$, we have
\[d_xh(V)=V+\Ad_x(V)+\dots+\Ad_x^{n-1}(V),\]
where $\Ad_x=d(L_x\circ R_{x^{-1}})\:\mathfrak{g}\to \mathfrak{g}$. 
Since the metric on $G$ is bi-invariant, we have $\Ad_x\in\SO(\mathfrak{g})$.
It remains to note that the linear transformation
\[V\mapsto V+T(V)+\dots+T^{n-1}(V)\]
can not revert orientation for any $T\in \SO_m$.
The last statement is an exercise in linear algebra.
\qeds

The idea of this solution is due to Murray Gerstenhaber and Oscar Rothaus 
\cite[see][]{gerstenhaber-rothaus}.
In fact the degree of $g$ is $n^k$, where $k$ is the rank of~$G$ \cite[see][]{hopf}.