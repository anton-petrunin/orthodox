\csname @openrightfalse\endcsname
\chapter{Comparison geometry}

In this chapter we consider Riemannian manifolds with curvature bounds.

This chapter is very demanding;
we assume that the reader is familiar with   
shape operator and second fundamental form, 
equations of Riccati and Jacobi,
comparison theorems,
and Morse theory.
The classical book \cite{cheeger-ebin} covers all the  necessary  material.

%%%%%%%%%%%%%%%%%%%%%%%%%%%%%%%%%%%%%%%%%
\subsection*{Geodesic immersion\hard}
\label{Geodesic immersion}

An isometric immersion $\iota\:N\looparrowright M$ from one Riemannian manifold to another is called \index{totally geodesic}\emph{totally geodesic} if it maps any geodesic in $N$ to a geodesic in $M$.

\begin{pr}
Let $M$ and $N$ be simply connected positively curved Riemannian manifolds and $\iota\:N\looparrowright M$ a totally geodesic immersion.
Assume that 
\[\dim N>\tfrac 12\cdot \dim M.\]
Prove that $\iota$ is an embedding.
\end{pr}

%%%%%%%%%%%%%%%%%%%%%%%%%%%%%%%%%%%%%%%%%%%%%%%%%%
\parit{Semisolution.}
Set $n=\dim N$, $m=\dim M$.

Choose a smooth increasing strictly concave function $\phi$.
Consider the function $f=\phi\circ\dist_N$,
where $\dist_N(x)$ denotes the distance from $x\in M$ to $N$.

Note that if $f$ is smooth at some point $x\in M$ 
then the Hessian of $f$ at $x$ (briefly $\Hess_xf$)
has at least $n+1$ negative eigenvalues.

Moreover, at any point $x\notin \iota(N)$ the same holds in the barrier sense\label{page:barrier sense}.
That is, there is a smooth function $h$ defined on $M$
such that $h(x)=f(x)$, $h(y)\ge f(y)$ for any $y$
and $\Hess_xh$ has at least $n+1$ negative eigenvalues.

Use that $m< 2\cdot n$ and the described property to prove the following
analog of Morse lemma for $f$.

\begin{cl}{$({*})$}
 Given $x\notin \iota(N)$, there is a neighborhood $U\ni x$ such that the set 
\[U_-=\set{z\in U}{f(z)<f(x)}\] is simply connected.
\end{cl}

Since $M$ is simply connected,
any closed curve in $\iota(N)$
can be contracted by a disc, say $s_0\:\mathbb D\to M$.

Applying the claim $({*})$, one can construct an $f$-decreasing homotopy $s_t$ that starts at $s_0$ and ends in $\iota(N)$;
that is, there is 
a homotopy $s_t\:\mathbb D\z\to M$, $t\in [0,1]$ 
such that $s_t(\partial \mathbb D)\subset \iota(N)$ for any $t$ 
and $s_1(\mathbb D)\subset \iota(N)$.
It follows that $\iota(N)$ is simply connected.

Finally assume that $a$ and $b$ are distinct points in $N$ such that $\iota(a)\z=\iota(b)$.
If $\gamma$ is a path from $a$ to $b$ in $N$ then the loop $\iota\circ\gamma$ is not contractible in $\iota(N)$.
Therefore if $\iota\:N\to M$ has a self-intersection,
then the image
$\iota(N)$ is not simply connected.
Hence the result follows.\qeds


The statement was proved by 
Fuquan Fang, 
S\'ergio Mendon\c{c}a 
and Xiaochun Rong \cite{FMR}.
The main idea was discovered by 
Burkhard Wilking \cite{wilking-2003}.

%%%%%%%%%%%%%%%%%%%%%%%%%%%%%%%%%%%%%%%%%
\subsection*{Geodesic hypersurface}
\label{Geodesic hypersurface}

The totally geodesic embedding is defined before the previous problem.

\begin{pr}
Assume a compact connected positively curved manifold $M$ has a totally geodesic embedded hypersurface.
Show that either $M$ or its double covering is homeomorphic to the sphere.
\end{pr}

%%%%%%%%%%%%%%%%%%%%%%%%%%%%%%%%%%%%%%%%%
\subsection*{If convex, then embedded}
\label{If convex then embedded} 

\begin{pr}
Let $M$ be a complete simply connected Riemannian manifold 
with non-positive curvature 
and dimension at least $3$.
Prove that any immersed locally convex
compact hypersurface $\Sigma$ in $M$ is embedded.
\end{pr}

Let us summarize some statements about complete simply connected Riemannian manifolds 
with non-positive curvature.

By the Cartan--Hadamard theorem, for any point $p\in M$
the exponential map $\exp_p\:\T_p\to M$ is a diffeomorphism.
In particular, $M$ is diffeomorphic to the Euclidean space of the same dimension.
Moreover, any geodesic in $M$ is minimizing,
and any two points in $M$ are connected by a unique minimizing geodesic,

Further, $M$ is a $\CAT(0)$ space; that is, it satisfies a global angle comparison which we are about to describe.
Let $[xyz]$ be a triangle in $M$;
that is, $[xyz]$ is formed by three distinct points $x,y,z$ pairwise connected by geodesics.
Consider its model triangle $[\tilde x\tilde y\tilde z]$ in the Euclidean plane;
that is, $[\tilde x\tilde y\tilde z]$ has the same side lengths as $[xyz]$.
Then each angle in $[xyz]$ cannot exceed the corresponding angle in $[\tilde x\tilde y\tilde z]$.
This inequality can be written as
\[\tilde\measuredangle(y\,^x_z)\ge\measuredangle\hinge yxz,\]
where $\measuredangle\hinge yxz$ denotes the angle of the hinge $\hinge yxz$ formed by two geodesics $[yx]$ and $[yz]$ 
and $\tilde\measuredangle(y\,^x_z)$ denotes the corresponding angle in the model triangle $[\tilde x\tilde y\tilde z]$.

From this comparison it follows that any connected closed locally convex sets in $M$ is globally convex.
In particular, if $\Sigma$ is embedded then it bounds a convex set.


%%%%%%%%%%%%%%%%%%%%%%%%%%%%%%%%%%%%%%%%%
\subsection*{Immersed ball\hard}
\label{Immersed ball}

\begin{pr}
Prove that any immersed locally convex
hypersurface $\iota\:\Sigma\looparrowright M$
in a compact positively curved manifold $M$ of dimension $m\ge 3$ is the boundary of an immersed ball. 
That is, there is an immersion of a closed ball $f\:\bar B^m\looparrowright M$ and a diffeomorphism $h\:\Sigma\to\partial \bar B^m$
such that $\iota=f\circ h$.
\end{pr}

%%%%%%%%%%%%%%%%%%%%%%%%%%%%%%%%%%%%%%%%%
\subsection*{Minimal surface in the sphere}
\label{minimal surface}\label{almgren} 

A  smooth $n$-dimensional surface $\Sigma$ in
an $m$-dimensional Riemannian manifold $M$ is called \index{minimal surface}\emph{minimal}
if it locally minimizes the $n$-dimensional area;
that is, sufficiently small regions of $\Sigma$ do not admit area decreasing deformations with fixed boundary.

The minimal surfaces can be also defined via mean curvature vector as follows.
Let $\T=\T\,\Sigma$ and $\mathrm{N}=\mathrm{N}\,\Sigma$ denote the tangent and the normal bundle respectively.
Let $s$ denotes the second fundamental form of $\Sigma$;
it is a quadratic from on $\T$ with values in $\mathrm{N}$,
see the remark after problem ``Hypercurve'' below. 
Given an orthonormal basis $(e_i)$ in $\T_x$, set 
$$H_x=\sum_i s(e_i,e_i).$$
The vector $H_x$ lies in the normal space $\mathrm{N}_x$
and does not depend on the choice of orthonormal basis $(e_i)$.
This vector $H_x$ is called the mean curvature vector at $x\in \Sigma$. 

We say that $\Sigma$ is \index{minimal surface}\emph{minimal} if $H\equiv 0$.

\begin{pr}
Let $\Sigma$ be a closed $n$-dimensional 
minimal surface
in the unit $m$-dimensional sphere $\mathbb{S}^m$.
Prove that
$\vol_n \Sigma\ge \vol_n \mathbb{S}^n$.
\end{pr}

%%%%%%%%%%%%%%%%%%%%%%%%%%%%%%%%%%%%%%%%%
\subsection*{Hypercurve}
\label{codim=2}

The Riemann curvature tensor $R$
can be viewed as an operator $\text{\bf R}$ on the space of tangent bi-vectors $\bigwedge^2 \T$;
it is uniquely defined by the identity
$$\langle\mathbf{R}(X\wedge Y),V\wedge W\rangle
=
\langle R(X,Y)V,W\rangle.$$
The operator $\mathbf{R}\:\bigwedge^2 \T\to \bigwedge^2 \T$ is called the \index{curvature operator}\emph{curvature operator} and it is said to be {}\emph{positive definite} if
$\langle\mathbf{R}(\phi),\phi\rangle>0$ for all nonzero
bi-vector $\phi\in\bigwedge^2 \T$.


\begin{pr}
Let $M^m\hookrightarrow \RR^{m+2}$ be a closed smooth $m$-dimensional
submanifold and let  $g$ be the  induced Riemannian metric on $M^m$.
Assume that sectional curvature of $g$ is positive.
Prove that the curvature operator of $g$ is positive definite.
\end{pr}

The second fundamental form for manifolds of arbitrary codimension which we are about to describe might help to solve this problem.

Let $M$ be a smooth submanifold in $\RR^m$.
Given a point $p\in M$, denote by $\T_p$ and $\mathrm{N}_p=\T_p^\bot$
the tangent and normal space of $M$ at $p$.
The \index{second fundamental form}\emph{second fundamental form}\label{page:second fundamental form} of $M$ at $p$ is defined by 
\[s(X,Y)=(\nabla_X Y)^\bot,\]
where $(\nabla_X Y)^\bot$ a denotes the orthogonal projection of covariant derivative $\nabla_X Y$ onto the normal bundle.

The curvature tensor of $M$ can be found from the second fundamental form using the following  formula
\[\langle R(X,Y)V,W\rangle=\langle s(X,W),s(Y,V)\rangle-\langle s(X,V),s(Y,W)\rangle,\]
which is a direct generalization of the formula for Gauss curvature of a surface.


%%%%%%%%%%%%%%%%%%%%%%%%%%%%%%%%%%%%%%%%%
\subsection*{Horo-sphere}
\label{Horosphere}

We say that a Riemannian manifold has negatively pinched sectional curvature if its sectional curvatures at all points in all sectional directions lie in an interval $[-a^2, -b^2]$, for fixed constants $a>b>0$.

Let $M$ be a complete Riemannian manifold
and $\gamma$ a ray in $M$; 
that is, $\gamma\:[0, \infty)\to M$ is a minimizing unit-speed geodesic.

The \label{page:Busemann function}\index{Busemann function}\emph{Busemann function} $\bus_\gamma\:M\to\RR$ is defined by
$$\bus_\gamma(p)=\lim_{t\to\infty}\left(|p-\gamma(t)|_M-t\right).$$
From the triangle inequality, 
the expression under the limit is non-increasing in $t$; 
therefore  the limit above is defined for any $p$.

A \index{horo-sphere}\emph{horo-sphere} in $M$ is defined as a level set of a Busemann function
on $M$.

We say that a complete Riemannian manifold $M$ has \index{polynomial volume growth}\emph{polynomial volume growth} if for some (and therefore any) $p\in M$ we have 
$$\vol B(p,r)_M\z\le C\cdot (r^k+1),$$ 
where $B(p,r)_M$ denotes the ball in $M$ and  $C$, $k$ are constants.

\begin{pr} Let $M$ be a complete simply connected manifold with negatively pinched sectional curvature
and $\Sigma\subset M$ an horo-sphere in $M$.
Show that
$\Sigma$ with the induced intrinsic metric 
has polynomial volume growth.
\end{pr}

%%%%%%%%%%%%%%%%%%%%%%%%%%%%%%%%%%%%%%%%%
\subsection*{Number of conjugate points}
\label{Number of conjugate points}

Recall that points $p$ and $q$ on a geodesic $\gamma$ are called \index{conjugate points}\emph{conjugate} if there exists a non-zero Jacobi field along $\gamma$ that vanishes at $p$ and $q$. 

\begin{pr}
Let $s\:N\to M$ be a Riemannian submersion.
Suppose $N$ has nonpositive sectional curvature.
Show that any point $p$ in $M$ has at most $k=\dim N-\dim M$ conjugate points on any geodesic $\gamma\ni p$.
\end{pr}

%%%%%%%%%%%%%%%%%%%%%%%%%%%%%%%%%%%%%%%%%
\subsection*{Minimal spheres}
\label{Minimal spheres}

Recall that two subsets $A$ and $B$ in a metric space $X$ are called \index{equidistant sets}\emph{equidistant} if the distance function $\dist_A\:X\to\RR$ is constant on $B$ and $\dist_B$ is constant on $A$.

The minimal surfaces are defined on page \pageref{minimal surface}.

\begin{pr}
Show that a 
$4$-dimensional
compact 
positively curved 
Riemannian manifold 
cannot contain an infinite number of  mutually
 equidistant minimal 2-spheres.
\end{pr}


%%%%%%%%%%%%%%%%%%%%%%%%%%%%%%%%%%%%%%%%%
\subsection*{Positive curvature and symmetry\thm}
\label{kleiner-hopf} 

\begin{pr}
Assume that $\mathbb S^1$ acts isometrically on a closed $4$-dimensional Riemannian manifold with positive sectional curvature.
Show that the action 
has at most $3$ isolated fixed points.
\end{pr}

The following statement might be useful.
If $(M,g)$ is a Riemannian manifold with sectional curvature $\ge \kappa$ that admits a continuous isometric action of a compact group $G$, 
then the quotient space $A=(M,g)/G$ is an Alexandrov space with curvature $\ge \kappa$;
that is, the conclusion of the Toponogov comparison theorem holds in $A$. 

For more on Alexandrov geometry see \cite{akp}.

%%%%%%%%%%%%%%%%%%%%%%%%%%%%%%%%%%%%%%%%%
\subsection*{Energy minimizer}
\label{Energy minimizer}

Let $F$ be a smooth map from a closed Riemannian manifold $M$ to a Riemannian manifold $N$.
The energy functional of $F$ is defined by
\[E(F)=\int\limits_M |d_xF|^2\cdot d_x\vol_M.\]
We assume that  
\[|d_xF|^2=\sum_{i,j}a_{i,j}^2,\]
where $(a_{i,j})$ denote the components 
of the differential $d_xF$ 
written in the orthonormal bases of the tangent spaces $\T_xM$ and $\T_{F(x)}N$.

\begin{pr}
Show that the identity map on $\RP^m$ is 
energy
minimizing in its homotopy class.
Here we assume that $\RP^m$ is equipped with the canonical metric.
\end{pr}


%%%%%%%%%%%%%%%%%%%%%%%%%%%%%%%%%%%%%%%%%
\subsection*{Curvature against injectivity radius\thm}
\label{scalar-curv}

\begin{pr} 
Let $(M,g)$ be a closed 
Riemannian $m$-dimensional manifold.
Assume average of sectional curvatures over all sectional directions of $(M,g)$ is $1$. 
Show that the injectivity radius of $(M,g)$ is at most $\pi$.
\end{pr}

Solutions of this and the previous problem use the fact that geodesic flow on the tangent bundle to a Riemannian manifold preserves the volume form; this is a corollary of Liouville's theorem about phase volume.

%%%%%%%%%%%%%%%%%%%%%%%%%%%%%%%%%%%%%%%%%
\subsection*{Approximation of a quotient}

\begin{pr}
Let $(M,g)$ be a compact Riemannian manifold 
and $G$ a compact Lie group acting by isometries on $(M,g)$.
Construct a sequence of metrics $g_n$ on a fixed manifold $N$ such that $(N,g_n)$ converges to the quotient space $(M,g)/G$ in the sense of Gromov--Hausdorff.
\end{pr}


%%%%%%%%%%%%%%%%%%%%%%%%%%%%%%%%%%%%%%%%%
\subsection*{Polar points\many}
\label{milka-polar} 

\begin{pr}
Let $M$ be a compact Riemannian manifold with sectional curvature at least $1$ 
and dimension at least $2$. 
Prove that for any point $p\in M$ there is a point $p^*\in M$ such that 
\[|p-x|_M+|x-p^*|_M\le \pi\]
for any $x\in M$.
\end{pr}

%%%%%%%%%%%%%%%%%%%%%%%%%%%%%%%%%%%%%%%%%
\subsection*{Isometric section\hard}
\label{Isometric section}

\begin{pr}
Let $M$ and $W$ be compact Riemannian manifolds,
$\dim W>\dim M$,
and $s\:W\to M$ a Riemannian submersion.
Assume that $W$ has positive sectional curvature.
Show that $s$ does not admit an isometric section;
that is, there is no isometric embedding $\iota\:M\hookrightarrow W$ such that $s\circ\iota(p)=p$ for any $p\in M$.
\end{pr}

%%%%%%%%%%%%%%%%%%%%%%%%%%%%%%%%%%%%%%%%%
\subsection*{Warped product}
\label{Warped product}
\label{page:warped product}

Let $(M,g)$ and $(N,h)$ be Riemannian manifolds 
and $f$ a smooth positive function defined on $M$.
Consider the product manifold $W\z=M\times N$.
Given a tangent vector 
$X\z\in \T_{(p,q)} W
\z=\T_p M\times \T_p N$, denote by 
$X_M\z\in \T M$ and $X_N\z\in \T N$ its projections.
Let us equip $W$ with the Riemannian metric defined by
\[s(X,Y)=g(X_M,Y_M)+f^2\cdot h(X_N,Y_N).\]
The obtained Riemannian manifold $(W,s)$ is called \index{warped product}\emph{warped product} of $M$ and $N$ with respect to $f\:M\to \RR$;
it can be written as  
\[(W,g)\z=(N,h)\times_f(M,g).\]

\begin{pr}
Let $M$ be an oriented 3-dimensional Riemannian manifold with positive scalar curvature 
and $\Sigma\subset M$ an oriented smooth hypersurface that is area minimizing in its homology class.

Show that there is a positive smooth function $f\:\Sigma\to \RR$
such that the warped product $\mathbb S^1\times_f \Sigma$
has positive scalar curvature;
here $\Sigma$ is equipped with the Riemannian metric
induced from $M$.
\end{pr}

%%%%%%%%%%%%%%%%%%%%%%%%%%%%%%%%%%%%%%%%%
\subsection*{No approximation\many}
\label{No approximation}

\begin{pr}
Prove that 
if $p\not=2$,
then $\RR^m$ 
equipped with the metric induced by the $\ell^p$-norm 
cannot be a
Gromov--Hausdorff limit of
$m$-dimensional Riemannian manifolds $(M_n,g_n)$ with $\Ric_{g_n}\z\ge C$ for some constant $C$.
\end{pr}

%%%%%%%%%%%%%%%%%%%%%%%%%%%%%%%%%%%%%%%%%
\subsection*{Area of spheres}
\label{Area of spheres}

\begin{pr}
Let $M$ be a complete non-compact Riemannian manifold 
with non-negative Ricci curvature and $p\in M$.
Show that there is $\eps>0$ such that 
$$\area\left[\partial B(p,r)\right]>\eps$$
for all sufficiently large $r$.
\end{pr}

%%%%%%%%%%%%%%%%%%%%%%%%%%%%%%%%%%%%%%%%%
\subsection*{Flat coordinate planes}
\label{Flat coordinate planes}

\begin{pr}
Let $g$ be a complete Riemannian metric on $\RR^3$ such that the coordinate planes $x=0$, $y=0$ and $z=0$ are flat and totally geodesic.
Assume the sectional curvature of $g$ is either non-negative or non-positive.
Show that in both cases $g$ is flat. 
\end{pr}

%%%%%%%%%%%%%%%%%%%%%%%%%%%%%%%%%%%%%%%%%
\subsection*{Two-convexity\many}
\label{Two-convexity}

An open subset $V$ with smooth boundary in the Euclidean space  
is called \index{two-convex set}\emph{two-convex} if at most one principal curvatures in the outward direction to $V$ is negative.

The two-convexity of $V$ is equivalent to the following property.
For any plane $\Pi$ and any closed curve $\gamma$ in the intersection  $V\cap \Pi$,
if $\gamma$ is contactable in $V$ then it is contactable in $\Pi\cap V$.

\begin{pr}
Let $K$ be a closed set bounded by a smooth surface
in $\RR^4$.
Assume that $K$ contains two coordinate planes 
$$\{(x,y,0,0)\in\RR^4\}
\quad
\text{and}
\quad
\{(0,0,z,t)\in\RR^4\}$$
in its interior 
and also belongs to the closed $1$-neighborhood of these two planes.

Show that the complement to $K$ is not two-convex.
\end{pr}


%%%%%%%%%%%%%%%%%%%%%%%%%%%%%%%%%%%%%%%%%???
\subsection*{Convex lens\thm}
\label{Convex lens}

\begin{pr} Let $D$ and $D'$ be two smooth discs with a common boundary that bound a convex set (a lens) $L$ in a positively curved 3-dimensional Riemannian manifold $M$.
Assume that the discs meet at a small angle.
Show that the integral 
\[\int\limits_{D}k_1\cdot k_2\]
is small; here $k_1$ and $k_2$ denote the principal curvatures of $D$.
\end{pr}

The expected solution uses the following relative version of the Bochner formula.
Let $M$ be a Riemannian manifold with boundary $\partial M$.
If $f\:M\to \RR$ is a smooth function that vanish on $\partial M$,
then 
\[\int\limits_M |\Delta f|^2
-|\Hess f|^2
-\langle\mathrm{Ric}(\nabla f),\nabla f\rangle
=\int\limits_{\partial M}
H\cdot|\nabla f|^2.\]
Here we denote by $\Ric$ the Ricci curvature of $M$ 
and by $H$ the mean curvature of $\partial M$, we assume that $H\ge 0$ is the boundary is convex.
%???REF



%%%%%%%%%%%%%%%%%%%%%%%%%%%%%%%%%%%%%%%%%%%%%%%%%%
\section*{Semisolutions}



%%%%%%%%%%%%%%%%%%%%%%%%%%%%%%%%%%%%%%%%%%%%%%%%%%
\parbf{Geodesic hypersurface.}
Let $\Sigma$ be the totally geodesic embedded hypersurface in the positively curved manifold $M$.
Without loss of generality, we can assume that $\Sigma$ is connected.%
\footnote{In fact, by Frankel's theorem [see page \pageref{page:frankel}] $\Sigma$ is connected.}

The complement $M\backslash\Sigma$ has one or two connected components.
First let us show that if the number of connected components is two, 
then $M$ is homeomorphic to a sphere.

By cutting $M$ along $\Sigma$ 
we get two manifolds
with geodesic boundaries.
It is sufficient to show that each of them is homeomorphic to a Euclidean ball.

Choose one of these manifolds; denote it by $N$.
Denote by $f\:N\z\to\RR$ the distance functions to the boundary $\partial N$.
By the Riccati equation $\Hess f\le 0$ at any smooth point,
and for any point the same holds in the barrier sense [defined on page \pageref{page:barrier sense}].
It follows that $f$ is concave.

Choose an increasing strictly concave function $\phi\:\RR\to\RR$.
Note that $\phi\circ f$ is strictly concave in the interior of $N$.

Choose a compact subset $K$ in the interior of $N$ and
smooth $\phi\circ f$ in a neighborhood of $K$ keeping it concave. 
This can be done by applying the smoothing theorem of Robert Greene and Hung-Hsi Wu \cite[Theorem~2]{greene-wu}.

After the smoothing, the obtained strictly concave function, say $h$, has a single critical point which is its maximum.
In particular by Morse lemma, we get that if the set  
\[N'_s=\set{x\in N}{h(x)\ge s}\]
is not empty and lies in $K$ then it is diffeomorphic to a Euclidean ball.

For appropriately chosen set $K$ and the smoothing $h$, the set $N'_s$ can be made arbitrarily close to $N$;
moreover, its boundary $\partial N'_s$ can be made $C^\infty$-close to $\partial N$.
It follows that $N$ are diffeomorphic to a Euclidean ball.
This finishes the proof of the first case.

Now assume $M\backslash\Sigma$ is connected.
In this case there is a double covering $\tilde M$ of $M$ that induce a double covering $\tilde\Sigma$ of $\Sigma$,
so $\tilde M$ contains a geodesic hypersurface $\tilde\Sigma$ that divides $\tilde M$ into two connected components. 
From the case which has already been considered, $\tilde M$ is homeomorphic to a sphere;
hence the second case follows.
\qeds

The problem was suggested by Peter Petersen.



%%%%%%%%%%%%%%%%%%%%%%%%%%%%%%%%%%%%%%%%%%%%%%%%%%
\parbf{If convex, then embedded.}
Set 
\[m=\dim \Sigma=\dim M-1.\]

Given a point $p$ on $\Sigma$, denote by $p_r$ the point at distance $r$ from $p$
that lies on the geodesic starting at $p$ in the outer normal direction to $\Sigma$.
Note that for fixed $r\ge 0$,
the points $p_r$ sweep an immersed locally convex hypersurface which we denote by $\Sigma_r$.

\begin{wrapfigure}{r}{60 mm}
\vskip0mm
\centering
\includegraphics{mppics/pic-302}
\end{wrapfigure}

Choose $z\in M$. 
Denote by $d$ the maximal distance from $z$ to the points in $\Sigma$.
Note that 
any point on $\Sigma_r$
lies on a distance at least $r-d$ from $z$.

By comparison, 
\[\measuredangle\hinge{p_r}zp\le \arcsin\tfrac dr.\]
In particular, for large $r$, 
each infinite geodesic starting at $z$ intersects $\Sigma_r$ transversally.

The space of geodesics starting at $z$ can be identified with the sphere $\mathbb{S}^m$.
Therefore the map that send a point $x\in \Sigma_r$ to a geodesic from $z$ thru $x$ induces a local diffeomorphism $\phi_r\:\Sigma\z\to\mathbb{S}^m$.

Since $m\ge 2$, the sphere $\mathbb{S}^m$ is simply connected.
Since $\Sigma$ is connected, the map $\phi_r$ is a diffeomorphism.
Therefore $\Sigma_r$ is star-shaped with a center at $z$.
In particular $\Sigma_r$ is embedded.
Since $\Sigma_r$ is locally convex, it bounds a convex region and is embedded.

Consider the set $W$ of reals $r\ge 0$ such that $\Sigma_r$ is embedded and bounds a convex region.
Note that $W$ is open and closed in $[0,\infty)$.
Therefore $W=[0,\infty)$, and, in particular, $\Sigma_0=\Sigma$ is embedded.\qeds

The problem is due to Stephanie Alexander \cite{alexander}.



%%%%%%%%%%%%%%%%%%%%%%%%%%%%%%%%%%%%%%%%%%%%%%%%%%
\parbf{Immersed ball.}
Equip $\Sigma$ with the induced intrinsic metric.
Denote by $\kappa$ the lower bound for principal curvatures of $\Sigma$.
Note that we can assume that $\kappa>0$.

Choose a sufficiently small $\eps=\eps(M,\kappa)>0$.
Given $p\in \Sigma$, denote by $\Delta(p)$ the $\eps$-ball in $\Sigma$ centered at $p$.
Consider the lift $\tilde h_p\:\Delta(p)\to \T_{h(p)}$ along the exponential map $\exp_{h(p)}\:\T_{h(p)}\to M$.
More precisely:
\begin{enumerate}
\item Connect each point $q\in \Delta(p)\subset \Sigma$ to $p$
by a minimizing geodesic  path $\gamma_q\:[0,1]\to \Sigma$
\item Consider the lifting $\tilde\gamma_q$ in $\T_{h(p)}$; 
that is, $\tilde\gamma_q$ is the curve such that $\tilde\gamma_q(0)=0$ 
and $\exp_{h(p)}\circ\,\tilde\gamma_q(t)=\gamma_q(t)$ for each $t\in[0,1]$.
 \item Set $\tilde h(q)=\tilde\gamma_q(1)$.
\end{enumerate}

Show that all the hypersurfaces $\tilde h_p(\Delta(p))\subset \T_{h(p)}$ have principal curvatures at least $\tfrac\kappa2$.

Use the same idea as in the solution of ``Immersed surface'' [page ~\pageref{Immersed surface}] to show that 
one can fix $\eps\z=\eps(M,\kappa)>0$ such that the restriction of $\tilde h_p|_{\Delta(p)}$ is injective.
Conclude that the restriction $h|_{\Delta(p)}$ is injective for any $p\in\Sigma$.
(Here we use that $m\ge 3$.)

Now consider locally equidistant surfaces $\Sigma_t$ in the inward direction for small $t$. 
The principal curvatures of $\Sigma_t$ remain at least $\kappa$ in the barrier sense;
that is, at any point $p$, the surface $\Sigma_t$ can be supported by a smooth surface with principal curvatures at least $\kappa$ at $p$.
By the same argument as above, any $\eps$-ball in $\Sigma_t$ is embedded.

We get a one parameter family of locally convex locally equidistant surfaces $\Sigma_t$
for $t$ in the maximal interval $[0,a]$,
where the surface $\Sigma_a$ degenerates to a point, say $p$. 

To construct the immersion $\partial \bar B^m\looparrowright M$,
take the point $p$ as the image of the center $\bar B^m$ 
and take the surfaces $\Sigma_t$ as the restrictions of the  embedding to the spheres;
the existence of the immersion follows from the Morse lemma.\qeds

\begin{wrapfigure}{r}{30 mm}
\vskip0mm
\centering
\includegraphics{mppics/pic-304}
\end{wrapfigure}

As you see from the picture, 
the analogous statement does not hold in the two-dimensional case.

The proof presented above was indicated in the lectures of Mikhael Gromov \cite{gromov-SGMC} and written rigorously by Jost Eschenburg \cite{eschenburg}.

A variation of Gromov's proof 
was obtained independently by Ben Andrews \cite{andrews}.
Instead of equidistant deformation, 
he uses the so-called \index{inverse mean curvature flow}\emph{inverse mean curvature flow};
this way one has to perform some calculations to show that convexity survives in the flow, 
but one does not have to worry about non-smoothness of the hypersurfaces $\Sigma_t$. 




%%%%%%%%%%%%%%%%%%%%%%%%%%%%%%%%%%%%%%%%%%%%%%%%%%
\parbf{Minimal surface in the sphere.}
Choose a  geodesic $n$-dimensional sphere $\tilde\Sigma=\mathbb{S}^n\subset \mathbb{S}^m$.

Given $r\in (0,\tfrac\pi2]$,
denote by $U_r$ and $\tilde U_r$ the closed tubular $r$-neighborhood 
of $\Sigma$ and $\tilde\Sigma$ in $\mathbb{S}^m$ respectively.

Note that 
\[U_{\frac\pi2}=\tilde U_{\frac\pi2}=\mathbb{S}^m.
\leqno({*})\]
Indeed, clearly $\tilde U_{\frac\pi2}=\mathbb{S}^m$.
If $U_{\frac\pi2}\ne\mathbb{S}^m$, fix $x\in \mathbb{S}^m\backslash U_r$.
Choose a closest point $y\in \Sigma$ to $x$.
Since $r=|x-y|_{\mathbb{S}^m}>\tfrac\pi2$ the $r$-sphere $\mathrm{S}_r\subset \mathbb{S}^m$ with center $x$ is concave.
Note that $\mathrm{S}_r$ supports $\Sigma$ at $y$;
in particular the mean curvature vector of $\Sigma$ at $y$ cannot vanish, a contradiction.


By the Riccati equation, 
\[H_r(x)\ge \tilde H_r\] 
for any $x\in \partial U_r$,
where $H_r(x)$ denotes the mean curvature of $\partial U_r$  at a point $x$
and $\tilde H_r$ is the mean curvature of $\partial\tilde U_r$,
the latter is the same at all points.

Set 
\begin{align*}
a(r)&=\vol_{m-1} \partial U_r,
&
\tilde a(r)&=\vol_{m-1} \partial\tilde U_r,
\\
v(r)&=\vol_m U_r,
&
\tilde v(r)&=\vol_m \tilde U_r.
\intertext{By the coarea formula,}
\tfrac d{dr} v(r)&\aall a(r),
&
\tfrac d{dr}\tilde v(r)&=\tilde a(r).
\end{align*}
Note that
\begin{align*}\tfrac d{dr}a(r)&\le \int\limits_{\partial U_r} H_r(x)\cdot d_x\vol_{m-1}\le
\\
&\le a(r)\cdot \tilde H_r
\end{align*}
and
\begin{align*}
\tfrac d{dr}\tilde a(r)
&= \tilde a(r)\cdot \tilde H_r.
\intertext{It follows that}
\frac {v''(r)}{v(r)}&\le \frac {\tilde v''(r)}{\tilde v(r)}
\end{align*}
for almost all $r$. 
Therefore
\[v(r)\le\frac{\area\Sigma}{\area \tilde\Sigma}\cdot \tilde v(r)\]
for any $r>0$.

According to $({*})$,
\[v(\tfrac\pi2)=\tilde v(\tfrac\pi2)=\vol\mathbb{S}^m.\]
Hence the result follows.\qeds

This problem is the geometric lemma in the proof given by Frederick Almgren of his isoperimetric inequality \cite{almgren}.
The argument is similar to 
the proof of isoperimetric inequality for manifolds with positive Ricci curvature
given by Mikhael Gromov \cite{gromov-apendix}.

%%%%%%%%%%%%%%%%%%%%%%%%%%%%%%%%%%%%%%%%%%%%%%%%%%
\parbf{Hypercurve.}
Choose $p\in M$.
Denote by $s$ 
the second fundamental form of $M$ at $p$.
Recall that $s$ is a symmetric bilinear form on the tangent space $\T_pM$ of $M$ with values in the normal space $\mathrm{N}_pM$ to $M$, see page~\pageref{page:second fundamental form}.

By the Gauss formula
\[\langle R(X,Y)Y,X\rangle=\langle s(X,X),s(Y,Y)\rangle-\langle s(X,Y),s(X,Y)\rangle,\]
Since the sectional curvature of $M$ is positive, 
we get
\[\<s(X,X),s(Y,Y)\> > 0\leqno({*})\]
for any pair of nonzero vectors $X,Y\in\T_pM$.

The normal space $\mathrm{N}_pM$ is two-dimensional.
By $({*})$ there is an orthonormal basis $e_1,e_2$ in $\mathrm{N}_pM$ 
such that the real-valued quadratic forms 
\begin{align*}
s_1(X,X)&=\<s(X,X),e_1\>,
&
s_2(X,X)&=\<s(X,X),e_2\>
\end{align*}
are positive definite.

Note that the curvature operators $\mathbf{R}_1$ and $\mathbf{R}_2$ 
defined by the formula
\[\mathbf{R}_{i}(X\wedge Y), V\wedge W\rangle 
=s_i(X,W)\cdot s_i(Y,V)-s_i(X,V)\cdot s_i(Y,W)\]
are positive.
Finally, note that $\mathbf{R}=\mathbf{R}_{1}+\mathbf{R}_{2}$ is the curvature operator of $M$ at $p$.\qeds

The problem is due to Alan Weinstein \cite{weinstein}.
Note that from \cite{micallef-moore}/\cite{boehm-wilking} it follows that
that the universal covering of $M$ is homeomorphic/\hskip0mm diffeomorphic to a standard sphere.



%%%%%%%%%%%%%%%%%%%%%%%%%%%%%%%%%%%%%%%%%%%%%%%%%%
\parbf{Horo-sphere.}
Set 
$m=\dim \Sigma=\dim M-1$.

Let $\bus\:M\to\RR$ be the Busemann function such that 
\[\Sigma=\bus^{-1}\{0\}.\]
Set  $\Sigma_r=\bus^{-1}\{r\}$, so $\Sigma_0=\Sigma$.

Let us equip each $\Sigma_r$ with the induced Riemannian metric.
Note that all $\Sigma_r$ have bounded curvature.
In particular, the unit balls in $\Sigma_r$ have volume bounded above by a universal constant, say $v_0$.
 
Given $x\in \Sigma$, denote by $\gamma_x$ 
the unit-speed geodesic
such that $\gamma_x(0)=x$ and $\bus(\gamma_x(t))=t$ for any $t$.
Consider the map $\phi_{r}\:\Sigma\to\Sigma_r$ defined by
$\phi_r\:x\mapsto \gamma_x(r)$.
In other words, $\phi_{r}$ is the closest point projection from $\Sigma$ to $\Sigma_r$.

Notice that $\phi_r$ is a bi-Lipschitz map with the Lipschitz constants $e^{a\cdot r}$ and $e^{b\cdot r}$.
In particular, the ball of radius $R$ in $\Sigma$ is mapped by $\phi_r$
to a ball of radius $e^{a\cdot r}\cdot R$ in $\Sigma_r$.
Therefore
\[\vol_m B(x,R)_\Sigma
\le 
e^{m\cdot b\cdot r}\cdot \vol_m B(\phi_r(x),e^{a\cdot r}\cdot R)_{\Sigma_r}\]
for all $R,r>0$.
Taking $R=e^{-a\cdot r}$, we get
\[\vol_m B(x,R)_\Sigma\le v_0\cdot R^{m\cdot \frac ba}\]
for any $R\ge1$. 
Hence the statement follows.
\qeds

The problem was suggested by Vitali Kapovitch.

There are examples of horo-spheres as above with a degree of polynomial growth higher than $m$.
For example, consider the horo-sphere $\Sigma$ in the
the complex hyperbolic space 
of real dimension $4$.
Clearly $m=\dim \Sigma=3$, but the degree of its volume growth is $4$.

In this case $\Sigma$ is isometric to the Heisenberg group.%
\footnote{\index{Heisenberg group}\emph{Heisenberg group}
is the group of $3\times3$ upper triangular matrices of the form
\[\begin{pmatrix}
 1 & a & c\\
 0 & 1 & b\\
 0 & 0 & 1\\
\end{pmatrix}\]
under the operation of matrix multiplication.} 
It is instructive to show that any such metric has volume  growth of degree $4$.

%%%%%%%%%%%%%%%%%%%%%%%%%%%%%%%%%%%%%%%%%%%%%%%%%%
\parbf{Number of conjugate points.}
Choose a geodesic $\gamma$ in $M$ and a point $p\in \gamma$.
Note that $\gamma$ can be lifted to a horizontal geodesic $\bar\gamma$ in $N$.
That is, $\gamma=s\circ\bar\gamma$ and $\bar\gamma$ is perpendicular to the fibers of~$s$ (in particular, $\gamma$ and $\bar\gamma$ have equal speeds).

Observe that each conjugate point of $p$ on $\gamma$ corresponds to a \index{focal point}\emph{focal points} on $\bar\gamma$ to the fiber $F$ over $p$ in $N$;
that is, $\bar\gamma$ lies in a family of geodesics $\bar\gamma_t$ that are perpendicular to $N$ 
such that the corresponding Jacobi field along $\bar\gamma$ vanish at $q$.

Note that $F$ has dimension $k=\dim N-\dim M$.
It remains to prove that any smooth $k$-dimensional submanifold $F$ in a complete nonpositively curved manifold $N$ has at most $k$ focal points on any geodesic $\bar \gamma$ that is perpendicular to $F$.\qeds

The problem inspired by the paper of Alexander Lytchak~\cite{lytchak:conjugate}.
Applying it together with the Poincar\'e recurrence theorem
leads to a solution of the following problem.

\begin{pr}
Let $s\:N\to M$ be a Riemannian submersion.
Suppose $N$ has nonpositive sectional curvature and $M$ is compact.
Show that $M$ has no conjugate points.
\end{pr}

In fact no compact negatively curved manifold $N$ admits a nontrivial Riemannian submersion $s\:N\to M$~\cite[see Theorem F in][]{zeghib}. 





%%%%%%%%%%%%%%%%%%%%%%%%%%%%%%%%%%%%%%%%%%%%%%%%%%
\parbf{Minimal spheres.}
Assuming the contrary,
we can choose a pair of sufficiently close minimal spheres $\Sigma$ and $\Sigma'$ in the 4-dimensional manifold $M$;
we can assume that the distance $a$ between $\Sigma$ and $\Sigma'$ is strictly smaller than the injectivity radius of the manifold.
Note that in this case there is a unique bijection $\Sigma\to \Sigma'$, denoted by $p\mapsto p'$ such that the distance $|p-p'|_M=a$ for any $p\in\Sigma$.

Let $\iota_p\:\T_p\to\T_{p'}$ be the parallel translation along the (necessary unique) minimizing geodesic $[pp']$.
Note that there is a pair $(p,p')$ such that $\iota_p(\T_p\Sigma)=\T_{p'}\Sigma'$.
Indeed, if this is not the case, then $\iota_p(\T_p\Sigma)\z\cap\T_{p'}\Sigma'$ forms a continuous line distribution over $\Sigma'$.
Since $\Sigma'$ is a two-sphere, the latter contradicts the hairy ball theorem.

Consider pairs of unit-speed geodesics $\alpha$ and $\alpha'$ 
in $\Sigma$ and $\Sigma'$  
that start at $p$ and $p'$ respectively
and go in the parallel directions, say $\nu$ and $\nu'$. %???direction???
Set $\ell_\nu(t)=|\alpha(t)-\alpha'(t)|$.

Use the second variation formula together with the lower bound on Ricci curvature
to show that $\ell_\nu''(0)$ has a negative average for all tangent directions $\nu$ to $\Sigma$ at $p$. 
In particular $\ell_\nu''(0)<0$ for some vector $\nu$ as above.
For the corresponding pair $\alpha$ and $\alpha'$,
it follows that there are points $v=\alpha(\eps)\in\Sigma$ near $p$ 
and $v'=\alpha'(\eps)\in\Sigma'$ near $p'$
such that 
\[|v-v'|<|p-p'|,\]
a contradiction.\qeds

Likely, any compact positively curved 
4-dimensional manifold
cannot contain a pair of equidistant spheres.
The argument above implies that the distance between such a pair has to exceed the injectivity radius of the manifold.


The problem was suggested by Dmitri Burago.
Here is a short list of classical problems with use the second variation formula in similar 	fashion:

\begin{pr}
Any compact even-dimensional orientable manifold with positive sectional curvature is
simply connected.
\end{pr}

This is called Synge's lemma \cite{synge}.

\begin{pr}
Any two compact minimal hypersurfaces in a Riemannian manifold with positive Ricci curvature must intersect.
\end{pr}

\begin{pr}
Let $\Sigma_1$ and $\Sigma_2$ be two compact geodesic submanifolds in a manifold with positive sectional curvature $M$ and 
\[\dim \Sigma_1+\dim \Sigma_2\ge \dim M.\] 
Then $\Sigma_1\cap\Sigma_2\ne\emptyset$.
\end{pr}

These two statements have been proved by Theodore Frankel \cite{frankel}.\label{page:frankel}

\begin{pr}
Let $(M,g)$ be a closed Riemannian manifold with negative Ricci curvature.
Prove that $(M,g)$ does not admit an isometric $\mathbb{S}^1$-action.
\end{pr}

This is a theorem of Salomon Bochner \cite{bochner}.

The problem ``Geodesic immersion'' [page \pageref{Geodesic immersion}] can be considered as further development of the idea.






%%%%%%%%%%%%%%%%%%%%%%%%%%%%%%%%%%%%%%%%%%%%%%%%%%
\parbf{Positive curvature and symmetry.}
Let $M$ be a 4-dimensional Riemannian manifold with isometric $\mathbb{S}^1$-action.
Consider the quotient space $X=M/\mathbb{S}^1$.
Note that $X$ is a positively curved 3-dimensional Alexandrov space.
In particular the angle $\measuredangle\hinge xyz$ between any two geodesics $[xy]$ and $[xz]$ is defined.
Further, for any non-degenerate triangle $[xyz]$ 
formed by the minimizing geodesics $[xy]$, $[yz]$ and $[zx]$  in $X$ we have
\[\measuredangle\hinge xyz+\measuredangle\hinge yzx+\measuredangle\hinge zxy> \pi.
\leqno({*})\]

Assume that $p\in X$ corresponds to a fixed point $\bar p\in M$ of the $\mathbb{S}^1$-action.
Each direction of a geodesic starting at $p$ in $X$ corresponds to $\mathbb{S}^1$-orbit of the induced isometric action $\mathbb{S}^1\z\acts\mathbb{S}^3$ on the sphere of unit vectors at $\bar p$.
Any such action is conjugate to the action $\mathbb{S}^1_{p,q}\z\acts\mathbb{S}^3\subset\CC^2$ induced by complex matrices 
$
\left(
\begin{smallmatrix}
z^p&0
\\
0&z^q
\end{smallmatrix}
\right)
$
with $|z|=1$ and some relatively prime positive integers $p,q$.
The possible quotient spaces $\Sigma_{p,q}=\mathbb{S}^3/\mathbb{S}^1_{p,q}$ 
have diameter $\tfrac\pi2$ and perimeter of any triangle in $\Sigma_{p,q}$ is at most $\pi$;
this is straightforward to check, but requires some work.

Therefore for any three geodesics $[px]$, $[py]$ and $[pz]$ in $X$ we have
\[\measuredangle\hinge pxy+\measuredangle\hinge pyz+\measuredangle\hinge pzx\le \pi.\leqno({*}{*})\]
and
\[\measuredangle\hinge pxy,\ \measuredangle\hinge pyz,\ \measuredangle\hinge pzx\le \tfrac\pi2.\leqno(\asterism)\]

Arguing by contradiction,
assume that there are 4 fixed points $q_1$, $q_2$, $q_3$ and $q_4$.
Connect each pair by a minimizing geodesic $[q_iq_j]$.

Denote by $\omega$ the sum of all 12 angles of the type  $\measuredangle\hinge{q_i}{q_j}{q_k}$.
By $(\asterism)$, each triangle $[q_iq_jq_k]$ is non-degenerate.
Therefore by $({*})$, we have
\[\omega>4\cdot\pi.\]
On the other hand, applying $({*}{*})$ at each vertex $q_i$, we have 
\[\omega\le 4\cdot\pi,\]
a contradiction.\qeds


The problem is due to 
Wu-Yi Hsiang 
and Bruce Kleiner 
\cite{hsiang-kleiner}.
The connection of this proof to Alexandrov geometry was noticed by Karsten Grove \cite{grove}.
An interesting new twist of the idea 
is given by 
Karsten Grove 
and Burkhard Wilking 
\cite{grove-wilking}.

%%%%%%%%%%%%%%%%%%%%%%%%%%%%%%%%%%%%%%%%%%%%%%%%%%
\parbf{Energy minimizer.}
Denote by $\mathcal{U}$ the unit tangent bundle over $\RP^m$
and by $\mathcal{L}$ the space of projective lines in $\ell\:\RP^1\to \RP^m$.
The spaces $\mathcal{U}$ and $\mathcal{L}$ 
have dimension $2\cdot m-1$ 
and $2\cdot(m-1)$
respectively.


According to Liouville's theorem about phase volume, the identity
\[\int\limits_{\mathcal{U}}f(v)\cdot d_v\vol_{2\cdot m-1}
=
\int\limits_{\mathcal{L}}d_\ell\vol_{2\cdot(m-1)}\cdot\int\limits_{\RP^1} f(\ell'(t))\cdot dt\]
holds for any integrable function $f\:\mathcal{U}\to\RR$.

Let $F\:\RP^m\to\RP^m$ be a smooth map.
Note that up to a multiplicative constant,
the energy of $F$ can be expressed the following way
\[\int\limits_{\mathcal{U}} |dF(v)|^2\cdot d_v\vol_{2m-1}
=
\int\limits_{\mathcal{L}}d_\ell\vol_{2\cdot(m-1)}\cdot\int\limits_{\RP^1} |[d(F\circ \ell)](t)|^2\cdot dt.\]

Notice that any noncontractible curve in $\RP^m$ has length at least $\pi$.
Therefore, by Bunyakovsky inequality, we get
\begin{align*}
\int\limits_{\RP^1} \left|[d(F\circ \ell)](t)\right|^2\cdot dt&\ge 
\tfrac1\pi\cdot \left(\,\int\limits_{\RP^1} \left|[d(F\circ \ell)](t)\right|\cdot dt\right)^2=
\\
&=\tfrac1\pi\cdot (\length F\circ\ell)^2\ge
\\
&\ge \pi.
\end{align*}
for any line $\ell\:\RP^1\to \RP^m$.
Hence the result follows.\qeds

\label{page:liouville}
The problem is due to Christopher Croke \cite{croke-energy}. 
He uses the same idea to show that the identity map on $\CP^m$ is energy minimizing in its homotopy class.
For $\mathbb S^m$, an analogous statement does not hold if $m\ge 3$.
In fact, 
if a closed Riemannian manifold $M$ 
has dimension at least 3 
and $\pi_1M=\pi_2M=0$,
then the identity map on $M$ is homotopic 
to a map with arbitrary small energy;
the latter was shown by Brian White \cite{white}.

The same idea is used to prove the so-called Loewner's inequality \cite{gromov-filling}.
\begin{pr}
Let $g$ be a Riemannian metric on $\RP^m$ that is conformally equivalent to the canonical metric $g_0$.
Assume that any noncontractible curve in $(\RP^m,g)$ has length at least $\pi$.
Then
\[\vol(\RP^m,g)\ge\vol(\RP^m,g_0).\]

\end{pr}

A more advanced application is the sharp isoperimetric inequality for 
4-dimensional Hadamard manifolds proved by Christopher Croke [see \ncite{croke-4d} and also \ncite{croke-eigenvalue}].





%%%%%%%%%%%%%%%%%%%%%%%%%%%%%%%%%%%%%%%%%%%%%%%%%%
\parbf{Curvature against  injectivity radius.}
We will show that 
if the injectivity radius of the manifold $(M,g)$ is at least $\pi$,
then the average of sectional curvatures on $(M,g)$ is at most $1$.
This is equivalent to the problem.

Choose a point $p\in M$ and two orthonormal vectors $U,V\in\T_p M$.
Consider the geodesic $\gamma$ in $M$ such that $\gamma'(0)=U$.

Set $U_t=\gamma'(t)\in \T_{\gamma(t)}$ 
and let $V_t\in \T_{\gamma(t)}$ be the parallel translation of $V=V_0$ along $\gamma$.


Consider the field $W_t=\sin t\cdot V_t$ on $\gamma$.
Set 
\begin{align*}
\gamma_\tau(t)&=\exp_{\gamma(t)} (\tau\cdot W_t),
\\
\ell(\tau)&=\length(\gamma_\tau|_{[0,\pi]}),
\\
q(U,V)&=\ell''(0).
\end{align*}
Note that
$$q(U,V)
=
\int\limits_{0}^\pi [(\cos t)^2-K(U_t,V_t)\cdot (\sin t)^2]\cdot dt,
\leqno({*})$$
where $K(U,V)$ is the sectional curvature 
in the direction spanned by $U$ and $V$. 

Since any geodesics of length $\pi$ is minimizing,
we get $q(U,V)\ge0$ for any pair of orthonormal vectors $U$ and $V$.
It follows that the average value of the right hand side in $({*})$ is non-negative.

By Liouville's theorem about phase volume, while taking the average of $({*})$, we can switch the order of integrals;
therefore  
\[0\le \tfrac\pi2\cdot(1-\bar{K}),\]
where $\bar{K}$ denotes the average of sectional curvatures on $(M,g)$.
Hence the result follows.\qeds

The problem illustrates the idea of Eberhard Hopf \cite{hopf-conjugate}
which was developed further by Leon Green \cite{green}.
Hopf used it to show that a metric on 2-dimensional torus without conjugate points must be flat
and Green showed that the average of sectional curvature on a closed manifold without conjugate points cannot be positive.









%%%%%%%%%%%%%%%%%%%%%%%%%%%%%%%%%%%%%%%%%%%%%%%%%%
\parbf{Approximation of a quotient.} The proof will use that for any Riemannian submersion $s\:M\to N$
the lower bound on sectional curvature of $M$ can non exceed the lower bound on sectional curvature of~$N$.

This statement follow from the O'Nail's formula \cite[Theorem 3.20]{cheeger-ebin} 
which gives the following relation between sectional curvatures of $M$ ad $N$
\[K_M(X,Y)=K_N(\bar X, \bar Y)+\tfrac34|[\bar X,\bar Y]^V|^2,\]
where $X,Y$ are orthonormal vector fields on $N$, $\bar X, \bar Y$ their horizontal lifts to $M$, $[{*},{*}]$ is the Lie bracket and ${*}^V$ is the projection to the vertical distribution of the submersion.
Indeed, since $\tfrac34|[\bar X,\bar Y]^V|^2\ge 0$, we have $K_M(X,Y)\ge K_N(\bar X, \bar Y)$.

\medskip

Note that $G$ admits an embedding into a compact connected Lie group $H$;
in fact we can assume that $H=\SO(n)$, for sufficiently large~$n$.

Suppose that the curvature of $(M,g)$ is bounded below by~$\kappa$.

The bi-invariant metric $h$ on $H$ is non-negatively curved.
Therefore for any positive integer $n$ the product $(H,\tfrac1n\cdot h)\times (M,g)$ is a Riemannian manifold with  curvature bounded below by~$\kappa$.

The diagonal action of $G$ on $(H,\tfrac1n\cdot h)\times (M,g)$ is isometric and free. 
Therefore 
the quotient $(H,\tfrac1n\cdot h)\times (M,g)/G$
is a Riemannian manifold, say $(N,g_n)$.
Note that the quotient map $(H,\tfrac1n\cdot h)\times (M,g)\to (N,g_n)$ is a Riemannian submersion.
Therefore $(N,g_n)$ has sectional curvature bounded below by $\kappa$.

It remains to observe that the spaces $(N,g_n)$ converge to $(M,g)/G$ as $n\z\to \infty$.\qeds

The used construction is called \index{Cheeger's trick}\emph{Cheeger's trick}.
The earliest use of this trick I found in \cite{GKM}; 
it was used there to show that Berger's spheres have positive curvature.
This trick is used in the construction of most of the known examples of positively and non-negatively curved manifolds
 \cite{cheeger,aloff-wallach,gromoll-meyer,eschenburg-spaces,bazajkin}.
 
The quotient space  $(M,g)/G$ has finite dimension and curvature bounded below in the sense of Alexandrov. 
It is expected that not all finite dimensional Alexandrov spaces admit approximation by Riemannian manifolds with curvature bounded below
[some partial results are discussed in \ncite{pwz,kapovitch}].








%%%%%%%%%%%%%%%%%%%%%%%%%%%%%%%%%%%%%%%%%%%%%%%%%%
\parbf{Polar points.}
Choose a unit-speed geodesic $\gamma$ that starts at $p$;
that is, $\gamma(0)=p$.
Apply the Toponogov comparison to show that $p^*=\gamma(\pi)$ is a solution. 
\qeds

\parit{Alternative proof.} 
Assume the contrary;
that is, for any $x\in M$ there is a point $x'$ such that 
\[|x-x'|_M+|p-x'|_M>\pi.\]

Given $x\in M$, denote by $f(x)$ a point that maximizes the following sum:
\[|x-f(x)|_M+|p-f(x)|_M.\]
Show that $f$ is uniquely defined and continuous.

Choose sufficiently small $\eps>0$.
Prove that the set $W_\eps=M\backslash B(p,\eps)$ 
is homeomorphic to a ball 
and the map $f$ sends $W_\eps$ into itself.

By Brouwer's fixed-point theorem, $x=f(x)$ for some $x$.
In this case 
\begin{align*}
|x-f(x)|_M+|p-f(x)|_M&=|p-x|_M\le
\\
&\le\pi,
\end{align*}
a contradiction.\qeds
 
The problem is due to Anatoliy Milka \cite{milka-poly}.





%%%%%%%%%%%%%%%%%%%%%%%%%%%%%%%%%%%%%%%%%%%%%%%%%%
\parbf{Isometric section.}
Arguing by contradiction, 
assume there is an isometric section $\iota\: M\z\to W$.
It makes possible to treat $M$ as a submanifold in $W$.

Given $p\in M$, denote by $\mathrm{N}^1_p$ the unit normal space to $M$ at $p$.
Given $v\in \mathrm{N}^1_p$ and a real number $k$,
set 
\[p^{k\cdot v}=s\circ\exp_{p} (k\cdot v).\]
Note that 
\[p^{0\cdot v}=p\ \ \text{for any}\ \  p\in M\ \ \text{and}\ \ v\in \mathrm{N}^1_p.\leqno({*})\]

Choose sufficiently small $\delta>0$.
By Rauch comparison \cite[Corollary 1.36]{cheeger-ebin}, 
if $w\in \mathrm{N}^1_q$ 
is the parallel translation of $v\in \mathrm{N}^1_q$ 
along a minimizing geodesic from $p$ to $q$ in $M$,
then 
\[|p^{k\cdot v}-q^{k\cdot w}|_M<|p-q|_M
\leqno({*}{*})\]
assuming that $|k|\le \delta$.
The same comparison implies that 
\[|p^{k\cdot v}-q^{k'\cdot w}|_M^2<|p-q|_M^2+ (k-k')^2
\leqno(\asterism)\]
assuming that $|k|,|k'|\le \delta$.

Choose $p$ and $v \in \mathrm{N}^1_p$ so that $r=|p-p^{\delta\cdot v}|$ 
takes the maximal possible value.
From $({*}{*})$ it follows that $r>0$.

Let $\gamma$ be the extension of the unit-speed minimizing geodesic from $p^{\delta\cdot v}$ to $p$;
denote by $v_t$ the parallel translation of $v$ to $\gamma(t)$ along $\gamma$. 

We can choose the parameter of $\gamma$ so that $p=\gamma(0)$, $p^{\delta\cdot v}=\gamma(-r)$.
Set $p_n=\gamma(n\cdot r)$, so $p=p_0$ and $p^{\delta\cdot v}=p_{-1}$. 
Choose a large integer $N$ and set $w_n=\delta\cdot(1-\tfrac nN)\cdot v_{n\cdot r}$, $q_n=p_n^{w_n}$, $x_n=\exp_{p_n} (w_n)$, and $q_n=p_n^{w_n}=s(x_n)$.

\begin{figure}[h!]
\vskip0mm
\centering
\includegraphics{mppics/pic-306}
\end{figure}

By $(\asterism)$, there is a constant $C$ independent of $N$ such that
\[|q_k-q_{k+1}|<r+\tfrac C{N^2}\cdot\delta^2.\]
Therefore 
\[|q_{k+1}-p_{k+1}|>|q_k-p_k|-\tfrac C{N^2}\cdot\delta^2.\]
By induction, we get 
\[|q_N-p_N|>r-\tfrac C{N}\cdot\delta^2.\]
Since $N$ is large we get
\[|q_N-p_N|>0.\]
Note that $w_N=0$;
therefore by $({*})$, we get $q_N=p_N^0=p_N$ --- a contradiction.\qeds

This is the core of Perelman's solution of Soul conjecture \cite{perelman}.

%%%%%%%%%%%%%%%%%%%%%%%%%%%%%%%%%%%%%%%%%%%%%%%%%%
\parbf{Warped product.}
Given $x\in \Sigma$, denote by $\nu_x$ the normal vector to $\Sigma$ at $x$ that agrees with the orientations of $\Sigma$ and $M$. %???remove,
Denote by $\kappa_x$ the non-negative principal curvature of $\Sigma$ at $x$;
since $\Sigma$ is minimal the other principal curvature has to be $-\kappa_x$.

Consider the warped product $W=\mathbb S^1\times_f\Sigma$ for some positive smooth function $f\:\Sigma\to \RR$.
Assume that a point $y\in W$ projects to a point $x\in\Sigma$.
Straightforward computations show that
\begin{align*}
\Sc_W(y)
&=\Sc_\Sigma(x)-2\cdot\frac{\Delta f(x)}{f(x)}=
\\
&=\Sc_M(x)-2\cdot\Ric(\nu_x)-2\cdot\kappa_x^2-2\cdot\frac{\Delta f(x)}{f(x)},
\end{align*}
where $\Sc$ and $\Ric$ denote the scalar and Ricci curvature respectively. 

Consider linear operator $L$ on the space of smooth functions on $\Sigma$ defined by 
\[(Lf)(x)= -[\Ric(\nu_x)+\kappa_x^2]\cdot f(x)-(\Delta f)(x)\]
It is sufficient to find a smooth function $f$ on $\Sigma$ such that
\[f(x)>0 \ \ \text{and}\ \ (Lf)(x)\ge 0\leqno({*})\]
for any $x\in \Sigma$.


Given a smooth function $f\:\Sigma\to \RR$,
extend the field $f(x)\cdot\nu_x$
on $\Sigma$ to a smooth field, say $v$, on whole $M$.
Denote by $\iota_t$ the flow along $v$ for time $t$ and set $\Sigma_t=\iota_t(\Sigma)$.

Denote by $H_t(x)$ the mean curvature of $\Sigma_t$ at $\iota_t(x)$.
Note that the value $(Lf)(x)$ is the derivative of
the function $t\mapsto H_t(x)$  at $t=0$.

Therefore the condition $({*})$
means that we can push $\Sigma$ into one of its sides 
so that its mean curvature does not increase in the first order.
Since $\Sigma$ is area minimizing,
such push can be obtained by increasing the pressure on one side of $\Sigma$.
(Read further if you are not convinced.)
\qeds

\parit{Formal end of proof.}
Denote by $\delta(f)$ the second variation of area of $\Sigma_t$;
that is, consider the area function $a(t)=\area\Sigma_t$ 
and set $\delta(f)=a''(0)$.
Direct calculations show that
\begin{align*}
\delta(f)
&=
\int\limits_{\Sigma} 
\left(-[\Ric(\nu_x)+\kappa_x^2]\cdot f^2(x)+|\nabla f(x)|^2\right)\cdot d_x\area=
\\
&=\int\limits_{\Sigma} 
(Lf)(x)\cdot f(x)\cdot d_x\area.\end{align*}
Since $\Sigma$ is area minimizing we get 
\[\delta(f)\ge 0\leqno({*}{*})\] for any $f$.

Choose a function $f$ that minimize $\delta(f)$ for all functions such that $\int_\Sigma f^2(x)\cdot d_x\area=1$.
Note that $f$ is an eigenfunction 
for the linear operator $L$;
in particular $f$ is smooth.
Denote by $\lambda$ the eigenvalue of $f$;
by $({*}{*})$,
$\lambda\ge 0$.

Show that $f(x)>0$ at any $x$.
Since $Lf=\lambda\cdot f$, the inequalities $({*})$ follow.\qeds


The problem is due to Mikhael Gromov and Blaine Lawson \cite{gromov-lawson}.
Earlier, in \cite{schoen-yau}, Shing-Tung  Yau and Richard Schoen showed that the same assumptions 
imply existence of conformal factor on $\Sigma$ that makes it positively curved.
Both statement are used the same way
to proof that $\TT^3$ does not admit a metric with positive scalar curvature.

Both statements admit straightforward generalization to higher dimensions
and they can be used to show the non-existence of a metric with positive scalar curvature on $\TT^m$ with $m\le 7$.
For $m=8$, the proof stops working 
since in this dimension the area minimizing hypersurfaces might have singularities.
For example, 
any domain in the cone in $\RR^8$
defined by the identity
\[x^2_1+x^2_2+x^2_3+x^2_4=x^2_5+x^2_6+x^2_7+x^2_8\]
is area minimizing among the hypersurfaces with the same boundary.





%%%%%%%%%%%%%%%%%%%%%%%%%%%%%%%%%%%%%%%%%%%%%%%%%%
\parbf{No approximation.}
Choose an increasing function $\phi\:(0,r)\to \RR$
such that 
\[\phi''+(n-1)\cdot(\phi')^2+C=0.\]

If $\Ric_{g_n}\ge C$, 
then the function 
$x\mapsto\phi(|q-x|_{g_n})$ is subharmonic.
Therefore for an arbitrary array of points $q_i$ 
and positive reals $\lambda_i$ the function $f_n\:M_n\to \RR$
defined by the formula
$$f(x)=\sum_i\lambda_i\cdot\phi(|q_i-x|_M)$$
is subharmonic.
In particular $f_n$ does not have a local minimum in $M_n$.

Passing to the limit as $n\to \infty$, we get that any function $f\:\mathbb{R}^m\z\to\mathbb{R}$
of the form 
$$f(x)=\sum_i\lambda_i\cdot\phi(|q_i-x|_{\ell_p})$$
does not have a local minimum in $\mathbb{R}^m$.

Let $e_i$ be the standard basis in $\RR^m$. 
If $p<2$, consider the sum 
$$f(x)=\sum\phi(|q-x|_{\ell_p}),$$
where $q=\pm\eps\cdot e_i$ for all singes and $i$'s.
Straightforward calculations show that if $\eps>0$ is small, then $f$
has a strict local minimum at $0$.

If $p>2$, one has to take the same sum for  $p=\sum_i\pm\eps\cdot e_i$ for all choices of signs.
In both case we arrive at a contradiction.
\qeds

The argument given here is very close to the proof of Abresch--Gromoll inequality \cite{abresch-gromoll}.
The solution admits a straightforward generalization which imples that if an $m$-dimensional  Finsler manifold $F$ is a Gromov--Hausdorff limit of $m$-dimensional Riemannian manifolds with uniform lower bound on Ricci curvature, then $F$ has to be Riemannian.

An alternative solution of this problem can be build on the almost splitting theorem proved by  Jeff Cheeger and Tobias Colding \cite{cheeger-colding}.





%%%%%%%%%%%%%%%%%%%%%%%%%%%%%%%%%%%%%%%%%%%%%%%%%%
\parbf{Area of spheres.}
Fix $r_0>0$.
Given $r>r_0$, choose a point $q$ on the distance $2\cdot r$ from $p$.

\begin{wrapfigure}{r}{35 mm}
\vskip-5mm
\centering
\includegraphics{mppics/pic-308}
\end{wrapfigure}

Note that any minimizing geodesic from $q$ to a point in $B=B(p,r_0)$
has to cross $S\z=\partial B(p,r)$.
The statement follows since  
\[\vol B\le C_m\cdot r_0\cdot \area S,\]
where $C_m$ is a constant depending only on the dimension $m=\dim M$.
This volume comparison inequality can be proved along the same lines as the Bishop--Gromov inequality.
\qeds


Applying the coarea formula, 
we see that volume growth of $M$ is at least linear; 
in particular $M$ has infinite volume.
The latter was proved independently 
by Eugenio Calabi 
and Shing-Tung Yau \cite{calabi,yau-ricci}.



%%%%%%%%%%%%%%%%%%%%%%%%%%%%%%%%%%%%%%%%%%%%%%%%%%
\parbf{Flat coordinate planes.}
Choose $\eps>0$ such that there is unique geodesic between any two points at distance $<\eps$ from the origin of $\RR^3$.

Consider three points $a$, $b$ and $c$ on the coordinate lines that are $\eps$-close 
to the origin.
The following observation is the key to the proof.

\begin{cl}{$({*})$}
There is a solid flat geodesic triangle in $(\RR^3,g)$ with vertices at $a$, $b$ and $c$.
\end{cl}

Since the coordinate planes are totally geodesic, 
the parallel translation along a coordinate line preserves the directions tangent to a coordinate plane.
Since the parallel translation preserves the angles between vectors, the angles between coordinate planes in $(\RR^3,g)$ are constant.

It follows that the angles of the triangle $[abc]$ coincide with its \emph{model angles},
that is, the angles in the plane triangle with the same sides.

Both curvature conditions imply that the triangle $[abc]$ bounds a solid flat geodesic triangle in   $(\RR^3,g)$.

Use the family of constructed flat triangles 
to show that at any $x$ point in the $\tfrac\eps{10}$-neighborhood of the origin
the sectional curvature 
vanishes in an open set of sectional directions.
The latter implies that the curvature is identically zero 
in this neighborhood.

Move the origin and apply the same argument locally.
This way we get that the curvature is identically zero everywhere.
\qeds

This problem is based on a lemma discovered by Sergei Buyalo [Lemma 5.8 in \ncite{buyalo}; see also \ncite{andersson-howard} and \ncite{panov-petrunin}].

%%%%%%%%%%%%%%%%%%%%%%%%%%%%%%%%%%%%%%%%%%%%%%%%%%
\parbf{Two-convexity.}
\textit{Morse-style solution.}
Choose $(x,y,z,t)$-coordinates in~$\RR^4$.

Consider a generic linear function $\ell\:\RR^4\to\RR$ that is close to the sum of coordinates $x+y+z+t$.
Note that $\ell$
has non-degenerate critical points on $\partial K$ and all its critical values are different.

For each $s$ consider the set 
$$W_s=\set{w\in \RR^4\backslash K}{\ell(w)<s}.$$
Note that $W_{-1000}$ contains a closed curve, say $\alpha$, that is contactable in $\RR^4\backslash K$, but not constructible in $W_{-1000}$.

Set $s_0$ to be the infimum of the values $s$ such that
the $\alpha$ is contactable in $W_s$.

Note that $s_0$ is a critical value of $\ell$ on $\partial K$;
denote by $p_0$ the corresponding critical point.
By 2-convexity of $\RR^4\backslash K$,
the index of $p_0$ has to be at most $1$.
On the other hand, a disc that contracts $\alpha$ cannot be moved lower $s_0$.
Therefore the index of $p_0$ has to be at least $2$, a contradiction.
\qeds

\parit{Alexandrov-style proof.}
Assume that the complement to $K$ is two-convex.

Note that two-convexity is preserved under linear transformation.
Apply a linear transformation of $\RR^4$ that makes the coordinate planes $\Pi_1$ and $\Pi_2$ not orthogonal.

According to the main result in \cite{ABB}, $W\z=\RR^4\backslash (\Int K)$ has non-positive curvature in the sense of Alexandrov.
In particular the universal metric covering $\tilde W$ of $W$ is a $\CAT(0)$ space.

By rescaling $\tilde W$ and passing to the limit we obtain that universal Riemannian covering $Z$ of $\RR^4$ branching in the planes $\Pi_1$ and $\Pi_2$ is a $\CAT(0)$ space.

Note that $Z$ is isometric to the Euclidean cone over universal covering $\Sigma$ of $\mathbb{S}^3$ branching in two great circles $\Gamma_i=\mathbb{S}^3\cap \Pi_i$ that are not orthogonal.
The shortest path in $\mathbb{S}^3$ between $\Gamma_1$ and $\Gamma_2$ traveled 4 times back and forth is shorter than $2\cdot\pi$ and it lifts to closed geodesic in $\Sigma$.
It follows that $\Sigma$ is not $\CAT(1)$ and therefore $Z$ is not $\CAT(0)$, a contradiction.\qeds

The Morse-style proof is based on an idea of Mikhael Gromov \cite[see \S\textonehalf{} in][]{gromov-SGMC}, where two-convexity was introduced.

Note that the $1$-neighborhood of these two planes has two-convex complement $W$ in the sense of the second definition;
that is, if a closed curve $\gamma$ lies in the plane $\Pi$
and is contactable in $W$, then it is contactable in $\Pi\cap W$.
Clearly the boundary of this neighborhood is not smooth
and as it follows from the problem, it cannot be smoothed in the class of two-convex sets. 

Two-convexity also shows up in comparison geometry --- the maximal open flat sets in the manifolds of nonnegative or nonpositive curvature are two convex \cite{panov-petrunin}.
%%%???+Davis-Charney

%%%%%%%%%%%%%???

\parbf{Convex lens.}
Before going into the proof, let us describe a straightforward idea which does not work.

By the Gauss formula, we get that 
\[\int\limits_{D}k_1\cdot k_2\le\int\limits_{D}K,\] 
where $K$ denotes the intrinsic curvature of $D$.
Therefore it would be sufficient to show that the right hand side is small;
however, the integral $\int_{D}K$ might be large for arbitrarily small angle between the discs; for example, if $M=\mathbb{S}^3$ it might be arbitrary close to $2\cdot \pi$.

\medskip

Denote by $\eps$ the maximal angle between the discs, we can assume that $\eps<\tfrac\pi2$.

Note that the function $h=\dist_{D'}$ is convex in $L$.
Moreover the gradient $\nabla_xh$ points outside of $L$ for any $x\in D$. 

Consider the restriction $f=h|_D$.
Note that $f$ is a concave function which vanishes on $\partial D$.

Assume that $f$ is smooth.
Since the discs are meeting at angle at most $\eps<\tfrac\pi2$,
we have that $|\nabla f|\le \sin\eps$ and 
\[(\Hess f)(v,v)+ \cos\eps \cdot \2(v,v)\le 0,\]
where $\2$ denotes the second fundamental form of $D$ in $M$.
It follows that
\begin{align*}
k_1\cdot k_2&=\det \2\le
\\
&\le \tfrac1{\cos^2\eps}\cdot \det(\Hess f)=
\\
&=\tfrac1{2\cdot\cos^2\eps}\cdot\left(|\Delta f|^2
-|\Hess f|^2\right).
\end{align*}


Applying the Bochner formula for $f$, we get that
\[\int\limits_D |\Delta f|^2
-|\Hess f|^2
-K\cdot|\nabla f|^2
=\int\limits_{\partial D}
\kappa\cdot|\nabla f|^2,\]
where $K$ and $\kappa$ denotes the curvature of $D$ and geodesic curvature of $\partial D$ in $D$ respectively.
By the Gauss--Bonnet formula, we get that
\[\int\limits_D 
K+\int\limits_{\partial D}\kappa=2\cdot\pi.\]
Therefore
\[\int\limits_Dk_1\cdot k_2\le \tfrac{\sin \eps}{\cos^2\eps}\cdot\pi.\]

If $f$ is not smooth, then one can smooth it using Greene--Wu construction \cite[Theorem~2]{greene-wu} and repeat the above argument for the obtained function.
\qedsf

This estimate was used by Nina Lebedeva and the author \cite{lebedeva-petrunin-curvature}.
For classical applications of Bochner's formula including the vanishing theorems and estimates for eigenvalues of Laplacian see \cite[][II \S 8 in]{lawson-michelsohn}.
