\csname @openrightfalse\endcsname
\chapter{Curvature free differential geometry}
\chaptermark{Curvature free}

This chapter we consider mostly Riemannian manifolds without curvature bounds.

The reader should be familiar with main technique including 
some simplectic geometry, in particular that geodesic flow preserves the volume on the tangent bungle. 

%%%%%%%%%%%%%%%%%%%%%%%%%%%%%%%%%%%%%%%%%
\subsection*{Besikovitch inequality}

\begin{pr}{}{Besikovitch inequality}
\label{Besikovitch inequality}
Let $g$ be a Riemannian metric on an $m$-dimensional cube $Q=[0,1]^m$ such that any curve connecting opposite faces has length at least $1$. 
Prove that $\vol(Q,g)\ge 1$, 
and equality holds if and only if $(Q,g)$ is isometric to the unit cube.
\end{pr}

%%%%%%%%%%%%%%%%%%%%%%%%%%%%%%%%%%%%%%%%%%%%%%%%%%
\parit{Semisolution.}
Set 
\[A_i=\set{(x_1,x_2,\dots,x_m)\in[0,1]^m}{x_i=0}.\]

Consider functions $f_i\:[0,1]^m\to\RR$ defined by
$f_i(x)=\dist_{A_i}x$.
Note that 
the map $\bm{f}\:([0,1]^m,g)\to\RR^m$
defined as
\[\bm{f}\:x\mapsto(f_1(x),f_2(x),\dots,f_m(x))\]
is Lipschitz.

Prove that Jacobian of  $\bm{f}$
is at most $1$
and $\bm{f}([0,1]^m)\supset [0,1]^m$.
Therefore 
\[\vol(Q,g)\ge \vol([0,1]^m)=1.\]

The equality case is left for the reader.
\qeds

The inequality was proved by Abram Besikovitch in \cite{besicovitch}.
It has number applications in Riemannian geometry.
For example using this inequality it is easy to solve the following problem.
\begin{itemize}
\item {\it Assume a metric $g$ on $\RR^m$ coincides with Euclidean outside of a bounded set $K$;
assume further that any geodesic which comes into $K$ goes out from $K$ the same way as if the metric would be Euclidean everywhere. 
Show that $g$ is flat.}
\end{itemize}

%%%%%%%%%%%%%%%%%%%%%%%%%%%%%%%%%%%%%%%%%
\subsection*{Minimal foliation\thm}

The minimal surface in Riemannian manifolds are defined on page \pageref{minimal surface}.

\begin{pr}{\thm}{Minimal foliation}\label{gromomorphic-curves} 
Consider $\mathbb{S}^2\times \mathbb{S}^2$ equipped with a Riemannian metric $g$ 
which is $C^\infty$-close to the product metric. 
Prove that there is a conformally equivalent metric $\lambda\cdot g$ and re-parametrization of $\mathbb{S}^2\times \mathbb{S}^2$
such that each sphere $\mathbb{S}^2\times x$ and $y\times \mathbb{S}^2$ forms a minimal surface 
in $(\mathbb{S}^2\times \mathbb{S}^2,\lambda\cdot g)$.
\end{pr}


The solution requires pseudo-holomorphic curves 
 introduced by Mikhael Gromov in \cite{gromov-pseudoholomorphic}.

%%%%%%%%%%%%%%%%%%%%%%%%%%%%%%%%%%%%%%%%%
\subsection*{Volume and convexity\thm}

A function $f$ defined on Riemannian manifold is called convex if for any geodesic $\gamma$, the composition $f\circ\gamma$ is a convex real-to-real function.

\begin{pr}{\thm}{Volume and convexity}
\label{Volume and convexity} 
Let
$M$ be a complete Riemannian manifold which admits a non-constant
convex function. Prove that $M$ has an infinite volume.
\end{pr}

The Liouville's theorem 
which states that geodesic flow on the tangent bundle to a Riemannian manifold preserves the volume form should help to solve this problem.


%%%%%%%%%%%%%%%%%%%%%%%%%%%%%%%%%%%%%%%%%
\subsection*{Sasaki metric}

Let $(M,g)$ be a Riemannian manifold.
The Sasaki metric is the most natural choice of metric on the tangent space $\T M$.
It is uniquely defined by the following properties:
\begin{enumerate}[(i)]
\item The natural projection $\tau\:\T M\to M$ is a Riemannian submersion.
\item The metric on each tangent space $\T_p\subset \T M$ is the Euclidean metric induced by $g$.
\item Assume $\gamma(t)$ is a curve in $M$ and $v(t)\in\T_{\gamma(t)}$ is a parallel vector field along $\gamma$. 
Note that $v(t)$ forms a curve in $\T M$ 
and $\T_{\gamma(t)}M$ forms a submanifold in $\T M$.
For the Sasaki metric, we have $\dot v(t)\perp \T_{\gamma(t)}M$ for any $t$.
\end{enumerate}

A more constructive way to describe Sasaki metric is given by identifying 
$\T_u[\T M]$ for any $u\in \T_p M$ with the direct sum of so called vertical and horizontal vectors $\T_p M\oplus \T_p M$.
The projection of this splitting defined by the differential of $\tau$
and the Levi-Civita connection.
Then $\T_u[\T M]$ is equipped with the metric  defined as 
\[\hat g(X,Y)=g(X^V,Y^V)+g(X^H,Y^H),\]
where $X^V,X^H\in\T_pM$ denotes the vertical and horizontal components of $X\in\T_u[\T M]$.



\begin{pr}{}{Sasaki metric}\label{pr:Sasaki metric}
Consider the tangent bundle $\T \mathbb{S}^2$ 
equipped with \hyperref[Sasaki metric]{Sasaki metric} $\hat g$ induced by a Riemannian metric $g$ on $\mathbb{S}^2$.
Show that $(\T \mathbb{S}^2, \hat g)$ lies on bounded Gromov--Hausdorff distance to the ray.
\end{pr}

%%%%%%%%%%%%%%%%%%%%%%%%%%%%%%%%%%%%%%%%%
\subsection*{Distant involution}

\begin{pr}{}{Distant involution}\label{Distant involution}
Construct a Riemannian metric $g$ on $\mathbb{S}^3$ and an involution $\iota\:\mathbb{S}^3\z\to\mathbb{S}^3$ such that $\vol (\mathbb{S}^3,g)$ is arbitrary small and 
\[|x\z-\iota(x)|_g>1\]
 for any $x\in\mathbb{S}^3$.
\end{pr}


%%%%%%%%%%%%%%%%%%%%%%%%%%%%%%%%%%%%%%%%%
\subsection*{Normal exponential map\easy}

Let $(M,g)$ be a Riemannian manifold;
denote by $\T M$ the tangent bundle over $M$ and by $\T_p=\T_pM$ the tangent space at point $p\in M$.

Given a vector $v\in\T_pM$ denote by $\gamma_v$ the geodesic in $(M,g)$
such that $\gamma(0)=p$ and $\gamma'(0)=v$.
The map $\exp\:\T M\to M$ defined by $v\mapsto \gamma_v(1)$ is called exponential map.

The restriction of $\exp$ to the $\T_p$ is called \index{exponential map}\emph{exponential map at} $p$ and denoted as $\exp_p$.

Given a smooth submanifold $S\subset M$;
denote by $\mathrm{N} S$ the normal bundle over $S$.
The restriction of $\exp$ to $\mathrm{N} S$ is called \index{normal exponential map}\emph{normal exponential map} of $S$ and denoted as $\exp_S$.

\begin{pr}{\easy}{Normal exponential map}\label{Normal exponential map}
Let $M,N$ be complete connected Riemannian manifolds.
Assume $N$ is immersed into $M$.
Show that the image  of the 
normal exponential map of $N$ is dense in $M$.
\end{pr}

%%%%%%%%%%%%%%%%%%%%%%%%%%%%%%%%%%%%%%%%%
\subsection*{Symplectic squeezing in the torus}




\begin{pr}{}{Symplectic squeezing in the torus}\label{Symplectic squeezing in the torus}
Let 
\[\omega=dx_1\wedge dy_1+ dx_2\wedge dy_2\]
be the standard symplectic form on $\RR^4$.
Assume $\ZZ^2$ is the the integer lattice in $(x_1,y_1)$ coordinate plane of $\RR^4$.

Show that an arbitrary bounded domain $\Omega\subset (\RR^4,\omega)$
admits a symplectic embedding into $(\RR^4,\omega)/\ZZ^2$. 
\end{pr}

%%%%%%%%%%%%%%%%%%%%%%%%%%%%%%%%%%%%%%%%%
\subsection*{Diffeomorphism test\easy}



\begin{pr}{\easy}{Diffeomorphism test}\label{Diffeomorphism test}
Let $M$ and $N$ be 
complete 
$m$-dimensional
simply connected 
Riemannian manifolds.
Assume $f\:M\to N$
is a smooth map such that 
$$|df(v)|\ge |v|$$
for any tangent vector $v$ of $M$.
Show that $f$ is a diffeomorphism.
\end{pr}

%%%%%%%%%%%%%%%%%%%%%%%%%%%%%%%%%%%%%%%%%
\subsection*{Volume of tubular neighborhoods}

\begin{pr}{}{Volume of tubular neighborhoods}\label{Volume of tubular neighborhoods}
Assume $M$ and $M'$ be isometric closed smooth submanifolds in $\RR^m$.
Show that for all small $r$ we have
$$\vol B_r(M)=\vol B_r(M'),$$
where $B_r(M)$ denotes the $r$-neighborhood of $M$.
\end{pr}

%%%%%%%%%%%%%%%%%%%%%%%%%%%%%%%%%%%%%%%%%
\subsection*{Disc\hard}

\begin{pr}{\hard}{Disc}\label{Disc}
Given a big real number $L$,
construct a Riemannian metric $g$ on the disc $\mathbb D$ 
with 
\[\diam(\mathbb D,g)\le 1
\ \ 
\text{and}
\ \ 
\length \partial\mathbb D\le 1  \]
such that any null-homotopy of the boundary in $(\mathbb D,g)$ 
has a curve of length at least $L$.
\end{pr}

%%%%%%%%%%%%%%%%%%%%%%%%%%%%%%%%%%%%%%%%%
\subsection*{Shortening homotopy}

\begin{pr}{}{Shortening homotopy}\label{short-homotopy}
Let $M$ be a compact Riemannian manifold with diameter $D$.
Assume that for some $L>D$,
there are no geodesic loops in $M$
with length in the interval $(L-D,L+ D]$.
Show that for any path $\gamma_0$ in $(M,g)$
there is a homotopy $\gamma_t$ rel. to the ends
such that 
\begin{enumerate}[a)]
\item $\length \gamma_1<L$;
\item $\length \gamma_t\le \length \gamma_0+2\cdot D$ for any $t\in[0,1]$.
 
\end{enumerate}
\end{pr}

%%%%%%%%%%%%%%%%%%%%%%%%%%%%%%%%%%%%%%%%%
\subsection*{Convex hypersurface}

\begin{pr}{}{Convex hypersurface}\label{Convex hypersurface}
Let $M$ be a hypersurface 
in a closed Riemannian $m$-dimensional manifold $W$.
Assume $M$ is geodesic and convex
and its injectivity radius is at least $1$.
Show that there is a point in $W$  which lies on a distance at least  
$\frac{1}{2\cdot(m+1)}$ from $M$.
\end{pr}

%%%%%%%%%%%%%%%%%%%%%%%%%%%%%%%%%%%%%%%%%
\subsection*{Almost constant function}

\begin{pr}{}{Almost constant function}\label{Almost constant function}
Assume $\eps>0$ is given.
Show that there is a positive integer $m$ such that
for any closed $m$-dimensional Riemannian manifold $M$
and any smooth $1$-Lipschitz function $f\:M\to\RR$ the following holds.
\begin{itemize}
\item For a random unit-speed geodesic $\gamma$ in $M$ 
the event 
\[|f\circ\gamma(0)-f\circ\gamma(1)|>\eps\]
happens with probability at most $\eps$.
\end{itemize}
Here {}\emph{random} means that $\gamma'(0)$ takes the random value in the unit tangent bundle of $M$ for the natural choice of distribution.
\end{pr}

%%%%%%%%%%%%%%%%%%%%%%%%%%%%%%%%%%%%%%%%%
\subsection*{Bounded curvature}

Denote by $\mathcal{R}$ the space of 
all Riemannian metrics on $\mathbb S^5$
with absolute value of sectional curvature at most $1$,
and injectivity radius at least $1$.

It is easy to see that any metric $g_1\in \mathcal{R}$ 
can be connected to the canonical metric $g_0$ on $\mathbb S^5$
by a continuous family of metrics $g_t$ where $t\in[0,1]$.
The one parameter family of metrics $g_t$
can be found among the metrics of the type 
\[g_t=a(t)\cdot g_0+b(t)\cdot g_1,\]
where $a,b\:[0,1]\to\RR$
are smooth functions such that $a(0)=1=b(1)$ and $a(1-s)=0=b(s)$ for $s\le \tfrac13$.
In order to keep the bounds on the curvature and injectivity radius,
the functions $a$ and $b$
suppose to take huge values in the middle of interval.

\begin{pr}{\thm}{Bounded curvature}\label{Bounded curvature}
Fix a fast growing function, say
\[\Phi(x)=1000^{1000\cdot (x+1000)}.\]

Show that there is a metric $g_1$ 
such that 
for any family $g_t$ as above
\[\max_{t\in[0,1]}\{\vol(g_t)\}
>
\Phi(\vol(g_1)).\]
\end{pr}

The solution requires Novikov theorem on on  the algorithmic undecidability of the problem of recognition of the sphere $\mathbb S^m$  for
$m\ge 5$. 
A detailed proof of this theorem can be found in \cite{nabutovsky-NovThm}.


\section*{Semisolutions}


%%%%%%%%%%%%%%%%%%%%%%%%%%%%%%%%%%%%%%%%%%%%%%%%%%
\parbf{Minimal foliation.}
First show that there is a self-dual harmonic 2-form on $(\mathbb{S}^2\times\mathbb{S}^2,g)$;
that is, a 2-form $\omega$ such that $d\omega=0$ and $\star\omega=\omega$,
where $\star$ denotes the Hodge star operator.

Fix $p\in \mathbb{S}^2\times\mathbb{S}^2$.
Use the identity $\star\omega_p=\omega_p$
to show that
there is a real number $\lambda_p$ and the isometry $\mathrm{J}_p\:\T_p\to\T_p$ 
such that
$\mathrm{J}_p\circ\mathrm{J}_p =-\id$ 
and 
$\omega(X,Y)=\lambda_p\cdot g(X,\mathrm{J}_pY)$ for any $X,Y\in \T_p$.

Consider canonical symplectic form $\omega_0$ on $\mathbb{S}^2\times\mathbb{S}^2$;
that is sum of pullbacks of volume forms on $\mathbb{S}^2$  
for the two projections $\mathbb{S}^2\times\mathbb{S}^2\to \mathbb{S}^2$.
Note that for the canonical metric on $\mathbb{S}^2\times\mathbb{S}^2$,
the form $\omega_0$ is harmonic and self-dual. 
Since $g$ is close to the standard metric,
we can assume that $\omega$ is close to $\omega_0$.
In particular $\lambda_p\ne0$ for any $p\in \mathbb{S}^2\times\mathbb{S}^2$.

It follows that $\omega$ defines symplectic structure on $\mathbb{S}^2\times\mathbb{S}^2$
and $\mathrm{J}$ is its pseudo-complex structure.
It remains to take the re-parametrization of $\mathbb{S}^2\times \mathbb{S}^2$
so that vertical and horizontal spheres will form pseudo-holomorphic curves in the homology classes of $\mathbb{S}^2\times x$ and $x\times \mathbb{S}^2$.
\qeds
 
\parit{Comments.} 
For general metric the form $\omega$ might vanish at some points;
if the metric is generic,
then it happens on disjoint circles \cite[see][]{honda}.







%%%%%%%%%%%%%%%%%%%%%%%%%%%%%%%%%%%%%%%%%%%%%%%%%%
\parbf{Volume and convexity.}
Assume the contrary; that is, there is a complete Riemannian manifold $M$
with finite volume which admits a convex function $f$.

Denote by $\tau\:\UU M\to M$ the unit tangent bundle over $M$. 
Clearly $\vol (\UU M)$ is finite.

Note that 
there is a nonempty bounded open set $\Omega\subset \UU M$
such that $df(u)>\eps$ for any $u\in \Omega$ and some fixed $\eps>0$.

Denote by $\phi^t$ the geodesic flow on $\UU M$.
By Liouville's theorem 
\[\vol[\phi^t(\Omega)]=\vol\Omega\] 
for any $t$.

Given $u\in \Omega$,
consider the function $h\:t\mapsto f\circ\tau\circ\phi^t(u)$.
Note that $h'(t)>\eps$ for any $t\ge 0$.
It follows that there is an infinite sequence of positive reals $t_1,t_2,\dots$
such that 
$$\phi^{t_i}(\Omega)\cap\phi^{t_j}(\Omega)=\emptyset$$ 
if $i\ne j$.
The latter implies that $\vol (\UU M)=\infty$,
a contradiction.
\qeds

The idea in the proof is the same as in 
Poincar\'e recurrence theorem.

The problem is due 
to Richard Bishop and Barrett O'Neill \cite[see][]{bishop-oneill},
it was generalized by
Shing-Tung Yau  in \cite{yau}.

%%%%%%%%%%%%%%%%%%%%%%%%%%%%%%%%%%%%%%%%%%%%%%%%%%
\parbf{Sasaki metric.}
Show that there is a constant $\ell$
such that for any two unit tangent vectors $v\in\T_p\mathbb{S}^2$ 
and $w\in T_q\mathbb{S}^2$
there is a path 
$\gamma\:[0,1]\to\mathbb{S}^2$ from $p$ to $q$
such that 
\[\length \gamma\le \ell\] 
and
$w$ is the parallel transformation of $v$ along $\gamma$.

Note that once it is proved, 
it follows that diameter of the set of all vectors of fixed length in $\T \mathbb{S}^2$ has diameter at most $\ell$;
in particular the map $\T\mathbb{S}^2\to[0,\infty)$ defined as $v\mapsto |v|$ 
preserves the distance with the maximal error $\ell$.
Hence the result follows.



%%%%%%%%%%%%%%%%%%%%%%%%%%%%%%%%%%%%%%%%%%%%%%%%%%
\begin{wrapfigure}{r}{31 mm}
\begin{lpic}[t(-5 mm),b(3 mm),r(0 mm),l(0 mm)]{pics/tripod}
\end{lpic}
\end{wrapfigure}

\parbf{Distant involution.}
Given $\eps>0$, construct a disc $D$ in the plane with 
$$\length\partial D<10\ \ \text{and}\ \ \area D<\eps$$
which admits an continuous involution $\iota$ such that 
$$|\iota(x)-x|\ge 1$$ 
for any $x\in\partial D$.
An example of $D$ can be guessed from the picture. 

Take the product $D\times D\subset \RR^4$;
it is homeomorphic to the 4-ball.
Note that 
$$\vol_3[\partial(D\times D)]=2\cdot\area D\cdot\length \partial D<20\cdot\eps.$$
The boundary $\partial(D\times D)$ homeomorphic to $\mathbb{S}^3$
and the restriction of the involution $(x,y)\mapsto (\iota(x),\iota(y))$ has the needed property.

It remains to smooth $\partial(D\times D)$.
\qeds

This example is given by Christopher Croke in \cite{croke}.

It is instructive to show that for $\mathbb{S}^2$ such thing is not possible.

Note that according to systolic inequality, 
the involution $\iota$ above cannot be made isometric \cite[see][]{gromov-filling}.



 
%%%%%%%%%%%%%%%%%%%%%%%%%%%%%%%%%%%%%%%%%%%%%%%%%%
\parbf{Normal exponential map.}
Assume the contrary; that is, there is a point $p\in M$ 
such that the image of normal exponential map to $N$
 does not touch $\eps$-neighborhood of $p$.

Show that given $R>0$ there is $\delta>0$ such that 
if $x\in N$ and $|p-x|_M<R$, 
then there is a unit speed curve in $N$
which moves to $p$ with velocity at least $\delta$.
(In fact, 
the value $\delta$ depends on $\eps$, $R$ and the curvature bounds in $B(p,R)$.)


Following this curve for sufficient time brings us to $p$;
that is, $p\in N$, a contradiction.
\qeds

\begin{wrapfigure}[7]{r}{25 mm}
\begin{lpic}[t(-7 mm),b(0 mm),r(0 mm),l(0 mm)]{pics/spiral}
\end{lpic}
\end{wrapfigure}


The problem was suggested by Alexander Lytchak.

From the picture, you should guess an example of immersion 
such that one point does not lie in the image of the corresponding normal exponential.
It might be interesting to see in more details 
which sets can be avoided by such images.



%%%%%%%%%%%%%%%%%%%%%%%%%%%%%%%%%%%%%%%%%%%%%%%%%%
\parbf{Symplectic squeezing in the torus.}
Equip $\RR^4$ with $(x_1,y_1,x_2,y_2)$-coordinates
so that 
\[\omega=dx_1\wedge dy_1+dx_2\wedge dy_2\]
is the symplectic form. 

The embedding will be given as a composition of a linear symplectomorphism $\lambda$ 
with the quotient map $\phi\:\RR^4\to \TT^2\times\RR^2$ by the integer $(x_1,y_1)$-lattice.
Clearly $\phi\circ\lambda$ preserves the symplectic structure,
it remains to find $\lambda$ such that the restriction $\phi\circ\lambda|_\Omega$
is injective.

Without loss of generality,
we can assume that $\Omega$ is a ball centered at the origin.
Choose an oriented 2-dimensional subspace $V$ subspace of $\RR^4$ 
such that the integral of $\omega$ over 
$\Omega\cap V$ is small positive number, say $\tfrac\pi4$. 

Note that there is a linear symplectomorphism $\lambda$
 which maps planes parallel to $V$ to planes
parallel to the $(x_1,y_1)$-plane, 
and that maps the disk $V\cap\Omega$ to a disk.
It follows that the the intersection of $\lambda(\Omega)$ 
with any plane parallel to the $(x_1,y_1)$-plane is a disk of radius at most $\tfrac 12$.
In particular $\phi\circ\lambda|_\Omega$
is injective.\qeds

This construction is given 
by Larry Guth in \cite{guth-symplectic}
and attributed to Leonid Polterovich.

Note that according to the Gromov's non-squeezing theorem in \cite{gromov-pseudoholomorphic}, 
an analogous statement with $\CC\times \DD$ as the target does not hold, here $\DD\subset \CC$ is the open disc with the induced symplectic structure.
In particular, it shows that
the projection of $\lambda(\Omega)$ as above 
to $(x_1,y_1)$-plane
cannot be made arbitrary small.

%%%%%%%%%%%%%%%%%%%%%%%%%%%%%%%%%%%%%%%%%%%%%%%%%%
\parbf{Diffeomorphism test.}
Since $N$ is simply connected, 
it is sufficient to show that $f\:M\to N$ is a covering map.

Note that $f$ is an open immersion.
Let $h$ be the pullback metric on $M$ for $f\:M\to N$.
Clearly $h\ge g$.
In particular $(M,h)$ is complete and the map $f\:(M,h)\to N$ is a local isometry. 

It remains to prove that any local isometry between complete connected Riemannian manifolds of the same dimension if a covering map.\qeds 

%???\parit{Comments.} 

%%%%%%%%%%%%%%%%%%%%%%%%%%%%%%%%%%%%%%%%%%%%%%%%%%
\parbf{Volume of tubular neighborhoods.}
Let us denote by $\mathrm{N} M$ and $\T M$ the normal and tangent bundle of $M$ in $\RR^m$.

Consider the the normal exponential map $\exp_M\:\mathrm{N} M\to\RR^m$
and denote by $J_V$ its Jacobian at $V\in \mathrm{N}_pM$.
Note that for all small $\eps>0$, we have
\[\vol B_\eps(M)=\int\limits_M d_p\vol_m\cdot\int\limits_{B(0,r)_{\mathrm{N}_pM}}J_V\cdot d_V\vol_{n-m}.\leqno{({*})}\]

Set $m=\dim M$.
Given $p\in M$, 
denote by $s_p\:\T_p\times \T_p\to \mathrm{N}_p$
the second fundamental form of $M$.
Recall that the curvature tensor of $M$ at $p$ can be expressed the following way
\[R_p(X\wedge Y), V\wedge W\rangle 
=\langle s_p(X,W), s_p(Y,V)\rangle-\langle s_p(X,V), s_p(Y,W)\rangle.\]

Given $V\in \mathrm{N}_p M$,
express $J_V$ in terms of $\<s_p(X,Y),V\>$.
Show that for small $r$ the integral
\[v(r)=\int\limits_{B(0,r)_{\mathrm{N}_pM}}J_V\cdot d_V\vol_{n-m}\]
is a polynomial 
of $r$ and its coefficients can be expressed in terms of the curvature tensor $R_p$.

It follows that the right hand side in $({*})$ can be expressed in terms of curvature tensor of $M$.
The problem follows since the curvature tensor can be expressed in terms of metric tensor of $M$.\qeds


The formula for volume of tubular neighborhood 
was given by Hermann Weyl in \cite{weyl}.

%%%%%%%%%%%%%%%%%%%%%%%%%%%%%%%%%%%%%%%%%%%%%%%%%%
{
\begin{wrapfigure}[9]{r}{33 mm}
\begin{lpic}[t(-5 mm),b(-0 mm),r(0 mm),l(0 mm)]{pics/tree(1)}
\end{lpic}
\end{wrapfigure}

\parbf{Disc.}
Show that given a positive integer $n$ one can construct a tree $T$ embedded into the disc such that any homotopy of the boundary of the disc to a point pass through a curve which intersects $n$ different edges.
(For the tree on the diagram $n=3$.)

Fix small $\eps>0$, say $\eps=\tfrac1{10}$.
Consider the disc with embedded tree $T$ as above.
We will construct a metric on the disc 
with diameter and length of its boundary below $1$
such that 
the distance between any two edges of $T$ of without common vertex 
is at least $\eps$.

To construct such a metric, 
fix a metric on the cylinder $\mathbb S^1\times [0,1]$ such that

}

\begin{itemize}
\item The $\eps$-neighborhoods of the boundary components 
have product metrics.
\item Any vertical segment $x\times[0,1]$ has length $\tfrac 12$.
\item One of the boundary component has length $\eps$.
\item The other boundary component has length $2\cdot m\cdot \eps$, 
where $m$ is the number of edges in $T$.
\end{itemize}
Equip $T$ with a length-metric so that each edge has length $\eps$
and glue the long boundary component of the cylinder to $T$ by piecewise isometry so that the resulting space is homeomorphic to disc and the tree corresponds to it-self.



According to the first construction,
for any null-homotopy of the boundary 
the least length is at least $n\cdot\tfrac{\eps}{10}$.
The obtained metric is not Riemannian, but is easy to smooth.
Since $n$ is arbitrary the result follows.
\qeds
 
This example was constructed by Sidney Frankel and Mikhail Katz in \cite{frankel-katz}.
 

%%%%%%%%%%%%%%%%%%%%%%%%%%%%%%%%%%%%%%%%%%%%%%%%%%
\parbf{Shortening homotopy.}
Set 
\[p=\gamma_0(0)\ \ \text{and}\ \  \ell_0=\length\gamma_0.\]

By compactness argument,
there exists $\delta>0$ 
such that no geodesic loops based at $p$ with has length in the interval $(L-D, L+D+\delta]$. 

Assume $\ell_0\ge L+\delta$.
Choose $t_0\in [0,1]$ such that
\[\length\left(\gamma_0|_{[0,t_0]}\right)=L+\delta\]
Let $\sigma$ be a the minimizing geodesic from $\gamma(t_0)$
to $p$.
Note that $\gamma_0$ is homotopic to the joint 
\[\gamma_0'=\gamma_0|_{[0,t_0]}*\sigma*\bar\sigma*\gamma|_{[t_0,1]},\]
where $\bar\sigma$ denotes the backward parametrization of $\sigma$.

Consider the loop $\lambda_0$ at $p$
formed by joint of $\gamma|_{[0,t_0]}$ and $\sigma$.
Applying a curve shortening process to $\lambda_0$, 
we get a curve shortening homotopy $\lambda_t$
rel. its ends 
from the loop $\lambda_0$ to a geodesic loop $\lambda_1$ at $p$.
From above, 
\[\length\lambda_1\le L-D.\]

The joint $\gamma_t=\lambda_t*\bar\sigma*\gamma|_{[t_0,1]}$
is a homotopy
from $\gamma_0'$ to an other curve $\gamma_1$.
From the construction it is clear that 
\begin{align*}
 \length \gamma_t&\le \length \gamma_0+2\cdot \length\sigma\le
 \\
 &\le \length \gamma_0+2\cdot D
\end{align*}
for any $t\in[0,1]$
and 
\begin{align*}
 \length \gamma_1&=\length\lambda_1+\length\sigma+\length\gamma|_{[t_0,1]}\le
\\ &\le L-D+D+\length\gamma-(L+\delta)=
\\ &=\ell_0 -\delta.
\end{align*}
Repeating the procedure few times we get we get curves $\gamma_2$, $\gamma_3,\dots,\gamma_n$
joint by the needed homotopies so that 
$\ell_{i+1}\le\ell_i-\delta$ and $\ell_n< L+\delta$,
where $\ell_i=\length\gamma_i$.

If $\ell_n\le L$, we are done.
Otherwise repeat the argument once more for $\delta'=\ell_n-L$.
\qeds

The problem is due to 
Alexander Nabutovsky 
and Regina Rotman \cite[see][]{nabutovsky-rotman}.

It is not at all easy to find an example of a manifold  which satisfy the above condition for some $L$;
they are found among the Zoll spheres
by Florent Balachev, Christopher Croke and Mikhail Katz \cite[see][]{balacheff-croke-katz}.

%%%%%%%%%%%%%%%%%%%%%%%%%%%%%%%%%%%%%%%%%%%%%%%%%%
\parbf{Convex hypersurface.}
Let $h$ be the maximal distance from points in $W$ to $M$.

Fix a fine triangulation of $W$ 
so that $M$ becomes a sub-complex.
Say, let us assume that the diameter of each simplex in $\tau$ is less than 
$\eps$.
We can assume that $\tau$ is a barycentric subdivision of an other triangulation, so all the vertices of $\tau$ can be colored into colors $(0,\dots, m+1)$
in such a way that the vertices of each simplex 
get different colors.
Denote by $\tau_i$ the maximal $i$-dimensional sub-complex of $\tau$ 
with all the vertices colored by $0,\dots, i$.

For each vertex $v$ in $\tau$ 
choose a point $v'\in M$ on the distance $\le h$.
Note that 
if $v$ and $w$ are the vertices of one simplex,
then
\[|v'-w'|_M<2\cdot h+\eps.\]

If $\tfrac{1}{2\cdot(m+1)}>h$, take $\eps<\tfrac{1}{2\cdot(m+1)}-h$.
Let us extend the map $v\mapsto v'$ 
to a continuous 
map $W\to M$.
The map is already defined on $\tau_0$.
Using the cone construction we can extend it to $\tau_1$;
we can do this since the distance between vertices in one simplex are below injectivity radius of $M$.
Repeat the cone construction recursively, to extend the map to $\tau_2,\dots,\tau_{m+1}=\tau$;
some distance estimates are needed here.

It follows that fundamental class of $M$ vanish in the homology ring of $M$, 
a contradiction. 
\qeds


This problem is a stripped version of the bound on filling radius given by Mikhael Gromov in \cite{gromov-filling}.  

%%%%%%%%%%%%%%%%%%%%%%%%%%%%%%%%%%%%%%%%%%%%%%%%%%
\parbf{Almost constant function.}
Given a positive integer $m$,
denote by $\delta_m$ 
the expected value $|x_1|$ of a unit vector 
$\bm{x}=(x_1,\dots,x_m)\z\in\RR^m$ 
with respect to the uniform distribution.

Observe that $\delta_m\to 0$ as $m\to\infty$.

Equip the unit tangent bundle $\UU M$ of $M$ with the natural probability measure.
Since $f$ is $1$-Lipschitz,
for a random vector $v$ in $\UU M$,
the  expected value of $|df(w)|$
is at most $\delta_m$.

Note that 
\begin{align*}
|f\circ \gamma(1)-f\circ\gamma(0)|
&=
\left|\int\limits_0^1df(\dot\gamma(t))\cdot dt\right|\le \\
&\le \int\limits_0^1\left|df(\dot\gamma(t))\right|\cdot dt.
\end{align*}

Assume $\dot\gamma(0)$
takes random value in $\UU M$.
By Liouville's theorem, the same holds for $\dot\gamma(t)$
for any fixed $t$.
Therefore
the expected value of
\[\int\limits_0^1|df(\dot\gamma(t))|\cdot dt\]
is at most $\delta_m$.

By Markov's inequality,
the probability of the event 
\[|f\circ \gamma(1)-f\circ\gamma(0)|>\eps\]
is at most $\tfrac{\delta_m}{\eps}$.
Hence the result follows.
\qeds

I learned the problem from Mikhael Gromov.
It gives an example 
of so called 
\index{concentration of measure}\emph{concentration of measure phenomenon}
introduced by Vitali Milman.
 
%%%%%%%%%%%%%%%%%%%%%%%%%%%%%%%%%%%%%%%%%%%%%%%%%%
\parbf{Bounded curvature.}
Show that there is an algorithm to estimate the Gromov--Hausdorff distance between two Riemannian manifolds given in any reasonable way.

Show that if two manifolds with bounded curvature are sufficiently close to each other then they are diffeomorphic.

Now assume contrary;
that is for any metric $g_1\in \mathcal{R}$ there is a path $g_t$ in $\mathcal{R}$
connecting the canonical metric $g_0$ to $g_1$ such that 
\[\vol g_t\le \Phi(\vol g_1).\]

If a 5-dimensional Riemannian manifold $M$ with curvature between $\pm1$ is diffeomorphic to $\mathbb{S}^5$ 
then it can be described by a metric $g_1$ in $\mathcal{R}$.
Let $g_t$ be the path as above.

Construct a finite set $F\subset \mathcal{R}$ 
which is sufficiently dense in the set of metrics in $\mathcal{R}$ with volume at most $\Phi(\vol g_1)$.
The path $g_t$ as above can be approximated by a sequence of metrics from $F$.

Using the algorithms above one can list all such sequences.
It implies existence of algorithm 
which recognize $\mathbb S^5$ among $5$-dimensional manifolds.
The later contradicts Novikov theorem.
\qeds

Instead of function $\Phi$ one can take any Turing computable function.
The problem and number of its generalizations 
are due to Alexander Nabutovsky \cite[see][]{nabutovsky-Disconnectedness}.