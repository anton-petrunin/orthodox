\csname @openrightfalse\endcsname
\chapter{Curves}

Recall that a \index{curve}\emph{curve} is a continuous map 
from a real interval into a space (for example, Euclidean plane)
and 
a {}\emph{closed curve} is a continuous map defined on a circle.
If the map is injective then the curve is called {}\emph{simple}.

We assume that the reader is familiar with related definitions including 
length of curve 
and its curvature.
The necessary material is covered in the first couple of lectures 
of a standard introduction to differential geometry, [see \ncite{hilbert-cohn-vossen}, \S26--27 or \ncite{toponogov-curves-and-surfaces}, Chapter 1].

\medskip

We give a practice problem with a solution --- after that, you are on your own.

\subsection*{Spiral}
\label{spiral}
The following problem states that 
if you drive on the plane and turn the steering wheel to the right all the time,
then you will not be able to come back to the same place.

\begin{pr}
Assume $\gamma$ is a smooth regular plane curve with strictly monotonic curvature. 
Show that $\gamma$ has no self-intersections.
\end{pr}

\begin{wrapfigure}{o}{25 mm}
\begin{lpic}[t(-4 mm),b(-2 mm),r(0 mm),l(0 mm)]{pics/kneser-log(1)}
\end{lpic}
\end{wrapfigure}

\parit{Semisolution.}
The trick is to show that the osculating circles of $\gamma$ are nested.

\medskip

Without loss of generality we may assume that the curve is parametrized by its length and its
curvature decreases.

Let $z(t)$ be the center of osculating circle at $\gamma(t)$
and $r(t)$ is its radius.
Note that 
\begin{align*}
z(t)&=\gamma(t)+\tfrac{\gamma''(t)}{|\gamma''(t)|^2},
&
r(t)&=\tfrac{1}{|\gamma''(t)|}.
\end{align*}

Straightforward calculations show that
\[|z'(t)|= r'(t).\]
Note that the curve $z(t)$ has no straight arcs;
therefore 
\[|z(t_1)-z(t_0)|<r(t_1)-r(t_0).\leqno({*})\]
if $t_1>t_0$.

Denote by $D_t$ the osculating disk of $\gamma$ at $\gamma(t)$;
it has center at $z(t)$ and radius $r(t)$.
By $({*})$, $D_{t_1}$ lies in the interior of $D_{t_0}$ for any $t_1>t_0$.
Hence the result follows.\qeds

This problem was considered by Peter Tait \cite[see][]{tait}
and later rediscovered by Adolf Kneser \cite[see][]{kneser}.
The osculating circles of the curve give a peculiar decomposition of an annulus into circles; it has the following property: if a smooth function is constant on each osculating circle it must be constant in the annulus \cite[see][Lecture 10]{fuchs-tabachnikov}.
The same idea leads to a solution of the following problem: %???

\begin{pr}
Assume that $\gamma$ is a smooth regular plane curve with strictly monotonic curvature. 
Show that no line can be tangent to $\gamma$ at two distinct points.
\end{pr} %???monotone or monotonic???



It is instructive to check that the 3-dimensional analog does not hold;
that is, there are self-intersecting smooth regular space curves with strictly monotonic curvature. 

Note that if the curve $\gamma(t)$ is defined for $t\in[0,\infty)$ and its curvature converges to $\infty$ as $t\to \infty$, 
then the problem implies the convergence of $\gamma(t)$ as $t\to\infty$.
The latter could be considered as a continuous analog of the Leibniz's test for alternating series.

%%%%%%%%%%%%%%%%%%%%%%%%%%%%%%%%%%%%%%%%%%%%%%%%%
{

\begin{wrapfigure}[6]{r}{27 mm}
\begin{lpic}[t(-5 mm),b(0 mm),r(0 mm),l(0 mm)]{pics/moon-in-puddle(1)}
\lbl{14.5,22.5;$F$}
\end{lpic}
\end{wrapfigure}

\subsection*{Moon in a puddle}
\label{moon-in-puddle}

\begin{pr}
A smooth closed simple plane curve with curvature less than $1$ bounds a figure $F$. 
Prove that $F$ contains a disk of radius~$1$.
\end{pr}

}

%%%%%%%%%%%%%%%%%%%%%%%%%%%%%%%%%%%%%%%%%%%%%%%%%
\subsection*{Spring in a tin}
\label{A spring in a tin} 

\begin{pr}
Let $\alpha$ be a closed smooth curve immersed
in a unit disk. 
Prove that the average absolute curvature of $\alpha$ is at least $1$, with
equality if and only if $\alpha$ is the unit circle possibly traversed more than once.
\end{pr}

%%%%%%%%%%%%%%%%%%%%%%%%%%%%%%%%%%%%%%%%%%%%%%%%%
\subsection*{Curve in a sphere}
\label{A curve in a sphere}

\begin{pr}
Show that if a closed curve on the unit sphere intersects every equator then its length is at least $2\cdot\pi$.
\end{pr}

%%%%%%%%%%%%%%%%%%%%%%%%%%%%%%%%%%%%%%%%%%%%%%%%%
{

\begin{wrapfigure}{r}{30 mm}
\begin{lpic}[t(2 mm),b(-1 mm),r(0 mm),l(0 mm)]{pics/tangent-eq(1)}
\end{lpic}
\end{wrapfigure}

\subsection*{Oval in an oval}
\label{Oval in oval} 

\begin{pr}
Consider two closed smooth strictly convex planar curves, one inside the other. 
Show that there is a chord of the outer curve that is tangent to the inner curve at its midpoint.
\end{pr}

}

%%%%%%%%%%%%%%%%%%%%%%%%%%%%%%%%%%%%%%%%%%%%%%%%%
\subsection*{Capture a sphere in a knot\hard}
\label{Capture a sphere in a knot}

The following formulation uses the notion of smooth isotopy of knots;
that is, a one parameter family of embeddings 
\[f_t\:\mathbb{S}^1\z\to \RR^3,\ \ t\in[0,1]\] 
such that the map $[0,1]\times \mathbb{S}^1\to\RR^3$ is smooth.


\begin{pr}
Show that one can not capture a sphere in a knot.

More precisely, let $B$ be the closed unit ball in $\RR^3$
and $f\:\mathbb{S}^1\z\to \RR^3\backslash B$ be a knot.
Show that there is a smooth isotopy 
$$f_t\:\mathbb{S}^1\to \RR^3\backslash B,\ \ \ t\in [0,1],$$ 
such that $f_0=f$,
the length of $f_t$ is non-increasing with respect to $t$
and $f_1(\mathbb{S}^1)$ can be separated from $B$ by a plane.
\end{pr}

%%%%%%%%%%%%%%%%%%%%%%%%%%%%%%%%%%%%%%%%%%%%%%%%%
{
\begin{wrapfigure}{r}{24 mm}
\begin{lpic}[t(-4 mm),b(-1 mm),r(0 mm),l(0 mm)]{pics/gehring-0(1)}
\end{lpic}
\end{wrapfigure}

\subsection*{Linked circles}
\label{linked-circles}

\begin{pr}
Suppose that two linked simple closed curves in $\RR^3$
lie at a distance at least $1$ from each other.
Show that the length of each curve is at least $2\cdot\pi$.
\end{pr}

}
%%%%%%%%%%%%%%%%%%%%%%%%%%%%%%%%%%%%%%%%%%%%%%%%%
\subsection*{Surrounded area}
\label{Surrounded area}

\begin{pr}
Consider two simple closed plane curves 
$\gamma_1,\gamma_2\:\mathbb S^1\to\RR^2$.
Assume 
\[|\gamma_1(v)-\gamma_1(w)|\le|\gamma_2(v)-\gamma_2(w)|\]
for any $v,w\in \mathbb S^1$.
Show that the area surrounded by $\gamma_1$ does not exceed the area surrounded by $\gamma_2$. 
\end{pr}



%%%%%%%%%%%%%%%%%%%%%%%%%%%%%%%%%%%%%%%%%%%%%%%%%
\subsection*{Crooked circle}

\label{Crooked circle}

\begin{pr}
Construct 
a bounded set in $\RR^2$
homeomorphic to an open disk
such that 
its boundary contains no simple curves.
\end{pr}

%%%%%%%%%%%%%%%%%%%%%%%%%%%%%%%%%%%%%%%%%%%%%%%%%
\subsection*{Rectifiable curve}
\label{Rectifiable curve}

For the following problem we need the notion of 
\index{Hausdorff measure}\emph{Hausdorff measure}.
Fix a compact set $X\subset\RR^2$ and $\alpha>0$.
Given $\delta>0$ consider the value
\[h(\delta)=\inf\left\{\,\sum_i(\diam X_i)^\alpha\,\right\}\]
where the infimum is taken over all finite coverings $\{X_i\}$ of $X$ 
such that $\diam X_i<\delta$ for each $i$.

Note that the function $\delta\mapsto h(\delta)$ is not decreasing in $\delta$.
In particular, $h(\delta)\to \mathcal{H}_\alpha(X)$ as $\delta\to 0$ for some (possibly infinite) value $\mathcal{H}_\alpha(X)$.
This value $\mathcal{H}_\alpha(X)$ is called $\alpha$-dimensional Hausdorff measure of $X$.

\begin{pr}
Let $X\subset \RR^2$ be a compact connected set
with finite 1-dimensional Hausdorff measure. 
Show that there is a rectifiable curve passing thru all the points in $X$.
\end{pr}

%%%%%%%%%%%%%%%%%%%%%%%%%%%%%%%%%%%%%%%%%%%%%%%%%
\subsection*{Typical convex curves}

Formally we do not need it in the problem, 
but it is worth noting that the curvature of a convex curve is defined almost everywhere;
it follows from the fact that monotonic functions are differentiable almost everywhere.

\begin{pr}
Show that \emph{most} of the convex closed curves in the plane
have vanishing curvature at every point where it is defined.
\end{pr}

We need to explain the meaning of word ``most'' in the formulation;
it use \index{Hausdorff distance}\emph{Hausdorff distance} and \index{G-delta set}\emph{G-delta sets}.

The Hausdorff distance $|A-B|_H$ between two closed bounded sets $A$ and $B$ in the plane is defined as the infimum of the positive numbers $r$ such that the $r$-neighborhood of $A$ contains $B$ and the $r$-neighborhood of $B$ contains $A$.

In particular we can equip the space of all closed plane curves with the Hausdorff metric.
The obtained metric space is locally compact.
The latter follows from the \index{selection theorem}\emph{selection theorem} \cite[see \S18 in][]{blaschke},
which states that the closed subsets of a fixed closed bounded set in the plane form a compact set with respect to the Hausdorff metric. 

A G-delta set in a metric space $X$ is defined as a countable intersection of open sets.
According to \index{Baire category theorem}\emph{Baire category theorem}, 
in locally compact metric spaces $X$,
the intersection of a countable collection of open dense set 
has to be dense.
(The same holds if $X$ is complete, but we will not need it.)

In particular, in $X$, 
the intersection of a finite or countable collection of G-delta dense sets is also a G-delta dense set. 
The later means that G-delta dense sets contain {}\emph{most} of $X$. 
This is the meaning of the word {}\emph{most} used in the problem.



\section*{Semisolutions}

%%%%%%%%%%%%%%%%%%%%%%%%%%%%%%%%%%%%%%%%%%%%%%%%%%

\parbf{Moon in a puddle.}
In the proof we will use the \emph{cut locus}
of $F$ with respect to its boundary\footnote{Also called \emph{medial axis}.};
it will be further denoted as $T$.
The cut locus can be defined as the closure
of the set of points $x\in F$ 
for which there exist two or more points in $\partial F$ minimizing the distance to $x$.



For each point $x\in T$, consider the subset $X\subset\partial F$ where minimal distance to $x$ is attained.
If $X$ is not connected then we say that $x$ is a \emph{cut point};
equivalently it means that for any sufficiently small neighborhood $U\ni x$, 
the complement $U\backslash T$ is disconnected.
If $X$ is connected 
then we say that $x$ is a \emph{focal point};
equivalently it means that the osculating circle to $\partial F$ at any point of $X$ is centered at $x$.


\begin{wrapfigure}{r}{28 mm}
\begin{lpic}[t(-0 mm),b(-4 mm),r(0 mm),l(0 mm)]{pics/moon-in-puddle-sol(1)}
\lbl[t]{21,10;$\partial F$}
\lbl[br]{8.5,30,;{\small $T$}}
\lbl[tr]{6,3;{\small $z$}}
\end{lpic}
\end{wrapfigure}

The trick is to show that $T$ contains a focal point, say $z$.
Since $\partial F$ has curvature of at most $1$, the radius of any osculating circle is at lest $1$.
Hence the distance from $\partial F$ to $z$ is at least 1,
and the statement will follow.

\medskip

After a small perturbation
of $\partial F$ we may assume that
$T$ is a graph embedded in
$F$ with finite number of edges.

Note that $T$ is a
deformation retract of $F$.
The retraction $F\to T$ can be obtained the following way:
(1) given a point $x\in F\setminus T$,
consider the (necessarily unique) point $\hat x\in \partial F$ that minimize the distance $|x-\hat x|$ and
(2) move $x$ along the extension of the line segment $[\hat x x]$ behind $x$ until it hits $T$.

In particular, $T$ is a tree.
Therefore $T$ has
an end vertex, say $z$.
The point $z$ is focal since there are arbitrary small neighborhoods $U$ of $z$ such that the complements $U\backslash T$ are connected.
\qeds

Note that we proved a slightly stronger statement, namely there are at least two points on $\partial F$ which osculating circles lie in $F$.
Note that these points are \index{vertex of curve}\emph{vertexes} of $\partial F$;
that is, they are critical points of its curvature.

Note further that inversion respects osculating circles.
That is, suppose $\gamma$ is an osculating circle of curve $\alpha$ at $t_0$;
assume $\gamma'$ and $\alpha'$ are inverions of $\gamma$ and $\alpha$,
then $\gamma'$ is an osculating circle of curve $\alpha'$ at $t_0$.
Therefore appling an inversion about a circle with the center in $F$, we also get a pair of osculating circles of $\partial F$ which surround $F$.
This way we obtain 4 osculating circles that lie on one side of $\partial F$.
The latter statement is a generalization of the four-vertex theorem.

The case of convex curves of this problem appears in a book of Wilhelm Blaschke \cite[see \S 24 in][]{blaschke}.
In full generality, the problem was discussed by Vladimir Ionin and German Pestov \cite[see][]{pestov-ionin}. %???include ref to my lectures 
A solution via curve shortening flow of a weaker statement 
was given by Konstantin Pankrashkin \cite[see][]{pankrashkin}.
The statement still holds if the curve fails to be smooth at one point.
A spherical version of the later statement 
was used by Dmitri Panov and me \cite[see][]{panov-petrunin-ramification}.

As you can see from the following problem, the 3-dimensional analog of this statement does not hold.

\begin{pr}
Construct a smooth embedding $\mathbb{S}^2\hookrightarrow \RR^3$ 
with all the principle curvatures between $-1$ and $1$
such that it does not surround a ball of radius 1.
\end{pr}

Such example can be obtained by fattening a nontrivial contractible 2-complex in $\RR^3$ 
[the Bing's house constructed in \ncite{bing} will do the job].
This problem is discussed by Abram Fet and Vladimir Lagunov in \cite{lagunov-2,lagunov-fet} 
and it was generalized to Riemannian manifolds with boundary by Stephanie Alexander and Richard Bishop \cite[see][]{alexander-bishop}.

A similar argument shows that for any Riemannian metric $g$ on the 2-sphere $\mathbb S^2$ 
and any point $p\in(\mathbb S^2,g)$ there is a minimizing geodesic $[pq]$ with conjugate ends.
On the other hand, for $(\mathbb S^3,g)$ this is not true.
Moreover there is a metric $g$ on $\mathbb{S}^3$ 
with sectional curvature bounded above by arbitrary small $\eps>0$ and $\diam(\mathbb{S}^3,g)\le 1$.
In particular, $(\mathbb S^3,g)$ has no minimizing geodesic with conjugate ends.
An example was originally constructed by Mikhael Gromov \cite[see][]{gromov-almost-flat}; 
a simplification was given by 
Peter Buser
and Detlef Gromoll \cite[see][]{buser-gromoll}.

%%%%%%%%%%%%%%%%%%%%%%%%%%%%%%%%%%%%%%%%%%%%%%%%%%
\parbf{Spring in a tin.}
To solve this problem,
you should imagine that you travel on a train along the curve $\alpha(t)$
and watch the position of the center of the disk in the frame of your wagon.

\medskip

Denote by $\ell$ the length of $\alpha$.
Equip the plane with complex coordinates so that $0$ is the center of the unit disk.
We can assume $\alpha$ is equipped with an $\ell$-periodic parametrization by arc length.

Consider the curve $\beta(t)=t-\tfrac{\alpha(t)}{\alpha'(t)}$.
Observe that 
\[\beta(t+\ell)=\beta(t)+\ell\] 
for any $t$.
In particular 
\[\length (\beta|_{[0,\ell]}) 
\ge 
|\beta(\ell)-\beta(0)|
=
\ell.\leqno({*})\]

Also 
\begin{align*}
|\beta'(t)|&=|\tfrac{\alpha(t)\cdot\alpha''(t)}{\alpha'(t)^2}|\le
\\
&\le|\alpha''(t)|.
\end{align*}
Since $|\alpha''(t)|$ is the absoloute curvature of $\alpha$ at $t$,
the result follows from $({*})$.\qeds

The statement was originally proved 
by Istv\'an F\'ary in \cite{fary};
number of different proofs are discussed by Serge Tabachnikov [see \ncite{tabachnikov} and also 19.5 in \ncite{fuchs-tabachnikov}].

Note that the same argument works for curves in the unit ball.

If instead of the disk, 
we have a region bounded by a closed convex curve $\gamma$, 
then it is still true that the average curvature of $\alpha$ is at least as big as average curvature of $\gamma$. 
The proof was given by Jeffrey Lagarias
and Thomas Richardson [see \ncite{lagarias-richardson} and also \ncite{nazarov-petrov}].


%%%%%%%%%%%%%%%%%%%%%%%%%%%%%%%%%%%%%%%%%%%%%%%%%%
\parbf{Curve in a sphere.} 
Let us present two solutions.
We assume that $\alpha$ is a closed curve in $\mathbb{S}^2$ of length $2\cdot\ell$ that intersects each equator.

\parit{A solution with the Crofton formula.}
Given a unit vector $u$ denote by $e_u$ the equator with pole at $u$.
Let $k(u)$ the number of intersections
between $\alpha$ and $e_u$.

Note that for almost all $u\in \mathbb{S}^2$, the value $k(u)$ is even or infinite.
Since each equator intersects $\alpha$, we get $k(u)\ge 2$ for almost all $u$.

Then we get
\begin{align*}
2\cdot\ell&=\tfrac14\cdot\int\limits_{\mathbb{S}^2}k(u)\cdot\, d_u\area\ge 
\\
&\ge\tfrac12\cdot\area\mathbb{S}^2=
\\
&=2\cdot\pi.
\end{align*}

The first identity above is called the \index{Crofton formula}\emph{Crofton formula}.
To prove this formula, start with the case when the curve is formed by one geodesic segment,
summing up we get it for broken lines
and by approximation it holds for all curves.
\qeds

\parit{A solution by symmetry.}
Let $\check\alpha$ be a sub-arc of $\alpha$ of length $\ell$, with endpoints $p$ and $q$. 
Let $z$ be the midpoint of a minimizing geodesic $[pq]$ in $\mathbb{S}^2$. 

Let $r$ be a point of intersection of $\alpha$ with the equator with pole at $z$. 
Without loss of generality we may assume that $r\in\check\alpha$. 

The arc $\check\alpha$ together with its reflection with respect to the point $z$ forms a closed curve of length $2\cdot \ell$ passing throgh both $r$ and its antipodal point $r^{*}$.
Therefore 
\[\ell=\length \check\alpha\ge |r-r^{*}|_{\mathbb S^2}=\pi.\]
Here $|r-r^{*}|_{\mathbb S^2}$ 
denotes the angle metric in the sphere $\mathbb S^2$.\qeds


The problem was suggested by Nikolai Nadirashvili.
It is nearly equivalent to the following: 

\begin{pr}
Show that total curvature of any closed smooth regular space curve is at least $2\cdot\pi$.
\end{pr}

A way more advanced problem is to show that any embedded circle of total curvature at most $4\cdot\pi$ is unknotted.
It was solved independently by Istv{\'a}n F{\'a}ry \cite[see][]{fary-knot} and John Milnor \cite[see][]{milnor}. 
Later many interesting generalizations and refinements were found including a generalization to singular spaces 
by Stephanie Alexander and Richard Bishop \cite[see][]{alexander-bishop:knot} and the
theorem on embedded minimal disk proved by Tobias Ekholm, 
Brian White
and Daniel Wienholtz \cite[see][]{EWW}.

%%%%%%%%%%%%%%%%%%%%%%%%%%%%%%%%%%%%%%%%%%%%%%%%%%
%%%%%%%%%%%%%%%%%%%%%%%%%%%%%%%%%%%%%%%%%%%%%%%%%%
\parbf{Oval in an oval.}
Choose the a chord that minimizes (or maximizes) the ratio in which it divides the bigger oval.

If the chord is not divided into equal parts, then you can rotate it slightly
to decrease the ratio.
Hence the problem follows.
\qeds

\begin{wrapfigure}{r}{51 mm}
\begin{lpic}[t(-0 mm),b(-3 mm),r(0 mm),l(0 mm)]{pics/tangent-eq-sol(1)}
\lbl[tr]{20,5;$u$}
\lbl[rt]{18,15;$r$}
\lbl[rt]{26,6;$l$}
\lbl[bl]{25,10;$x_u$}
\end{lpic}
\end{wrapfigure}


\parit{Alternative solution.}
Given a unit vector $u$, denote by $x_u$ the point on the inner curve
with outer normal vector $u$.
Draw a chord of outer curve that is tangent to the inner curve at $x_u$;
denote by $r=r(u)$ and $l=l(u)$ the lengths of the segments of this chord to the right and to the left of $x_u$, respectively.


Arguing by contradiction, assume $r(u)\ne l(u)$ for all $u\in\mathbb{S}^1$.
Since the functions $r$ and $l$ are continuous,
we can assume that 
$$r(u)>l(u)\ \ \text{for all}\ \ u\in\mathbb{S}^1.\leqno{({*})}$$

Prove that
each of the following two integrals 
\begin{align*}
\tfrac12\cdot\int\limits_{\mathbb{S}^1}r^2(u)\cdot du
\quad\text{and}\quad
\tfrac12\cdot\int\limits_{\mathbb{S}^1}l^2(u)\cdot du
\end{align*}
give the area between the curves.
In particular, 
the integrals are equal. 
The latter contradicts $({*})$.\qeds



This is a problem by Serge Tabachnikov \cite[see][]{tabachnikov-mi}.
A closely related {}\emph{equal tangents problem} is discussed by the same author in \cite{tabachnikov-tan}.

%%%%%%%%%%%%%%%%%%%%%%%%%%%%%%%%%%%%%%%%%%%%%%%%%%
\parbf{Capture a sphere in a knot.}
We can assume that the knot lies on the sphere $\partial B$.

Fix a M\"obius transformation 
$m\:\mathbb{S}^2\to\mathbb{S}^2$ close to the identity and not a rotation.

Note that $m$ is a conformal map;
that is, there is a function $u$ defined on $\mathbb{S}^2$ 
as 
\[u(x)=\lim_{y,z\to x}\frac{|m(y)-m(z)|}{|y-z|}.\]
(The function $u$ is called the \emph{conformal factor} of $m$.)

Since the area is preserved, 
we get 
$$\frac1{\area \mathbb{S}^2}\cdot\int\limits_{\mathbb{S}^2} u^2=1.$$ 
By Bunyakovsky inequality, 
$$\frac1{\area \mathbb{S}^2}\cdot\int\limits_{\mathbb{S}^2} u<1.$$ 

It follows that after a suitable rotation of $\mathbb{S}^2$, 
the map $m$ decreases the length of the knot.

Iterate this construction and pass to the limit as $m\to\id$.
This way you get a continuous one parameter family of M\"obius transformations which shorten the length of the knot.
Therefore it drifts the knot to a single hemisphere and allows the ball to escape. 
\qeds


This is a problem by Zarathustra Brady, 
the given solution is based on the idea of David Eppstein \cite[see][]{zeb}.



%%%%%%%%%%%%%%%%%%%%%%%%%%%%%%%%%%%%%%%%%%%%%%%%%%

\parbf{Linked circles.} 
Denote the linked circles by $\alpha$ and $\beta$. 

Fix a point $x\in\alpha$. 
Note that there is a point $y\in\alpha$ such that the line segment 
$[xy]$ intersects $\beta$, say at the point $z$. 
Indeed, if this is not the case, 
applying a homothety with center $x$ to $\alpha$, would shrink it to $x$ without crossing $\beta$.
The latter contradicts that $\alpha$ and $\beta$ are linked. 

\begin{wrapfigure}{r}{54 mm}
\begin{lpic}[t(-0 mm),b(-0 mm),r(0 mm),l(0 mm)]{pics/gehring(1)}
\lbl[br]{4,31;$x$}
\lbl[l]{37.3,12;$y$}
\lbl[rw]{29,19;$z$}
\lbl[bl]{12.2,3.2;$\alpha$}
\lbl[br]{49,11;$\beta$}
\lbl[bl]{18,11;$\alpha^{*}$}
\end{lpic}
\end{wrapfigure}

Let $\alpha^{*}$ be the image of $\alpha$ under the central projection onto the unit sphere around $z$.
Clearly
$$\length \alpha\ge \length\alpha^{*}.$$

Note that $\alpha^{*}$ passes thru two antipodal points of the sphere, the
one corresponding to $x$ and the one corresponding to $y$.
Therefore 
$$\length \alpha^{*}\ge 2\cdot\pi.$$
Hence the result follows.\qeds


This problem was proposed by Frederick Gehring \cite[see 7.22 in][]{gehring};
solutions and generalizations are surveyed in \cite{mateljevic}. 
The presented solution is attributed to Marvin Ortel in \cite{CJKSW} and it is very close to the solution given by Michael Edelstein and Binyamin Schwarz \cite[see][]{edelstein-schwatz}.

%%%%%%%%%%%%%%%%%%%%%%%%%%%%%%%%%%%%%%%%%%%%%%%%%%
\parbf{Surrounded area.}
Let $C_1$ and $C_2$ be the compact regions bounded by $\gamma_1$ and $\gamma_2$ correspondingly.

By Kirszbraun theorem, 
any 1-Lipschitz map $X\to \RR^2$ defined on $X\subset \RR^2$
can be extended to a 1-Lipschitz map on the whole $\RR^2$.
In particular, there is a 1-Lipschitz map $f\:\RR^2\to\RR^2$ 
such that $f(\gamma_2(v))\z=f(\gamma_1(v))$ for any $v\in\mathbb S^1$.

Note that $f(C_2)\supset C_1$.
Hence the statement follows.\qeds


The Kirszbraun theorem appears in his thesis \cite[see][]{kirszbraun}
and was rediscovered later by Frederick Valentine \cite[see][]{valentine}.
An interesting survey is given by 
Ludwig Danzer, Branko Gr{\"u}nbaum and Victor Klee \cite[see][]{danzer-grunbaum-klee}.





%%%%%%%%%%%%%%%%%%%%%%%%%%%%%%%%%%%%%%%%%%%%%%%%%%
\parbf{Crooked circle.}
A continuous function $f\:[0,1]\to [0,1]$
will be called $\eps$-crooked 
if $f(0)=0$, $f(1)=1$ 
and for any segment $[a,b]\subset [0,1]$ 
one can choose $a\le x\le y\le b$ 
such that
\[|f(y)-f(a)|\le\eps\ \ \text{and}\ \ |f(x)-f(b)|\le\eps.\]

\begin{center}
\begin{lpic}[t(-0 mm),b(4 mm),r(0 mm),l(0 mm)]{pics/crooked(1)}
\lbl[t]{8,0;$\eps=\tfrac12$}
\lbl[t]{33,0;$\eps=\tfrac13$}
\lbl[t]{57,0;$\eps=\tfrac14$}
\lbl[t]{80,0;$\eps=\tfrac15$}
\end{lpic}
\end{center}

A sequence of $\tfrac1n$-crooked maps can be constructed recursively. 
Guess the construction from the diagram.

{

\begin{wrapfigure}{r}{46 mm}
\begin{lpic}[t(-2 mm),b(0 mm),r(0 mm),l(0 mm)]{pics/circrooked(1)}
\lbl[tr]{35,33;$\gamma_0$}
\lbl[bl]{38,36;$\gamma_1$}
\end{lpic}
\end{wrapfigure}

Now, start with the unit circle, 
$\gamma_0(t)=(\cos 2\pi t,\sin 2 \pi t)$.
Fix a sequence of positive numbers $\eps_n$ converging to zero very fast. 
Construct recursively a sequence of simple closed curves $\gamma_n\:[0,1]\z\to\RR^2$ such that $\gamma_{n+1}$ runs outside of the disk bounded by $\gamma_n$
and 
\[|\gamma_{n+1}(t)-\gamma_n\circ f_n(t)|<\eps_n,\]
for some $\eps_n$-crooked function $f_n$.
(On the diagram you can see an attempt to draw the first iteration.)

Denote by $D$ the union of all disks bounded by the curves $\gamma_n$.
Clearly $D$ is homeomorphic to an open disk.
For the right choice of the sequence $\eps_n$, 
the set $D$ is bounded.
By construction, the boundary of $D$ contains no simple curves. \qeds

In fact, the only curves in the boundary of the constructed set are constant. Compare to the problem \emph{Simple path} on page~\pageref{Simple path}.

The proof use the so called \emph{pseudo-arc} 
constructed by Bronis\l{}aw Knaster \cite[see][]{knaster}.
The proof resembles construction of the Cantor set.
Here are few similar problems:

}

\begin{pr}
 Construct three distinct open sets in $\RR$ with identical boundaries.
\end{pr}

\begin{pr}
 Construct three open disks in $\RR^2$ having the same boundary.
\end{pr}

These disks are called \index{lakes of Wada}\emph{lakes of Wada}; it is described by Kuniz\^{o} Yoneyama \cite[see][]{yoneyama}.

\begin{pr}
 Construct a Cantor set in $\RR^3$ with non simply connected complement.
\end{pr}

This example is called \index{Antoine's necklace}\emph{Antoine's necklace} \cite[see][]{antoine}.

\begin{pr}
 Construct an open set in $\RR^3$ with fundamental group isomorphic to the additive group of rational numbers.
\end{pr}

More advanced examples include
\emph{Whitehead manifold}, 
\emph{Dogbone space}, 
\emph{Casson handle};
see also the problem ``Conic neighborhood'' on page \pageref{Conic neighborhood}.





%%%%%%%%%%%%%%%%%%%%%%%%%%%%%%%%%%%%%%%%%%%%%%%%%%
\parbf{Rectifiable curve.}
The 1-dimensional Hausdorff measure will be denoted as $\mathcal{H}_1$. 

Set $L=\mathcal{H}_1(K)$.
Without loss of generality, we may assume that $K$ has diameter $1$.

Since $K$ is connected, we get 
\[\mathcal{H}_1(B(x,\eps)\cap K)\ge\eps\leqno(*)\]
for any $x\in K$ and $0<\eps<\tfrac12$.

Let $x_1,\dots, x_n$ be a maximal set of points in $K$ with 
\[|x_i-x_j|\z\ge\eps\] for all $i\ne j$. 
From $(*)$ we have $n\le2\cdot L/\eps$.

Note that there is a tree $T_\eps$ with vertices $x_1,\dots, x_n$ and straight edges with length at most $2\cdot\eps$ each.
Therefore the total length of $T_\eps$ is below $2\cdot n\cdot\eps\le 4\cdot L$.
By construction, 
$T_\eps$ is $\eps$-close to $K$ in the Hausdorff metric.

Clearly, there is a closed curve $\gamma_\eps$ whose image is $T_\eps$ and its length is twice the total length of $T_\eps$;
that is, 
\[\length\gamma_\eps\le 8\cdot L.\]

Passing to a partial limit of $\gamma_\eps$ as $\eps\to 0$,
we get the needed curve. \qeds

In terms of measure, the optimal bound is $2\cdot L$;
if in addition the diameter $D$ is known then it is $2\cdot L-D$.
The problem is due to 
Samuel Eilenberg 
and Orville Harrold 
\cite[see][]{eilenberg-harrold};
it also appears in the book of Kenneth Falconer \cite[see Exercise 3.5 in][]{falconer}.



%%%%%%%%%%%%%%%%%%%%%%%%%%%%%%%%%%%%%%%%%%%%%%%%%%
\parbf{Typical convex curves.}
Denote by $\mathfrak{C}$ the space of all closed convex curves in the plane equipped with the Hausdorff metric.
Recall that $\mathfrak{C}$ is locally compact.
In particular, by the Baire theorem, a countable intersection of everywhere dense open sets is everywhere dense.

Note that if a curve $\gamma\in\mathfrak{C}$ 
has nonzero second derivative at some point $p$,
then it lies between two circles with one of them tangent to the other from inside at $p$.

\begin{wrapfigure}[10]{o}{38 mm}
\begin{lpic}[t(-0 mm),b(0 mm),r(0 mm),l(0 mm)]{pics/typical-curve(1)}
\lbl[rt]{20,20;$\gamma$}
\end{lpic}
\end{wrapfigure}

Fix these two circles.
It is straightforward to check that there is $\eps>0$ such that 
the Hausdorff distance from any convex curve $\gamma$ squeezed between the circles 
to any convex $n$-gon is at least $\frac{\eps}{n^{100}}$.

Fix a countable dense set of convex polygons $\mathfrak{p}_1,\mathfrak{p}_2,\dots$ in $\mathfrak{C}$.
Denote by $n_i$ the number of sides in $\mathfrak{p}_i$.
For any positive integer $k$,
consider the set $\Omega_k\subset\mathfrak{C}$ defined as 
\[\Omega_k
=
\set{\xi\in \mathfrak{C}}{|\xi-\mathfrak{p}_i|_{H}<\tfrac1{k\cdot n_i^{100}}\quad\text{for some}\quad i},\]
where $|{*}-{*}|_H$ denotes the Hausdorff distance 

From above we get that $\gamma\notin\Omega_k$ for some $k$. 

Note that $\Omega_k$ is open and everywhere dense in $\mathfrak{C}$.
Therefore 
\[\Omega=\bigcap_k\Omega_k\]
is a G-delta dense set.
Hence the statement follows.\qeds

This problem states that typical convex curves have an unexpected property.
In fact, this is a very common situation --- it is hard to see the typical objects and these objects often have surprising properties.

For example, as it was proved by
Bernd Kirchheim, 
Emanuele Spadaro 
and 
L{\'a}szl{\'o} Sz{\'e}kelyhidi,
typical 1-Lipschitz maps from the plane to itself preserve the length of all curves \cite[see][]{KSS}.
The same way one could show that the boundaries of typical open sets in the plane contain no nontrivial curves, 
altho the construction of a concrete example is not trivial;
see ``Crooked circle'', page \pageref{Crooked circle}.
More problems of that type are surveyed by Tudor Zamfirescu \cite[see][]{zamfirescu}.







