\chapter{Curves}


Recall that a \index{curve}\emph{curve} is a continuous map defined on real interval
and 
a {}\emph{closed curve} is a continuous map defined on a circle.
If the map is injective then the curve is called {}\emph{simple}.

We assume that the reader is familiar with related definitions including 
length of curve 
and its curvature.
The necessary material is covered in the first couple of lectures 
of a standard introduction to differential geometry, 
see \cite[][\S26--27]{hilbert-cohn-vossen}
or  
\cite[][Chapter 1]{toponogov-curves-and-surfaces}.

\medskip

We give a practice problem with a solution, 
after that you are on your own.

\subsection*{Spiral}

The following problem states that 
if you drive on the plane and turn the steering wheel to the right all the time,
then you will not be able to come back to the same place.

\begin{pr}{}{Problem}\label{spiral}
Assume $\gamma$ is a smooth regular plane curve with strictly monotonic curvature. 
Show that $\gamma$ has no self-intersections.
\end{pr}

\parit{Solution.}
Without loss of generality we may assume that the curve is parametrized by its length and its
curvature decreases.

\begin{wrapfigure}{o}{25 mm}
\begin{lpic}[t(-0 mm),b(-2 mm),r(0 mm),l(0 mm)]{pics/kneser-log(1)}
\end{lpic}
\end{wrapfigure}

Let $z(t)$ be the center of osculating circle at $\gamma(t)$
and $r(t)$ is its radius.
That is 
\begin{align*}
z(t)&=\gamma(t)+\tfrac{\ddot\gamma(t)}{|\ddot\gamma(t)|^2},
&
r(t)&=\tfrac{1}{|\ddot\gamma(t)|}.
\end{align*}

Straightforward calculations show that
\[|\dot z(t)|\le \dot r(t).\]
It follows that the osculating discs are nested;
that is, 
$D_{t_1}\supset D_{t_0}$ for $t_1>t_0$.
Hence the result follows.\qeds

This problem gives a continuous analog of the Leibniz's test for alternating series.
It was considered by Peter Tait in \cite{tait}
and later rediscovered by Adolf Kneser in \cite{kneser};
see also \cite{ovsienko-tabachnikov}.

It is instructive to check that 3-dimensional analog does not hold.

%%%%%%%%%%%%%%%%%%%%%%%%%%%%%%%%%%%%%%%%%%%%%%%%%
{

\begin{wrapfigure}[6]{r}{27 mm}
\begin{lpic}[t(-5 mm),b(0 mm),r(0 mm),l(0 mm)]{pics/moon-in-puddle(1)}
\lbl{14.5,22.5;$F$}
\end{lpic}
\end{wrapfigure}

\subsection*{The moon in the puddle}

\begin{pr}{}{The moon in the puddle}\label{moon-in-puddle}
A smooth closed simple plane curve with curvature less than $1$ bounds a figure $F$. 
Prove that $F$ contains a disc of radius~$1$.
\end{pr}

}

%%%%%%%%%%%%%%%%%%%%%%%%%%%%%%%%%%%%%%%%%%%%%%%%%
\subsection*{A spring in a tin}

\begin{pr}{\many}{A spring in a tin}\label{A spring in a tin} 
Let $\alpha$ be a closed smooth immersed curve
inside a unit disc. 
Prove that the average absolute curvature of $\alpha$ is at least $1$, with
equality if and only if $\alpha$ is the unit circle possibly traversed more than once.
\end{pr}

%%%%%%%%%%%%%%%%%%%%%%%%%%%%%%%%%%%%%%%%%%%%%%%%%
\subsection*{A curve in a sphere}
\label{A curve in a sphere}

\begin{pr}{\many}{A curve in a sphere} 
Show that if a closed curve on the unit sphere intersects every equator then it has length at least $2\cdot\pi$.
\end{pr}

%%%%%%%%%%%%%%%%%%%%%%%%%%%%%%%%%%%%%%%%%%%%%%%%%
{

\begin{wrapfigure}{r}{30 mm}
\begin{lpic}[t(2 mm),b(-1 mm),r(0 mm),l(0 mm)]{pics/tangent-eq(1)}
\end{lpic}
\end{wrapfigure}

\subsection*{Oval in oval}


\begin{pr}{}{Oval in oval}\label{Oval in oval} 
Consider two closed smooth strictly convex planar curves, one inside the other. 
Show that there is a chord of the outer curve, which is tangent to the inner curve at its midpoint.
\end{pr}

}

%%%%%%%%%%%%%%%%%%%%%%%%%%%%%%%%%%%%%%%%%%%%%%%%%
\subsection*{Capture a sphere in a knot\hard}

The following formulation use the notion of smooth isotopy of knots;
that is, one parameter of embeddings 
\[f_t\:\mathbb{S}^1\z\to \RR^3,\ \  t\in[0,1]\] 
such that the map $[0,1]\times \mathbb{S}^1\to\RR^3$ is smooth.


\begin{pr}{}{Capture a sphere in a knot}\label{Capture a sphere in a knot}
Show that one can not capture a sphere in a knot.

More precisely, let $B$ be the closed unit ball in $\RR^3$
and $f\:\mathbb{S}^1\z\to \RR^3\backslash B$ be a knot.
Show that there is a smooth isotopy 
$$f_t\:\mathbb{S}^1\to \RR^3\backslash B,\ \ \ t\in [0,1],$$ 
such that $f_0=f$,
the length of $f_t$ does not increase in $t$
and $f_1(\mathbb{S}^1)$ can be separated from $B$ by a plane.
\end{pr}

%%%%%%%%%%%%%%%%%%%%%%%%%%%%%%%%%%%%%%%%%%%%%%%%%
\subsection*{Linked circles}

\begin{pr}{}{Linked circles}\label{linked-circles}
Suppose that two linked  simple closed curves in $\RR^3$
lie at a distance at least $1$ from each other.
Show that the length of each curve is at least $2\cdot\pi$.
\end{pr}

%%%%%%%%%%%%%%%%%%%%%%%%%%%%%%%%%%%%%%%%%%%%%%%%%
\subsection*{Surrounded area}

\begin{pr}{\easy}{Surrounded area}\label{Surrounded area}
Consider two simple closed plane curves  
$\gamma_1,\gamma_2\:\mathbb S^1\to\RR^2$.
Assume 
\[|\gamma_1(v)-\gamma_1(w)|\le|\gamma_2(v)-\gamma_2(w)|\]
for any $v,w\in \mathbb S^1$.
Show that the area surrounded by $\gamma_1$ does not exceed the area surrounded by $\gamma_2$. 
\end{pr}

%%%%%%%%%%%%%%%%%%%%%%%%%%%%%%%%%%%%%%%%%%%%%%%%%
\subsection*{Fat curve}

\begin{pr}{\easy}{Fat curve}\label{Fat curve}
Construct a simple plane curve with positive Lebesgue measure.
\end{pr}


%%%%%%%%%%%%%%%%%%%%%%%%%%%%%%%%%%%%%%%%%%%%%%%%%
\subsection*{Crooked circle}

\begin{pr}{}{Crooked circle}\label{Crooked circle} 
Construct 
a bounded open disc in $\RR^2$ 
such that 
its boundary contains no simple curve.
\end{pr}

%%%%%%%%%%%%%%%%%%%%%%%%%%%%%%%%%%%%%%%%%%%%%%%%%
\subsection*{Rectifiable curve}

For the following problem we need the notion of 
\index{Hausdorff measure}\emph{Hausdorff measure}.
Fix a compact set $X\subset\RR^2$ and $\alpha>0$.
Given $\delta>0$ consider the value
\[h(\delta)=\inf\left\{\,\sum_i(\diam X_i)^\alpha\,\right\}\]
where the infimum is taken for all coverings of $X$ by $\{X_i\}$
such that $\diam X_i<\delta$ for each $i$.

Note that the function $\delta\mapsto h(\delta)$ is not decreasing in $\delta$.
In particular, the there is a (possibly infinite) limit, say $h$, of $h(\delta)$ as $\delta\to0$.
This value $h$ is called $\alpha$-dimensional Hausdorff measure of $X$.

\begin{pr}{}{Rectifiable curve}\label{Rectifiable curve}
Let $X\subset \RR^2$ be a compact connected set
with finite 1-dimensional Hausdorff measure. 
Show that $X$ is an image of rectifiable curve.
\end{pr}

%%%%%%%%%%%%%%%%%%%%%%%%%%%%%%%%%%%%%%%%%%%%%%%%%
\subsection*{Typical convex functions}

Recall that \index{G-delta set}\emph{G-delta set} is defined as a countable intersection of open sets.
According to \index{Baire category theorem}\emph{Baire category theorem}, 
in a complete metric spaces,
any intersection of countable collection of dense open set 
has to be dense.

In particular, in a complete metric spaces, 
the intersection of a finite or countable collection of G-delta dense sets is also G-delta dense. 
The later means that G-delta dense sets contains {}\emph{most} of the points of a complete metric space. 
This is the meaning of the word {}\emph{most} used in the following problem.

\begin{pr}{\easy}{Typical convex functions}\label{Most of the convex functions}
Consider the space of convex $1$-Lipschitz functions defined on $[0,1]$,
equipped with the metric induced by sup-norm.

Show that most of these functions have vanishing the second derivative at every point where it is defined.
\end{pr}


\section*{Semisolutions}

%%%%%%%%%%%%%%%%%%%%%%%%%%%%%%%%%%%%%%%%%%%%%%%%%%

\parbf{The moon in the puddle.}
Consider the {\it cut locus} $T$
of $F$ with respect to $\partial F$;
it is defined as the closure
of the set of points $x\in F$ 
such that there are two or more points in $\partial F$ which minimize distance to $x$.

\begin{wrapfigure}{o}{28 mm}
\begin{lpic}[t(-0 mm),b(-0 mm),r(0 mm),l(0 mm)]{pics/moon-in-puddle-sol(1)}
\lbl[t]{21,10;$\partial F$}
\lbl[br]{8.5,30,;{\small $T$}}
\lbl[tr]{6,4;{\small $z_0$}}
\end{lpic}
\end{wrapfigure}

Note that after a small perturbation
of $\partial F$ we may assume that
$T$ is a graph embedded in
$F$ with finite number of edges.

Note that $T$ is a
deformation retract of $F$.
The retraction can be obtained by moving each point $y\in F\setminus T$ to $T$
along the line segment from $y$ to $T$ 
which extends behind $y$ the (necessary unique) shortest segment from $y$ to $\partial F$.

In particular, $T$ is a tree.
Therefore $T$  has
at least two end vertices;
denote one of them by $z$.

Prove that the disc of radius $1$ centered at $z$ lies completely in $F$.\qeds

The problem discussed by German Pestov and Vla\-di\-mir Ionin in \cite{pestov-ionin}.
An other solution via curve shortening flow 
was given by Konstantin Pankrashkin in  \cite{pankrashkin}.
The statement still holds if the curve fails to be smooth at one point.
A spherical version of the later statement 
was used by Dmitri Panov and me 
in \cite{panov-petrunin-ramification}.

The 3-dimensional analog of this statement does not hold.
Namely, there is a smooth embedding $\mathbb{S}^2\hookrightarrow \RR^3$ 
with all the principle curvatures between $-1$ and $1$
such that it does not surround a ball of radius 1.
Such example can be obtained by fattening a nontrivial contractible 2-complex in $\RR^3$ 
\cite[the Bing's house constructed in][will do the job]{bing}.
This problem is discussed by Vladimir Lagunov in \cite{lagunov-2} 
and it was generalized to Riemannian manifolds with boundary by Stephanie Alexander and Richard Bishop \cite[see][]{alexander-bishop}.

A similar argument shows that
for any point $p\in (\mathbb S^2,g)$ there is a minimizing geodesic $[pq]$ with conjugate ends.
On the other hand, 
for $(\mathbb S^3,g)$ this is not true.
Related examples discussed after the hint for ``Almost flat manifold'', page \pageref{page-sol:almost-flat}.



 


%%%%%%%%%%%%%%%%%%%%%%%%%%%%%%%%%%%%%%%%%%%%%%%%%%
\parbf{A spring in a tin.}
Denote by $\ell$ the length of $\alpha$.

Equip the plane with complex coordinates so that $0$ is the center of the unit disc.
We can assume that $\alpha$ equipped with $\ell$-periodic parametrization by length.

Consider the curve $\beta(t)=t-\tfrac{\alpha(t)}{\dot\alpha(t)}$.
Note that 
\[\beta(t+\ell)=\beta(t)+\ell\] 
for any $t$.
In particular 
\[\length (\beta|_{[0,\ell]}) 
\ge 
|\beta(\ell)-\beta(0)|
=
\ell.\]

Note that 
\begin{align*}
|\dot\beta(t)|&=|\tfrac{\alpha(t)\cdot\ddot\alpha(t)}{\dot\alpha(t)^2}|\le
\\
&\le|\ddot\alpha(t)|.
\end{align*}
Since $|\ddot\alpha(t)|$ is the curvature of $\alpha$ at $t$,
we get the result.\qeds

The statement was originally proved 
by Istv\'an F\'ary in \cite{fary};
number of different proofs are discussed by Serge Tabachnikov in \cite{tabachnikov}, see also \cite[19.5 in ][]{fuchs-tabachnikov}.

If instead of a disc, 
we have a region bounded by closed convex curve $\gamma$, 
then it is still true that the average curvature of $\alpha$ is at least as big as average curvature of $\gamma$. 
The proof was given by Jeffrey Lagarias
and Thomas Richardson in \cite{lagarias-richardson}, see also \cite{nazarov-petrov}.


%%%%%%%%%%%%%%%%%%%%%%%%%%%%%%%%%%%%%%%%%%%%%%%%%%
\parbf{A curve in a sphere.} 
Let us present two solutions, both by contradiction.
We assume that $\alpha$ is a closed curve in $\mathbb{S}^2$ of length $2\cdot\ell$ which intersects each equator.

\parit{A solution with the Crofton formula.}
Note that we can assume that $\alpha$ is a broken line.

Given a unit vector $u$ denote by $e_u$ the equator with pole at $u$.
Let $k(u)$ the number of intersections
of the $\alpha$ and $e_u$.

Note that for almost all $u\in \mathbb{S}^2$, the value $k(u)$ is even.
Since each equator intersects $\alpha$, we get $k(u)\ge 2$ for almost all $u$.

Then we get
\begin{align*}
2\cdot\ell&=\tfrac14\cdot\int\limits_{\mathbb{S}^2}k(u)\cdot\, d_u\area\ge 
\\
&\ge\tfrac12\cdot\area\mathbb{S}^2=
\\
&=2\cdot\pi.
\end{align*}
The first identity above is called \index{Crofton formula}\emph{Crofton formula}.
(Prove this formula first for a curve formed by one geodesic segment,
summing up we get it for broken lines
and by approximation it holds for all curves.)
\qeds

\parit{A solution by symmetry.}
Let $\check\alpha$ be a sub-arc of $\alpha$ of length $\ell$, with endpoints $p$ and $q$.  
Let $z$ be the midpoint of a minimizing geodesic $[pq]$ in $\mathbb{S}^2$.  

Let $r$ be a point of intersection of $\alpha$ with the equator with pole at $z$.  
Without loss of generality we may assume that $r\in\check\alpha$. 

The arc $\check\alpha$ together with its reflection in the point $z$ 
form a closed curve of length $2\cdot \ell$ 
that passes through $r$ and its antipodal point $r^{*}$.
Therefore 
\[\ell=\length \check\alpha\ge |r-r^{*}|_{\mathbb S^2}=\pi.\]
Here $|r-r^{*}|_{\mathbb S^2}$ 
denotes the angle metric in the sphere $\mathbb S^2$.\qeds


The problem was suggested by Nikolai Nadirashvili.

%%%%%%%%%%%%%%%%%%%%%%%%%%%%%%%%%%%%%%%%%%%%%%%%%%
%%%%%%%%%%%%%%%%%%%%%%%%%%%%%%%%%%%%%%%%%%%%%%%%%%
\begin{wrapfigure}{r}{51 mm}
\begin{lpic}[t(-3 mm),b(-3 mm),r(0 mm),l(0 mm)]{pics/tangent-eq-sol(1)}
\lbl[tr]{20,5;$u$}
\lbl[rt]{18,15;$r$}
\lbl[rt]{26,6;$l$}
\lbl[bl]{25,10;$x_u$}
\end{lpic}
\end{wrapfigure}

\parbf{Oval in oval.}
Show that the chord which minimizes (or maximizes) the ratio, 
in which it divides the bigger oval, 
solves the problem.\qeds


\parit{Alternative solution.}
Given a unit vector $u$, denote by $x_u$ the point on the inner curve
with outer normal vector $u$.
Draw a chord of outer curve which is tangent to the inner curve at $x_u$;
denote by $r=r(u)$ and $l=l(u)$ the lengths of this chord at the right and left from $x_u$.


Arguing by contradiction, assume $r(u)\ne l(u)$ for any $u\in\mathbb{S}^1$.
Since the functions $r$ and $l$ are continuous,
we can assume that 
$$r(u)>l(u)\ \ \text{for any}\ \ u\in\mathbb{S}^1.\leqno{({*})}$$

Prove that
each of the following two integrals 
\begin{align*}
\tfrac12\cdot\int\limits_{\mathbb{S}^1}r^2(u)\cdot du
\quad\text{and}\quad
\tfrac12\cdot\int\limits_{\mathbb{S}^1}l^2(u)\cdot du
\end{align*}
gives 
the area between the curves.
In particular, 
the integrals are equal to each other. 
The latter contradicts $({*})$.\qeds



This is a problem of Serge Tabachnikov \cite[see][]{tabachnikob-mi}.
A closely related, so called {}\emph{equal tangents problem} is discussed by the same author in \cite{tabachnikov-tan}.

%%%%%%%%%%%%%%%%%%%%%%%%%%%%%%%%%%%%%%%%%%%%%%%%%%
\parbf{Capture a sphere in a knot.}
We can assume that the knot is given by a diagram on the sphere.

Fix a M\"obius transformation $\mathbb{S}^2\to\mathbb{S}^2$ which is not an isometry.
Denote by $u$ its conformal factor. 
Since the M\"obius transformation preserves total area, 
we get 
$$\frac1{\area \mathbb{S}^2}\cdot\int\limits_{\mathbb{S}^2} u^2=1.$$ 
Therefore, 
$$\frac1{\area \mathbb{S}^2}\cdot\int\limits_{\mathbb{S}^2} u<1.$$ 
It follows that after a suitable rotation of $\mathbb{S}^2$, 
the length of the knot decreases.

The same way one gets 
a continuous one parameter family of M\"obius transformations which moves the knot in a hemisphere 
and allows the ball to escape. \qeds


This is a problem of Zarathustra Brady, 
the given solution is based on the idea of David Eppstein \cite[see][]{zeb}.



%%%%%%%%%%%%%%%%%%%%%%%%%%%%%%%%%%%%%%%%%%%%%%%%%%
\parbf{Linked circles.} 
Denote the linked circles by $\alpha$ and $\beta$. %???+PIC

Fix a point $x\in\alpha$. 
Note that one can find another point $y\in\alpha$ such that the interval 
$[xy]$ intersects $\beta$, say at the point $z$. 
Otherwise we can move each point of $\alpha$ along the line segment to $x$.
This deformation of $\alpha$ will not cross $\beta$;
the latter contradicts that $\alpha$ and $\beta$ are linked. 

\begin{wrapfigure}{o}{54 mm}
\begin{lpic}[t(-0 mm),b(-0 mm),r(0 mm),l(0 mm)]{pics/gehring(1)}
\lbl[br]{4,31;$x$}
\lbl[l]{37.3,12;$y$}
\lbl[rw]{29,19;$z$}
\lbl[bl]{12.2,3.2;$\alpha$}
\lbl[br]{49,11;$\beta$}
\lbl[bl]{18,11;$\alpha^{*}$}
\end{lpic}
\end{wrapfigure}

Consider the curve $\alpha^{*}$ which is the central projection of $\alpha$ 
from $z$ onto the unit sphere around $z$.
Clearly
$$\length \alpha\ge \length\alpha^{*}.$$

Note that $\alpha^{*}$ passes through two antipodal points of the sphere,
one corresponds to $x$ and the other to $y$.
Therefore 
$$\length \alpha^{*}\ge 2\cdot\pi.$$
Hence the result follows.\qeds


This is the simplest case of so called \index{Gehring's problem}\emph{Gehring's problem}. 
The solution above was given by Michael Edelstein and Binyamin Schwarz in \cite{edelstein-schwatz};
later the same solution was rediscovered few times.





%%%%%%%%%%%%%%%%%%%%%%%%%%%%%%%%%%%%%%%%%%%%%%%%%%
\parbf{Surrounded area.}
Let $C_1$ and $C_2$ be the compact regions bounded by $\gamma_1$ and $\gamma_2$ correspondingly.

By Kirszbraun theorem, 
any 1-Lipschitz map $X\to \RR^2$ defined on $X\subset \RR^2$
can be extended to a 1-Lipschitz map on the whole $\RR^2$.
In particular, there is a 1-Lipschitz map $f\:\RR^2\to\RR^2$ 
such that $f(\gamma_2(v))=f(\gamma_1(v))$ for any $v\in\mathbb S^1$.

Note that $f(C_2)\supset C_1$.
Whence the statement follows.\qeds


The Kirszbraun theorem appears in his thesis \cite[see][]{kirszbraun}
and rediscovered later by Frederick Valentine in \cite{valentine}.
An interesting survey is given by 
Ludwig Danzer, Branko Gr{\"u}nbaum  and Victor Klee
in \cite{danzer-grunbaum-klee}.


%%%%%%%%%%%%%%%%%%%%%%%%%%%%%%%%%%%%%%%%%%%%%%%%%%
\begin{wrapfigure}{r}{40 mm}
\begin{lpic}[t(-0 mm),b(-0 mm),r(0 mm),l(0 mm)]{pics/serpinski-cemetery(1)}
\end{lpic}
\end{wrapfigure}

%%%%%%%%%%%%%%%%%%%%%%%%%%%%%%%%%%%%%%%%%%%%%%%%%%
\parbf{Fat curve.}
Modify your favorite space filling curve 
keeping its area nearly the same 
and removing the self-intersections.

It works for the \index{Sierpi\'nski curve}\emph{Sierpi\'nski curve} 
which can be constructed as a limit of 
recursively defined sequence of curves;
the 8th iteration is on the diagram.\qeds 

It seems that the existence of such curves was first observed 
by William Osgood \cite[see][]{osgood}.


%%%%%%%%%%%%%%%%%%%%%%%%%%%%%%%%%%%%%%%%%%%%%%%%%%
\parbf{Crooked circle.}
A continuous function $f\:[0,1]\to [0,1]$
will be called $\eps$-crooked 
if $f(0)=0$, $f(1)=1$ 
and for any segment $[a,b]\subset [0,1]$ 
one can choose $a\le x\le y\le b$ 
such that
\[|f(y)-f(a)|\le\eps\ \ \text{and}\ \ |f(x)-f(b)|\le\eps.\]

A sequence of $\tfrac1n$-crooked maps can be constructed recursively. 
Guess the construction from the diagram.



\begin{center}
\begin{lpic}[t(-0 mm),b(4 mm),r(0 mm),l(0 mm)]{pics/crooked(1)}
\lbl[t]{8,0;$\eps=\tfrac12$}
\lbl[t]{33,0;$\eps=\tfrac13$}
\lbl[t]{57,0;$\eps=\tfrac14$}
\lbl[t]{80,0;$\eps=\tfrac15$}
\end{lpic}
\end{center}


Now, start with the unit circle, 
$\gamma_0(t)=(\cos \tfrac{t}{2\cdot\pi},\sin \tfrac{t}{2\cdot\pi})$.
Fix a sequence of positive numbers $\eps_n$ which converges to zero very fast. 
Construct recursively a sequence of simple closed curves $\gamma_n\:[0,1]\to\RR^2$.
Such that $\gamma_{n+1}$ runs outside of the disc bounded by $\gamma_n$
and 
\[|\gamma_{n+1}(t)-\gamma_n\circ f_n(t)|<\eps_n,\]
for some $\eps_n$-crooked function $f_n\:[0,1]\to[0,1]$.

\begin{wrapfigure}{o}{46 mm}
\begin{lpic}[t(-2 mm),b(0 mm),r(0 mm),l(0 mm)]{pics/circrooked(1)}
\lbl[tr]{35,33;$\gamma_0$}
\lbl[bl]{38,36;$\gamma_1$}
\end{lpic}
\end{wrapfigure}

On the diagram you see an attempt to draw the first iteration.

Denote by $D$ the union of all discs bounded by $\gamma_n$.
Show that $D$ is homeomorphic to an open disc 
and for the right choice of the $\eps_n$ the set $D$ is bounded and its
boundary contains no simple curves.\qeds

Our proof repeats the construction of pseudo-arc 
given by Bronis\l{}aw Knaster in \cite{knaster}.
This construction is ubiquitous;
here are few problems which use it:
\begin{itemize}
\item {\it Construct three disjoint non-empty open sets in $\RR$ which have the same boundary.}
\item {\it Construct three open discs in $\RR^2$ which have the same boundary.}
(These discs are called \index{lakes of Wada}\emph{lakes of Wada}; it is  described by Kuniz\^{o} Yoneyama in \cite{yoneyama}.)
\item {\it Construct a Cantor set in $\RR^3$ with non simply connected complement.}
(This example was is called  \index{Antoine's necklace}\emph{Antoine's necklace};
it is constructed in \cite{antoine}.)
\item {\it Construct an open set in $\RR^3$ with fundamental group isomorphic to the additive group of rational numbers.}
\end{itemize}
More advanced examples include
\emph{Whitehead manifold}, 
\emph{Dogbone space}, 
\emph{Casson handle};
see also the problem ``Conic neighborhood'' on page \pageref{Conic neighborhood}.





%%%%%%%%%%%%%%%%%%%%%%%%%%%%%%%%%%%%%%%%%%%%%%%%%%
\parbf{Rectifiable curve.}
The 1-dimensional Hausdorff measure will be denoted as $\mathcal{H}_1$. 

Set $L=\mathcal{H}_1(K)$.
Without loss of generality, we may assume that $K$ has diameter $1$.

Prove that 
\[\mathcal{H}_1(B(x,\eps)\cap K)\ge\eps\leqno(*)\]
for any $x\in K$ and $0<\eps<\tfrac12$.

Let $x_1,\dots, x_n$ be a maximal set of points in $K$ such that 
\[|x_i-x_j|\z\ge\eps\] for all $i\ne j$. 
From $(*)$ we have $n\le2\cdot L/\eps$.

Construct a curve $\gamma_\eps$ such that (1) $\gamma_\eps$ is passing through all $x_i$, (2) $\length\gamma_\eps\le10\cdot L$ and (3) $\gamma_\eps$ lies in $\eps$-neighborhood of $K$.
We can assume that $\gamma_\eps$ is parametrized by length.

The needed curve can be obtained as 
a partial limit of $\gamma_\eps$. \qeds


This is a problem of Kenneth Falconer
\cite[see Exercise 3.5 in][]{falconer}.



%%%%%%%%%%%%%%%%%%%%%%%%%%%%%%%%%%%%%%%%%%%%%%%%%%
\parbf{Typical convex functions.}
Denote by $\mathfrak{F}$ the space of all convex 1-Lipschitz functions defined on $[0,1]$ with the sup-norm.
Note that $\mathfrak{F}$ is a complete metric space.

Prove that if a function $f\in\mathfrak{F}$ has nonzero derivative at some point then there is $\eps>0$ such that
\[|f-g|> \frac{\eps}{n^{23}}\]
if $g\in\mathfrak{F}$  can be presented as a maximum of $n$ linear functions.

Note that there is a countable set of piecewise linear convex functions $g_1,g_2,\dots$ which is everywhere dense in $\mathfrak{F}$.
Denote by $n_i$ the number of linear intervals of $g_i$.
For any positive integer $k$,
consider the set $\Omega_k\subset\mathfrak{F}$ defined as 
\[\Omega_k
=
\set{|f-g_i|<\frac1{k\cdot n_i^{23}}}{\text{for some}\ i}.\]

From above we get that if the second derivative of $f$ does not vanishing at some point then $f\notin\Omega_k$ for large $k$. 

Note that $\Omega_k$ is open and everywhere dense in $\mathfrak{F}$.
Therefore 
\[\Omega=\bigcap_k\Omega_k\]
is a G-delta dense set.
Hence the statement follows.\qeds
  


This problem states that typical convex functions is very far from what we used to work with.

This is a typical answer for such question --- typically we do not see the typical objects.
For example, according to the result of 
Bernd Kirchheim, 
Emanuele Spadaro  
and 
L{\'a}szl{\'o} Sz{\'e}kelyhidi proved in \cite{KSS}
a typical 1-Lipschitz maps from the plane to itself preserves the length of all curves.
The same way one could show that the typical open discs in the plane contain no simple curves in their boundary, 
although the construction of one such is not trivial, 
see ``Crooked circle'' above.

More problems of that type are surveyed by Tudor Zamfirescu in \cite{zamfirescu}.







