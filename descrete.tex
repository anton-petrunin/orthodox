\csname @openrightfalse\endcsname
\chapter{Discrete geometry}

In this chapter we consider geometrical problems with strong combinatoric flavor.
No special prerequisite is needed.

%%%%%%%%%%%%%%%%%%%%%%%%%%%%%%%%%%%%%%%%%
\subsection*{Round circles in 3-sphere}\label{Round circles}

\begin{pr}
Suppose that $\mathcal{C}$ a finite collection of pairwise linked round circles in the unit 3-sphere. 
Prove that there is an isotopy of $\mathcal{C}$
which moves all of them into great circles.
\end{pr}


\parit{Semisolution.}
For each circle in $\mathcal{C}$ consider the plane containing it.
Note that the circles are linked 
if and only if 
the corresponding planes intersect at a single point inside the unit sphere $\mathbb{S}^3\subset \RR^4$.

Take the intersection of the planes with the sphere of radius $R\ge 1$,
rescale and pass to the limit as $R\to\infty$.  
This way we get needed isotopy.\qeds

The problem was discussed 
by Genevieve Walsh \cite[see][]{walsh}.
The same idea was used by Michael Freedman and  Richard Skora to show that any link made from round circles which are not linked pairwise, then  it is the trivial link \cite[see Lemma 3.2 in ][]{freedman-skora}.
In particular, Borromean rings can not be realized by round circles.

%%%%%%%%%%%%%%%%%%%%%%%%%%%%%%%%%%%%%%%%%
\subsection*{Box in a box}\label{box-in-box}

\begin{pr}
Assume that a rectangular parallelepiped with sizes $a,b,c$ 
lies inside another rectangular parallelepiped with sizes $a',b',c'$. 
Show that 
\[a'+b'+c'\ge a+b+c.\]

\end{pr}

%%%%%%%%%%%%%%%%%%%%%%%%%%%%%%%%%%%%%%%%%
\subsection*{Harnack's circles}\label{Harnack}

\begin{pr}
Prove that a smooth algebraic curve of degree $d$ in $\RP^2$ consists of at most $n=\tfrac12\cdot(d^2-3\cdot d+4)$ connected components.
\end{pr}

%%%%%%%%%%%%%%%%%%%%%%%%%%%%%%%%%%%%%%%%%
\subsection*{Two points on each line}\label{2pts-on-line}

\begin{pr}
Construct a set in the Euclidean plane, 
which intersects each line at exactly 2 points. 
\end{pr}

%%%%%%%%%%%%%%%%%%%%%%%%%%%%%%%%%%%%%%%%%
\subsection*{Balls without gaps}
\label{Balls without gaps}

\begin{pr}
Let $B_1,\dots,B_n$ be the balls  
of radiuses $r_1,\dots,r_n$ 
in a Euclidean space.
Assume that no hyperplane divides the balls into two
non-empty sets without intersecting at least one of the balls. 
Show that the balls
$B_1,\dots,B_n$ can be covered by a ball of radius
$r=r_1+\dots+r_n$.

\end{pr}

\subsection*{Covering lemma}

\begin{pr}
Let $\{B_i\}_{i\in F}$ be any finite collection of  balls in $m$-dimensional Euclidean space. 
Show that there is a subcollection of pairwise disjoint balls $\{B_i\}_{i\in G}$, $G\subset F$
such that
\[\vol \left(\bigcup_{i\in F}B_i\right)\le 3^m\cdot \vol \left(\bigcup_{i\in G}B_i\right).\]
\end{pr}



{

%%%%%%%%%%%%%%%%%%%%%%%%%%%%%%%%%%%%%%%%%
\begin{wrapfigure}[7]{r}{27 mm}
\begin{lpic}[t(-0 mm),b(-4 mm),r(0 mm),l(0 mm)]{pics/kiss(1)}
\lbl[]{12.5,11;{\color{white} $W_0$}}
\lbl[]{8.5,18;$W_1$}
\lbl[]{16.5,18;$W_2$}
\lbl[]{20.5,11;$W_3$}
\lbl[]{16.5,4;$W_4$}
\lbl[]{8.5,4;$W_5$}
\lbl[]{4.5,11;$W_6$}
\end{lpic}
\end{wrapfigure}

\subsection*{Kissing number\easy}\label{pr:Kissing number}


Let  $W_0$ be a convex body in $\RR^m$.
We say that $k$ is the \index{kissing number}\emph{kissing number} of $W_0$ (briefly $k=\mathop{\rm kiss}W_0$)
if $k$ is the maximal integer such that there are $k$ bodies $W_1,\dots,W_k$ such that 
(1) each $W_i$ is congruent to $W_0$,
(2) $W_i\cap W_0\not=\emptyset$ for each $i$ 
and (3) no pair $W_i,W_j$ has common interior points.

}

As you may guess from the diagram, the kissing number of round disc in the plane is $6$.

\begin{pr}
Show that for any convex body $W_0$ in $\RR^m$
$$\mathop{\rm kiss}W_0\ge \mathop{\rm kiss}B,$$
where $B$ denotes the unit ball in $\RR^m$.
\end{pr}

%%%%%%%%%%%%%%%%%%%%%%%%%%%%%%%%%%%%%%%%%
\subsection*{Monotonic homotopy}
\label{mono-homotopy}

\begin{pr}
Let $F$ be a finite set and $h_0,h_1\: F\to\RR^m$ be two maps.
Consider $\RR^m$ as a subspace of $\RR^{2\cdot m}$.
Show that there is a homotopy  $h_t\:F\z\to\RR^{2\cdot m}$ from $h_0$ to $h_1$ such that  the function 
\[t\z\mapsto |h_t(x)-h_t(y)|\] 
is monotonic for any pair $x,y\in F$.
\end{pr}

%%%%%%%%%%%%%%%%%%%%%%%%%%%%%%%%%%%%%%%%%
\subsection*{Cube}\label{Cube}

\begin{pr}
Half of the vertices 
of an $m$-dimensional cube
are colored in white and the other half in black.
Show that the cube has at least $2^{m-1}$ edges which connect the vertices of different colors. 
\end{pr}

%%%%%%%%%%%%%%%%%%%%%%%%%%%%%%%%%%%%%%%%%
\subsection*{Geodesic loop}\label{Geodesic loop}

\begin{pr}
Show that the surface of cube in $\RR^3$
does not admit a geodesic loop with the base point at a vertex.
\end{pr}

%%%%%%%%%%%%%%%%%%%%%%%%%%%%%%%%%%%%%%%%%
\subsection*{Right and acute triangles}\label{Right and acute triangles}

\begin{pr}
Let $x_1,\dots,x_n\in\RR^m$
be a collection of points such that any triangle $[x_ix_jx_k]$ is right or acute.
Show that $n\le 2^m$.
\end{pr}



%%%%%%%%%%%%%%%%%%%%%%%%%%%%%%%%%%%%%%%%%
\subsection*{Right-angled polyhedron\thm}\label{Right-angled polyhedron}

A polyhedron is called {}\emph{right-angled} if all its dihedral angles are right.

\begin{pr}
Show that in all sufficiently large dimensions, there is no compact convex hyperbolic right-angled polyhedron. 
\end{pr}

Let us give a short summary of Dehn--Sommerville equations 
which can help to solve this problem.

Assume $P$ is a \index{simple polyhedron}\emph{simple} Euclidean $m$-dimensional polyhedron;
that is, every vertex of $P$ exactly $m$ facets are meeting.
Denote by $f_k$ the number of $k$-dimensional faces of $P$;
the array of integers $(f_0,\dots f_m)$ is called $f$-vector of $P$.

Fix an order of the vertices $v_1,\dots, v_{f_0}$
of $P$ so that for some linear function $\ell$, we have $\ell(v_i)<\ell(v_j)\ \Leftrightarrow\ i<j$.
The \index{index of vertex}\emph{index} of the vertex $v_i$ 
is defined as the number of edges $[v_iv_j]$ of $P$ such that $i<j$. 
The number of vertices of given index $k$ will be denoted as $h_k$.
The array of integers $(h_0,\dots h_m)$ is called $h$-vector of $P$.
Clearly $h_0=h_m=1$ and 
\[h_k\ge 0\quad\text{for all}\quad k.\leqno({*})\]

Each $k$-face of $P$ contains unique vertex which maximize $\ell$;
if the vertex has index $i$,
then $i\ge k$ and
then it is the maximal vertex for exactly $\tfrac{i!}{k!\cdot (i-k)!}$
faces of dimension $k$.
This observation can be packed in the following polynomial identity 
\[\sum_k h_k\cdot (t+1)^k=\sum_k f_k\cdot t^k.\]

Note that the identity above implies that $h$-vector does not depend on the choice of $\ell$.
In particular, the $h$ vector is the same for the reversed order;
that is,
\[h_k=h_{m-k}\leqno({*}{*})\]
for any $k$.

The identities $({*}{*})$ are called Dehn--Sommerville equations.
It gives the complete list of linear equations for $h$-vectors (and therefore $f$-vectors) of simple polyhedrons.

Note that the Dehn--Sommerville equations $({*}{*})$ 
as well as the inequalities $({*})$ can be rewritten in terms of 
$f$-vectors.



\section*{Semisolutions}
%%%%%%%%%%%%%%%%%%%%%%%%%%%%%%%%%%%%%%%%%%%%%%%%%%


\parbf{Box in a box.} 
Let $\Pi$ be a parallelepiped
with dimensions $a$, $b$ and $c$.
Denote by $v(r)$ the volume of  $r$-neighborhoods of $\Pi$,
 
Note that for all positive $r$ we have
\[v_{\Pi}(r)=w_3(\Pi)+w_2(\Pi)\cdot r+w_1(\Pi)\cdot r^2+w_0(\Pi)\cdot r^3,\leqno({*})\]
where 
\begin{itemize}
\item $w_0(\Pi)=\tfrac43\cdot \pi$ is the volume of unit ball,
\item $w_1(\Pi)=\pi\cdot (a+b+c)$,
\item $w_2(\Pi)=2\cdot(a\cdot b+b\cdot c+c\cdot a)$ is the surface area of $\Pi$,
\item $w_3(\Pi)=a\cdot b\cdot c$ is the volume of $\Pi$,
\end{itemize}

Assume $\Pi'$ be an other parallelepiped
with dimensions $a'$, $b'$ and~$c'$.
If $\Pi\subset \Pi'$,
then $v(r)_{\Pi}\le v_{\Pi'}(r)$ for any $r$.
For $r\to\infty$, these inequalities imply
\[a+b+c\le a'+b'+c'.\qedsin\]

\parit{Alternative proof.}
Note that the average length of projection of $\Pi$ to a line is
$\Const\cdot(a+b+c)$ for some $\Const>0$.
(In fact $\Const=\tfrac12$, but we will not need it.)

Since $\Pi\subset \Pi'$,
the average length of projection of $\Pi$
can not exceed the average length of projection of $\Pi'$.
Hence the statement follows.
\qeds

The problem was discussed by Alexander Shen \cite[see][]{shen}.

A formula analogous to $({*})$
holds for arbitrary convex body $B$ in arbitrary dimension $m$.
It was discovered by Jakob Steiner \cite[see][]{steiner}.
The coefficient $w_i(B)$ in the polynomial with different normalization constants 
appear under different names most commonly
\emph{intrinsic volumes} and
\emph{quermassintegrals}.
Up to a normalization constant
they also can be defined as the average 
of area of projections of $B$ to the $i$-dimensional planes.
In particular, 
if $B'$ and $B$ are convex bodies such that $B'\subset B$,
then $w_i(B')\le w_i(B)$ for any $i$.
This generalize our problem quite a bit.
Further generalizations lead to so called \index{mixed volumes}\emph{mixed volumes} \cite[see][]{burago-zalgaller}.

The equality $w_1(\Pi)=\pi\cdot (a+b+c)$ still holds for all parallelepipeds, not only rectangular ones.
In particular, if one parallelepiped 
lies inside another then sum of all edges of the first one cannot exceed the sum for the second.


%%%%%%%%%%%%%%%%%%%%%%%%%%%%%%%%%%%%%%%%%%%%%%%%%%
\parbf{Harnack's circles.}
Let $\sigma\subset \RP^2$ be a algebraic curve of degree~$d$.
Consider the complexification $\Sigma\subset \CP^2$ of~$\sigma$.
Without loss of generality, we may assume that $\Sigma$ is regular.

Note that all regular complex algebraic curves of degree $d$ in $\CP^2$
are isotopic to each other in the class of regular algebraic curves of degree $d$.
Indeed, the set equation of degree $d$ which correspond to singular curves have the real codimesion 2.
Therefore the set of equation of degree $d$ which correspond to regular curves is connected.
In particular one can construct isotopy from one regular curve to an other by changing continuously the parameters of their equations.

In particular it follows that all regular complex algebraic curves of degree $d$ in $\CP^2$ have the same genus,
denote it by $g$.
Perturbing a singular curve formed by $d$ lines in $\CP^2$,
we can see that 
\[g=\tfrac12\cdot(d-1)\cdot(d-2).\]

The real curve $\sigma$ forms the fixed point set in $\Sigma$ 
by the complex conjugation. 
In particular $\sigma$ divides $\Sigma$ into two symmetric surfaces with boundary formed by $\sigma$.
It follows that each connected component of $\sigma$ adds one to the genus of $\Sigma$.
Hence the result follows.
\qeds
 
The inequality was originally proved 
by Axel Harnack using a different method \cite[see][]{harnack}.
The idea to use complexification is due to Felix Klein \cite[see][]{klein}.
This problem is a background for the Hilbert's 16th problem. 

%%%%%%%%%%%%%%%%%%%%%%%%%%%%%%%%%%%%%%%%%%%%%%%%%%
\parbf{Two points on each line.}
Take any complete ordering of the set of all lines 
so that each beginning interval has cardinality less than continuum.

Assume we have a set of points $X$ of cardinality less than continuum such that each line intersects $X$ at most $2$ points and cardinality of $X$ is less than continuum.

Choose the least line $\ell$ in the ordering which intersect $X$ 
by $0$ or $1$ point.
Note that the set of all lines intersecting $X$ at two points has cardinality less than continuum.
Therefore we can choose a point on $\ell$ and add it to $X$ so that the remaining lines are not overloaded.

It remains to apply well ordering principle.
\qeds

This problem has endless list of variations.
The following problem look similar but far more involved;
a solution follows from the proof of Paul Monsky that a square cannot be cut into triangles with equal areas \cite[see][]{monsky}.

\begin{pr}
Subdivide the plane into three everywhere dense sets $A$, $B$ and $C$ such that each line meets exactly two of these sets.
\end{pr}


%%%%%%%%%%%%%%%%%%%%%%%%%%%%%%%%%%%%%%%%%%%%%%%%%%
\parbf{Balls without gaps.} 
Assume that each ball has the mass proportional to its radius.
Denote by $z$  the center of mass of the balls.
It is sufficient to show the following.
\begin{cl}{$({*})$}
The ball $B(z,r)$ contains all $B_1,\dots,B_n$.
\end{cl}

Assume this is not the case.
Then there is a line $\ell$ thru $z$, 
such that the orthogonal projection of some ball $B_i$ to $\ell$ 
does not lie in the projection of $B$ completely.
(This projection reduces the problem to one-dimensional case.)

Note that the projection of all balls $B_1,\dots,B_n$ has to be connected and it contain a line segment longer than $r$ on one side from $z$. 
In this case, the center of mass of balls projects inside of this segment, a contradiction.
\qeds

The statement was conjectured by Paul Erd\H{o}s.
The solution is given by Adolph and Ruth Goodmans
[see \ncite{goodman-goodman} and also \ncite{hadwiger}].

%%%%%%%%%%%%%%%%%%%%%%%%%%%%%%%%%%%%%%%%%%%%%%%%%%

\parbf{Covering lemma.} 
The required collection $\{B_i\}_{i\in G}$ is constructed using the \emph{greedy algorithm}. 
We choose the balls one by one;
on each step we take the largest ball which does not intersect those which we choose already.

\medskip

Note that each ball in the original collection $\{B_i\}_{i\in F}$ intersects a ball in $\{B_i\}_{i\in G}$ with larger radius.
Therefore 
\[\bigcup_{i\in F}B_i\subset \bigcup_{i\in G}3\cdot B_i,\leqno({*})\]
where $3\cdot B_i$ denotes the ball with the came center as $B_i$ and with three times larger radius.
Hence the statement follows.
\qeds



The constant $3^n$ is not optimal.
The optimal constant is at least $2^n$, but its value is not known and maybe no one is willing to know.

The inclusion $({*})$ is called \emph{Vitali covering lemma}.
The following statement is called \emph{Besikovitch covering lemma};
it has a similar proof.

\begin{pr}
For any positive integer $m$ there is a positive integer $M$ such that 
any finite collection of balls $\{B_i\}_{i\in F}$ in the $m$-dimensional Euclidean space 
contains a subcollection $\{B_i\}_{i\in G}$
such that center of any ball from $\{B_i\}_{i\in F}$ lies in one of the balls from $\{B_i\}_{i\in G}$
and the collection $\{B_i\}_{i\in G}$ can be subdivided into $M$ subcollections of pairwise disjoint balls.
\end{pr}

Both lemmas used to prove so called \emph{covering theorems} in measure theory,
which state that ``undesirable sets'' have vanishing measure.
Their applications overlap but not identical, \emph{Vitali covering theorem} works for nice measures in arbitrary metric spaces while \emph{Besikovitch covering theorem} work in nice metric spaces for arbitrary Borel measures.

More precisely, Vitali works in arbitrary metric space for so called \index{doubling measure}\emph{doubling measure} $\mu$;
which means that 
\[\mu B(x,2\cdot r)\le C\cdot \mu B(x,r)\] 
for some fixed constant $C$ and any ball $B(x,r)$ in the metric space.
On the other hand, Besikovitch works for all Borel measures in the so called \emph{directionally limited} metric spaces \cite[see 2.8.9 in][]{federer};
these include Alexandrov spaces with curvature bounded below.




%%%%%%%%%%%%%%%%%%%%%%%%%%%%%%%%%%%%%%%%%%%%%%%%%%

\begin{wrapfigure}{o}{42 mm}
\begin{lpic}[t(-0 mm),b(0 mm),r(0 mm),l(0 mm)]{pics/kiss-sol(1)}
\lbl[]{21,5;{\color{white} $W$}}
\lbl[t]{28,19;$W_j$}
\lbl[t]{10,19;$W_i$}
\lbl[t]{40,3;$\Pi_j$}
\lbl[tl]{3,1;$\Pi_i$}
\end{lpic}
\end{wrapfigure}

\parbf{Kissing number.}
Fix the dimension $m$.
Set $n=\mathop{\rm kiss} B$.
Let $B_1,\dots, B_n$ the copies of the ball $B$  
which touch $B$ and have no common interior points.
For each $B_i$ consider the vector $v_i$ from the center of $B$ to the center of $B_i$.
Note that $\measuredangle(v_i,v_j)\ge \tfrac\pi3$ if $i\ne j$.

For each $i$,
consider supporting hyperplane $\Pi_i$
to $W$
with outer normal vector $v_i$.
Denote by $W_i$ the reflection of $W$ in $\Pi_i$.

Note that $W_i$ and $W_j$ have no common interior points if $i\ne j$;
the latter gives the needed inequality.
\qeds

The proof is given by 
Charles Halberg, 
Eugene Levin 
and Ernst Straus 
\cite[see][]{halberg-levin-straus}.
It is not known if the same inequality holds for the orientation-preserving version of kissing number.



%%%%%%%%%%%%%%%%%%%%%%%%%%%%%%%%%%%%%%%%%%%%%%%%%%
\parbf{Monotonic homotopy.}
Note that we can assume
that $h_0(F)$ and $h_1(F)$ both lie in the coordinate $m$-spaces of $\RR^{2\cdot m}=\RR^m\times \RR^m$;
that is,
$h_0(F)\z\subset\RR^m\times\{0\}$
and $h_1(F)\subset  \{0\}\times\RR^m$.

Direct calculations show that the following homotopy is monotonic
\[h_t(x)=\bigl(h_0(x)\cdot \cos\tfrac{\pi\cdot t}2
\,,\,
 h_1(x)\cdot\sin\tfrac{\pi\cdot t}{2}\bigr).\qedsin\] 
\medskip

This homotopy was discovered by Ralph Alexander \cite[see][]{ralexander}.
It has number of applications, 
one of the most beautiful is the given 
by K\'aroly Bezdek 
and Robert Connelly 
in the proof of 
Kneser--Poulsen  
and Klee--Wagon conjectures in the two-dimensional case \cite[see][]{bezdek-connelly}.

The dimension $2\cdot m$ is optimal;
that is, for any positive integer $m$,
there are two maps $h_0,h_1\:F\to \RR^m$ which cannot be connected by a monotonic homotopy $h_t\:F\to\RR^{2\cdot m-1}$.
The latter was shown by Maria Belk and Robert Connelly \cite[see][]{belk-connelly}



%%%%%%%%%%%%%%%%%%%%%%%%%%%%%%%%%%%%%%%%%%%%%%%%%%
\parbf{Cube.}
Consider the cube $[-1,1]^m\subset \RR^m$.
Any vertex this cube has the form $\bm{q}=(q_1,\dots,q_m)$,
where  $q_i=\pm1$.

For each vertex $\bm{q}$,
consider the intersection of the corresponding octant with the unit sphere;
that is, the set
\[V_{\bm{q}}=\set{(x_1,\dots,x_m)\in\mathbb{S}^{m-1}}{q_i\cdot x_i\ge 0\ \text{for each}\ i}.\]

Let $\mathcal{A}\subset\mathbb{S}^{m-1}$ be the union of all the sets $V_{\bm{q}}$ for all black $\bm{q}$.
Note that 
\[\vol_{m-1}\mathcal{A}=\tfrac12\cdot\vol_{m-1}\mathbb{S}^{m-1}.\]

By spherical isoperimetric inequality,
\[\vol_{m-2}\partial\mathcal{A}
\ge \vol_{m-2}\mathbb{S}^{m-2}.\] 

It remains to observe that
\[\vol_{m-2}\partial\mathcal{A}
=
\tfrac k{2^{m-1}}\cdot\vol_{m-2}\mathbb{S}^{m-2},\]
where $k$ is the number of edges of the cube with one end black and the other in white.
\qeds

The problem was suggested by Greg Kuperberg.

%%%%%%%%%%%%%%%%%%%%%%%%%%%%%%%%%%%%%%%%%%%%%%%%%%
\parbf{Geodesic loop.}
Assume such loop exists; denote it by $\gamma$ and let $v$ be its base vertex.

Denote by $\xi$ and $\zeta$ the directions of exit and the entrance of the loop.
Let $\alpha$ be the angle between $\xi$ and $\zeta$
measured in the tangent cone to the surface of cube at $v$.

Note that $\alpha=\tfrac\pi2$.
It can be seen from the Gauss--Bonnet formula since each vertex of the cube has curvature $\tfrac\pi2$.
Alternatively, it can be proved by the unfolding of $\gamma$ on the plane.

It follows that there is a rotational symmetry of cube with order 3 which fix $v$ and sends $\xi$ to $\zeta$.
The later leads to a contradiction.
\qeds

The same idea can be used to solve the following harder problems.

\begin{pr}
Show the same for the surface of higher dimensional cube.
\end{pr}

\begin{pr}
 Show the same for the surface tetrahedron, octahedron and icosahedron.
\end{pr}


%The same statement holds for tetrahedron, octahedron and icosahedron.
%In this case $\alpha$ is a multiple of $\tfrac\pi3$; it implies existence of rotational symmetry of which fix $v$ and sends $\xi$ to $\zeta$ and leads to a contradiction the same way.

For the dodecahedron such loop exists;
its development shown on the diagram.
The vertices of a cube inscribed in the dodecahedron are circled.

\begin{center}
\begin{lpic}[t(-0 mm),b(0 mm),r(0 mm),l(0 mm)]{pics/geodesic-loop(1)}
\end{lpic}
\end{center}

The problem suggested by Jaros{\l}aw K\k{e}dra.

%%%%%%%%%%%%%%%%%%%%%%%%%%%%%%%%%%%%%%%%%%%%%%%%%%
\parbf{Right and acute triangles.}
Denote by $K$ the convex hull of $\{x_1,\z\dots,x_n\}$.
Without loss of generality we can assume that $K$ is $m$-dimensional. 
Note that for any distinct points $x_i$ and $x_j$
and any interior point $z$ in $K$
we have 
\[\measuredangle \hinge{x_i}{x_j}z<\tfrac\pi2.\leqno({*})\]
Indeed, if $({*})$ does not hold, then $\langle x_j-x_i,z-x_i\rangle<0$.
Since $z\in K$ we have $\langle x_j-x_i,x_k-x_i\rangle<0$ for some vertex $x_k$.
That is, $\measuredangle \hinge{x_i}{x_j}{x_k}<\tfrac\pi2$, a contradiction.

Denote by $h_i$ the homothety with center at $x_i$ and coefficient $\tfrac12$.
Set $K_i=h_i(K)$.

{

\begin{wrapfigure}[10]{r}{31 mm}
\begin{lpic}[t(-4 mm),b(0 mm),r(0 mm),l(0 mm)]{pics/acute-triangles(1)}
\lbl[t]{2,-1;$x_i$}
\lbl[t]{20,-1;$x_j$}
\lbl[b]{13,9;$z$}
\lbl[t]{25,17;$z_i$}
\lbl[t]{7.4,17;$z_j$}
\lbl[b]{15,17;$K$}
\end{lpic}
\end{wrapfigure}

Let us show that $K_i$ and $K_j$ have no common interior points.
Assume contrary; 
that is, \[z=h_i(z_i)=h_j(z_j);\]
for some interior points $z_i$ and $z_j$ in $K$.
Note that 
\[
\measuredangle\hinge{x_i}{x_j}{z_i}
+
\measuredangle\hinge{x_j}{x_i}{z_j}
=
\pi,
\]
which contradicts $({*})$.

}

Note that $K_i\subset K$ for any $i$;
it follows that 
\begin{align*}
\tfrac n{2^m}\cdot \vol K
&=\sum_{i=1}^n\vol K_i\le
\\
&\le\vol K.
\end{align*}
Hence the result follows.
\qeds

The problem was posted by Paul Erd{\H{o}}s 
and solved by Ludwig Danzer and Branko Gr\"unbaum \cite[see][]{erdos,danzer-guenbaum}.

Grigori Perelman noticed that the same proof works for a similar problem for Alexandrov space \cite[see][]{perelman-Erdos};
the later led to interesting connections to the crystallographic groups \cite[see][]{lebedeva}.

The upper bound for the number of points with only acute triangles grows exponentially with $m$;
the later was shown by Paul Erd\H{o}s and Zolt\'an F\"uredi \cite[see][]{erdos-fueredi};
the proof use so called \index{probabilistic method}\emph{probabilistic method}.


%%%%%%%%%%%%%%%%%%%%%%%%%%%%%%%%%%%%%%%%%%%%%%%%%%
\parbf{Right-angled polyhedron.}
Let $P$ be a right-angled hyperbolic polyhedron of dimension $m$.
Note that $P$ is simple; 
that is, exactly $m$ facets meet at each vertex of $P$.

From the projective model of hyperbolic plane, 
one can see that for any simple compact hyperbolic polyhedron there is a simple Euclidean polyhedron with the same combinatorics. 
In particular Dehn--Sommerville equations hold for $P$.

Denote by $(f_0,\dots f_m)$ and $(h_0,\dots h_m)$ the $f$- and $h$-vectors of $P$.
Recall that $h_i\ge 0$ for any $i$ and $h_0=h_m=1$.
By Dehn--Sommerville equations, we get
\[f_2> \tfrac{m-2}4\cdot f_1.
\leqno({*})\]

Since $P$ is hyperbolic, each 2-dimensional face of $P$ has at least 5 sides.
It follows that
\[f_2\le \tfrac{m-1}5\cdot f_1.\]
The latter contradicts $({*})$ for $m\ge 6$.
\qeds
 
The proof above 
is the core of proof of nonexistence of compact hyperbolic Coxeter's polyhedra of large dimensions 
given by Ernest Vinberg \cite[see][]{vinberg, vinberg-strong}.

Playing a bit more with the same inequalities, 
one gets nonexistence of  right-angled hyperbolic polyhedra,
in all dimensions starting from $5$.
In the 4-dimensional case,
there is a regular right-angled  hyperbolic \index{120-cells}\emph{120-cells} --- a 4-dimensional uncle of dodecahedra.


