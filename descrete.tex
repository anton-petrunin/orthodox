\csname @openrightfalse\endcsname
\chapter{Discrete geometry}

In this chapter we consider geometrical problems with strong combinatoric flavor.
No special prerequisite is needed.

%%%%%%%%%%%%%%%%%%%%%%%%%%%%%%%%%%%%%%%%%
\subsection*{Round circles in $\mathbb{S}^3$}

\begin{pr}{\easy}{Round circles in $\mathbb{S}^3$}\label{Round circles}
Suppose that you have a finite collection of pairwise linked round circles in the unit 3-sphere, 
not necessarily all of the same radius. 
Prove that there is an isotopy in the space of such collections of circles 
which moves all of them into great circles.
\end{pr}


\parit{Semisolution.}
For each circle consider the containing it plane in $\RR^4$.
Note that the circles are linked 
if and only if 
the corresponding planes intersect at a single point inside $\mathbb{S}^3$.

Take the intersection of the planes with the sphere of radius $R\ge 1$,
rescale and pass to the limit as $R\to\infty$.  
This way we get needed isotopy.\qeds

The problem was discussed 
in the thesis of Genevieve Walsh \cite[see][]{walsh}.

%%%%%%%%%%%%%%%%%%%%%%%%%%%%%%%%%%%%%%%%%
\subsection*{Box in a box}

\begin{pr}{}{Box in a box}\label{box-in-box} 
Assume that a parallelepiped with sizes $a,b,c$ 
lies inside another parallelepiped with sizes $a',b',c'$. 
Show that 
\[a'+b'+c'\ge a+b+c.\]

\end{pr}

%%%%%%%%%%%%%%%%%%%%%%%%%%%%%%%%%%%%%%%%%
\subsection*{Harnack's circles}

\begin{pr}{}{Harnack's circles}\label{Harnack}
Prove that a smooth algebraic curve of degree $d$ in $\RP^2$ consists of at most $n=\tfrac12\cdot(d^2-3\cdot d+4)$ connected components.
\end{pr}

%%%%%%%%%%%%%%%%%%%%%%%%%%%%%%%%%%%%%%%%%
\subsection*{Two points on each line}

\begin{pr}{}{Two points on each line}\label{2pts-on-line}
Construct a set in the Euclidean plane, 
which intersects each line at exactly 2 points. 
\end{pr}

{

%%%%%%%%%%%%%%%%%%%%%%%%%%%%%%%%%%%%%%%%%
\begin{wrapfigure}[7]{r}{27 mm}
\begin{lpic}[t(-0 mm),b(-4 mm),r(0 mm),l(0 mm)]{pics/kiss(1)}
\lbl[]{12.5,11;{\color{white} $W_0$}}
\lbl[]{8.5,18;{\color{white} $W_1$}}
\lbl[]{16.5,18;{\color{white} $W_2$}}
\lbl[]{20.5,11;{\color{white} $W_3$}}
\lbl[]{16.5,4;{\color{white} $W_4$}}
\lbl[]{8.5,4;{\color{white} $W_5$}}
\lbl[]{4.5,11;{\color{white} $W_6$}}
\end{lpic}
\end{wrapfigure}

\subsection*{Kissing number\easy}


Let  $W_0$ be a convex body in $\RR^m$.
We say that $k$ is the \index{kissing number}\emph{kissing number} of $W_0$ (briefly $k=\mathop{\rm kiss}W_0$)
if $k$ the maximal integer such that there are $k$ bodies $W_1,W_2,\dots,W_k$ such that each $W_i$ is congruent to $W_0$,
$W_i\cap W_0\not=\emptyset$ for each $i$ 
and have no pair $W_i,W_j$ has common interior points.

}

Say, as you may guess from the diagram, the kissing number of round disc in the plane is $6$.

\begin{pr}{\easy}{Kissing number}\label{pr:Kissing number}
Show that for any convex body $W_0$ in $\RR^m$
$$\mathop{\rm kiss}W_0\ge \mathop{\rm kiss}B,$$
where $B$ denotes the unit ball in $\RR^m$.
\end{pr}

%%%%%%%%%%%%%%%%%%%%%%%%%%%%%%%%%%%%%%%%%
\subsection*{Monotonic homotopy}
\label{mono-homotopy}

\begin{pr}{}{Monotonic homotopy} 
Let $F$ be a finite set and $h_0,h_1\: F\to\RR^m$ be two maps.
Consider $\RR^m$ as a subspace of $\RR^{2\cdot m}$.
Show that there is a homotopy  $h_t\:F\z\to\RR^{2\cdot m}$ from $h_0$ to $h_1$ such that for any $x,y\in F$ the function 
\[t\z\mapsto |h_t(x)-h_t(y)|\] 
is monotonic.
\end{pr}

%%%%%%%%%%%%%%%%%%%%%%%%%%%%%%%%%%%%%%%%%
\subsection*{Cube}

\begin{pr}{}{Cube}\label{Cube}
Half of the vertices 
of an $m$-dimensional cube
are colored in white and the other half in black.
Show that the cube has at least $2^{m-1}$ edges which connect the vertices of different colors. 
\end{pr}

%%%%%%%%%%%%%%%%%%%%%%%%%%%%%%%%%%%%%%%%%
\subsection*{Geodesic loop}

\begin{pr}{}{Geodesic loop}\label{Geodesic loop}
Show that the surface of cube in $\RR^3$
does not admit a geodesic loop with the base point at a vertex.
\end{pr}

%%%%%%%%%%%%%%%%%%%%%%%%%%%%%%%%%%%%%%%%%
\subsection*{Right and acute triangles}

\begin{pr}{}{Right and acute triangles}\label{Right and acute triangles}
Let $x_1,\dots,x_n\in\RR^m$
be a collection of points such that any triangle $[x_ix_jx_k]$ is right or acute.
Show that $n\le 2^m$.
\end{pr}



%%%%%%%%%%%%%%%%%%%%%%%%%%%%%%%%%%%%%%%%%
\subsection*{Right-angled polyhedron\thm}

\begin{pr}{\thm}{Right-angled polyhedron}\label{Right-angled polyhedron}
Show that in all sufficiently large dimensions, there is no compact convex hyperbolic polyhedron with right dihedral angles. 
\end{pr}

Let us give a short summary of Dehn--Sommerville equations 
which can help you to solve this problem.

Assume $P$ is a \index{simple polyhedron}\emph{simple} Euclidean $m$-dimensional polyhedron;
that is, every vertex of $P$ exactly $m$ facets are meeting.
Denote by $f_k$ the number of $k$-dimensional faces of $P$;
the array of integers $(f_0,f_1,\dots f_m)$ is called $f$-vector of $P$.

Fix an order of the vertices $v_1,v_2,\dots v_{f_0}$
of $P$ so that for some linear function $\ell$, we have $\ell(v_i)>\ell(v_j)\ \Leftrightarrow\ i<j$.
The \index{index of vertex}\emph{index} of the vertex $v_i$ 
is defined as the number of edges $[v_iv_j]$ such that $j<i$. 
The number of vertices of given index $k$ will be denoted as $h_k$.
The array of integers $(h_0,h_1,\dots h_m)$ is called $h$-vector of $P$.
Clearly $h_0=h_m=1$ and $h_k\ge 0$ for all $k$.

Each $k$-face of $P$ contains unique vertex which maximize $\ell$;
if the vertex has index $i$,
then $i\ge k$ and
then it is the maximal vertex for exactly $\tfrac{i!}{k!\cdot (i-k)!}$
faces of dimension $k$.
This observation can be packed in the following polynomial identity 
\[\sum_k h_k\cdot (t+1)^k=\sum_k f_k\cdot t^k.\]

Note that the identity above implies that $h$-vector does not depend on the choice of order of the vertices.
In particular, the $h$ vector is the same for the reversed order;
that is
\[h_k=h_{m-k}\]
for any $k$.
These identities are called Dehn--Sommerville equations.
It gives the complete list of linear equations for $h$-vectors (and therefore $f$-vectors) of simple polyhedrons.

%%%%%%%%%%%%%%%%%%%%%%%%%%%%%%%%%%%%%%%%%
\subsection*{Balls without gaps}
\label{Balls without gaps}
\begin{pr}{}{Gaps between balls}
Let $B_1,\dots,B_n$ be the balls  
of radii $r_1,\dots,r_n$ 
in a Euclidean space.
Assume that no hyperplane divides the balls into two
non-empty sets without intersecting at least one of the balls. 
Show that the balls
$B_1,\dots,B_n$ can be covered by a ball of radius
$r=r_1+\dots+r_n$.

\end{pr}



\section*{Semisolutions}
%%%%%%%%%%%%%%%%%%%%%%%%%%%%%%%%%%%%%%%%%%%%%%%%%%


\parbf{Box in a box.}
Let $\Pi$ be a parallelepiped
with dimensions $a$, $b$ and $c$.
Denote by $v(r)$ the volume of  $r$-neighborhoods of $\Pi$,
 
Note that for all positive $r$ we have
\[v(r)=w_3+w_2\cdot r+w_1\cdot r^2+w_0\cdot r^3,\leqno({*})\]
where 
\begin{itemize}
\item $w_0=\tfrac43\cdot \pi$ is the volume of unit ball,
\item $w_1=\pi\cdot (a+b+c)$,
\item $w_2=2\cdot(a\cdot b+b\cdot c+c\cdot a)$ is the surface area of $\Pi$,
\item $w_3=a\cdot b\cdot c$ is the volume of $\Pi$,
\end{itemize}

Assume $\Pi'$ be an other parallelepiped
with dimensions $a'$, $b'$ and $c'$.
For the volume $v'(r)$ the volume of  $r$-neighborhoods of $\Pi'$ we have a formula similar $({*})$.

Note that 
if $\Pi\subset \Pi'$,
then $v(r)\le v'(r)$ for any $r$.
Checking this inequality for $r\to\infty$,
we get 
\[a+b+c\le a'+b'+c'.\]
\qedsf


The problem was discussed 
by Alexander Shen in \cite{shen}.

A formula analogous to $({*})$
holds for arbitrary convex body $B$ in arbitrary dimension $m$.
The coefficient $w_i(B)$ in the polynomial with different normalization constants 
appear under different names most commonly
\emph{intrinsic volume} and
\emph{quermassintegral}.
They also can be defined as the average 
of area of projections of $B$ to the $i$-dimensional planes.
In particular 
if $B'$ and $B$ are convex bodies such that $B'\subset B$,
then $w_i(B')\le w_i(B)$ for any $i$.
This generalize our problem quite a bit.
Further generalizations lead to so called \index{mixed volumes}\emph{mixed volumes} \cite[see][]{burago-zalgaller}.


%%%%%%%%%%%%%%%%%%%%%%%%%%%%%%%%%%%%%%%%%%%%%%%%%%
\parbf{Harnack's circles.}
Let $\sigma\subset \RP^2$ be a smooth algebraic curve of degree $d$.
Consider the complexification $\Sigma\subset \CP^2$ of $\sigma$.
Without loss of generality, we may assume that $\Sigma$ is regular.

Prove that all regular complex algebraic curves of degree $d$ in $\RP^2$
are homeomorphic to each other; 
denote by $g$ their genus.
Perturbing a singular curve formed by  $d$ lines in $\CP^2$,
we get that 
\[g=\tfrac12\cdot(d-1)\cdot(d-2).\]

The real curve $\sigma$ forms the fixed point set in $\Sigma$ 
by the complex conjugation. 
Use it to show that $\Sigma\backslash\sigma$ has at most 2 connected components.
Hence the result follows.
\qeds
 
The inequality was originally proved 
by Axel Harnack with a different method \cite[see][]{harnack}.
The idea to use complexification is due to Felix Klein \cite[see][]{klein}.
This problem formed the background to Hilbert's 16th problem \cite[see][]{hilbert-problems}. 






%%%%%%%%%%%%%%%%%%%%%%%%%%%%%%%%%%%%%%%%%%%%%%%%%%
\parbf{Two points on each line.}
Take any complete ordering of the set of all lines 
so that each beginning interval has cardinality less than continuum.

Assume we have a set of points $X$ such that each line intersects $X$ at most $2$ points and cardinality of $X$ is less than continuum.

Choose the least line $\ell$ in the ordering which intersect $X$ 
by $0$ or $1$ point.
Note that the set of all lines intersecting $X$ at two points has cardinality less than continuum.
Therefore we can choose a point on $\ell$ and add it to $X$ so that the remaining lines are not overloaded.

It remains to apply well ordering principle.
\qeds

The following problem look similar but far more involved;
a solution follows from the proof that a square cannot be cut into triangles of equal area given by Paul Monsky in \cite{monsky}.

\begin{itemize}
\item {\it Subdivide the plane into three everywhere dense sets $A$, $B$ and $C$ such that each line meets exactly two of these sets.}
\end{itemize}









%%%%%%%%%%%%%%%%%%%%%%%%%%%%%%%%%%%%%%%%%%%%%%%%%%
\parbf{Kissing number.}
Let $n=\mathop{\rm kiss} B^m$
and $B_1,B_2,\dots, B_n$ the copies of $B$ 
which touch $B$ and have no common interior points.
For each $B_i$ consider the vector $v_i$ from the center of $B$ to the center of $B_i$.
Note that $\measuredangle(v_i,v_j)\ge \tfrac\pi3$ if $i\ne j$.

For each $i$,
consider supporting hyperplane $\Pi_i$
to $W$
with outer normal vector $v_i$.
Denote by $W_i$ the reflection of $W$ in $\Pi_i$.

Prove that $W_i$ and $W_j$ have no common interior points if $i\ne j$;
the latter gives the needed inequality.
\qeds

The proof is given by 
Charles Halberg, 
Eugene Levin 
and Ernst Straus 
in \cite{halberg-levin-straus}.
It is expected that the same inequality holds for the orientation-preserving version of kissing number.



%%%%%%%%%%%%%%%%%%%%%%%%%%%%%%%%%%%%%%%%%%%%%%%%%%
\parbf{Monotonic homotopy.}
Note that we can assume
that $h_0(F)$ and $h_1(F)$ both lie in the coordinate $m$-spaces of $\RR^{2\cdot m}=\RR^m\times \RR^m$;
that is,
$h_0(F)\z\subset\RR^m\times\{0\}$
and $h_1(F)\subset  \{0\}\times\RR^m$.

Show that the following homotopy is monotonic
\[h_t(x)=\bigl(h_0(x)\cdot \cos\tfrac{\pi\cdot t}2
\,,\,
 h_1(x)\cdot\sin\tfrac{\pi\cdot t}{2}\bigr).\] 
\qedsf

This homotopy was discovered by Ralph Alexander in \cite{ralexander}.
It has number of applications, 
one of the most beautiful is the given 
by K\'aroly Bezdek 
and Robert Connelly \cite{bezdek-connelly} 
in their proof of 
Kneser--Poulsen  
and Klee--Wagon conjectures in dimension $2$.

The dimension $2\cdot m$ is optimal;
that is, for any positive integer $m$,
there are two maps $h_0,h_1\:F\to \RR^m$ which cannot be connected by a monotonic homotopy $h_t\:F\to\RR^{2\cdot m-1}$.
The latter was shown by Maria Belk and Robert Connelly in \cite{belk-connelly}



%%%%%%%%%%%%%%%%%%%%%%%%%%%%%%%%%%%%%%%%%%%%%%%%%%
\parbf{Cube.}
Consider the cube $[-1,1]^m\subset \RR^m$.
Any vertex this cube has the form $\bm{q}=(q_1,q_2,\dots,q_m)$,
where  $q_i=\pm1$.

For each vertex $\bm{q}$,
consider the intersection of the corresponding octant with the unit sphere;
that is, the set
\[V_{\bm{q}}=\set{(x_1,x_2,\dots,x_m)\in\mathbb{S}^{m-1}}{q_i\cdot x_i\ge 0\ \text{for each}\ i}.\]

Consider the set $\mathcal{A}\subset\mathbb{S}^{m-1}$
formed by the union of all the sets $V_{\bm{q}}$ for $\bm{q}\in A$.
Note that 
\[\vol_{m-1}\mathcal{A}=\tfrac12\cdot\vol_{m-1}\mathbb{S}^{m-1}\]
and 
\[\vol_{m-2}\partial\mathcal{A}
=
\tfrac k{2^{m-1}}\cdot\vol_{m-2}\mathbb{S}^{m-2},\]
where $k$ is the number of edges of the cube with one end in $A$ and the other in $B$.

It remains to  show that 
\[\vol_{m-2}\partial\mathcal{A}
\ge \vol_{m-2}\mathbb{S}^{m-2}.\]
The latter follows from the isoperimetric inequality for $\mathbb{S}^m$. 
\qeds

The problem was suggested by Greg Kuperberg \cite[see][]{One-step}.

%%%%%%%%%%%%%%%%%%%%%%%%%%%%%%%%%%%%%%%%%%%%%%%%%%
\parbf{Geodesic loop.}
Assume such loop exists; denote it by $\gamma$ and let $v$ be its base point.

Denote by $\xi$ and $\zeta$ the directions of exit and the entrance of the loop.
Let $\alpha$ be the angle between $\xi$ and $\zeta$
measured in the tangent cone to the surface of cube at $v$.

Prove that $\alpha=\tfrac\pi2$.
To do this you can use Gauss--Bonnet theorem or unfolding.

It follows that there is a rotational symmetry of cube which fix $v$ and sends $\xi$ to $\zeta$.
The later leads to a contradiction.
\qeds

For the surface of a higher dimensional cube,
the same idea provides existence of symmetry of the cube which fix $v$ and swaps $\xi$ and $\zeta$.
From this point one can arrive to a contradiction, but the work is harder that above.

The same statement holds for tetrahedron, octahedron and icosahedron.
In this case $\alpha$ is a multiple of $\tfrac\pi3$;
it implies existence of rotational symmetry of which fix $v$ and sends $\xi$ to $\zeta$ and leads to a contradiction the same way.

For the dodecahedron such loop exists;
its development shown on the diagram.
(The vertexes of a cube inscribed in the dodecahedron are circled.)

\begin{center}
\begin{lpic}[t(-0 mm),b(0 mm),r(0 mm),l(0 mm)]{pics/geodesic-loop(1)}
\end{lpic}
\end{center}

The problem suggested by Jaros{\l}aw K\k{e}dra.

%%%%%%%%%%%%%%%%%%%%%%%%%%%%%%%%%%%%%%%%%%%%%%%%%%
\parbf{Right and acute triangles.}
Denote by $K$ the convex hull of $\{x_1,\z\dots,x_n\}$.
Without loss of generality we can assume that $K$ is nondegenrate polytope. Prove the following claim.
\qeds

For any distinct $x_i$ and $x_j$
and any interior point $z$ in $K$
we have 
\[\measuredangle \hinge{x_i}{x_j}z<\tfrac\pi2.\]

\medskip

Denote by $h_i$ the homothety with center at $x_i$ and coefficient $\tfrac12$.
Set $K_i=h_i(K)$.

Let us show that $K_i$ and $K_j$ have no comon interior point.
Assume contrary; 
that is, $z=h_i(z_i)=h_j(z_j)$;
for some interior points $z_i$ and $z_j$ in $K$.
Note that 
\[
\measuredangle\hinge{x_i}{x_j}{z_i}
+
\measuredangle\hinge{x_j}{x_i}{z_j}
=
\pi,
\]
which contradicts the claim.

Note that $K_i\subset K$ for any $i$;
it follows that 
\begin{align*}
\tfrac m{2^n}\cdot \vol K
&=\sum_{i=1}^m\vol K_i\le
\\
&\le\vol K.
\end{align*}
Hence the result follows.
\qeds

The problem was posted by Paul Erd{\H{o}}s in \cite{erdos}
and solved by Ludwig Danzer and Branko Gr\"unbaum in \cite{danzer-guenbaum}.
Under the name {}\emph{Angels in Space},
this problem appears in the {}\emph{connoisseur's collection}  \cite{winkler} of Peter Winkler.

Grigori Preleman noticed that the same proof works for a similar problem for Alexandrov space \cite[see][]{perelman-Erdos};
the later led to interesting connections to the crystallographic groups \cite[see][]{lebedeva}.

The upper bound for the number of points with only acute triangles grows exponentially with $m$;
the later was shown by Paul Erd\H{o}s and Zolt\'an F\"uredi in \cite{erdos-fueredi};
the proof use so called \index{probabilistic method}\emph{probabilistic method}.


%%%%%%%%%%%%%%%%%%%%%%%%%%%%%%%%%%%%%%%%%%%%%%%%%%
\parbf{Right-angled polyhedron.}
Let $P$ be a right-angled hyperbolic polyhedron of dimension $m$.
Note that $P$ is simple; that is, exactly $m$ facets meet at each vertex of $P$.

From the projective model of hyperbolic plane, 
one can see that for any simple compact hyperbolic polyhedron there is a simple Euclidean polyhedron with the same combinatorics. 
In particular Dehn--Sommerville equations hold for $P$.

Denote by $(f_0,f_1,\dots f_m)$ and $(h_0,h_1,\dots h_m)$ the $f$- and $h$-vectors of $P$.
Recall that $h_i\ge 0$ for any $i$ and $h_0=h_m=1$.
By Dehn--Sommerville equations, we get
\[f_2> \tfrac{m-2}4\cdot f_1.
\leqno({*})\]

Since $P$ is hyperbolic, each 2-dimensional face of $P$ has at least 5 sides.
It follows that
\[f_2\le \tfrac{m-1}5\cdot f_1.\]
The latter contradicts $({*})$ for $m\ge 6$.
\qeds
 
The proof above 
is the core of proof of nonexistence of compact hyperbolic Coxeter's polyhedra of large dimensions 
given by Ernest Vinberg in \cite{vinberg}, 
see also \cite{vinberg-strong}.

Playing a bit more with the same inequalities, 
one gets nonexistence of  right-angled hyperbolic polyhedra,
in all dimensions starting from $5$.
In 4-dimensional case, an example of a bonded right-angled hyperbolic polyhedron
can be found among regular \index{120-cells}\emph{120-cells} --- the 4-dimensional brothers of dodecahedra.


%%%%%%%%%%%%%%%%%%%%%%%%%%%%%%%%%%%%%%%%%%%%%%%%%%
\parbf{Balls without gaps.} 
Assume that each ball has the mass proportional to its radius.
Denote by $z$  the center of mass of the balls.

Show that all balls lie in the ball $B(z,r)$.
\qeds

The statement was conjectured by Paul Erd\H{o}s.
The solution is given by Adolph and Ruth Goodmans in
\cite{goodman-goodman}.
A variation was given later by Hugo Hardwiger in \cite{hadwiger}.


