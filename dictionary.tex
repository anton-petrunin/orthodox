\chapter{Dictionary}

\begin{description}





\item{\bf Doubling}\refstepcounter{thm}\label{Doubling} 
of a metric space $V$ along  a closed subset $A\subset V$
is the metric space $W$ which obtained by gluing two copies of $V$ along the corresponding points of $A$.

More precisely, consider the minimal equivalence relation $\sim$ on the set $V\times\{1,2\}$,  such that $(a,1)\sim (a,2)$ for any $a\in A$.
Then  $W$ 
is the set $(V\times\{1,2\})/\sim$, 
equipped with the metric such that 
\begin{align*}
|(x,i)-(y,i)|_W&=|x-y|_V
\intertext{and}
|(x,1)-(y,2)|_W&=\inf\set{|x-a|_V+|y-a|_V}{a\in A}
\end{align*}

for any $x,y\in V$.

For a manifold with boundary,
the doubling is usually assumed to be taken along  its boundary;
in this case the resulting space is a manifold without boundary.

\item{\bf Euclidean cone.}\refstepcounter{thm}\label{Euclidean cone} 
Let $\Sigma$ be a metric space with diameter $\le \pi$. 
A metric space $K$ is called Euclidean cone over $\Sigma$
if its underling set 
coincides with the quotient 
$\Sigma\times [0,\infty)/{\sim}$
by the minimal equivalence relation $\sim$ such that $(x,0)\sim(y,0)$ for any $x,y\in \Sigma$
and the metric is defined by cosine rule;
that is,
$$|(x,a)-(y,b)|^2_K=a^2+b^2-2\cdot a\cdot b\cdot\!\cos|x-y|_\Sigma.$$


\item{\bf Exponential map.}\refstepcounter{thm}\label{Exponential map} 
Let $(M,g)$ be a Riemannian manifold;
denote by $\T M$ the tangent bundle over $M$ and by $\T_p=\T_pM$ the tangent space at point $p\in M$.

Given a vector $v\in\T_pM$ denote by $\gamma_v$ the geodesic in $(M,g)$
such that $\gamma(0)=p$ and $\gamma'(0)=v$.
The map $\exp\:\T M\to M$ defined by $v\mapsto \gamma_v(1)$ is called exponential map.

The restriction of $\exp$ to the $\T_p$ is called \emph{exponential map at} $p$ and denoted as $\exp_p$.

Given a smooth submanifold $S\subset M$;
denote by $\mathrm{N} S$ the normal bundle over $S$.
The restriction of $\exp$ to $\mathrm{N} S$ is called \emph{normal exponential map} of $S$ and denoted as $\exp_S$. 

\item{\bf Geodesic.}\refstepcounter{thm}\label{Geodesic}  
Let $X$ be a metric space and $\II$ be a real interval.
A locally isometric immersion $\gamma\:\II\looparrowright X$ is called \emph{unit-speed geodesic}.
In other words, $\gamma$ is a unit-speed geodesic
if for any $t_0\in\II$ we have 
$$|\gamma(t)-\gamma(t')|_X=|t-t'|$$ 
for all $t,t'\in\II$ sufficiently close to $t_0$.

If the condition holds for any $t,t'\in\II$, then $\gamma$ is called \emph{minimizing}.
A minimizing geodesic from point $p$ to point $q$ usually denoted $[pq]$.

Any linear re-parametrization of $\gamma$ is called \emph{geodesic}.

\item{\bf Heisenberg group}\refstepcounter{thm}\label{Heisenberg group}
is the group of $3\times3$ upper triangular matrices of the form
\[\begin{pmatrix}
 1 & a & c\\
 0 & 1 & b\\
 0 & 0 & 1\\
\end{pmatrix}\]
under the operation of matrix multiplication. The elements $a$, $b$ and $c$ usually assumed to be real,
but they can be taken from any commutative ring with identity.


\item{\bf  Length-metric space.}\refstepcounter{thm}\label{Length-metric space} 
A complete metric space $X$ is called \emph{length-metric space}  if the distance between any pair of points in $X$ is equal to the infimum of lengths of curves connecting these points. 




\item{\bf Length-preserving map.}\refstepcounter{thm}\label{Length-preserving map} 
A continuous map $f\:X\to Y$ between 
length-metric spaces 
$X$ and $Y$ is a length-preserving map if for any path $\alpha\:[0,1]\to X$, we have 
$$\length(\alpha)=\length(f\circ\alpha).$$



\item{\bf Path}\refstepcounter{thm}\label{Path} --- continuous map from the real interval $[0,1]$ into a topological space. 




\item{\bf Proper metric space.}
\refstepcounter{thm}\label{Proper metric space} 
A metric space $X$ is called \emph{proper} if any closed bounded set in $X$ is compact.




\item{\bf Saddle surface.}\refstepcounter{thm}\label{Saddle surface} A smooth surface $\Sigma$ in $\RR^3$ is saddle 
(correspondingly strictly saddle) 
if  the product of the principle curvatures at each point is $\le 0$ (correspondingly $<0$).

It admits the following generalization to non-smooth case and arbitrary dimension of the ambient space:
A surface $\Sigma$ in $\RR^m$ is saddle if the restriction $\ell|_\Sigma$ of any linear function $\ell\:\RR^m\to\RR$ has no strict local minimums at interior points of $\Sigma$.

One can generalize it further to an arbitrary ambient space, using  convex functions instead of linear functions in the above definition.




\item{\bf Second fundamental form.}\refstepcounter{thm}\label{Second fundamental form} 


\item{\bf Short map}\refstepcounter{thm}\label{Short map} --- the same as 1-Lipschitz 
or distance non-expanding map.

==========

Assume $\gamma_v\:\RR\to M$ is a geodesic with tangent vector $v\in \T_p$;
that is, such that $\gamma_v(0)=p$ and $\gamma'(0)=v$.
Then 
\[s(v,v)=(f\circ\gamma_v)''(0).\]

This property can be also used to define second fundamental form via the identity
$$s(v,w)=\tfrac 12\cdot[s(v+w,v+w)-s(v,v)-s(w,w)].$$


\end{description}