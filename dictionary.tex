\chapter{Dictionary}

\begin{description}

\item{\bf Asymptotic line}\refstepcounter{thm}\label{Asymptotic line} on the surface $\Sigma\subset \RR^3$
is a curve always tangent to an \emph{asymptotic direction} of $\Sigma$; 
that is, the direction in which the normal curvature of $\Sigma$ is zero.

\item{\bf Busemann function.}\refstepcounter{thm}\label{Busemann function} 
Let $X$ be a metric space
and $\gamma$ is a ray in $X$; 
that is, $\gamma\:[0, \infty)\to X$ is a \hyperref[Geodesic]{\emph{minimizing unit-speed geodesic}}.

The Busemann function $b_\gamma\:X\to\RR$ is defined by
$$b_\gamma(p)=\lim_{t\to\infty}\left(|p-\gamma(t)|_X-t\right).$$
From the triangle inequality, 
the expression under the limit is non-increasing in $t$; 
therefore  the limit above is defined for any $p$.

\item{\bf Curvature operator.}\refstepcounter{thm}\label{Curvature operator}
The Riemannian curvature tensor $R$
can be viewed as an operator $\text{\bf R}$ on the space of tangent bi-vectors $\bigwedge^2 \T$;
it is uniquely defined by identity
$$\langle\mathbf{R}(X\wedge Y),V\wedge W\rangle
=
\langle R(X,Y)V,W\rangle,$$
The operator $\mathbf{R}\:\bigwedge^2 \T\to \bigwedge^2 \T$ is called \emph{curvature operator} and it is said to be \emph{positive definite} if
$\langle\mathbf{R}(\phi),\phi\rangle>0$ for all non zero
bi-vector $\phi\in\bigwedge^2 \T$.

\item{\bf Dehn twist.}\refstepcounter{thm}\label{Dehn twist}
Let $\Sigma$ be a surface and $\gamma\:\RR/\ZZ\to\Sigma$ be non-contractible closed \hyperref[Simple curve]{\emph{simple curve}}.
Let $U_\gamma$ be a neighborhood of $\gamma$ which admits a homeomorphism $h\:U_\gamma\to \RR/\ZZ\times (0,1)$.
Dehn twist along $\gamma$ is a homeomorphism $f\:\Sigma\to\Sigma$
which is identity outside of $U_\gamma$ and 
such that
$$h\circ f\circ h^{-1}\:(x,y)\mapsto(x+y,y).$$

If $\Sigma$ is oriented 
and $h$ is orientation preserving
then the Dehn twist described above is called \emph{positive}.

\item{\bf Dehn--Sommerville equations}\refstepcounter{thm}\label{Dehn--Sommerville equations}.
Assume $P$ is a \emph{simple} Euclidean $m$-dimensional polyhedron;
that is, every vertex of $P$ exactly $m$ facets are meeting.
Denote by $f_k$ the number of $k$-dimensional faces of $P$;
the array of integers $(f_0,f_1,\dots f_m)$ is called $f$-vector of $P$.

Fix an order of the vertices $v_1,v_2,\dots v_{f_0}$
of $P$ so that for some linear function $\ell$, we have $\ell(v_i)>\ell(v_j)\ \Leftrightarrow\ i<j$.
The \emph{index} of the vertex $v_i$ 
is defined as the number of edges $[v_iv_j]$ such that $j<i$. 
The number of vertices of given index $k$ will be denoted as $h_k$.
The array of integers $(h_0,h_1,\dots h_m)$ is called $h$-vector of $P$.
Clearly $h_0=h_m=1$ and $h_k\ge 0$ for all $k$.

Each $k$-face of $P$ contains unique vertex which maximize $\ell$;
if the vertex has index $i$ then $i\ge k$ and
then it is the maximal vertex for exactly $\tfrac{i!}{k!\cdot (i-k)!}$
faces of dimension $k$.
This observation can be packed in the following polynomial identity 
\[\sum_k h_k\cdot (t+1)^k=\sum_k f_k\cdot t^k.\]

Note that the identity above implies that $h$-vector does not depend on the choice of order of the vertices.
In particular, the $h$ vector is the same for the reversed order;
that is
\[h_k=h_{m-k}\]
for any $k$.
These identities are called Dehn--Sommerville equations.
It gives the complete list of linear equations for $h$-vectors (and therefore $f$-vectors) of simple polyhedrons.

\item{\bf Doubling}\refstepcounter{thm}\label{Doubling} 
of a metric space $V$ in a closed subset $A\subset V$
is the metric space $W$ which obtained by gluing two copies of $V$ along the corresponding points of $A$.

More precisely, consider the minimal equivalence relation $\sim$ on the set $V\times\{1,2\}$,  such that $(a,1)\sim (a,2)$ for any $a\in A$.
Then  $W$ 
is the set $(V\times\{1,2\})/\sim$, 
equipped with the metric such that 
\begin{align*}
|(x,i)-(y,i)|_W&=|x-y|_V
\intertext{and}
|(x,1)-(y,2)|_W&=\inf\set{|x-a|_V+|y-a|_V}{a\in A}
\end{align*}

for any $x,y\in V$.

For a manifold with boundary,
the doubling is usually assumed to be taken in its boundary;
in this case the resulting space is a manifold without boundary.

\item{\bf Euclidean cone.}\refstepcounter{thm}\label{Euclidean cone} 
Let $\Sigma$ be a metric space with diameter $\le \pi$. 
A metric space $K$ is called Euclidean cone over $\Sigma$
if its underling set 
coincides with the quotient 
$\Sigma\times [0,\infty)/{\sim}$
by the minimal equivalence relation $\sim$ such that $(x,0)\sim(y,0)$ for any $x,y\in \Sigma$
and the metric is defined by cosine rule;
that is,
$$|(x,a)-(y,b)|^2_K=a^2+b^2-2\cdot a\cdot b\cdot\!\cos|x-y|_\Sigma.$$

\item{\bf Energy functional.}\refstepcounter{thm}\label{Energy functional} Let $F$ be a smooth map from a closed Riemannian manifold $M$ to a Riemannian manifold $N$.
Then energy functional of $F$ is defined as
\[E(F)=\int\limits_M |d_xF|^2\cdot d_x\vol_M.\]
If $(a_{i,j})$ denote the components 
of the differential $d_xF$ 
written in the orthonormal bases in $\T_xM$ and $\T_{F(x)}N$
then 
\[|d_xF|^2=\sum_{i,j}a_{i,j}^2.\]

\item{\bf Equidistant subsets.}\refstepcounter{thm}\label{Equidistant subsets} 
Two subsets $A$ and $B$ in a metric space $X$ are called equidistant if the distance function $\dist_A\:X\to\RR$ is constant on $B$ and $\dist_B$ is constant on $A$.

\item{\bf Exponential map.}\refstepcounter{thm}\label{Exponential map} 
Let $(M,g)$ be a Riemannian manifold;
denote by $\T M$ the tangent bundle over $M$ and by $\T_p=\T_pM$ the tangent space at point $p\in M$.

Given a vector $v\in\T_pM$ denote by $\gamma_v$ the geodesic in $(M,g)$
such that $\gamma(0)=p$ and $\gamma'(0)=v$.
The map $\exp\:\T M\to M$ defined by $v\mapsto \gamma_v(1)$ is called exponential map.

The restriction of $\exp$ to the $\T_p$ is called \emph{exponential map at} $p$ and denoted as $\exp_p$.

Given a smooth submanifold $S\subset M$;
denote by $\mathrm{N} S$ the normal bundle over $S$.
The restriction of $\exp$ to $\mathrm{N} S$ is called \emph{normal exponential map} of $S$ and denoted as $\exp_S$. 

\item{\bf Geodesic.}\refstepcounter{thm}\label{Geodesic}  
Let $X$ be a metric space and $\II$ be a real interval.
A locally isometric immersion $\gamma\:\II\looparrowright X$ is called \emph{unit-speed geodesic}.
In other words, $\gamma$ is a unit-speed geodesic
if for any $t_0\in\II$ we have 
$$|\gamma(t)-\gamma(t')|_X=|t-t'|$$ 
for all $t,t'\in\II$ sufficiently close to $t_0$.

If the condition holds for any $t,t'\in\II$ then $\gamma$ is called \emph{minimizing}.
A minimizing geodesic from point $p$ to point $q$ usually denoted $[pq]$.

Any linear re-parametrization of $\gamma$ is called \emph{geodesic}.

\item{\bf Heisenberg group}\refstepcounter{thm}\label{Heisenberg group}
is the group of $3\times3$ upper triangular matrices of the form
\[\begin{pmatrix}
 1 & a & c\\
 0 & 1 & b\\
 0 & 0 & 1\\
\end{pmatrix}\]
under the operation of matrix multiplication. The elements $a$, $b$ and $c$ usually assumed to be real,
but they can be taken from any commutative ring with identity.

\item{\bf Kissing number.}\refstepcounter{thm}\label{Kissing number}
Let  $W_0$ be a convex body in $\RR^m$.
The kissing number of $W_0$ is the maximal integer $k$ such that there are $k$ bodies $W_1,W_2,\dots,W_k$ such that each $W_i$ is congruent to $W_0$,
for each $i$ we have $W_i\cap W_0\not=\emptyset$ and have no pair $W_i,W_j$ has common interior points.

\item{\bf  Length-metric space.}\refstepcounter{thm}\label{Length-metric space} 
A complete metric space $X$ is called {\it length-metric space}  if the distance between any pair of points in $X$ is equal to the infimum of lengths of curves connecting these points. 

\item{\bf Macro-dimension.}\refstepcounter{thm}\label{Macro-dimension}
Let $X$ be a locally compact metric space $a>0$ and $m$ is an integer.
We say that the macro-dimension  of $X$ at the scale $a$ is at most $m$
if there is a continuous map $f$ from $X$ to an $m$-dimensional simplicial complex $K$
such that for any $k\in K$ the inverse image $f^{-1}(\{k\})$ has diameter less than $a$.

If macro-dimension of $X$ at the scale $a$ is at most $m$,
but not at most $m-1$, 
we say that $m$ is the macro-dimension of $X$ at the scale $a$.

Equivalently, the macro-dimension of $X$ on scale $a$ can be defined as 
the least integer $m$ such that $X$ admits an open covering with diameter of each set less than $a$ 
and such that each point in $X$ is covered by at most $m+1$ sets in the cover.


\item{\bf Length-preserving map.}\refstepcounter{thm}\label{Length-preserving map} 
A continuous map $f\:X\to Y$ between 
\hyperref[Length-metric space]{\emph{length-metric spaces}} 
$X$ and $Y$ is a length-preserving map if for any path $\alpha\:[0,1]\to X$, we have 
$$\length(\alpha)=\length(f\circ\alpha).$$

\item{\bf Minimal surface.}\refstepcounter{thm}\label{Minimal surface} 
Let $\Sigma$ be a $k$-dimensional smooth surface in
a Riemannian manifold $M$
and $\T=\T\,\Sigma$ and $\mathrm{N}=\mathrm{N}\,\Sigma$ correspondingly tangent and normal bundle.
Let $s\:\T\otimes \T\to \mathrm{N}$ denotes the \hyperref[Second fundamental form]{\emph{second fundamental form}} of $\Sigma$.
Let  $e_i$ is an orthonormal basis for $\T_x$, 
set $H_x=\sum_i s(e_i,e_i)\in \mathrm{N}_x$; 
it is the mean curvature vector at $x\in \Sigma$. 

We say that $\Sigma$ is \emph{minimal} if $H\equiv 0$.

\item{\bf Nil-manifolds}\refstepcounter{thm}\label{Nil-manifolds} form the minimal class of manifolds which includes a point, and has the following property:  
the total space of any principle $\mathbb{S}^1$-bundle over a nil-manifold is a nil-manifold. 

Any nil-manifold is diffeomorphic to the quotient of a connected nilpotent Lie group by a lattice.

The celebrated Gromov's theorem states that almost flat manifolds admit a finite cover by a nil-manifold.



\item{\bf Polyhedral space}\refstepcounter{thm}\label{Polyhedral space}
is a complete length-metric space which admits a finite triangulation 
such that each simplex is globally isometric to a simplex in a Euclidean space.

A point in a polyhedral space is called \emph{regular} if it has a neighborhood isometric to an open set in a Euclidean space;
otherwise it called \emph{singular}.

Often finiteness of the triangulation is relaxed to \emph{local finiteness}.

If one exchange Euclidean space to sphere or hyperbolic space,
one gets definition of \emph{spherical} and correspondingly \emph{hyperbolic polyhedral spaces}.
To define regular/singular points in spherical or hyperbolic space,
one has to exchange in the above definition Euclidean space to unit sphere or hyperbolic space with curvature $-1$.


\item{\bf Polynomial volume growth.}\refstepcounter{thm}\label{Polynomial volume growth} A Riemannian manifold $M$ has polynomial volume growth if for some (and therefore any) $p\in M$, 
we have 
$$\vol B(p,r)\z\le C\cdot (r^k+1),$$ 
where $B(p,r)$ is the ball in $M$ and  $C$, $k$ are real constants.

\item{\bf Proper metric space.}
\refstepcounter{thm}\label{Proper metric space} 
A metric space $X$ is called \emph{proper} if any closed bounded set in $X$ is compact.

\item{\bf Piecewise distance preserving map.}%
\refstepcounter{thm}%
\label{Piecewise distance preserving map} 
Let $P$ and $Q$ be polyhedral spaces, a map $f\:P\to Q$ is called piecewise linear isometry if there is a triangulation $\mathcal{T}$ of $P$ such that at any simplex $\Delta\in \mathcal{T}$ the restriction $f|_\Delta$ is distance preserving.

\item{\bf  PL-homeomorphism}\refstepcounter{thm}\label{PL-homeomorphism} or piecewise linear homeomorphism.
A map $h\:P\to Q$ between polyhedral spaces $P$ and $Q$ is called PL-homeomorphism if it is a homeomorphism and both spaces $P$ and $Q$ admit triangulations such that each simplex of $P$ is mapped to a simplex of $Q$ by an affine map.

\item{\bf  Quasi-isometry.}\refstepcounter{thm}\label{Quasi-isometry} A map $f\:X\to Y$ is called a quasi-isometry if there is a positive real constant $C$ such that $f(X)$ is a $C$-net in $Y$ and
$$\tfrac{1}{C}\cdot|x-y|_X-C
\le 
|f(x)-f(y)|_Y\le C\cdot|x-y|_X+C.$$

Note that a quasi-isometry is not assumed to be continuous, for example any map between compact metric spaces is a quasi-isometry.

\item{\bf Saddle surface.}\refstepcounter{thm}\label{Saddle surface} A smooth surface $\Sigma$ in $\RR^3$ is saddle 
(correspondingly strictly saddle) 
if  the product of the principle curvatures at each point is $\le 0$ (correspondingly $<0$).

It admits the following generalization to non-smooth case and arbitrary dimension of the ambient space:
A surface $\Sigma$ in $\RR^m$ is saddle if the restriction $\ell|_\Sigma$ of any linear function $\ell\:\RR^m\to\RR$ has no strict local minima at interior points of $\Sigma$.

One can generalize it further to an arbitrary ambient space, using  convex functions instead of linear functions in the above definition.


\item{\bf Sasaki metric.}\label{Sasaki metric}
Let $(M,g)$ be a Riemannian manifold.
The Sasaki metric is the most natural choice of metric on the tangent space $\T M$.
It is uniquely defined by the following properties:
\begin{enumerate}[(i)]
\item The natural projection $\tau\:\T M\to M$ is a Riemannian submersion.
\item The metric on each tangent space $\T_p\subset \T M$ is the Euclidean metric induced by $g$.
\item Assume $\gamma(t)$ is a curve in $M$ and $v(t)\in\T_{\gamma(t)}$ is a parallel vector field along $\gamma$. 
Note that $v(t)$ forms a curve in $\T M$ 
and $\T_{\gamma(t)}M$ forms a submanifold in $\T M$.
For the Sasaki metric, we have $\dot v(t)\perp \T_{\gamma(t)}M$ for any $t$.
\end{enumerate}

A more constructive way to describe Sasaki metric is given by identifying 
$\T_u[\T M]$ for any $u\in \T_p M$ with the direct sum of so called vertical and horizontal vectors $\T_p M\oplus \T_p M$.
The projection of this splitting defined by the differential of $\tau$
and the Levi-Civita connection.
Then $\T_u[\T M]$ is equipped with the metric  defined as 
\[\hat g(X,Y)=g(X^V,Y^V)+g(X^H,Y^H),\]
where $X^V,X^H\in\T_pM$ denotes the vertical and horizontal components of $X\in\T_u[\T M]$.

\item{\bf Second fundamental form.}\refstepcounter{thm}\label{Second fundamental form} 
Assume $f\:M\looparrowright \RR^m$ be an immersion of smooth manifold $M$.
Given a point $p\in M$ denote by $\T_p$ and $\mathrm{N}_p=\T_p^\bot$
the tangent and normal spaces of $L$ at $p$.
The second fundamental form for $f$ at $p\in M$ is defined as $$s(v,w)=(\nabla_v w)^\bot,\eqno({*})$$ 
where $(\nabla_v w)^\bot$ denotes the orthogonal projection of covariant derivative $\nabla_v w$ onto the normal bundle.

Assume $\gamma_v\:\RR\to M$ is a geodesic with tangent vector $v\in \T_p$;
that is, such that $\gamma_v(0)=p$ and $\gamma'(0)=v$.
Then 
\[s(v,v)=(f\circ\gamma_v)''(0).\]

This property can be also used to define second fundamental form via the identity
$$s(v,w)=\tfrac 12\cdot[s(v+w,v+w)-s(v,v)-s(w,w)].$$

The formula $({*})$ can be used to define the second fundamental form for smooth immersions from into Riemannian manifold.

\item{\bf Short map}\refstepcounter{thm}\label{Short map} --- the same as 1-Lipschitz 
or distance non-expanding map.

\item{\bf Simple curve}\refstepcounter{thm}\label{Simple curve} --- an image of a continuous injective map of a real segment or a circle in a topological space.

\item{\bf Sub-Riemannian metric.}\refstepcounter{thm}\label{Sub-Riemannian metric}
Let $(M,g)$ is a Riemannian manifold.

Assume that in the tangent bundle $\T M$ 
a choice of sub-bundle $H$ is given;
the sub-bundle $H$ which will be called  \emph{horizontal distribution}.
The tangent vectors which lie in $H$ will be called \emph{horizontal}.
A piecewise smooth curve will be called \emph{horizontal}
if all its tangent vectors are horizontal.

The sub-Riemannian distance between points $x$ and $y$ is defined as infimum of lengths of horizontal curves connecting $x$ to $y$.

Alternatively, the distance can be defined as a limit of Riemannian distances 
for the metrics 
\[g_\lambda(X,Y)=g(X^H,Y^H)+\lambda\cdot g(X^V,Y^V)\] 
as $\lambda\to \infty$,
where $X^H$ denotes the horizontal part of $X$;
that is, the orthogonal projection of $X$ to $H$
and $X^V$ denotes the vertical part of $X$;
so, $X^V+X^H=X$.

One usually adds a condition which ensure that any curve in $M$ can be arbitrary well approximated by a horizontal curve with the same endpoints.
(In particular this ensures that the distance will not take infinite values.)
The most common condition is so called  \emph{complete non-integrability};
it means that for any $x\in M$, 
one can choose a basis in $T_xM$
from the vectors of the following type:
$A(x)$, $[A,B](x)$, $[A,[B,C]](x)$, $[A,[B,[C,D]]](x),\dots$ where all vector fields $A,B,C,D, \dots$ are horizontal.


\item{\bf Total curvature.}
\refstepcounter{thm}\label{Total curvature} 
Let $\gamma\:[a,b]\to\RR^m$ be a curve.
The total curvature of $\gamma$ is defined as supremum of sum of external angles for broken lines inscribed in $\gamma$. 
Namely, 
$$\sup\set{\sum_{i=1}^{n-1}\alpha_i}{a=t_0<t_1<\dots<t_n=b},$$
where $\alpha_i=\pi-\measuredangle \hinge{\gamma(t_{i})}{\gamma(t_{i-1})}{\gamma(t_{i+1})}$.

\item{\bf Warped product.}
\refstepcounter{thm}
\label{def:Warped product} 
Let $(M,g)$ and $(N,h)$ be Riemannian manifolds 
and $f$ be a smooth positive function defined on $M$.
Consider the product manifold $W=M\times N$.
Given a tangent vector 
$X\z\in \T_{(p,q)} W
=\T_p M\times \T_p N$ denote by 
$X_M\z\in \T M$ and $X_N\z\in \T N$ its projections.
Let us equip $W$ with the Riemannian metric defined as
\[s(X,Y)=g(X_M,Y_M)+f^2\cdot h(X_N,Y_N).\]
The obtained Riemannian manifold $(W,s)$ is called warped product of $M$ and $N$ with respect to $f\:M\to \RR$;
it can be written as  $(W,g)=(N,h)\times_f(M,g)$.

\end{description}