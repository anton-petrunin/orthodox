\csname @openrightfalse\endcsname
\chapter{Metric geometry}

In this chapter, we consider metric spaces.
All the nececary material could be found in the first three chapters of the textbook \cite{bbi} by Dmitry Burago, Yuri  Burago and Sergei Ivanov. 

Let us fix few standard notations.
\begin{itemize}
\item The distance between two points $x$ and $y$ in a metric space $X$
will be denoted as 
\[|x-y|_X.\]
\item A metric space $X$ is called \index{length space}\emph{length space} if for any $\eps>0$ two points $x,y\in X$ can be connected by urve $\alpha$
such that
\[\length\alpha<|x-y|_X+\eps.\]
\end{itemize}

%%%%%%%%%%%%%%%%%%%%%%%%%%%%%%%%%%%%%%%%%
\subsection*{Embedding of a compact}

\begin{pr}{}{Embedding of a compact}\label{compact} 
Prove that any compact metric space 
is isometric to 
a subset of a compact length spaces.
\end{pr}

%%%%%%%%%%%%%%%%%%%%%%%%%%%%%%%%%%%%%%%%%%%%%%%%%%
\parit{Semisolution.}
Let $K$ be a compact metric space.
Denote by $B(K)$ the space of bounded functions on $K$
equipped with sup norm; 
that is, 
\[|f|=\sup_{x\in K}|f(x)|.\]

Note that the map $\phi\:K\to B(K)$, defied by $x\mapsto \dist_x$
is a distance preserving embedding.

Denote by $W$ the linear convex hull of the image $\phi(K)\subset B(K)$ 
with the metric induced from $B(K)$.
It remains to show that $W$ forms a compact length space.
\qeds

The map $\phi$ is called \index{Kuratowski embedding}\emph{Kuratowski embedding},
it was constructed in \cite{kuratowski},
although essentially the same map 
was described by Maurice Fr\'echet 
in the same paper he introduced metric spaces \cite[see][]{frechet}.


%%%%%%%%%%%%%%%%%%%%%%%%%%%%%%%%%%%%%%%%%
\subsection*{Non-contracting map\easy}

\begin{pr}{\easy}{Non-contracting map}\label{Noncontracting map}
Let $K$  be a compact metric space and
\[f\:K\z\to K\] 
be a non-contracting map.
Prove that $f$ is an isometry.
\end{pr}


%%%%%%%%%%%%%%%%%%%%%%%%%%%%%%%%%%%%%%%%%
\subsection*{Horo-compactification\easy}

Let $X$ be a metric space.
Denote by $C(X,\RR)$ the space of continuous real-valued functions
equipped with the compact open topology.

Fix a point $z_0\in X$.
Given a point $z\in X$, let $f_z\in C(X,\RR)$ be the function defined as 
\[f_z(x)=|z-x|_X-|z-x_0|_X.\]
Let $F_X\:X\to C(X,\RR)$ be the map 
defined as $F_X\:z\mapsto f_z$.

Denote by $\bar X$ 
the closure of $F_X(X)$ in $C(X,\RR)$;
note that $\bar X$ is compact.
That is, 
if $F_X$ is an embedding, 
then $\bar X$ forms a compactification of $X$,
which is called \index{horo-compactification}\emph{horo-compactification}.
The complement 
$\partial_\infty X=\bar X\backslash F_X(X)$ 
is called \index{horo-absolute}\emph{horo-absolute} of $X$.
The compactification
and absolute
was introduced by Mikhael Gromov in \cite{gromov-hyperbolic}.


\begin{pr}{\easy}{Horo-compactification}\label{Horo-compactification}
Construct a proper metric space $X$
such that $F_X$ is not an embedding.
Show that there are no such examples among proper length spaces.
\end{pr}

%%%%%%%%%%%%%%%%%%%%%%%%%%%%%%%%%%%%%%%%%
\subsection*{Ball and sphere}

\begin{pr}{}{Ball and sphere}\label{3-sphere is close to a ball}
Construct a sequence of Riemannian metrics on $\mathbb{S}^3$ 
which converges in the Gromov--Hausdorff topology 
to the unit ball in $\RR^3$.
\end{pr}

%%%%%%%%%%%%%%%%%%%%%%%%%%%%%%%%%%%%%%%%%
\subsection*{Macro-dimension\easy}

Let $X$ be a locally compact metric space  $m$ is an integer
and $a>0$.
We say that the macro-dimension  of $X$ at the scale $a$ is at most $m$
if there is a continuous map $f$ from $X$ to an $m$-dimensional simplicial complex $K$
such that for any $k\in K$ the inverse image $f^{-1}(\{k\})$ has diameter less than $a$.

If macro-dimension of $X$ at the scale $a$ is at most $m$,
but not at most $m-1$, 
we say that $m$ is the \index{macro-dimension}\emph{macro-dimension} of $X$ at the scale $a$.

Equivalently, the macro-dimension of $X$ on scale $a$ can be defined as 
the least integer $m$ such that $X$ admits an open covering with diameter of each set less than $a$ 
and such that each point in $X$ is covered by at most $m+1$ sets in the cover.

\begin{pr}{\easy}{Macro-dimension}\label{macro-dimension} 
Let $M$ be a simply connected Riemannian manifold with the following property: 
any closed curve is null-homotopic 
in its own  1-neighborhood. 
Prove that the macro-dimension of $M$ on the scale $100$ is at most $1$.
\end{pr}

%%%%%%%%%%%%%%%%%%%%%%%%%%%%%%%%%%%%%%%%%
\subsection*{No short embedding\hard}

\begin{pr}{\hard}{No short embedding}\label{weird-metric} 
Construct a length-metric $d$ on $\RR^3$,
such that for any open set $U\subset  \RR^3$,
there is no short embeddings $(U,d)\to \RR^3$,
where $\RR^3$ equipped with the canonical metric.
\end{pr}

%%%%%%%%%%%%%%%%%%%%%%%%%%%%%%%%%%%%%%%%%
\subsection*{Sub-Riemannian sphere\thm}

Let us define sub-Riemannian metric.

Fix a Riemannian manifold $(M,g)$.
Assume that in the tangent bundle $\T M$ 
a choice of sub-bundle $H$ is given;
the sub-bundle $H$ which will be called  \index{horizontal distribution}\emph{horizontal distribution}.
The tangent vectors which lie in $H$ will be called {}\emph{horizontal}.
A piecewise smooth curve will be called {}\emph{horizontal}
if all its tangent vectors are horizontal.

The sub-Riemannian distance between points $x$ and $y$ is defined as infimum of lengths of horizontal curves connecting $x$ to $y$.

Alternatively, the distance can be defined as a limit of Riemannian distances 
for the metrics 
\[g_\lambda(X,Y)=g(X^H,Y^H)+\lambda\cdot g(X^V,Y^V)\] 
as $\lambda\to \infty$,
where $X^H$ denotes the horizontal part of $X$;
that is, the orthogonal projection of $X$ to $H$
and $X^V$ denotes the vertical part of $X$;
so, $X^V+X^H=X$.

One usually adds a condition which ensure that any curve in $M$ can be arbitrary well approximated by a horizontal curve with the same endpoints.
(In particular this ensures that the distance will not take infinite values.)
The most common condition is so called  {}\emph{complete non-integrability};
it means that for any $x\in M$, 
one can choose a basis in $T_xM$
from the vectors of the following type:
$A(x)$, $[A,B](x)$, $[A,[B,C]](x)$, $[A,[B,[C,D]]](x),\dots$ where all vector fields $A,B,C,D, \dots$ are horizontal.

\begin{pr}{\thm}{Sub-Riemannian sphere}\label{sub-Riemannian} 
Prove that any sub-Riemannian metric 
on the $\mathbb{S}^m$ is isometric to the intrinsic metric of a hypersurface in $\RR^{m+1}$.
\end{pr}


It will be hard to solve the problem without knowing proof of Nash--Kuiper theorem on length preserving $C^1$-embeddings.
The original papers of John Nash 
and Nicolaas Kuiper \cite[see][]{nash,kuiper} are very readable.

%%%%%%%%%%%%%%%%%%%%%%%%%%%%%%%%%%%%%%%%%
\subsection*{Length-preserving map\thm}

A continuous map $f\:X\to Y$ between metric spaces is called \index{length-preserving}\emph{length-preserving} if it preserves the length of curves; that is for any curve $\alpha$ in $X$ we have
\[\length(f\circ\alpha)=\length\alpha.\]

\begin{pr}{\thm}{Length-preserving map}\label{two2one} 
Show that there is no length-preserving map $\RR^2\to \RR$.
\end{pr}

The Rademacher's theorem on differentiability of Lipschitz maps should help to solve this problem. 



%%%%%%%%%%%%%%%%%%%%%%%%%%%%%%%%%%%%%%%%%
\subsection*{Fixed segment}

\begin{pr}{}{Fixed segment}\label{Fixed segment}
Let $\rho(x,y)=\|x-y\|$ be a metric on $\RR^m$ induced by a norm $\|{*}\|$.

Assume that $f\:(\RR^m,\rho)\to(\RR^m,\rho)$ is an isometry which fixes two distinct points.
Show that $f$ fixes the line segment between them.
\end{pr}

%%%%%%%%%%%%%%%%%%%%%%%%%%%%%%%%%%%%%%%%%
\subsection*{Pogorelov's construction\easy}

\begin{pr}{}{Pogorelov's construction}\label{Pogorelov's construction}
Let $\mu$ be a regular centrally symmetric finite measure on $\mathbb{S}^2$ which is positive on every open set.
Given two points $x,y\in \mathbb{S}^2$,
set 
\[\rho(x,y)=\mu[B(x,\tfrac \pi2)\backslash B(y,\tfrac\pi2)].\]

Show that $\rho$ is a length-metric on $\mathbb{S}^2$
and moreover, geodesics in this metric formed by arcs of great circles.
\end{pr}

%%%%%%%%%%%%%%%%%%%%%%%%%%%%%%%%%%%%%%%%%
\subsection*{Straight geodesics}

\begin{pr}{}{Straight geodesics}\label{Straight geodesics}
Let $\rho$ be a length-metric on $\RR^m$, 
which is bi-Lipschitz equivalent to the canonical metric.
Assume that every geodesic $\gamma$ in $(\RR^d,\rho)$ is linear 
(that is, $\gamma(t)=v+w\cdot t$ for some $v,w\in\RR^m$).
Show that $\rho$ is induced by a norm on $\RR^m$.
\end{pr}

%%%%%%%%%%%%%%%%%%%%%%%%%%%%%%%%%%%%%%%%%
\subsection*{Hyperbolic space}

A map $f\:X\to Y$ between metric spaces is called a \index{quasi-isometry}\emph{quasi-isometry} if there is a positive real constant $C$ such that $f(X)$ is a $C$-net in $Y$ and
$$\tfrac{1}{C}\cdot|x-y|_X-C
\le 
|f(x)-f(y)|_Y\le C\cdot|x-y|_X+C.$$

Note that a quasi-isometry is not assumed to be continuous, for example any map between compact metric spaces is a quasi-isometry.

\begin{pr}{}{Hyperbolic space}\label{Hyperbolic space}
Construct a quasi-isometry
from the hyperbolic $3$-space 
to a subset 
of the product of two hyperbolic planes.
\end{pr}

%%%%%%%%%%%%%%%%%%%%%%%%%%%%%%%%%%%%%%%%%
\subsection*{A homeomorphism near quasi-isometry\thm}

The quasi-isometry is defined few lines above.

\begin{pr}{\thm}{A homeomorphism near quasi-isometry}\label{hom-near-QI} 
Let $f\:\RR^m\to\RR^m$ be a quasi-isometry.
Show that there is a (bi-Lipschitz) homeomorphism 
$h\:\RR^m\to\RR^m$ on a bounded distance from $f$;
that is, there is 
$$|f(x)-h(x)|\le C$$
for any $x\in\RR^m$ and a real constant $C$.
\end{pr}

The solution requires a corollary of the theorem  proved by Laurence Siebenmann in \cite{siebenmann}.
It states that 
if $V_1, V_2\subset\RR^m$ are open
and the two embedding $f_1\:V_1\to\RR^m$ and $f_2\:V_2\to\RR^m$ 
are sufficiently close to each other 
on the overlap $U=V_1\cap V_2$, 
then
there is an embedding $f$ defined on an open set $W'$
which is slightly smaller than $W=V_1\cup V_2$
and such that $f$ is sufficiently close to each $f_1$ and $f_2$ at the points where they are defined.

The  bi-Lipschitz version requires 
an analogous statement in the category of bi-Lipschitz embeddings;
it was proved by
Dennis Sullivan in \cite{sullivan}.
This result can be used to prove a that the homeomorphism is bi-Lipschitz.

%%%%%%%%%%%%%%%%%%%%%%%%%%%%%%%%%%%%%%%%%
\subsection*{A family of sets with no section}

\begin{pr}{}{A family of sets with no section}\label{hausdorff-section} 
Construct a one parameter family of closed sets $C_t$ in $\mathbb{S}^1$, $t\in [0,1]$
which is continuous in the Hausdorff topology, 
but which does not admit a {}\emph{section};
that is, there is no continuous 
map $c\:[0,1]\to \mathbb{S}^1$ such that $c(t)\in C_t$ for any $t$.
\end{pr}

\subsection*{Spaces with isometric balls}

\begin{pr}{}{Spaces with isometric balls}
Construct a pair of locally compact length spaces $X$ and $Y$ 
which are not isometric,
but for some fixed points $x_0\in X$,  $y_0\in Y$ and any radius $R$
the ball $B(x_0,R)$ in $X$ is 
isometric to the ball $B(y_0,R)$ in $Y$.
\end{pr}



\section*{Semisolutions}
%%%%%%%%%%%%%%%%%%%%%%%%%%%%%%%%%%%%%%%%%%%%%%%%%%



%%%%%%%%%%%%%%%%%%%%%%%%%%%%%%%%%%%%%%%%%%%%%%%%%%
\parbf{Non-contracting map.}
Given any pair of point $x_0,y_0\in K$, 
consider two sequences $x_0,x_1,\dots$ and $y_0,y_1,\dots$
such that 
and $x_{n+1}=f(x_n)$ and $y_{n+1}=f(y_n)$ for each $n$.

Since $K$ is compact, 
we can choose an increasing sequence of integers $n_k$
such that both sequences $(x_{n_i})_{i=1}^\infty$ and $(y_{n_i})_{i=1}^\infty$
converge.
In particular, both of these sequences  are Cauchy;
that is,
\[
|x_{n_i}-x_{n_j}|_K, |y_{n_i}-y_{n_j}|_K\to 0
\ \ 
\text{as}
\ \ \min\{i,j\}\to\infty.
\]


Since $f$ is non-contracting, we get
\[
|x_0-x_{|n_i-n_j|}|
\le 
|x_{n_i}-x_{n_j}|.
\]

It follows that  
there is a sequence $m_i\to\infty$ such that
\[
x_{m_i}\to x\ \ \text{and}\ \ y_{m_i}\to y\ \ \text{as}\ \ i\to\infty.
\leqno({*})\]

Set \[\ell_n=|x_n-y_n|_K.\]
Since $f$ is non-contracting, $(\ell_n)$ is a non-decreasing sequence.

By $({*})$, it follows that $\ell_{m_i}\to\ell_0$ as $m_i\to\infty$.
It follows that $(\ell_n)$ is a constant sequence.

In particular 
\[|x_0-y_0|_K=\ell_0=\ell_1=|f(x_0)-f(y_0)|_K\]
for any pair of points $(x_0,y_0)$ in $K$.
That is, $f$ is distance preserving, in particular injective.

From $({*})$, we also get that $f(K)$ is everywhere dense.
Since $K$ is compact $f\:K\to K$ is surjective. Hence the result follows.\qeds


This is a basic lemma in the introduction to Gromov--Hausdorff distance \cite[see 7.3.30 in][]{bbi}.
The proof presented here is different, 
it was given by Travis Morrison, 
a students in my MASS class at Penn State, Fall 2011.


%%%%%%%%%%%%%%%%%%%%%%%%%%%%%%%%%%%%%%%%%%%%%%%%%%
\parbf{Horo-compactification.}
For the first part of problem, take $X$ to be the set of non-negative integers with the metric $\rho$ defined as 
$\rho(m,n)=m+n$ for $m\ne n$.

Now assume $X$ is geodesic and $F_X$ is not an embedding.
Then there is a sequence of points $z_1,z_2,\dots$ 
and a point $z_\infty$,
such that $f_{z_n}\to f_{z_\infty}$ in $C(X,\RR)$
as $n\to \infty$, 
while $|z_n-z_\infty|_X>\eps$ 
for some fixed $\eps>0$ and all $n$.

Choose $\bar z_n$ on the a geodesic $[z_\infty z_n]$ such that $|z_\infty-z_n'|=\eps$.
Note that 
\begin{align*}
f_{z_n}(z_\infty)-f_{z_n}(\bar z_n)&=\eps
\intertext{and}
f_{z_\infty}(z_\infty)-f_{z_n}(\bar z_n)&=-\eps
\end{align*}
for any $n$.
In particular $f_{z_n}\not\to f_{z_\infty}$ in $C(X,\RR)$,
a contradiction.\qeds

I learned this problem from Linus Kramer and Alexander Lytchak;
the example was also mentioned in the lectures of Anders Karlsson
and attributed to Uri Bader \cite[see 2.3 in][]{karlsson}.





%%%%%%%%%%%%%%%%%%%%%%%%%%%%%%%%%%%%%%%%%%%%%%%%%%
\parbf{Ball and sphere.}
Make fine burrows in the standard 3-ball which do not change its topology,
but at the same time a come sufficiently close to any point in the ball.

Consider the doubling of obtained ball in its boundary.
Clearly the obtained space is homeomorphic to $\mathbb{S}^3$.
Prove that the burrows can be made 
so that it is sufficiently close to the original ball 
in the Gromov--Hausdorff metric.

It remains to smooth the obtained space slightly 
to get a genuine Riemannian metric with needed property.\qeds


This construction is a stripped version of theorem proved by Steven Ferry and Boris Okunin in \cite{ferry-okun}.
The theorem states that Riemannian metrics on a smooth closed manifold $M$ with $\dim M\ge 3$ 
can approximate given compact length space $X$ 
if and only if 
there is a continuous map $M\to X$
which is surjective on the fundamental groups. 

The two-dimensional case is quite different.
There is no sequence of Riemannian metrics on
$\mathbb{S}^2$ which converge in the Gromov--Hausdorff topology to the unit disc.
In fact, 
if $X$ is a Gromov--Hausdorff limit of $(\mathbb{S}^2,g_n)$,
then any point $x_0\in X$ either admits a neighborhood homeomorphic to $\RR^2$ or it is a cut point;
that is $X\backslash\{x_0\}$ is disconnected \cite[see 3.32 in][]{gromov-MetStr}.

%%%%%%%%%%%%%%%%%%%%%%%%%%%%%%%%%%%%%%%%%%%%%%%%%%
\parbf{Macro-dimension.}
Choose a point $p\in M$,
denote by $f$ the distance function from $p$.

Let us cover $M$ by the connected components of the inverse images 
$f^{-1}((n-1,n+1))$.
Clearly any point in $M$ is covered by at most two such components.
It remains to show that each of these components has diameter less than $100$.

Assume the contrary; let $x$ and $y$ be two points in such connected component 
and $|x-y|_M\ge 100$.
Connect $x$ to $y$ by a curve $\tau$ in the component.
Consider the closed curve $\sigma$ formed by two geodesics $[px]$, $[py]$ and $\tau$.

%%%???+PIC

Prove that $\sigma$ can be divided into 4 arcs $\alpha$, $\beta$, $\gamma$ and $\delta$
in such a way that the minimal distance from $\alpha$ to $\gamma$ as well as the minimal distance from $\beta$ to $\delta$ is at least $10$.

Use the last statement to show that $\sigma$ 
cannot be shrank 
by a disc it its $1$-neighborhood;
the latter contradicts the assumption.\qeds

The problem was discussed it a talk by Nikita Zinoviev around 2004.


%%%%%%%%%%%%%%%%%%%%%%%%%%%%%%%%%%%%%%%%%%%%%%%%%%
\parbf{No short embedding.}
Consider a chain of disjoint circles $c_0,\dots,c_n$ in $\RR^3$;
that is, $c_i$ and $c_{i-1}$ are linked for each $i$. 


\begin{center}
\begin{lpic}[t(-0 mm),b(0 mm),r(0 mm),l(0 mm)]{pics/chain(1)}
\lbl[t]{5,0;$c_0$}
\lbl[t]{11,0;$c_1$}
\lbl[t]{31,0;$\cdots$}
\lbl[t]{54,0;$c_n$}
\end{lpic}
\end{center}


Assume that $\RR^3$ is equipped with a length-metric $\rho$,
such that the total length of the circles is $\ell$
and $U$ is an open set containing all the circles $c_i$.
Note that for any short homeomorphism $f\:(U,\rho)\to\RR^3$ the distance from $f(c_0)$ to $f(c_n)$ is less than $\ell$.

Let us show that the $\rho$-distance from $c_0$ to $c_n$ might be much larger than $\ell$.
Fix a line segment $[ab]$ in $\RR^3$.
Modify 
the length-metric on $\RR^3$ in arbitrary small neighborhood of $[ab]$
so that there is a chain $(c_i)$ of circles as above,
which goes from $a$ to $b$ 
such that
(1) the total length, say $\ell$, 
of $(c_i)$ is arbitrary small,
but 
(2) the obtained metric $\rho$ 
is arbitrary close to the canonical, say
\[\bigl|\rho(x,y)-|x-y|\bigr|<\eps\]
for any two points $x,y\in\RR^3$
and fixed in advanced small $\eps>0$.
The construction of $\rho$ 
is done by shrinking the length of each circle
and expanding the length in the normal directions  
to the circles in their small neighborhood.
The latter is made in order to make impossible to use the circles $c_i$ as a shortcut;
that is, one spends more time to go from one circle to an other 
than saves on going along the circle.

Set $a_n=(0,\tfrac1n,0)$ and $b_n=(1,\tfrac1n,0)$.
Note that the line segments $[a_nb_n]$ are disjoint and converging
to $[a_\infty b_\infty]$
where $a_\infty=(0,0,0)$ and $b_\infty=(1,0,0)$.

Apply the above construction in non-overlapping convex neighborhoods of $[a_nb_n]$ 
and for a sequences 
$\eps_n$ and $\ell_n$ 
which converge to zero very fast.

The obtained length-metric $\rho$ is still close to the canonical,
but for any open set $U$ containing $[a_\infty b_\infty]$
the space $(U,\rho)$ does not admit 
a short homeomorphism to $\RR^3$.
Indeed, 
if such homeomorphism $h$ exists, 
then 
from the above construction,
we get 
\begin{align*}
|h(a_\infty)-f(b_\infty)|
&\le 
|h(a_n)-f(b_n)|
+
\\
&\ \ \ \ \ +
|h(a_\infty)-f(a_n)|
+
|h(b_n)-f(b_\infty)|
\le
\\
&\le
\ell_n+\tfrac2n+100\cdot\eps_n.
\end{align*}
The right hand side converges to $0$ as $n\to\infty$.
Therefore 
\[h(a_\infty)=f(b_\infty),\] 
a contradiction.

It remains to performs similar construction countably many times so a bad segment as $[a_\infty b_\infty]$ above
appears in any open set of $\RR^3$.\qeds



The problem is due
Dmitri Burago, 
Sergei Ivanov 
and David Shoenthal \cite[see][]{BIS}.

%%%%%%%%%%%%%%%%%%%%%%%%%%%%%%%%%%%%%%%%%%%%%%%%%%
\parbf{Sub-Riemannian sphere.}
Prove that there is a non-decreasing sequence of Riemannian metric tensors
$g_0\le g_1\le ...$ such that the induced metrics converge to the given sub-Riemannian metrics.
The metric $g_0$ can be assumed to be a metric on round sphere.

Applying the construction as in Nash--Kuiper theorem,
one can produce a sequence of smooth embeddings $h_n\:\mathbb{S}^m\to \RR^{m+1}$ with the induced metrics $g_n'$
such that $|g_n'-g_n|\to 0$.

Moreover, assume we assign a positive real number $\eps(h)$ for any smooth embedding $h\:\mathbb{S}^m\to\RR^{m+1}$.
Then we can assume that 
\[|h_{n+1}(x)-h_n(x)|<\eps(h_n)\] for any $x\in \mathbb{S}^m$ and $n$.

Show that for a right choice of function $\eps(h_n)$,
the sequence $h_n$ converges, say to $h_\infty$, 
and the metric induced by $h_\infty$ coincides with the given sub-Riemannian metric.\qeds


The problem appeared 
on this list first rediscovered later by Enrico Le Donne in \cite{le-donne}.
Similar construction described in the lecture notes by Allan Yashinski and me \cite[see][]{petrunin-yashinsky} 
which is aimed for undergraduate students. 
Yet \cite{petrunin-paths} is closely relevant.

The same idea can be used to prove the following.
\begin{itemize}
\item Let $M$ be a Riemannian manifold diffeomorphic to the $n$-sphere. 
Show that there is a Riemannian manifold $M'$ arbitrary close to $M$ in Lipschitz metric and vanishing Weyl curvature tensor.
\end{itemize}


%%%%%%%%%%%%%%%%%%%%%%%%%%%%%%%%%%%%%%%%%%%%%%%%%%
\parbf{Length-preserving map.}
The solution uses Rademacher's theorem,
which states that any Lipschitz function $f\:\RR^n\to\RR$ 
is differentiable almost everywhere.
The Rademacher's theorem appears in \cite{rademacher}. \qeds


Assume there is a length-preserving map $f\:\RR^2\to \RR$.

Note that $f$ is Lipschitz.
Therefore by Rademacher's theorem, $f$ is differentiable almost everywhere.

Fix a unit vector $u$.
Prove that, for almost all $x$, the length of curve 
$\alpha\:t\mapsto x+t\cdot u$, $t\in[0,1]$ can be expressed as the integral
\[\int\limits_0^1 (d_{\alpha(t)}f)(u) \cdot dt.\]

It follows that $|d_xf(v)|=|v|$ for almost all $x,v\in\RR^2$;
in particular $d_xf$ is defined and has rank 2 at some point $x$, a contradiction. \qeds 


The idea above can be also used to solve the following problem.

\begin{itemize}
\item {\it Assume $\rho$ is a metric on $\RR^2$ 
which is induced by a norm.
Show that $(\RR^2,\rho)$ admits 
a length-preserving map
to $\RR^3$ 
if and only if 
$(\RR^2,\rho)$ is isometric to the Euclidean plane.}
\end{itemize}








%%%%%%%%%%%%%%%%%%%%%%%%%%%%%%%%%%%%%%%%%%%%%%%%%%
\parbf{Fixed segment.}
Note that it is sufficient to show that 
if 
\[f(a)=a\ \ \text{and}\ \ f(b)=b\]
for some $a,b\in\RR^m$,
then 
\[f(\tfrac{a+b}2)=\tfrac12\cdot(f(a)+f(b)).\]

(This statement is not trivial since in general
metric midpoint of $a$ and $b$ in $(\RR^m,d)$ 
are not defined uniquely.)

Without loss of generality, we can assume that $b+a=0$.

Set $f_0=f$.
Consider the recursively defined sequence of isometries $f_0$, $f_1,\dots$ defined recursively
\[f_{n+1}(x)= -f_n^{-1}(-f_n(x)).\]

Note that $f_n(a)=a$ and $f_n(b)=b$ for any $n$ and 
$$|f_{n+1}(0)|=2\cdot|f_n(0)|.$$
The latter condition implies that 
if $f(0)\ne 0$,
then $|f_n(0)|\to\infty$ as $n\to\infty$.
On the other hand, since $f_n$ is isometry and $f(a)=a$,
we get $|f_n(0)|\le 2\cdot |a|$, a contradiction.\qeds


The problem is a stripped version of Mazur--Ulam theorem proved in  \cite{mazur-ulam};
it states that any isometry of $(\RR^m,d)$ to itself 
forms an affine map. 

The solution above
is the main step in the proof of this theorem 
given by Jussi V\"ais\"al\"a's in \cite{vaisala}.

%%%%%%%%%%%%%%%%%%%%%%%%%%%%%%%%%%%%%%%%%%%%%%%%%%

\parbf{Pogorelov's construction.}
Positivity and symmetry of $\rho$ is evident.

The triangle inequality follows since
\[[B(x,\tfrac \pi2)\backslash B(y,\tfrac\pi2)]
\cup 
[B(y,\tfrac\pi2)\backslash B(z,\tfrac\pi2)]
\supset 
B(x,\tfrac \pi2) \backslash B(z,\tfrac\pi2).
\leqno(*)\]

Note that we get equality in $(*)$ if and only if $y$ lies on the great circle arc from $x$ to $z$.
Therefore the second statement follows.\qeds


This construction was given by 
Aleksei Pogorelov in \cite{pogorelov}.
It is closely related to the construction given 
by David Hilbert in \cite{hilbert}
which was the motivating example of his 4th problem \cite[see][]{hilbert-problems}.


%%%%%%%%%%%%%%%%%%%%%%%%%%%%%%%%%%%%%%%%%%%%%%%%%%
\parbf{Straight geodesics.}
From uniqueness of straight segment between given points in $\RR^m$,
it follows that any straight line in $\RR^m$ forms a geodesic in $(\RR^m,\rho)$.

Set 
\[\|\bm{v}\|_{\bm{x}}=\rho(\bm{x},(\bm{x}+\bm{v})).\]
Note that 
\[ \|\lambda\cdot\bm{v}\|_{\bm{x}}
=
|\lambda|\cdot\|\bm{v}\|_{\bm{x}}\]
for any $\bm{x},\bm{v}\in\RR^m$ and $\lambda\in\RR$.

Prove that 
\[
\|\lambda\cdot\bm{v}\|_{\bm{x}}
-
\|\lambda\cdot\bm{v}\|_{\bm{x}'}
\le 
\Const\cdot |\bm{x}-\bm{x'}|\]
for any $\bm{x},\bm{x'},\bm{v}\in\RR^m$, 
$\lambda\in\RR$
and some fixed $\Const\in\RR$.

Passing to the limit as $\lambda\to\infty$, 
we get
$\|\bm{v}\|_{\bm{x}}$ does not depend on $\bm{x}$;
hence the result follows.\qeds


The idea in the proof is due to Thomas Foertsch
and Viktor Schroeder \cite[see][]{foertsch-schroeder}.
A more general statement was proved by Petra Hitzelberger and Alexander Lytchak in \cite{hitzelberger-lytchak}.
Namely they show that 
if any pair of points in a geodesic metric space $X$ can be separated by an affine function,
then $X$ is isometric to a convex subset a normed vector space.


%%%%%%%%%%%%%%%%%%%%%%%%%%%%%%%%%%%%%%%%%%%%%%%%%%
\parbf{Hyperbolic space.}
Note that $2$-dimensional hyperbolic space 
can be viewed as $(\RR^2,g)$, where 
\[g(x,y)=\left(\begin{matrix}
     1&0
     \\
     0&e^{x}
    \end{matrix}\right).\]
The same way $3$-dimensional hyperbolic space 
can be viewed as $(\RR^3,h)$, where 
where 
\[h(x,y,z)=\left(\begin{matrix}
     1&0&0
     \\
     0&e^{x}&0
     \\
     0&0&e^{x}
    \end{matrix}\right).\]

Prove that the map $\RR^3\to \RR^4$ defined as
$$(x,y,z)\mapsto (x,y,x,z)$$
is a quasi-isometry from $(\RR^3,h)$ to its image in $(\RR^2,g)\times (\RR^2,g)$.\qeds


In the proof we used that horosphere in the hyperbolic space is isometric to the Euclidean plane.
This observation appears already in the book of Nikolai Lobachevsky \cite[see 34 in][]{lobachevsky}.

%%%%%%%%%%%%%%%%%%%%%%%%%%%%%%%%%%%%%%%%%%%%%%%%%%
\parbf{A homeomorphism near quasi-isometry.}\\
Let $M\ge 1$ and $A\ge 0$.
Define $(M,A)$-quasi-isometry
as a map $f\:X\to Y$ between metric spaces $X$ and $Y$ such that for any $x,y\in X$,
 we have
\[\tfrac1M\cdot |x-y|-A\le |f(x)-f(y)|\le M\cdot |x-y|+A\]
and any point in $Y$ lies on the distance at most $A$ from a point in the image $f(X)$.

{\sloppy
Note that $(M,0)$-quasi-isometry is a $[\tfrac1M,M]$-bi-Lipschitz map.
Moreover,
if $f_n\:\RR^m\to\RR^m$ is a  $(M,\frac1n)$-quasi-isometry 
for each $n$, 
then any partial limit of $f_n$ as $n\to\infty$
is a $[\tfrac1M,M]$-bi-Lipschitz map.

}

It follows that given $M\ge 1$ and $\eps>0$ there is $\delta>0$ such that 
for any $(M,\delta)$-quasi-isometry $f\:\RR^m\to\RR^m$ and any $p\in \RR^m$
there is an $[\tfrac1M,M]$-bi-Lipschitz map $h\:B(p,1)\to \RR^m$
such that
\[|f(x)-h(x)|<\eps\]
for any $x\in B(p,1)$.

Applying rescaling, we can get the following equivalent formulation. 
Given $M\ge 1$, $A\ge 0$ and $\eps>0$
there is big enough $R>0$ such that for any $(M,A)$-quasi-isometry 
$f\:\RR^m\to\RR^m$ and any $p\in\RR^m$ there is a $[\tfrac1M,M]$-bi-Lipschitz map $h\:B(p,R)\to \RR^m$
such that 
\[|f(x)-h(x)|<\eps\cdot R\]
for any $x\in B(p,R)$.

Now cover $\RR^m$ by balls
$B(p_n,R)$, construct a $[\tfrac1M,M]$-bi-Lipschitz map $h_n\:B(p_n,R)\to \RR^m$ for each $n$.

The maps $h_n$ are $2\cdot \eps\cdot R$ close to each other on the overlaps of their domains of definition.
This makes possible to deform slightly each $h_n$ so that they agree on the overlaps.
This can be done by Siebenmann's Theorem.
If instead you apply Sullivan's theorem, you get a bi-Lipschitz homeomorphism $h\:\RR^m\to\RR^m$.\qeds


The problem was suggested by Dmitri Burago.





%%%%%%%%%%%%%%%%%%%%%%%%%%%%%%%%%%%%%%%%%%%%%%%%%%
\parbf{A family of sets with no section.}
Identify $\mathbb{S}^1$ with $[0,1]/(0\sim 1)$.
Consider one parameter family of Cantor sets $K_t$
formed by all possible sums $\sum_{n=1}^\infty a_n\cdot t^n$,
where $a_i$ is $0$ or $1$ and $t\in[0,\tfrac12]$.

Note that $K_{\frac12}=\mathbb{S}^1$.

Denote by $\rho_\alpha\:\mathbb{S}^1\to\mathbb{S}^1$ 
the rotation by angle $\alpha$.
Set 
\[Z_t=\left[\begin{aligned}
             Z_t&=\rho_{\frac1{1-2\cdot t}}(K_t)&&\text{if}\ t\in[0,\tfrac12),
\\
Z_{t}&=\mathbb{S}^1&&\text{if}\ t=\tfrac12.
            \end{aligned}
\right.
\]


Prove that the family of sets $Z_t$
is a continuous in the Hausdorff topology and it does not have a section.\qeds

 The problem is suggested by Stephan Stadler.

It is instructive to check that any Hausdorff continuous family of closed sets in $\RR$ admits a continuous section.

\begin{wrapfigure}{r}{33 mm}
\begin{lpic}[t(-5 mm),b(0 mm),r(0 mm),l(0 mm)]{pics/comb(1)}
\end{lpic}
\end{wrapfigure}

\parbf{Spaces with isometric balls.} 
Take the upper half-plane and cut it along a ``dyadic comb'' shown on the diagram. 
Equipp the obtained space with the intrinsic metric induced from the $\ell_\infty$-norm on the plane. 
(Few concentric balls in this metric are shown on the diagram.)

To construct a comb we have to make infinite number of desisions --- for each tooth size we have to dicide on which side from the origin will be closest tooth of that size. 

Show that the needed spaces $X$ and $Y$ can be obtained this way for a pair of different combs.
