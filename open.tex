%%%%%%%%%%%%%%%%%%%%%%%%%%%%%%%%%%%%%%%%%

\chapter{Open puzzles}

Puzzles are contagious problems --- they might be pleasant or unpleasant, but it is hard to stop to think about them.
It does not really matter if the puzzle has a solution, altho knowing that a simple solution exists might make it more contagious.
Often, when I think about a puzzle, I see many things to grip, but I do not have enuf hands.

A book is not a good format for open puzzles --- most likely status of half of the problems will change in a year,
but I include it anyway. 
Send me an email if you solve one of the problems, or have a good idea how to solve it.


\subsection*{Group generated by central symmetries}
\label{Group generated by central symmetries}

\begin{pr}
Consider Lobachevsky space of sufficiently large dimension.
Does it admit a discrete cocompact isometric action generated by central symmetries?
\end{pr}

A closely related statement proved by Ernest Vinberg \cite{vinberg, vinberg-strong};
it says that there are no such actions generated by reflections.

\subsection*{Milnor's cartography problem}
\label{Milnor's cartography problem}

\begin{pr}
Let $\Omega$ be a round disc of radius $\alpha<\tfrac\pi2$ on the unit sphere $\mathbb{S}^2$
and $\Omega'$ be a convex domain in $\mathbb{S}^2$ with the same area as $\Omega$. Show that there is $(1,\tfrac{\alpha}{\sin\alpha})$-bi-Lipschitz map from $\Omega'$ to the plane.
\end{pr}

The question was asked by John Milnor \cite{milnor-cartography}.
The polar coordinates with the origin at the center of $\Omega$ provide an example of $(1,\tfrac{\alpha}{\sin\alpha})$-bi-Lipschitz map and this constant $\tfrac{\alpha}{\sin\alpha}$ is optimal.

If in the formulation of the problem, one exchanges ``area''  above to ``perimeter'',
then the answer is yes.
It can be proved using the idea of Gary Lawlor \cite{lawlor}.

\subsection*{Convex hull in CAT(0)}
\label{Convex hull in CAT(0)}

The \emph{closed convex hull} of a subset $K$ in a metric space is defined as the intersection of all closed convex sets containing $K$.


\begin{pr}
Let $X$ be complete $\mathop{\rm CAT}(0)$-space and $K\subset X$ be a compact subset.
Is it true that closed  convex hull of $K$ is compact?
\end{pr}


Note that the space is not assumed to be locally compact; otherwise the statement is evident.
I believe there is a counterexample, maybe even in case of negatively pinched curvature (in the sense of Alexandrov).

This problem was stated by Eva Kopeck\'a and Simeon Reich in \cite[see][]{kopecka-reich}, 
but the problem is known in mathematical folklore for long time. 



\subsection*{Two-convexity}
\label{Two-convexity}

Note that an open connected set $\Omega\subset\mathbb R^n$ is convex (in the usual sense) if together with two sides of triangle $\Omega$ contains whole triangle.
This observation motivates the definition of two-convexity used in the following problem.


\begin{pr}
Let $\Omega$ be an open set in $\mathbb R^n$.
Assume that $\Omega$ is two-convex; that is, if 3 faces of a 3-simplex belong to $\Omega$ then whole simplex lies in $\Omega$.

Is it true that any component of intersection of $\Omega$ with any 2-plane is simply connected?
\end{pr}


If the boundary of $\Omega$ is a smooth hypersurface then the answer is ``yes''.
Indeed, the above property implies that at most one principle curvature of the boundary is negative.
Then the statement follows easily from a Morse-type argument \cite[see Lefschetz theorem in Section~$\tfrac12$ in][]{gromov-SGMC}.
This argument, shows that any component of intersection of $\Omega$ with any affine subspace (not necessary 2-dimensional) is simply connected.

The answer is ``yes'' if $n=3$; in this case one can mimic the Morse-type argument.

The statement would follow if one could approximate any $\Omega$ by two-convex domains with smooth boundary. However, the example below shows that such approximation does not exist for $n\ge 4$.


\parit{Example.} We will construct two-convex simply connected open set $\Omega$ in $\mathbb R^4$ such that intersection $L_{t_0}$ of $\Omega$ with some hyperplane is not simply connected.
(Note that this not a counterexample;
it only shows that it is impossible to prove it using smoothing, as indicated in the comments.)

Set 
$$\Pi=\{\,(x,y,z,t)\in\mathbb R^4\mid\,y< x^2\}.$$
Let $\Pi'$ be the image of $\Pi$ under a generic rotation of $\mathbb R^4$,
say $(x,y,z,t)\z\mapsto(z,t,x,y)$.

Note that $\Pi$ is open and two-convex.
Therefore $\Omega=\Pi\cap \Pi'$ is also open two-convex set.
One can choose coordinates so that $\Omega$ is an epigraph for a function $f\colon\mathbb R^3\to\mathbb R$ like
$$f=\max\{\alpha_1-\beta_1^2,\alpha_2-\beta_2^2 \},$$
where $\alpha_i$ and $\beta_i$ are linear functions.
In particular $\Omega$ is contactable.

Let $L_{t_0}$ be the intersection of $\Omega$ with hyperplane $t=t_0$;
it is a complement of two convex parabolic cylinders in general position. 
If these cylinders have a point of intersection then $\pi_1 L_{t_0}=\mathbb Z$.
 
In particular, the function $f$ cannot be approximated by smooth functions which Hessian has at most one negative eigenvalue value at all points.
 
\subsection*{Homeomorphism of torus}

Let $\TT$ denotes the 2-dimensional torus.

A homeomorphism $h\:\TT\to \TT$ is called \emph{simple} if it supported on en embedded open disc;
that is, there is an embedded open disc $\Delta\subset \TT$ such that $h$ is identity outside of $\Delta$.

The \emph{disc norm} of a homeomorphism $h\:\TT\to \TT$ is defined as the least number simple homeomorphisms $h_1,\dots,h_n$ such that
\[h=h_1\circ\dots\circ h_n.\]

\begin{pr}
Is there a homeomorphism of the 2-dimensional torus with arbitrary large finite disc norm?
\end{pr}
I learned this problem from Dmitri Burago.

\subsection*{Open projection}

Let us denote by $\DD^n$ a closed $n$-dimensional ball.

Recall that a map is called open if the image of any open set is open.

\begin{pr}
Is it possible to construct an embedding $\DD^4\hookrightarrow \mathbb{S}^2\times
\mathbb R^2$
such that the projection  $\DD^4\to \mathbb{S}^2$ is an open map?
\end{pr}


It is easy to construct an embedding $\DD^3\hookrightarrow \mathbb{S}^3$ such that
its composition with Hopf fibration $f_3:\DD^3\to \mathbb{S}^2$ is open.
Therefore there is an open map $f_3\:\DD^3\to\mathbb{S}^2$.

Composing $f_3$ with any open map $\DD^4\to \DD^3$,
one gets an open map $f_4:\DD^4\to \mathbb{S}^2$.

The map $f_3$ is not a projection of embedding  $\DD^3\hookrightarrow \mathbb{S}^2\times\mathbb R$.
(We have $f_3^{-1}(p)=\mathbb{S}^1$ for some $p\in \mathbb{S}^2$ and $\mathbb{S}^1$ cannot be embedded in $\mathbb R$.)  

The question was asked by $\eps$-$\delta$ (an anonymous mathematician)  \cite{epsilon-delta}.

\subsection*{Braid space}

Denote by $\mathbb{B}^n$ the universal covering of $\CC^n$ infinitely branching along the lines $z_i=z_j$ for $i\ne j$.
We suppose that $\CC^n$ is equipped with canonical Euclidean metric, 
and $\mathbb{B}^n$ is equipped with lifted length metric.

In other words, if $\mathbb{W}^n$ denoted the complement of $\CC^n$ to the lines $z_i=z_j$,
then the braid space $\mathbb{B}^n$ is the completion of the universal metric covering of $\mathbb{W}^n$.

\begin{pr}
Is it true that $\mathbb{B}^n$ is $\mathrm{CAT}(0)$ for any $n$? 
\end{pr}

The answer is yes for $n\le 3$.
While the case $n=2$ is trivial.
The case $n=3$ follows from the theorem of Ruth Charney and Michael Davis about branched covering of 3-sphere \cite{charney-davis};
another proof is given by Dmitri Panov and the author \cite{panov-petrunin-ramification}


\subsection*{Quotients of Hilbert space}

\begin{pr} Suppose $R$ is a compact simply connected Riemannian manifold that is isometric to a quotient of the Hilbet space by group of isometris (or more generally $R$ is the target of Riemannian submersion from a Hilbet space).
Is it true that $R$ is isometric to a double quotient? That is, is it true that $R$ is a quotient of compact Lie group $G$ by a group of isometries?
 
\end{pr}

Any double quotient can appear as a quotient of the Hilbert space by group of isometries.
The following proof is taken from \cite{lebedeva-petrunin-zolotov} and it this statement was suggested by Alexander Lytchak.
Another proof follows from the construction of Chuu-Lian Terng and Gudlaugur Thorbergsson given in \cite[Section 4]{terng-thorbergsson}.

Denote by $G^n$ the direct product of $n$ copies of $G$.
Consider the map $\phi_n\:G^n\to G/\!\!/H$ defined by
\[\phi_n\:(\alpha_1,\dots,\alpha_n)\mapsto [\alpha_1\cdots\alpha_n]_H,\]
where $[x]_H$ denotes the $H$-orbit of $x$ in $G$.

Note that $\phi_n$ is a quotient map for the action of $H\times G^{n-1}$ on $G^n$ defined by
\[(\beta_0,\dots,\beta_n)\cdot(\alpha_1,\dots,\alpha_n)=(\gamma_1\cdot \alpha_1\cdot\beta_1^{-1},\beta_1\cdot\alpha_2\cdot\beta_2^{-1},\dots,\beta_{n-1}\cdot\alpha_n\cdot\beta_n^{-1}),\]
where $\beta_i\in G$ and $(\beta_0,\beta_n)\in H<G\times G$.

Denote by $\rho_n$ the product metric on $G^n$ rescaled with factor $\sqrt{n}$.
Note that the quotient $(G^n,\rho_n)/(H\times G^{n-1})$ is isometric to $G/\!\!/H\z=(G,\rho_1)/\!\!/H$.
Let $\phi_n\:(G^n,\rho_n)\to G/\!\!/H$ be the corresponding quotient map; 
clearly $\phi_n$ is a submetry. 

As $n\to\infty$ the curvature of $(G^n,\rho_n)$ converges to zero and its injectivity radius goes to infinity.
Therefore the ultra-limit of $(G^n,\rho_n)$ with marked identity element is a Hilbert space $\HH$ and the submetries $\phi_n$ ultra-converge to a submetry $\phi\:\HH\to G/\!\!/H$.
It remains to apply 


\subsection*{Nested convex surfaces}

\begin{pr}
Describe the Riemannian metrics on $\mathbb{S}^n$ that are isometric to a smooth nested convex surfaces;
that is, a complete smooth convex hypersurface in a complete smooth convex hypersurface in ... in a Euclidean space.
\end{pr}

If $n=2$, then by Alexandrov embedding theorem these are all Riemannian metrics with nonnegative curvature.

Direct calculations show that the metric has \emph{nonnegative cosectional curvature};
the latter means that at each point the curvature tensor can be expressed as a linear combination of the curvature tensors of $\mathbb{S}^2\times \RR^{n-2}$ with positive coefficients.
It might happen that any metric with nonnegative cosectional curvature on $\mathbb{S}^n$ is isometric to a nested convex surface.
If one drops completeness from the definition of nested convex surface
and assume that the cosectional curvature is strictly positive,
then we get the answer is yes; see \cite{petrunin-poly}.



\subsection*{Maximal finite subgroups}

\begin{pr}
Suppose that a group $\Gamma$ acts by isometrically and totally discontinuously on the Euclidean space.
Show that $\Gamma$ contains at most $2^n$ maximal finite subgroups up to conjugation.
\end{pr}

Let $F$ be a maximal finite subgroup in $\Gamma$.
Note that the fixed points of $F$ is an affine subset of $\mathbb{E}^n$ and it is mapped to a singular set $S_F$ in the quotient space $X=\mathbb{E}^n/\Gamma$.
Moreover, (1) a conjugation of $F$ does not change the set $S_F$ and, since $F$ is maximal, (2) $S_F$ is a \emph{simple} singular set 
that is, $S_F$ does not have a proper singular subsets.
So instead of counting subgroups up to conjugation, we may count simple singular sets in $X$.

It is expected that the maximal number of simple singular set appears if all of them have dimension 0;
that is, each set $S_F$ is an isolated point in $X$.
If this is indeed true, then the statement would follow.
The latter was shown by  Nina Lebedeva \cite{lebedeva};
that is, she proved that the number of such points indeed can not exceed $2^n$.
This is done by modifying a solution of the following problem of Paul Erdős given by Ludwig Danzer and Branko Gr{\"u}nbaum \cite{danzer-guenbaum}.
\emph{Show that one can chose at most $2^m$ points in $\mathbb{E}^m$ such that any triangle with the vertexes in the chosen points is acute or right.}



\subsection*{Zero-curvature along geodesics}

\begin{pr}
Let $M$ be a surface of genus 2 with a Riemannian metric of nonpositive curvature.
Show that with probability 1, a random geodesic in $M$ runs into the set of negative curvature.
\end{pr}

\subsection*{Ambrose problem}

\begin{pr}
Let $(M,g)$ be a simply connected complete Riemannian manifold and $p\in M$.
Denote by $h$ the pullback of the metric tenor $g$ to the tangent space $\T_pM$ along the exponential map $\exp_p\:\T_pM\to M$;
in general $h$ might be degenerate at some points.
Does $h$ defines the manifold $(M,g)$ and the point $p$ up to isometry? 
\end{pr}

This simple-looking question was asked a long time ago by Warren Ambrose \cite{ambrose}.
Affirmative answers were obtained in three cases: (1) in dimension 2, (2) for analytic metrics, and (3) for generic metrics; see \cite{hebla, itoh, ardoy} and the references therein.




\subsection*{Shortcut in a connected set}

\begin{pr}
Let $p$ and $q$ be a pair of points in a compact connected subset $K\subset \RR^n$.
Is it possible to connect $p$ to $q$ by a curve $\gamma$ the length of $\gamma\backslash K$ is arbitrary small?
\end{pr}

Be aware that there are examples of compact connected set which contains no nontrivial paths; for example the so called \emph{pseudoarc}.

In the case $n=2$, an affirmative answer was obtained by Taras Banakh \cite{banakh}.
His argument is based on the following claim

\begin{itemize}
 \item Suppose $K$ is a connected compact set in the plane and $p,q\in K$.
 Then given $\eps>0$ there is a sequence of compact connected subsets $K_0,K_2,\dots K_n$
 with pairs of points $p_i,q_i\in K_i$ for each $i$ such that 
(1) $\diam K_i<\eps$ for each $i$,
(2) $p_0=p$, $q_n=q$, and (3) if $\ell_i$ denotes the line segment $[q_{i-1}q_i]$, then 
 \[\length \ell_1+\dots+\length \ell_n<\eps.\]
\end{itemize}

Assume that the claim is proved.
Let us apply it recursively to each $K_i$ choosing much smaller value $\eps$; each time we can assume that the number $n$ is taken to be minimal for the given $\eps$.
Consider the closure $L$ of the set of all points $p_i$ and $q_i$ for all iterations; it is a closed subset $L\subset K$.
By adding to $L$ the line segments $\ell_i$ for all the iterations we obtain a curve connecting $p$ to $q$ that spends arbitrary small length outside of $L$.

Whence the 2-dimensional case follows.
It remains to prove the claim.

Consider a coordinate grid $\Gamma$ on the plane with step $\tfrac\eps2$.
Since the intersection $\Gamma\cap K$ is compact, it can be covered by a finite collection of closed line segments $\{I_1,\dots,I_n\}$ in the grid such that the 
\[\sum_i\length(I_i\backslash K)<\eps.\]

Take a minimal collection of sets $\{K_i\}$ such that each set $K_i$ is a connected component of intersection of $K$ with a closed square of the grid and the union 
\[\biggl(\bigcup_i I_i\biggr)\cup\biggl(\bigcup_j K_j\biggr)\]
is a connected set containing $p$ and $q$. 
Observe that the collection $\{K_i\}$ is finite and it satisfies the contitions in the claim assuming it is ordered right.
(The choice of point $p_i$ and $q_i$ is straightforward;
the line segments $\ell_j$ can be chosen to lie in the union of $\{I_i\}$.)



