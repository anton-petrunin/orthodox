\csname @openrightfalse\endcsname
\chapter{Unsolved puzzles}

Here is a collection of puzzles that I fail to solve,
but it was pleasant to think about them.

%%%%%%%%%%%%%%%%%%%%%%%%%%%%%%%%%%%%%%%%%
\subsection*{Group generated by central symmetries}
\label{Group generated by central symmetries}

\begin{pr}
Consider Lobachevsky space of sufficiently large dimension.
Does it admit a discrete cocompact isometric action generated by central symmetries?
\end{pr}

It seems that there is no example of cocompact group of isomtries generated by finite order elements in any Lobachevsky space of sufficiently large dimension.
A closely related statement proved by Ernest Vinberg \cite[see][]{vinberg, vinberg-strong};
it says that there are no such actions generated by reflections.

\subsection*{Milnor's cartography problem}
\label{Milnor's cartography problem}

\begin{pr}
Let $\Omega$ be a round disc of radius $\alpha<\tfrac\pi2$ on the unit sphere $\mathbb{S}^2$
and $\Omega'$ be a convex domain in $\mathbb{S}^2$ with the same area as $\Omega$. Show that there is $(1,\tfrac{\alpha}{\sin\alpha})$-bi-Lipschitz map from $\Omega'$ to the plane.
\end{pr}

The problem appears in a paper of John Milnor \cite[see][]{milnor-cartography},
where it is noted that the polar coordinates with the origin at the center of $\Omega$ provide an example of $(1,\tfrac{\alpha}{\sin\alpha})$-bi-Lipschitz map and this constant $\tfrac{\alpha}{\sin\alpha}$ is optimal.

If in the formulation of the problem, one exchanges ``area''  above to ``perimeter'',
then the answer is yes. The statement follows from the idea from of Lawlor \cite[see][]{lawlor}.

\subsection*{Convex hull in CAT(0)}
\label{Convex hull in CAT(0)}

Recall that the \emph{closed convex hull} of a subset $K$ in a metric space is defined as the intersection of all closed convex sets containing $K$.


\begin{pr}
Let $X$ be complete $\mathop{\rm CAT}(0)$-space and $K\subset X$ be a compact subset.
Is it true that convex hull of $K$ is compact?
\end{pr}


Note that the space is not assumed to be locally compact; otherwise the statement is evident.
I believe there is a counterexample, maybe even in case of negatively pinched curvature (in the sense of Alexandrov).

This problem was stated by Eva Kopeck\'a and Simeon Reich in \cite[see][]{kopecka-reich}, 
but the problem is known in mathematical folklore for long time. 



\subsection*{Two-convexity}
\label{Two-convexity}

Note that an open connected set $\Omega\subset\mathbb R^n$ is convex (in the usual sense) if together with two sides of triangle $\Omega$ contains whole triangle.
This observation motivates the definition of two-convexity used in the following problem.


\begin{pr}
Let $\Omega$ be an open set in $\mathbb R^n$.
Assume $\Omega$ is two-convex; that is if 3 faces of a 3-simplex belong to $\Omega$ then whole simplex lies in $\Omega$.

Is it true that any component of intersection of $\Omega$ with any 2-plane is simply connected?
\end{pr}


If the boundary of $\Omega$ is a smooth hypersurface then the answer is ``yes''.
Indeed, the above property implies that at most one principle curvature of the boundary is negative.
Then the statement follows easily from a Morse-type argument \cite[see Lefschetz theorem in Section~$\tfrac12$ in][]{gromov-SGMC}.
In fact this argument, shows that in this case any component of intersection of $\Omega$ with any affine subspace (not necessary 2-dimensional) is simply connected.

The answer is ``yes'' if $n=3$; in this case one can mimic the Morse-type argument.

The statement would follow if one could approximate any $\Omega$ by two-convex domains with smooth boundary. However, the example below shows that such approximation does not exist for $n\ge 4$.


\parit{Example.} We will construct two-convex simply connected open set $\Omega$ in $\mathbb R^4$ such that intersection $L_{t_0}$ of $\Omega$ with some hyperplane is not simply connected.
(Note that this not a counterexample;
it only shows that it is impossible to prove it using smoothing, as indicated in the comments.)

Set 
$$\Pi=\{\,(x,y,z,t)\in\mathbb R^4\mid\,y< x^2\}.$$
Let $\Pi'$ be the image of $\Pi$ under a generic rotation of $\mathbb R^4$,
say $(x,y,z,t)\mapsto(z,t,x,y)$.

Note that $\Pi$ is open and two-convex.
Therefore $\Omega=\Pi\cap \Pi'$ is also open two-convex set.
One can choose coordinates so that $\Omega$ is an epigraph for a function $f\colon\mathbb R^3\to\mathbb R$ like
$$f=\max\{\alpha_1-\beta_1^2,\alpha_2-\beta_2^2 \},$$
where $\alpha_i$ and $\beta_i$ are linear functions.
In particular $\Omega$ is contactable.

Let $L_{t_0}$ be the intersection of $\Omega$ with hyperplane $t=t_0$;
it is a complement of two convex parabolic cylinders in general position. 
If these cylinders have a point of intersection then $\pi_1 L_{t_0}=\mathbb Z$.
 
(In particular, the function $f$ can not be approximated by smooth functions which Hessian has at most one negative eigenvalue value at all points.
 
\subsection*{Homeomorphism of torus}

Let $\TT$ denotes the 2-dimensional torus.

A homeomorphism $h\:\TT\to \TT$ is called \emph{simple} if it supported on en embedded open disc;
that is, there is an embedded open disc $\Delta\subset \TT$ such that $h$ is identity outside of $\Delta$.

The \emph{disc norm} of a homeomorphism $h\:\TT\to \TT$ is defined as the least number simple homeomorphisms $h_1,\dots,h_n$ such that
\[h=h_1\circ\dots\circ h_n.\]

The following problem is stated by Dmitri Burago.

\begin{pr}
Is there a homeomorphism of the 2-dimensional torus with arbitrary large finite disc norm?
\end{pr}



\subsection*{Open projection}

Let us denote by $\DD^n$ closed $n$-dimensional ball.

Recall that a map is called open if the image of any open set is open.

\begin{pr}
Is it possible to construct an embedding $\DD^4\hookrightarrow \mathbb{S}^2\times
\mathbb R^2$
such that the projection  $\DD^4\to \mathbb{S}^2$ is an open map?
\end{pr}


It is easy to construct an embedding $\DD^3\hookrightarrow \mathbb{S}^3$ such that
its composition with Hopf fibration $f_3:\DD^3\to \mathbb{S}^2$ is open.
Therefore there is an open map $f_3\:\DD^3\to\mathbb{S}^2$.

Composing $f_3$ with any open map $\DD^4\to \DD^3$,
one gets an open map $f_4:\DD^4\to \mathbb{S}^2$.

The map $f_3$ is not a projection of embedding  $\DD^3\hookrightarrow \mathbb{S}^2\times\mathbb R$.
(We have $f_3^{-1}(p)=\mathbb{S}^1$ for some $p\in \mathbb{S}^2$ and $\mathbb{S}^1$ can not be embedded in $\mathbb R$.)  
