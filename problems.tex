\documentclass[twoside]{book}
\usepackage{problems}
\usepackage{amsrefs}
\usepackage[colorlinks=true,
citecolor=black,
linkcolor=black,
anchorcolor=black,
filecolor=black,
menucolor=black,
urlcolor=black,
pdftitle={Exercises in Orthodox Geometry},
pdfauthor={Anton Petrunin}
]{hyperref}


\begin{document}
\title{Exercises in Orthodox Geometry}
\author{Edited by A.~Petrunin}

\date{}
\maketitle

\null\vfill\noindent{\begin{lpic}[t(-0 mm),b(-0 mm),r(0 mm),l(0 mm)]{pics/by-sa(0.5)}\end{lpic} This work is licensed under the Creative Commons Attribution-ShareAlike 4.0 International License. To view a copy of this license, visit http://creativecommons.org/licenses/by-sa/4.0/.}

\ 

\noindent{\tt
http://www.createspace.com/5230762 
\\
\\
ISBN-13: 978-1506028590
\\
ISBN-10: 1506028594 
}

%\begin{abstract}
%This collection is oriented to graduate students who want to learn fast simple tricks in geometry.
%\end{abstract}

\tableofcontents

\ 

The meaning of signs next to number of the problem:
\begin{itemize}
\item[$\circ$] --- easy problem;%\easy
\item[$*$] --- the solution requires at least two ideas;%\hard
\item[{\small$+$}] --- the solution requires knowledge of a theorem;%\thm
\item[$\sharp$] --- there are interesting solutions based on different ideas.%\many
\end{itemize}


\parbf{Acknowledgments.} 
Very special thanks to 
Christopher Croke,
Bogdan Georgiev,
Jouni Luukkainen,
Alexander Lytchak,
Rostislav Matve\-yev, 
Peter Petersen, 
Idzhad Sabitov,
Serge Tabachnikov 
as well as the students in my classes for their interest in this list and for correcting number of mistakes.
I'm also thankful to everyone who shared the problems
and/or took part in the discussion of this list on \href{http://mathoverflow.net/questions/8247}{math\textit{overflow}}.






%%%%%%%%%%%%%%%%%%%%%%%%%%%%%%%%%%%%%%%%%%%%%%%%%%%%%%%%%%%%
%%%%%%%%%%%%%%%%%%%%%%%%%%%%%%%%%%%%%%%%%%%%%%%%%%%%%%%%%%%%
%%%%%%%%%%%%%%%%%%%%%%%%%%%%%%%%%%%%%%%%%%%%%%%%%%%%%%%%%%%%
\chapter{Curves}

\begin{pr}{}{Geodesic for birds}\label{liberman}
Let $f\:\RR^2\to\RR$ be a $\ell$-Lipschitz function.
Let $W\subset \RR^3$ be the epigraph of $f$;
that is,
$$W=\set{(x,y,z)\in\RR^3}{z\ge f(x,y)}.$$
Equip $W$ with the induced intrinsic metric.

Show that any \hyperref[Geodesic]{\emph{geodesic}} in $W$ 
 has  \hyperref[Total curvature]{\emph{total curvature}} at most $2\cdot\ell$. 
\end{pr}

\begin{pr}{}{Spiral}\label{spiral}
Let $\gamma$ be a plane curve with strictly monotonic curvature
function. 
Prove that $\gamma$ has no self-intersections.%
\footnote{In other words, if you drive on the plane and turn the steering wheel to the right all the time then you can not come back to the same place.}
\end{pr}

{\begin{wrapfigure}[5]{o}{27 mm}
%\begin{center}
\begin{lpic}[t(-12 mm),b(0 mm),r(0 mm),l(0 mm)]{pics/moon-in-puddle(1)}
%\lbl[b]{24,41;x}
\end{lpic}
%\end{center}
\end{wrapfigure}

\begin{pr}{}{The moon in the puddle}\label{moon-in-puddle}
A smooth closed \hyperref[Simple curve]{\emph{simple}} plane curve with curvature less than $1$ bounds a figure $F$. 
Prove that $F$ contains a disc or radius~$1$.
\end{pr}



\begin{pr}{\many}{A curve in a sphere}\label{curve-in-S^2} 
Show that any closed curve on the unit sphere which intersects every equator has length at least $2\cdot\pi$.
\end{pr}

}



\begin{pr}{\many}{A spring in a tin}\label{A spring in a tin} 
Let $\alpha$ be a closed smooth immersed curve
inside a unit disc. 
Prove that the average absolute curvature of $\alpha$ is at least $1$, with
equality if and only if $\alpha$ is the unit circle possibly traversed more than once.
\end{pr}

\begin{pr}{\easy}{Surrounded area}\label{Surrounded area}
Let $\gamma_1,\gamma_2\:\mathbb S^1\to\RR^2$ be two simple closed plane curves.
Assume 
\[|\gamma_1(v)-\gamma_1(w)|\le|\gamma_2(v)-\gamma_2(w)|\]
for any $v,w\in \mathbb S^1$.
Show that the area surrounded by $\gamma_1$ does not exceed the area surrounded by $\gamma_2$. 
\end{pr}


\begin{pr}{\easy}{Convex figures}\label{Convex figures}
Consider the set of all convex figures $\mathfrak{C}$
in the plane 
equipped with Hausdorff distance.
Show that the set of smooth figures\footnote{A convex figure in the plane is said to be \emph{smooth}
if it has unique supporting line at every boundary point.}
 forms a G-delta dense subset in $\mathfrak{C}$.
\end{pr}

\begin{pr}{}{Complicated circle}\label{Complicated circle} 
Construct 
a bounded open disc in $\RR^2$ 
such that 
its boundary does not contain a 
\hyperref[Simple curve]{\emph{simple curve}}.
\end{pr}

\begin{pr}{\easy}{Fat curve}\label{Fat curve}
Construct a \hyperref[Simple curve]{\emph{simple plane curve}} with positive Lebesgue measure.
\end{pr}

\begin{pr}{}{Rectifiable curve}\label{Rectifiable curve}
Assume $X$ is a compact connected set in $\RR^2$
with finite 1-dimensional Hausdorff measure. 
Show that $X$ can be presented as the image of a rectifiable curve.
\end{pr}


\begin{pr}{}{Capture a sphere in a knot}\label{Capture a sphere in a knot}
Let $B$ be the closed unit ball in $\RR^3$
and $f\:\mathbb{S}^1\z\to \RR^3\backslash B$ be a knot.
Show that there is an ambient isotopy 
$$H_t\:\RR^3\backslash B\to \RR^3\backslash B,\ \ \ t\in [0,1],$$ 
such that $H_0=\id$,
the length of $H_t\circ f$ does not increase in $t$
and $H_1(f(\mathbb{S}^1))$ can be disjointed from $B$ by a plane.
\end{pr}

\begin{pr}{}{Linked circles}\label{linked-circles}
Suppose that two linked  simple closed curves in $\RR^3$
lie at a distance at least $1$ from each other.
Show that the length of each curve is at least $2\cdot\pi$.
\end{pr}


\begin{wrapfigure}{o}{30 mm}
%\begin{center}
\begin{lpic}[t(-6 mm),b(-1 mm),r(0 mm),l(0 mm)]{pics/tangent-eq(1)}
%\lbl[b]{24,41;x}
\end{lpic}
%\end{center}
\end{wrapfigure}


\begin{pr}{}{Oval in oval}\label{Oval in oval} %???+PIC
Consider two closed smooth strictly convex planar curves, one inside another. 
Show that there is a chord of the outer curve, which is tangent to the inner curve at its midpoint.
\end{pr}


%%%%%%%%%%%%%%%%%%%%%%%%%%%%%%%%%%%%%%%%%%%%%%%%%%%%%%%%%%%%
%%%%%%%%%%%%%%%%%%%%%%%%%%%%%%%%%%%%%%%%%%%%%%%%%%%%%%%%%%%%
%%%%%%%%%%%%%%%%%%%%%%%%%%%%%%%%%%%%%%%%%%%%%%%%%%%%%%%%%%%%
\chapter{Surfaces}

\begin{pr}{\easy}{Convex hat}\label{Convex hat}
Let $\Sigma$ be a smooth closed convex surface 
in $\RR^3$ 
and $\Pi$ be a plane which cuts from $\Sigma$ a disc $\Delta$.
Assume that the reflection of $\Delta$ in $\Pi$ lies inside $\Sigma$.
Show that $\Delta$ is \emph{convex} in the intrinsic metric  of $\Sigma$;
that is, 
if the ends of a minimizing geodesic in $\Sigma$ 
lie in $\Delta$ then whole geodesic lies in $\Delta$.
\end{pr}

\begin{pr}{}{Unbended geodesic}\label{Unbended geodesic} %???+PIC
Let $\Sigma$ be a smooth closed convex surface 
in $\RR^3$ 
and $\gamma\:[0,\ell]\z\to \Sigma$ be a unit speed minimizing geodesic in $\Sigma$.
Set $p=\gamma(0)$, $q=\gamma(\ell)$ and 
$$p_t=\gamma(t)-t\cdot\gamma'(t),$$ 
where $\gamma'(t)$ denotes the velocity vector of $\gamma$ at $t$.

Show that for any $t\in (0,\ell)$,
one \emph{can not see}  $q$ from $p_t$;
that is, the line segment $[p_tq]$ intersects $\Sigma$ at a point distinct from $q$.
\end{pr}

\begin{pr}{}{Corkscrew geodesic}\label{Corkscrew geodesic}
Given a line $\ell$ in $\RR^3$
construct a closed convex body $K$  
with a minimizing geodesic $\sigma\:[a,b]\to\partial K$ in the surface of $K$ 
which \emph{rotates 1000 times} around $\ell$; 
that is, if $\phi(t)$ is continuous azimuth-function of $\sigma(t)$ in the cylindrical coordinates with axis at $\ell$ then $|\phi(b)-\phi(a)|=2000\cdot\pi$.
\end{pr}

{

\begin{wrapfigure}{r}{21 mm}
\begin{lpic}[t(-8 mm),b(-4 mm),r(0 mm),l(0 mm)]{pics/long-geodesic(1)}
\end{lpic}
\end{wrapfigure}

\begin{pr}{}{Long geodesic}\label{Long geodesic}
Assume that the surface of convex body $B$ in $\RR^3$
admits an arbitrary long closed simple geodesic.
Show that $B$ is a tetrahedron with equal opposite sides.
\end{pr}

}

\begin{pr}{}{Simple geodesic}\label{Simple geodesic}
Let $\Sigma$ be a complete unbounded convex surface in $\mathbb R^3$.
Show that there is a two-sided infinite geodesic in $\Sigma$ with no self-intersections.
\end{pr}


\begin{pr}{}{A minimal surface}%
\label{min-surf}
Let $\Sigma$ be a \hyperref[Minimal surface]{\emph{minimal surface}} in $\RR^3$ which has boundary on a unit sphere.
Assume $\Sigma$ passes through the center of the sphere.
Show that area of $\Sigma$ is at least $\pi$.
\end{pr}

\begin{pr}{\hard}{Half-torus}\label{half-torus}
Consider torus $T$;
that is, a surface of revolution generated by revolving a circle in $\RR^3$ about an axis coplanar with the circle.
Let $\gamma\subset T$ be one of the circles in $T$ which separates positive and negative curvature\footnote{The circle $\gamma$ has to be tangent to a plane}
and $\Omega$ be an neighborhood of $\gamma$ in $T$.

%PIC???

Assume $\iota\:\Omega\to\RR^3$ 
is a smooth length-preserving embedding which is sufficiently close to the identity map.
Show that $\iota(\gamma)$ is a congruent to $\gamma$.
\end{pr}



\begin{pr}{}{Asymptotic line}\label{asymptotic-line}
Let $\Sigma\subset \RR^3$ be the graph $z\z=f(x,y)$
of smooth function $f$ 
and $\gamma$ be a closed smooth \hyperref[Asymptotic line]{\emph{asymptotic line}} in $\Sigma$.
Assume $\Sigma$ is \hyperref[Saddle surface]{\emph{strictly saddle}} in a neighborhood of $\gamma$.
Prove that the projection of $\gamma$ to $x y$-plane can not be star-shaped.
\end{pr}


\begin{pr}{\easy}{Non-contractible geodesics}\label{torus}
Give an example of a non-flat metric on $2$-torus such that it has no contractible geodesics.
\end{pr}

\begin{pr}{\hard}{The last problem of Poincar\'e}\label{The last problem of Poincare}
Let $f\:\CC\to \CC$ be an area preserving homeomorphism
such that 
\[f(z)
=
\left[
\begin{aligned}
&z-i&&\text{if}&&\Re(z)\le -1,
\\
&z+i&&\text{if}&&\Re(z)\ge 1.
\end{aligned}
\right.
\] 
and $f(z+i)=f(z)+i$ for any $z\in\CC$.

Show that $f$ has a fixed point.
\end{pr}


\begin{pr}{}{Periodic asymptote}\label{Asymptotic geodesic}
Let $\Sigma$ be a smooth surface with non-positive curvature
and $\gamma$ be a geodesic in $\Sigma$.
Assume that $\gamma$ is not periodic
and the curvature of $\Sigma$ vanish at every point of $\gamma$.
Show that $\gamma$ does not have a periodic asymptote;
that is, there is no periodic geodesic $\delta$ such that the distance from $\gamma(t)$ to $\delta$  converges to $0$ as $t\to\infty$. 
\end{pr}

\begin{pr}{}{Immersed surface}\label{Immersed surface}
Let $\Sigma$ be a smooth connected immersed surface in $\RR^3$ with strictly positive Gauss curvature and nonempty boundary $\partial\Sigma$.
Assume $\partial\Sigma$ lies in a plane $\Pi$
and whole $\Sigma$ lies on one side from $\Pi$.
Prove that $\Sigma$ is an embedded disc.
\end{pr}

\begin{pr}{}{Two discs}\label{Two discs}
Let $\Sigma_1$ and $\Sigma_2$ be two smoothly embedded open discs in $\mathbb R^3$ 
which have a common closed smooth curve $\gamma$.
Show that there is a pair of points  $p_1\in \Sigma_1$ and $p_2\in \Sigma_2$ with parallel tangent planes.
\end{pr}




\chapter{Comparison geometry}


\begin{pr}{\easy}{Geodesic hypersurface}\label{Geodesic hypersurface}
Prove that if a compact connected positively curved manifold $M$ admits a totally geodesic embedded hypersurface then $M$ or its double cover is homeomorphic to the sphere.
\end{pr}


\begin{pr}{}{If convex then embedded}\label{If convex then embedded} Let $M$ be a complete simply connected Riemannian manifold with non-positive curvature with dimension at least $3$.
Prove that any  immersed locally convex
compact hypersurface in $M$ is embedded.
\end{pr}

\begin{pr}{\hard}{Immersed ball}\label{Immersed ball}
Prove that any immersed locally convex
hypersurface $\iota\:\Sigma\looparrowright M$
in a compact positively curved manifold $M$ of dimension $m\ge 3$, is the boundary of an immersed ball. 
That is, there is an immersion of a closed ball $f\:\bar B^m\looparrowright M$ and a diffeomorphism $h\:\Sigma\to\partial \bar B^m$
such that $\iota=f\circ h$.
\end{pr}

\begin{pr}{}{Almgren's inequalities}\label{almgren} 
Let $\Sigma$ be a closed $n$-dimensional 
\hyperref[Minimal surface]{\emph{minimal surface}} 
in $\mathbb{S}^m$.
Prove that
$\vol_n \Sigma\ge \vol_n \mathbb{S}^n$.
\end{pr}


\begin{pr}{}{Hypercurve}\label{codim=2}
Let $M^m\hookrightarrow \RR^{m+2}$ be a closed smooth $m$-dimensional
submanifold and let  $g$ be the  induced Riemannian metric on $M^m$.
Assume that sectional curvature of $g$ is positive.
Prove that \hyperref[Curvature operator]{curvature operator} of $g$ is positive definite.
\end{pr}

\begin{pr}{}{Horosphere}\label{Horosphere} Let $M$ be a complete simply connected manifold with negatively pin\-ched sectional curvature\footnote{that is $-a^2\le K\le -b^2$, for fixed constants $0<a<b$ and the curvature $K$ in any sectional  direction of $M$}. 
And let $\Sigma\subset M$ be an horosphere%
\footnote{that is, $\Sigma$ is a level set of a 
\hyperref[Busemann function]{\emph{Busemann function}}
in $M$} in $M$.
 Prove that
$\Sigma$ with the  induced intrinsic metric has \hyperref[Polynomial volume growth]{\emph{polynomial volume growth}}.
\end{pr}

\begin{pr}{}{Minimal spheres}\label{Minimal spheres}
Show that a 
$4$-dimensional
compact 
positively curved 
Riemannian manifold 
can not contain infinite number of  mutually
 \hyperref[Equidistant subsets]{\emph{equidistant}} \hyperref[Minimal surface]{\emph{minimal}} 2-spheres.
\end{pr}

\begin{pr}{\hard}{Geodesic immersion}
\label{Geodesic immersion}
Let $(M,g)$ be a simply connected positively curved manifold and $\iota\:N\looparrowright M$ be a totally geodesic immersion.
Assume that 
\[\dim N>\tfrac 12\cdot \dim M.\]
Prove that $\iota$ is an embedding.
\end{pr}


\begin{pr}{\thm}{Positive curvature and symmetry}\label{kleiner-hopf} 
Prove that effective isometric $\mathbb S^1$-action 
on a $4$-dimensional positively curved closed Riemannian manifold  
has at most $3$ isolated fixed points.
\end{pr}

\begin{pr}{}{Energy minimizer}\label{Energy minimizer}
Show that the identity map on $\RP^m$ is 
\hyperref[Energy functional]{\emph{energy}}
minimizing in its homotopy class.
Here we assume that $\RP^m$ is equipped with canonical metric.
\end{pr}

\begin{pr}{\thm}{Curvature vs. injectivity radius}\label{scalar-curv} 
Let $(M,g)$ be a closed 
Riemannian $m$-dimensional manifold.
Assume average of sectional curvatures of $(M,g)$ is $1$. 
Show that the injectivity radius of $(M,g)$ is at most $\pi$.
\end{pr}

\begin{pr}{}{Almost flat manifold}\label{almost-flat}
Show that for any $\eps>0$ there is $m=m(\eps)$ such that there is a compact
$m$-dimensional manifold $M$ which admits a Riemannian metric with diameter $\le 1$ and sectional
curvature $|K|<\eps$,
but does not admit a finite covering by a \hyperref[Nil-manifolds]{\emph{nil-manifold}}.
\end{pr}

\begin{pr}{\easy}{Lie group}\label{lie-nonneg}
Show that the space of non-negatively curved left invariant metrics 
on a given compact Lie group is contractible.
\end{pr}

\begin{pr}{\many}{Polar points} \label{milka-polar} Let $(M,g)$ be a compact Riemannian manifold with sectional curvature $\ge 1$. 
Prove that for any point $p\in M$ there is a point $p^*\in M$ such that 
\[|p-x|_g+|x-p^*|_g\le \pi\]
for any $x\in M$.
\end{pr}

\begin{pr}{\hard}{Isometric section}\label{Isometric section}
Let $M$ and $W$ be compact Riemannian manifolds,
$\dim W>\dim M$
and $s\:W\to M$ be a Riemannian submersion.
Assume that $W$ has positive sectional curvature.
Show that $s$ does not admit an isometric section;
that is, there is no isometric embedding $\iota\:M\hookrightarrow W$ such that $s\circ\iota(p)=p$ for any $p\in M$.
\end{pr}

\begin{pr}{}{Warped product}\label{Warped product}
Assume $M$ is an oriented 3-dimensional Riemannian manifold with positive scalar curvature 
and $\Sigma\subset M$ is an oriented smooth hypersurface which is area minimizing in its homology class.

Show that there is a positive smooth function $f\:\Sigma\to \RR$
such that the \hyperref[def:Warped product]{\emph{warped product}} $\mathbb S^1\times_f \Sigma$
has positive scalar curvature;
here $\Sigma$ is equipped with the Riemannian metric
induced from $M$.
\end{pr}



\begin{pr}{\many}{No approximation}\label{No approximation}
Prove that if $p\not=2$ then $\RR^m$ 
equipped with the metric induced by the $\ell^p$-norm 
can not be a
Gromov--Hausdorff limit of
Riemannian $m$-dimensional manifolds $(M_n,g_n)$ such that $\Ric_{g_n}\z\ge C$ for some fixed constant $C\in\RR$.
\end{pr}


\begin{pr}{}{Area of spheres}\label{Area of spheres}
Let $M$ be a complete non-compact Riemannian manifold manifold 
with non-negative Ricci curvature and $p\in M$.
Then there is $\eps>0$ such that 
$$\area\left[\partial B(p,r)\right]>\eps$$
for all sufficiently large $r$.
\end{pr}

\begin{pr}{}{Curvature hollow}\label{Curvature hollow}
Construct a Riemannian metric $g$ on $\RR^3$ 
which is Euclidean outside of an open bounded set $\Omega$ 
and scalar curvature of $g$ is negative in $\Omega$.
\end{pr}


\begin{pr}{}{Flat coordinate planes}\label{Flat coordinate planes}
Assume $g$ be a Riemannian metric on $\RR^3$,
such that the coordinate planes $x=0$, $y=0$ and $z=0$ are flat and totally geodesic.
Assume $g$ has sectional curvature $\ge 0$ or $\le 0$.
Show that in both cases $g$ is flat. 
\end{pr}

\begin{pr}{\many}{Two-convexity}\label{Two-convexity}
Let $K$ be a closed set bounded by a smooth surface $W$
in $\RR^4$.
Assume $K$ contains two coordinate planes $$\{(x,y,0,0)\in\RR^4\}\ \ 
\text{and}
\ \ \{(0,0,z,t)\in\RR^4\}$$
in its interior 
and lies in the closed $1$-neighborhood of these two planes.

Show that the complement of $K$ can not be two-convex;
that is at some point of $W$ at least two principle curvatures in the outward direction to $K$
have positive sign.
\end{pr}





\chapter{Curvature free differential geometry}



\begin{pr}{\thm}{Minimal foliation}\label{gromomorphic-curves} 
Consider $\mathbb{S}^2\times \mathbb{S}^2$ equipped with a Riemannian metric $g$ 
which is $C^\infty$-close to the product metric. 
Prove that there is a conformally equivalent metric $\lambda\cdot g$ and re-parametrization of $\mathbb{S}^2\times \mathbb{S}^2$
such that each sphere $\mathbb{S}^2\times x$ and $y\times \mathbb{S}^2$ forms a 
\hyperref[Minimal surface]{\emph{minimal surface}} 
in $(\mathbb{S}^2\times \mathbb{S}^2,\lambda\cdot g)$.
\end{pr}


\begin{pr}{\thm}{Volume and convexity}
\label{Volume and convexity} 
Let
$M$ be a complete Riemannian manifold which admits a non-constant
convex function. Prove that $M$ has infinite volume.
\end{pr}

\begin{pr}{}{Besikovitch inequality}\label{Besikovitch inequality}
Let $g$ be a Riemannian metric on a $m$-dimensional cube $Q=[0,1]^m$ such that any curve connecting opposite faces has length $\ge 1$. 
Prove that $\vol(Q,g)\ge 1$ and equality holds if and only if $(Q,g)$ is isometric to the unit cube.
\end{pr}

\begin{pr}{}{Sasaki metric}\label{pr:Sasaki metric}
Consider the tangent bundle $\T \mathbb{S}^2$ 
equipped with \hyperref[Sasaki metric]{Sasaki metric} $\hat g$ induced by a Riemannian metric $g$ on $\mathbb{S}^2$.
Show that $(\T \mathbb{S}^2, \hat g)$ lies on bounded Gromov--Hausdorff distance to the ray.
\end{pr}

\begin{pr}{}{Distant involution}\label{Distant involution}
Construct a Riemannian metric $g$ on $\mathbb{S}^3$ and involution $\iota\:\mathbb{S}^3\to\mathbb{S}^3$ such that $\vol (\mathbb{S}^3,g)$ is arbitrary small and 
\[|x\z-\iota(x)|_g>1\]
 for any $x\in\mathbb{S}^3$.
\end{pr}




\begin{pr}{\easy}{Normal exponential map}\label{Normal exponential map}
Let $M,N$ be complete connected Riemannian manifolds.
Assume $N$ is immersed into $M$.
Show that the image  of the 
\hyperref[Exponential map]{\emph{normal}} 
\hyperref[Exponential map]{\emph{exponential}} 
\hyperref[Exponential map]{\emph{map}} of $N$ is dense in $M$.
\end{pr}


\begin{pr}{}{Symplectic squeezing in the torus}\label{Symplectic squeezing in the torus}
Let 
\[\omega=dx_1\wedge dy_1+ dx_2\wedge y_2\]
be the standard symplectic form on $\RR^4$
and $\ZZ^2$ the the integer lattice in $(x_1,y_1)$ coordinate plane.

Show that arbitrary bounded domain $\Omega\subset (\RR^4,\omega)$
admits a symplectic embedding into $(\RR^4,\omega)/\ZZ^2$. 
\end{pr}


\begin{pr}{\easy}{Diffeomorphism test}\label{Diffeomorphism test}
Let $M$ and $N$ be 
complete 
$m$-dimensional
simply connected 
Riemannian manifolds.
Assume $f\:M\to N$
is a smooth map such that 
$$|df(v)|\ge |v|$$
for any tangent vector $v$ of $M$.
Show that $f$ is a diffeomorphism.
\end{pr}

\begin{pr}{}{Volume of tubular neighborhoods}\label{Volume of tubular neighborhoods}
Assume $M$ and $M'$ be isometric closed smooth submanifolds in $\RR^m$.
Show that for all small $r$ we have
$$\vol B_r(M)=\vol B_r(M'),$$
where $B_r(M)$ denotes $r$-neighborhood of $M$.
\end{pr}

\begin{pr}{\hard}{Disc}\label{Disc}
Given a big real number $L$,
construct a Riemannian metric $g$ on the disc $\mathbb D$ 
with 
\[\diam(\mathbb D,g)\le 1
\ \ 
\text{and}
\ \ 
\length \partial\mathbb D\le 1  \]
such that any null-homotopy of the boundary in $(\mathbb D,g)$ 
has a curve of length at least $L$.
\end{pr}

\begin{pr}{}{Shortening homotopy}\label{short-homotopy}
Let $M$ be a compact Riemannian manifold with diameter $D$.
Assume that for some $L>D$,
there are no geodesic loops in $M$
with length in the interval $(L-D,L+ D]$.
Show that for any path $\gamma_0$ in $(M,g)$
there is a homotopy $\gamma_t$ rel. to the ends
such that 
\begin{enumerate}[a)]
\item $\length \gamma_1<L$;
\item $\length \gamma_t\le \length \gamma_0+2\cdot D$ for any $t\in[0,1]$.
 
\end{enumerate}
\end{pr}

\begin{pr}{}{Convex hypersurface}\label{Convex hypersurface}
Let $M$ be a hypersurface 
in a closed Riemannian $m$-dimensional manifold $W$.
Assume $M$ is geodesic and convex
and its injectivity radius is at least $1$.
Show that there is a point in $W$  which lies on the distance at least  
$\frac{1}{2\cdot(m+1)}$ from $M$.
\end{pr}

\begin{pr}{}{Almost constant function}\label{Almost constant function}
Assume $\eps>0$ is given.
Show that there is a positive integer $m$ such that
for any closed $m$-dimensional Riemannian manifold $M$
and any smooth $1$-Lipschitz function $f\:M\to\RR$ the following holds.
\begin{itemize}
\item For a random unit-speed geodesic $\gamma$ in $M$ 
the event 
\[|f\circ\gamma(0)-f\circ\gamma(1)|>\eps\]
happens with probability at most $\eps$.
\end{itemize}
Here \emph{random} means that $\gamma'(0)$ takes the random value in the unit tangent bundle of $M$ for the natural choice of distribution.
\end{pr}

\begin{pr}{\thm}{Bounded curvature}\label{Bounded curvature}
Set 
\[\Phi(x)=1000^{1000\cdot (x+1000)}.\]

Denote by $\mathcal{R}$ the space of 
all Riemannian metrics on $\mathbb S^5$
with absolute value of sectional curvature at most $1$,
and injectivity radius at least $1$.

Show that any metric $g_1\in \mathcal{R}$ 
can be connected to the canonical metric $g_0$ on $\mathbb S^5$
by a continuous family of metrics $g_t$ where $t\in[0,1]$.

Show that there is a metric $g_1$ 
such that 
for any family $g_t$ as above
\[\max_{t\in[0,1]}\{\vol(g_t)\}
>
\Phi(\vol(g_1)).\]
\end{pr}






\chapter{Metric geometry}


\begin{pr}{\easy}{Non-contracting map}\label{Noncontracting map}
Let $K$  be a compact metric space and
\[f\:K\z\to K\] 
be a non-contracting map.
Prove that $f$ is an isometry.
\end{pr}




\begin{pr}{}{Embedding of a compact}\label{compact} 
Prove that any compact metric space 
is isometric to 
a subset of a compact \hyperref[Length-metric space]{\emph{length-metric spaces}}.
\end{pr}

\begin{pr}{\easy}{Horo-compactification}\label{Horo-compactification}
Let $X$ be a metric space.
Denote by $C(X,\RR)$ the space of continuous real-valued functions on $X$ equipped with compact open topology.

Fix a point $z_0$.
Given a point $z\in X$, let $f_z\in C(X,\RR)$ be the function defined as 
\[f_z(x)=|z-x|_X-|z-x_0|_X.\]
Let $F_X\:X\to C(X,\RR)$ be the map 
defined as $F\:z\mapsto f_z$.

Construct a proper metric space $X$
such that $F_X$ is not an embedding.

Show that there are no such examples among proper length-metric spaces.
\end{pr}




\begin{pr}{}{Disc and 2-sphere}\label{2-sphere is far from a ball}
Show that there is no sequence of Riemannian metrics on
$\mathbb{S}^2$ which converge in Gromov--Hausdorff topology to the unit disc.
\end{pr}


\begin{pr}{}{Ball and 3-sphere}\label{3-sphere is close to a ball}
Construct a sequence of Riemannian metrics on $\mathbb{S}^3$ 
which converges in Gromov--Hausdorff topology 
to the unit ball in $\RR^3$.
\end{pr}

\begin{pr}{\easy}{Macro-dimension}\label{macrodimension} 
Let $M$ be a simply connected Riemannian manifold with the following property: 
any closed curve is null-homotopic 
in its own  1-neighborhood. 
Prove that \hyperref[Macro-dimension]{\emph{the macro-dimension of $M$ on the scale $100$}} is at most $1$.
\end{pr}


{\begin{pr}{\hard}{No short embedding}\label{weird-metric} 
Construct a length-metric $d$ on $\RR^3$,
such that for any open set $U\subset  \RR^3$,
there is no \hyperref[Short map]{\emph{short}} embeddings $(U,d)\to \RR^3$,
where $\RR^3$ equipped with the canonical metric.
\end{pr}

\begin{pr}{\thm}{Sub-Riemannian sphere}\label{sub-Riemannian} 
Prove that any \hyperref[Sub-Riemannian metric]{\emph{sub-\hskip0mm Riemannian metric}} 
on the $\mathbb{S}^m$ is isometric to the intrinsic metric of a hypersurface in $\RR^{m+1}$.
\end{pr}

\begin{pr}{\thm}{Length-preserving map}\label{two2one} 
Show that there is no \hyperref[Length-preserving map]{\emph{length-preserving map}} $\RR^2\to \RR$.
\end{pr}



\begin{pr}{}{Hyperbolic space}\label{Hyperbolic space}
Construct a \hyperref[Quasi-isometry]{\emph{quasi-isometry}}
from the hyperbolic $3$-space 
to a subset 
of the product of two hyperbolic planes.
\end{pr}

\begin{pr}{}{Fixed segment}\label{Fixed segment}
Let $\rho(x,y)=\|x-y\|$ be a metric on $\RR^m$ induced by a norm $\|{*}\|$.

Assume that $f\:(\RR^m,\rho)\to(\RR^m,\rho)$ is an isometry which fixes two distinct point.
Show that $f$ fixes the line segment between them.
\end{pr}

\begin{pr}{}{Pogorelov's construction}\label{Pogorelov's construction}
Let $\mu$ be a regular centrally symmetric finite measure on $\mathbb{S}^2$ which is positive on every open set.
Given two points $x,y\in \mathbb{S}^2$,
set 
\[\rho(x,y)=\mu[B(x,\tfrac \pi2)\backslash B(y,\tfrac\pi2)].\]

Show that $\rho$ is a length-metric on $\mathbb{S}^2$
and moreover, geodesics in this metric formed by arcs of grate circles.
\end{pr}

\begin{pr}{}{Straight geodesics}\label{Straight geodesics}
Let $\rho$ be a length-metric on $\RR^m$, 
which is bi-Lipschitz equivalent to the canonical metric.
Assume that every \hyperref[Geodesic]{\emph{ge\-o\-de\-sic}} $\gamma$ in $(\RR^d,\rho)$ is linear 
(that is, $\gamma(t)=v+w\cdot t$ for some $v,w\in\RR^m$).
Show that $\rho$ is induced by a norm on $\RR^m$.
\end{pr}

\begin{pr}{\thm}{A homeomorphism near quasi-isometry}\label{hom-near-QI} 
Let $f\:\RR^m\to\RR^m$ be a \hyperref[Quasi-isometry]{\emph{quasi-isometry}}.
Show that there is a (bi-Lipschitz) homeomorphism 
$h\:\RR^m\to\RR^m$ on a bounded distance from $f$;
that is, there is 
$$|f(x)-h(x)|\le C$$
for any $x\in\RR^m$ and a real constant $C$.
\end{pr}

\begin{pr}{}{A family of sets with no section}\label{hausdorff-section} 
Construct a one parameter family of closed sets $C_t$ in $\mathbb{S}^1$, $t\in [0,1]$
which is continuous in Hausdorff topology, 
but which does not admit a \emph{section};
that is, there is no continuous 
map $c\:[0,1]\to \mathbb{S}^1$ such that $c(t)\in C_t$ for any $t$.
\end{pr}






\chapter{Actions and coverings}


\begin{pr}{}{Bounded orbit}\label{Bounded orbit} Let $X$ be a 
\hyperref[Proper metric space]{\emph{proper metric space}} 
and $\iota\:X\to X$ is an isometry.
Assume that for some $x\in X$, the the sequence $x_n\z=\iota^n(x)$, $n\in\ZZ$ has a converging subsequence.
Prove that $x_n$ is bounded.
\end{pr}

\begin{pr}{}{Finite action}\label{Finite action}
Show that for any nontrivial continuous action of a finite group on the unit sphere
there is an orbit which does not lie in the interior of a hemisphere.
\end{pr}


\begin{pr}{}{Covers of figure eight}\label{figure-eight-1}
Let $(\Phi,d)$ be a ``figure eight''; 
that is,
a metric space which
is obtained by gluing together all four ends of two unit segments.

Prove that any compact \hyperref[Length-metric space]{\emph{length-metric spaces}} $K$ is a
Gromov--Hausdorff limit of a sequence of
metric covers  $(\widetilde \Phi_n, \tilde d/n)\to
(\Phi,d/n)$.
\end{pr}

\begin{pr}{\hard}{Diameter of \textit{m}-fold cover}\label{m-fold-cover}
Let $X$ be a \hyperref[Length-metric space]{\emph{length-metric space}}
and $\tilde X$ be a connected $m$-fold cover of $X$ 
equipped with induced intrinsic metric.
Prove that
$$\diam\tilde X\le m\cdot \diam X.$$

\end{pr}

\begin{pr}{\easy}{Symmetric square}\label{Symmetric square} Let $X$ be a connected topological space.
Note that $X{\times} X$ admits natural $\ZZ_2$-action by $(x,y)\mapsto (y,x)$.
Show that fundamental group of $X{\times} X/\ZZ_2$ is commutative.
\end{pr}

{
\begin{wrapfigure}[3]{o}{21 mm}
%\begin{center}
\begin{lpic}[t(-10 mm),b(-5 mm),r(0 mm),l(0 mm)]{pics/sierpinski-triangle(1)}
%\lbl[b]{24,41;x}
\end{lpic}
%\end{center}
\end{wrapfigure}

\begin{pr}{\easy}{Sierpinski triangle}\label{Sierpinski triangle} Find the homeomorphism group of Sierpinski triangle.
\end{pr}

\begin{pr}{}{Latices in a Lie group}\label{Boys and girls in a Lie group}
Let $L$ and $M$ be two discrete subgroups
of a connected Lie group $G$ and $h$ be a left
invariant metric on $G$.
Equip the groups $L$ and $M$ 
with the induced left invariant metric from $G$.
Assume $L\backslash G$ and $M\backslash G$ are compact and moreover
$$\vol(L\backslash (G,h))
=
\vol(M\backslash (G,h)).$$
Prove that there is bi-Lipschitz one-to-one mapping
(not necessarily a homomorphism)
$f\:L
\to
M$.
\end{pr}

}

\begin{pr}{\easy}{Piecewise Euclidean quotient}\label{Piecewise Euclidean quotient}
Let $\Gamma$ be a finite subgroup of $\SO(m)$; 
denote by $P$ the quotient $\RR^m/\Gamma$ equipped with induced
\hyperref[Polyhedral space]{\emph{polyhedral metric}}.
Assume $P$ is \hyperref[PL-homeomorphism]{\emph{PL-homeomorphic}} to $\RR^m$.
Show that $\Gamma$ is generated by rotations  around subspaces of codimension $2$.
\end{pr}

\begin{pr}{\easy}{Subgroups of free group}\label{Subgroups of free group} 
Show that every finitely generated subgroup of the free group 
is an intersection of subgroups of finite index.
\end{pr}

\begin{pr}{\easy}{Lengths of generators of the fundamental group}\label{Lengths of generators of the fundamental group}
Let $M$ be a compact Riemannian manifold and $p\in M$.
Show that the fundamental group $\pi_1(M,p)$
is generated by the homotopy classes of loops with length at most $2\cdot\diam M$.
\end{pr}

\begin{pr}{}{Number of generators}\label{Number of generators}
Let $M$ be a complete connected Riemannian manifold with non-negative sectional curvature.
Show that the minimal number of generators of the fundamental group $\pi_1 M$
can be bounded above in terms of $\dim M$.
\end{pr}

\chapter{Topology}

\begin{pr}{}{Immersed disks}\label{Immersed disks} 
Construct two \emph{essentially different} smooth immersions of the disk 
into the plane which coincide near the boundary. 

Two immersions $f_1,f_2\:D\looparrowright \RR^2$ are called essentially different 
if there is no diffeomorphism $h\:D\z\to D$ such that
$f_1=f_2\circ h$.
\end{pr}

\begin{pr}{\easy}{Positive Dehn twist}\label{Positive Dehn twist} Let $\Sigma$ be an oriented surface with non empty boundary.
Prove that any composition of \hyperref[Dehn twist]{\emph{positive Dehn twists}} of $\Sigma$ is not homotopic to identity \emph{rel} boundary.
\end{pr}

\begin{pr}{}{Function with no critical points}\label{Function with no critical points}
Given $m\ge 2$, construct a smooth function $f\:\RR^m\to \RR$ 
with no critical points in the unit ball $B^m$ 
such that the restriction $f|_{B^m}$ does not factor through a linear function;
that is $f|_{B^m}$ can not be presented as a composition
$\ell\circ\phi$,
where $\ell\:\RR^m\to\RR$ is a linear function 
and $\phi\:B^m\to\RR^m$ is a smooth embedding.
\end{pr}

\begin{pr}{}{Conic neighborhood}\label{Conic neighborhood}  
Let $p$ be a point in a topological space $X$.
We say that an open neighborhood $U_p\ni p$ is conic
if there is a homeomorphism from a cone
to $U_p$ which sends its vertex to $p$.
Show that any two conic neighborhoods of $p$ are homeomorphic to each other.
\end{pr}

\begin{pr}{}{No $C^0$-knots}\label{No knots}
Prove that the set of smooth embeddings $f\:\mathbb{S}^1\z\to\RR^3$ equipped with $C^0$-topology 
forms a connected space.
\end{pr}

\begin{pr}{}{Stabilization}\label{Simple stabilization}
Construct two compact subsets $K_1, K_2\subset\RR^2$ such that
$K_1$ is not homeomorphic to $K_2$, but $K_1\times[0,1]$ is homeomorphic to $K_2\z\times[0,1]$.
\end{pr}

\begin{pr}{}{Isotropy}\label{Isotropy}
Let $K_1$ and $K_2$ be compact subsets of the coordinate subspace $\RR^m$ in $\RR^{2\cdot m}$.
Show that there is a homeomorphism 
\[h\:\RR^{2\cdot m}\z\to \RR^{2\cdot m}\] 
such that $K_2=h(K_1)$.
Moreover, $h$ can be chosen to be isotopic to the identity map.
\end{pr}



\begin{pr}{}{Homeomorphism of cube}\label{Homeomorphism of cube}
Let $\square^m$ be a cube in $\RR^m$.
Assume that a homeomorphism $h\:\square^m\to\square^m$ sends each face of $\square^m$ to itself.
Extend $h$ to a homeomorphism $f\:\RR^m\to\RR^m$ which coincides with the identity map outside of a bounded set.    
\end{pr}


\begin{pr}{\easy}{Finite topological space}\label{Finite topological space}
Given a finite topological space $F$ 
construct a finite simplicial complex $K$
whic admits a weak homotopy equivalence  $K\to F$. 
\end{pr}

\begin{pr}{\easy}{Dense homeomorphism}\label{Dense homeomorphism}
Let $\mathcal{H}$ be the set of all homeomorphisms $\mathbb {S}^2\to\mathbb {S}^2$ equipped with $C^0$-metric.
Show that there is a homeomorphism $h\in \mathcal{H}$ such that its conjugations $a\circ h\circ a^{-1}$ for all $a\in\mathcal{H}$ form a dense set in $\mathcal{H}$.
 
\end{pr}












\chapter{Piecewise linear geometry}



\begin{pr}{}{Triangulation of 3-sphere}\label{4-poly}
Construct a triangulation of $\mathbb{S}^3$ 
such with $100$ vertices
such that any two vertices are connected by an edge.
\end{pr}

\begin{pr}{}{Spherical arm lemma}\label{Spherical arm lemma}
Let $A=a_1a_2\dots a_n$ and $B=b_1b_2\dots b_n$ be two simple spherical polygons 
with equal corresponding sides.
Assume $A$ lies in a hemisphere and $\measuredangle a_i\ge\measuredangle b_i$ for each $i$.
Show that $A$ is congruent to $B$.
\end{pr}



\begin{pr}{}{Folding problem} \label{Folding problem}
Let $P$ be a compact $2$-dimensional 
\hyperref[Polyhedral space]{\emph{polyhedral}}
\hyperref[Polyhedral space]{\emph{space}}. 
Construct a 
\hyperref[Piecewise distance preserving map]{\emph{piecewise distance preserving map}} 
$f\:P\to \RR^2$.
\end{pr}

\begin{pr}{}{Piecewise linear extension} \label{iso-kirzhbraun}
Prove that any \hyperref[Short map]{\emph{short map}} from a finite subset $F\subset \RR^2$
to 
$\RR^2$ can be extended to a 
\hyperref[Piecewise distance preserving map]{\emph{piecewise distance preserving map}} 
$\RR^2\to\RR^2$.
\end{pr}


\begin{pr}{}{Minimal polyhedron}\label{Minimal polyhedron}
By polyhedral disc in $\RR^3$
we understand a triangulation of a plane polygon with a map in $\RR^3$ which is affine on each triangle.
The area of the polyhedral disc is defined as the sum of areas of the images of the triangles in the triangulation.

Consider the  class of polyhedral discs glued from $n$ triangles in $\RR^3$ 
with fixed broken line as the boundary.
Let $\Sigma_n$ be a surface of minimal area in this class.
Show that $\Sigma_n$ is  \hyperref[Saddle surface]{\emph{saddle}}.
\end{pr}


\begin{pr}{}{Coherent triangulation}\label{Coherent triangulation} 
A triangulation of a convex polygon is called coherent if there is a convex function which is linear on each triangle and changes the gradient if you come trough any edge of the triangulation.
Find a non-coherent triangulation of triangle.
\end{pr}

\begin{pr}{}{Characterization of polytope}
\label{conic neighborhoods}
Let $P$ be a compact subset of the Euclidean space.
Assume for every point $x\in P$
there is a \emph{cone} $K_x$ with tip at $x$ and $\eps>0$
such that 
$$B(x,\eps)\cap P
=
B(x,\eps)\cap K_x.$$
Show that $P$ is a polytope; 
that is, $P$ is a union of finite collection of simplices.
\end{pr}

\begin{pr}{\hard}{A sphere with one edge}\label{panov-S^3} 
Given  a \hyperref[Polyhedral space]{\emph{spherical polyhedral space}} $P$,
denote by $P_s$ the subset of its 
\hyperref[Polyhedral space]{\emph{singular points}}.

Construct spherical polyhedral space $P$ which is homeomorphic to $\mathbb{S}^3$ and such that $P_s$ is formed by a knotted circle.
Show that in such an example the total length of $P_s$ can be arbitrary large and the angle around $P_s$ can be made strictly less than $2\cdot\pi$.
\end{pr}

\begin{pr}{}{Triangulation of a torus}\label{Triangulation of a torus}
Show that torus does not admit a triangulation 
such that one vertex has 5 edges,
one has 7 edges and 
all other vertexes have 
6 edges. 
\end{pr}

\begin{pr}{\easy}{Unique geodesics imply $\mathrm{CAT}(0)$}\label{Unique geodesics imply CAT}
Let $P$ be a polyhedral space.
Assume that any two points in $P$ 
are connected by unique geodesic.
Show that $P$ is a $\mathrm{CAT}(0)$ space.
\end{pr}

\begin{pr}{\easy}{No simple geodesics}\label{No simple geodesics}
Construct a convex polyhedron $P$ which surface 
does not have a closed simple geodesic.
\end{pr}





\chapter{Discrete geometry}



\begin{pr}{}{Box in a box}\label{box-in-box} 
Assume that a parallelepiped with sizes $a,b,c$ 
lies inside another with parallelepiped sizes $a',b',c'$. 
Show that 
\[a'+b'+c'\ge a+b+c.\]

\end{pr}

%\begin{pr}{\hard}{Besicovitch's set}\label{Besicovitch's set}
%Show that one can cut a unit plane disc along a finite number of radii into sectors and then move each sector by a parallel translation in such a way that their union have arbitrary small area.
%\sign{\cite{besicovitch}}
%\end{pr}




\begin{pr}{\easy}{Round circles in $\mathbb{S}^3$}\label{Round circles}
Suppose that you have a finite collection of pairwise linked round circles in the unit 3-sphere, 
not necessarily all of the same radius. 
Prove that there is an isotopy in the space of such collections of circles 
which moves all of them into great circles.
\end{pr}

\begin{pr}{}{Harnack's circles}\label{Harnack}
Prove that a smooth algebraic curve of degree $d$ in $\RP^2$ consists of at most $n=\tfrac12\cdot(d^2-3\cdot d+4)$ connected components.
\end{pr}

\begin{pr}{}{Two points on each line}\label{2pts-on-line}
Construct a set in the Euclidean plane, which intersect each line at exactly 2 points. 
\end{pr}





\begin{pr}{\easy}{Kissing number}\label{pr:Kissing number}
Show that for any convex body $W$ in $\RR^m$
$$\mathop{\rm kiss}W\ge \mathop{\rm kiss}B^m,$$
where $\mathop{\rm kiss}W$ denotes the \hyperref[Kissing number]{\emph{kissing number}}
of $W$ and $B$ denotes the unit ball in $\RR^m$.
\end{pr}

\begin{pr}{}{Monotonic homotopy}\label{mono-homotopy} 
Let $F$ be a finite set and $h_0,h_1\: F\to\RR^m$ be two maps.
Consider $\RR^m$ as a subspace of $\RR^{2\cdot m}$.
Show that there is a homotopy  $h_t\:F\to\RR^{2\cdot m}$ from $h_0$ to $h_1$ such that for any $x,y\in F$ the function 
\[t\z\mapsto |h_t(x)-h_t(y)|\] 
is monotonic.
\end{pr}

\begin{pr}{}{Cube}\label{Cube}
Assume the $2^m$ vertices 
of $m$-dimensional cube are divided into 
two sets $A$ and $B$ with the same number of vertices in each.
Show that there are at least $2^{m-1}$ edges with the ends in the different sets. 
\end{pr}

\begin{pr}{\thm}{Right-angled polyhedron}\label{Right-angled polyhedron}
Show that in all sufficiently large dimensions, there is no compact convex hyperbolic polyhedron with right dihedral angles. 
\end{pr}
 


\appendix
\chapter{Semisolutions}

\section*{Curves and surfaces}

%%%%%%%%%%%%%%%%%%%%%%%%%%%%%%%%%%%%%%%%%%%%%%%%%%
\parbf{\ref{liberman}.} 
\textit{Geodesic for birds.}
Consider a geodesic 
\[t\mapsto(x(t),y(t),z(t))\] 
in $W$;
assume it is defined in the interval $\II\subset \RR$.
Let us denote by $\phi$ the variation of turn;
it is a measure on $\II$.
We need to estimate $\phi(\II)$.

Denote by $s=s(t)$ the natural parameter of the plane curve \[t\mapsto (x(t),y(t)).\]

Prove that the function $f\:s\mapsto z$ is concave.

Given a semiopen interval $\mathbb J=(a,b]\subset \II$,
set
$\mu(\mathbb J)=f^+(a)-f^+(b)$,
where $f^+$ denotes right derivatives.
The function $\mu$ extends to a measure which could be also written as
\[\mu=\tfrac{dz^2}{d^2s}\cdot ds.\]
if $\tfrac{dz^2}{d^2s}$ understood in the sense of distribution.
 
Note that $|\tfrac{dz}{ds}|\le \ell$.
In particular $\mu(\II)\le 2\cdot\ell$.

Further note that $\phi\le \sqrt{1+\ell^2}\cdot\mu$.
In particular, 
$$\phi(\II)\le 2\cdot\ell\cdot\sqrt{1+\ell^2}.$$

A straightforward improvement of these estimates gives 
$$\phi(\II)\le 2\cdot\ell.$$
This bound is optimal, check for example $f(x,y)=-\ell\cdot\sqrt{x^2+y^2}$.

\parit{Comments.}
The problem appears in the paper of Berg, \cite{berg}.
The main observation 
in this proof (concavity of $s\mapsto z$) 
was used earlier by Liberman in \cite{liberman}
in order to bound on the variation of turn of a geodesic on a convex surface.



%%%%%%%%%%%%%%%%%%%%%%%%%%%%%%%%%%%%%%%%%%%%%%%%%%
\parbf{\ref{spiral}.}
\textit{Spiral.}
Without loss of generality we may assume that the curvature of $\gamma$ decreases in $t$.

\begin{wrapfigure}{r}{50mm}
%\begin{center}
\begin{lpic}[t(-7mm),b(0mm),r(0mm),l(0mm)]{pics/kneser-log(.5)}
%\lbl[b]{24,41;x}
\end{lpic}
%\end{center}
\end{wrapfigure}

Let $z(t)$ be the center of osculating circle at $\gamma(t)$
and $r(t)$ is its radius.
Prove that 
$$|z'(t)|\le r'(t).$$

Conclude that the osculating discs are nested;
that is, $D_{t_1}\supset D_{t_0}$ for $t_1>t_0$.
Hence the result follows.

\parit{Comments.}
The problem can be considered as a continuous analog of the Leibniz's test for alternating series.

It seems that the problem first discovered by Tait in \cite{tait}
and later reproved by Knesser in \cite{kneser};
see also \cite{ovsienko-tabachnikov}.


%%%%%%%%%%%%%%%%%%%%%%%%%%%%%%%%%%%%%%%%%%%%%%%%%%
%???+PIC
\parbf{\ref{moon-in-puddle}.}
\textit{The moon in the puddle.}
Consider the {\it cut locus} $W$
of $F$ with respect to $\partial F$;
it is defined as the closure
of the set of points $x\in F$ 
such that there are two or more points in $\partial F$ which minimize distance to $x$.

Note that after a small perturbation
of $\partial F$ we may assume that
$W$ is a graph embedded in
$F$ with finite number of edges.

Note that $W$ is a
deformation retract of $F$.
The retraction can be obtained by moving each point $y\in F\setminus W$ to $W$
along the geodesic from the closest point to $y$ on $\partial F$ which pass through $y$.

In particular, $W$ is a tree.
Therefore $W$  has
at least two end vertices;
Denote one of them by $z$.

Prove that the disc of radius $1$ centered at $z$ lies completely in $F$.

\parit{Comments.} A spherical version of this problem was used by Panov and me in \cite{panov-petrunin-ramification}.
 


%%%%%%%%%%%%%%%%%%%%%%%%%%%%%%%%%%%%%%%%%%%%%%%%%%
\parbf{\ref{3D-moon-in-puddle}.} 
\textit{Closed surface.}
The solution should be guessed from the picture.

%???CHANGE PIC
\begin{lpic}[t(-0mm),b(0mm),r(0mm),l(-5mm)]{pics/bing-1(.2)}
\end{lpic}
\begin{lpic}[t(-0mm),b(0mm),r(0mm),l(-5mm)]{pics/bing-2(.2)}
\end{lpic}
\begin{lpic}[t(-0mm),b(0mm),r(0mm),l(-5mm)]{pics/bing-3(.2)}
\end{lpic}

\begin{lpic}[t(-0mm),b(0mm),r(0mm),l(-5mm)]{pics/bing-6(.2)}
\end{lpic}
\begin{lpic}[t(-0mm),b(0mm),r(0mm),l(-5mm)]{pics/bing-5(.2)}
\end{lpic}
\begin{lpic}[t(-0mm),b(0mm),r(0mm),l(-5mm)]{pics/bing-4(.2)}
\end{lpic}

\parit{Comments.}
This solution is based on so called \emph{Bing's House}, see \cite{bing}.

%%%%%%%%%%%%%%%%%%%%%%%%%%%%%%%%%%%%%%%%%%%%%%%%%%
\parbf{\ref{curve-in-S^2}.}
\textit{A curve in a sphere.}
Let $\alpha$ be a closed curve in $\mathbb{S}^2$ of length $2\cdot\ell$ which intesects each equator.

\parit{A solution with Crofton formula.}
Note that we can assume that $\alpha$ is a broken line.

Given a unit vector $u$ denote by $e_u$ the equator with pole at $u$.
Let $k(u)$ the number of intersections
of the $\alpha$ and $e_u$.

Note that for almost all $u\in \mathbb{S}^2$, the value $k(u)$ is even.
Since each equator intersects $\alpha$, we get $k(u)\ge 2$ for almost all $u$.

Then we get
\begin{align*}
2\cdot\ell&=\tfrac14\cdot\int\limits_{\mathbb{S}^2}k(u)\cdot\, d_u\area\ge 
\\
&\ge\tfrac12\cdot\area\mathbb{S}^2=
\\
&=2\cdot\pi.
\end{align*}
The first identity above is called \emph{Crofton formula};
prove it first for one geodesic segment in $\alpha$ and then sum it up for all segments in $\alpha$.

\parit{Solution with symmetry.}
Let $\check\alpha$ be a subarc of $\alpha$ of length $\ell$, with endpoints $p$ and $q$.  
Let $z$ be the midpoint of a minimizing geodesic $[pq]$ in $\mathbb{S}^2$.  

Let $r$ be a point of intersection of $\alpha$ with the equator with pole at $z$.  
Without loss of generality we may assume that $r\in\check\alpha$. 

The arc $\check\alpha$ together with its reflection in $z$ form a closed curve of length $2\cdot \ell$ that passes through $r$ and its antipodal point $r'$.
Therefore 
\[\ell=\length \check\alpha\ge |r-r'|_{\mathbb S^2}=\pi.\]

\parit{Comments.} 
The problem was suggested by Nikolai Nadirashvili;
it is a the first step in the proof of Reshetnyak's majorization theorem for $\CAT[1]$ spaces, see \cite{akp}.
 


%%%%%%%%%%%%%%%%%%%%%%%%%%%%%%%%%%%%%%%%%%%%%%%%%%
\parbf{\ref{A spring in a tin}.} \textit{A spring in a tin.}
Let $\alpha$ be a closed curve in the unit disc;
denote by $\ell$ its length.

Let us equip the plane with complex coordinates so that $0$ is the center of the unit disc.
We can assume that $\alpha$ equipped with $\ell$-periodic parametrization by length.

Consider the curve $\beta(t)=t-\tfrac{\alpha(t)}{\alpha'(t)}$.
Note that 
\[\beta(t+\ell)=\beta(t)+\ell\] 
for any $t$.
In particular 
\[\length (\beta|_{[0,\ell]}) 
\ge 
|\beta(\ell)-\beta(0)|
=
\ell.\]

Note that 
\begin{align*}
|\beta'(t)|&=|\tfrac{\alpha(t)\cdot\alpha''(t)}{\alpha'(t)^2}|\le
\\
&\le|\alpha''(t)|.
\end{align*}
Since $|\alpha''(t)|$ is the curvature of $\alpha$ at $t$,
we get the result.

\parit{Comment.}
This proof admits straightforward generalization to the higher dimensions.

If instead of a disc we have a region bounded by closed convex curve $\gamma$ then it is still true that the average absolute curvature of $\alpha$ is at least as big as average absolute curvature of $\gamma$. 
The proof is not that simple, see \cite{tabachnikov} and the reference there in.


%%%%%%%%%%%%%%%%%%%%%%%%%%%%%%%%%%%%%%%%%%%%%%%%%%
\parbf{\ref{Convex hat}.}
\textit{Convex hat.}
Let $\gamma$ be a minimizing geodesic with the ends in $\Delta$.

Assume $\gamma\backslash\Delta\ne\emptyset$.
Denote by $\gamma'$ the curve formed by $\gamma\cap \Delta$ 
and the reflection on $\gamma\backslash\Delta$ in $\Pi$.
Note 
\[\length\gamma'=\length\gamma\]
and $\gamma'$ runs partly in and partly outside of the surface, but does not get inside of $\Sigma$.

Denote by $\gamma''$ the closest point projection of $\gamma'$ on $\Sigma$.
The curve $\gamma''$ lies in $\Sigma$ has the same ends as $\gamma$.

It remains to note that 
\[\length\gamma''<\length\gamma;\]
the later leads to a contradiction.

 



%%%%%%%%%%%%%%%%%%%%%%%%%%%%%%%%%%%%%%%%%%%%%%%%%%
\parbf{\ref{Unbended geodesic}.} 
\textit{Unbended geodesic.}
Let $W$ be the closed unbounded set formed by $\Sigma$ and its exterior points.

Prove that for any $x\in\Sigma$ the distance $|x - p_t|$ is nondecreasing in $t$.

Use the later statement, to prove the same for $|x - p_t|_W$,
where $|x - p_t|_W$ stays for the intrinsic distance from $x$ to $p_t$ in $W$.

Prove that 
\[|q - p|_W=|q - p|_\Sigma\] 
for any $p,q\in\Sigma$.


Conclude that the distance $|q - p_t|_W=|q - p|_\Sigma$
for any $t$.
It follows that the curve 
$$\gamma_t(\tau)=\left[
\begin{aligned}
&(\tau-t)\cdot\gamma'(\tau)&&\text{if}&&\tau< t;
\\
&\gamma(\tau)&&\text{if}&&\tau> t.
\end{aligned}
\right.$$
is a minimizing geodesic from $p_t$ to $q$ in the intrinsic metric of $W$. 

If $q$ is visible from $p_t$ for some $t$ then the line segment $[qp_t]$ intersects $\Sigma$ only at $q$.
From above, 
$\gamma_t$  coinsides with the line segment $[qp_t]$ which is impossible.

\parit{Comment.}
This  observation was used by Milka
to generalize Alexandrov's comparison theorem for convex surfaces, see \cite{milka-geod}.

%%%%%%%%%%%%%%%%%%%%%%%%%%%%%%%%%%%%%%%%%%%%%%%%%%
\parbf{\ref{min-surf}.} 
\textit{A minimal surface.}
Without loss of generality we may assume that the sphere is centered at $0\in\RR^3$.

Consider the restriction $h$ of the function $x\mapsto |x|^2$ to the surface $\Sigma$.
Prove that $\Delta_\Sigma h\le 2$ and apply apply the divergence theorem for $\nabla_\Sigma h$.
It follows that the function
\[f\:r\mapsto \frac{\area(\Sigma\cap B(0,r))}{r^2}
\]
is non-decreasing in the interval $(0,1)$.
Hence the result follows.

\parit{Comments.}
We described a partial case of so called \emph{monotonicity formula}.

Note that if we assume in addition that the surface is a disc,
then the statement holds for any saddle surface. 
Indeed, denote by $S_r$ the sphere of radius $r$ concentrated with the unit sphere. 
Then according to Problem~\ref{curve-in-S^2}, 
$\length \Sigma\cap S_r\ge 2\cdot\pi\cdot r$.
Then coarea formula leads to the solution.

On the other hand there are saddle surfaces homeomorphic to the cylinder
may have have arbitrary small area in the ball. 

If $\Sigma$ does not pass through the center and we only know the distance $r$ from center to $\Sigma$ then optimal bound is expected to be $\pi\cdot(1-r^2)$.
This is known if $\Sigma$ is topological disc, see \cite{alexander-osserman}.
An analogous result for area-minimizing submanifolds holds for all dimensions and codimensions, see \cite{alexander-hoffman-osserman}.






%%%%%%%%%%%%%%%%%%%%%%%%%%%%%%%%%%%%%%%%%%%%%%%%%%
\parbf{\ref{half-torus}.} 
\textit{Half-torus.}
Let $K$ be the convex hull of $\Omega'$.
Consider the boundary curve $\gamma'$ of $\partial K\cap \Omega'$ in $\Omega'$.

First note that the Gauss curvature of $\Omega'$ has to vanish at the points of $\gamma'$;
in other words, $\gamma'$ is the image of $\gamma$ under the length-preserving map.
Indeed since $\gamma'$ lies on convex part, 
the Gauss curvature at the points of $\gamma'$ has to be nonnegative. 
On the other hand $\gamma'$ bounds a flat disc in $\partial K$;
therefore its integral intrinsic curvature has to be $2{\cdot}\pi$.
If the Gauss curvature is positive at some point of $\gamma'$ then total intrinsic curvature of $\gamma'$ has to be $<2{\cdot}\pi$, a contradiction.

Now prove that $\gamma'$ is an asymptotic line.
(Assume that the asymptotic direction goes transversely to $\gamma'(t)$ and conclude $\gamma(t)\notin\partial K$.)

Without loss of generality, we can assume that the length of $\gamma$ is $2{\cdot}\pi$ and its intrinsic curvature is $\equiv 1$.
Therefore, as the space curve,
$\gamma'$ has to be a curve with constant curvature $1$ and it should be closed.
Any such curve is congruent to a flat circle.

\parit{Comments.} It is not known if $\Omega'$ is congruent to $\Omega$.

The solution presented above is based on my answer 
to the question of O'Rourke, see \cite{rourke}.
Here are some related statements.
\begin{itemize}
\item Half-torus is second order rigid;
this was proved in
\cite{rembs} 
and \cite[p. 135]{efimov}.
\item Any second order rigid surface does not admit analytic deformation (see \cite[p. 121]{efimov})
and for the surfaces of revolution, the assumption of analyticity can be removed (see \cite{sabitov}).
\end{itemize}






%%%%%%%%%%%%%%%%%%%%%%%%%%%%%%%%%%%%%%%%%%%%%%%%%%
\parbf{\ref{asymptotic-line}.} 
\textit{Asymptotic line.}
Arguing by contradiction, assume that the projection $\bar\gamma$
of $\gamma$ on $x y$-plane is star shaped with respect to the origin.

Consider the function 
$$h(t)=(d_{\bar\gamma(t)}f)(\gamma(t)).$$
Prove that $h'(t)\ne 0$.
In particular $h(t)$ is a strictly monotonic function of $\mathbb{S}^1$, a contradiction.

\parit{Comments.}
The problem discussed by Panov in \cite{panov-curves}.


%%%%%%%%%%%%%%%%%%%%%%%%%%%%%%%%%%%%%%%%%%%%%%%%%%
\parbf{\ref{torus}.}
\textit{Non-contractible geodesics.}
Take a torus of revolution $T$;
the rotations of the circle produce a family closed geodesics which we will call \emph{meridians}.

Note that a geodesic on $T$ is either a meridian
or it is transversal to all the meridians.
No closed curve of these types can be contractible. 

\parit{Comments.} This problem appears from Gromov's book \cite{gromov-MetStr},
where it is attributed to Y. Colin de Verdi\`ere.



%%%%%%%%%%%%%%%%%%%%%%%%%%%%%%%%%%%%%%%%%%%%%%%%%%
\parbf{\ref{Convex figures}.}
\textit{Convex figures.}
Consider the set $\Omega_n$ of all convex figures $F\subset\RR^2$ 
such that for any $x\z\in\partial F$ there are $y,z\z\in F$ such that
$\measuredangle \hinge xyz>\pi-\tfrac1n$.

%%%%%%%%%%%%%%%%%%%%%%%%%%%%%%%%%%%%%%%%%%%%%%%%%%
\begin{wrapfigure}{r}{29mm}
\begin{lpic}[t(-4mm),b(-3mm),r(0mm),l(0mm)]{pics/hilbertcurve5(.25)}
\end{lpic}
\end{wrapfigure}

Prove that $\Omega_n$ 
is open and dense in $\mathfrak{C}$.
Finally note that the intersection
$\bigcap_n\Omega_n$
forms the subset of all smooth figures in $\mathfrak{C}$.  

\parit{Comments.} 
Number of similar problems surveyed by Zamfirescu in \cite{zamfirescu}.


\parbf{\ref{Fat curve}.}
\textit{Fat curve.} 
Modify your favorite space filling curve 
to keep area nearly the same and removing self-intersections.

Say, you can modify the Hilbert curve which can be constructed as a limit of recursively defined sequence of curve;
see the 5-th iteration on the diagram. 



%%%%%%%%%%%%%%%%%%%%%%%%%%%%%%%%%%%%%%%%%%%%%%%%%%
\parbf{\ref{Rectifiable curve}.}
\textit{Rectifiable curve.}
The 1-dimensional Hausdorff measure will be denoted as $\mathcal{H}_1$. 

Set $L=\mathcal{H}_1(K)$.
Without loss of generality, we may assume that $K$ has diametr $1$.

Assume that $0<\eps<\tfrac12$.
Prove that 
\[\mathcal{H}_1(B(x,\eps)\cap K)\ge\eps\leqno(*)\]
for any $x\in K$.

Let $x_1,\dots, x_n$ be a maximal set of points in $K$ such that 
\[|x_i-x_j|\z\ge\eps\] for all $i\ne j$. 
From $(*)$ we have $n\le2\cdot L/\eps$.

Construct a curve $\gamma_\eps$ such that (1) $\gamma_\eps$ is passing through all $x_i$, (2) $\length\gamma_\eps\le10\cdot L$ and (3) $\gamma_\eps$ lies in $\eps$-neighborhood of $K$.
We can assume that $\gamma_\eps$ is parametrized by length.

The needed curve can be obtained by passing to 
a partial limit of $\gamma_\eps$
 as $\eps\to 0$. 

\parit{Comments.}
This is an exercise  in the Falconer's book \cite[Ex. 3.5]{falconer}.



%%%%%%%%%%%%%%%%%%%%%%%%%%%%%%%%%%%%%%%%%%%%%%%%%%
\parbf{\ref{Capture a sphere in a knot}.}
\textit{Capture a sphere in a knot.}
We can assume that the knot is given by a diagram on the the sphere.

Fix a M\"obius transformation $\mathbb{S}^2\to\mathbb{S}^2$ which is not an isometry.
Denote by $u$ its conformal factor. 
Since the M\"obius transformation preservs total area, 
we get 
$$\frac1{\area \mathbb{S}^2}\cdot\int\limits_{\mathbb{S}^2} u^2=1.$$ 
Therefore, 
$$\frac1{\area \mathbb{S}^2}\cdot\int\limits_{\mathbb{S}^2} u<1.$$ 
It follows that after a suitable rotation of $\mathbb{S}^2$, 
the length of the knot decreases.

Similar argument gives a continuous one parameter family of M\"obius transformations which moves the knot in a hemisphere 
and allows the ball to escape. 

\parit{Comments.}
This is a question of Brady, see \cite{zeb}, 
the idea in the solution is due to David Eppstein.



%%%%%%%%%%%%%%%%%%%%%%%%%%%%%%%%%%%%%%%%%%%%%%%%%%
\parbf{\ref{linked-circles}.}
\textit{Linked circles.} 
Fix a point $x\in\alpha$. 
Note that one can find another point $x'\in\alpha$ such that the interval 
$[xx']$ intersects $\beta$, say at the point $z$. 
Otherwise we can move each point of $\alpha$ along the line segment to $x$.
This deformation of $\alpha$ will not cross $\beta$;
the later contradicts that $\alpha$ and $\beta$ are linked. 


Consider the curve $\alpha'$ which is the central projection of $\alpha$ 
from $z$ onto the unit sphere around $z$;
clearly
$$\length \alpha\ge \length\alpha'.$$

Note that $\alpha'$ passes through two antipodal points of the sphere;
therefore 
$$\length \alpha'\ge 2\cdot\pi.$$
Hence the result follows.

\parit{Comments.}
This is the simplest case of so called \emph{Gehring's problem}. 
The solution above was given by Edelstein and Schwatz in \cite{edelstein-schwatz};
later the same solution was rediscovered few times.


%%%%%%%%%%%%%%%%%%%%%%%%%%%%%%%%%%%%%%%%%%%%%%%%%%
\parbf{\ref{Oval in oval}.}
\textit{Oval in oval.}
Show that the chord which minimize (or maximize) the ratio in which it divides the bigger oval solves the problem.

\begin{wrapfigure}{o}{51mm}
\begin{lpic}[t(-1mm),b(-0mm),r(0mm),l(0mm)]{pics/tangent-eq-sol(1)}
\lbl[tr]{20,5;$u$}
\lbl[rt]{18,15;$r$}
\lbl[rt]{26,6;$l$}
\lbl[bl]{25,10;$x_u$}
\end{lpic}
\end{wrapfigure}


%%%%%%%%%%%%%%%%%%%%%%%%%%%%%%%%%%%%%%%%%%%%%%%%%%
\parbf{\ref{Oval in oval}.}
\textit{Oval in oval.}
Given a unit vector $u$, denote by $x_u$ the point on the inner curve
with outer normal vector $u$.
Draw a chord of outer curve which is tangent to the inner curve at $x_u$;
denote by $r=r(u)$ and $l=l(u)$ the lengths of this chord at the right and left from $x_u$.


Arguing by contradiction, assume $r(u)\ne l(u)$ for any $u\in\mathbb{S}^1$.
Since the functions $r$ and $l$ are continuous,
we can assume that 
$$r(u)>l(u)\ \ \text{for any}\ \ u\in\mathbb{S}^1.\leqno{({*})}$$

Prove that
each of the following two integrals 
\begin{align*}
\tfrac12\cdot\int\limits_{\mathbb{S}^1}r^2(u)\cdot du
\quad\text{and}\quad
\tfrac12\cdot\int\limits_{\mathbb{S}^1}l^2(u)\cdot du
\end{align*}
give 
the area between the curves.
In particular 
the integrals are equal to eachother. 
The later contradicts $({*})$.

\parit{Comments.} This is a problem of Tabachnikov, see \cite{tabachnikob-mi}.
A closely related, so called \emph{equal tangents problem} is discussed by the same author in \cite{tabacnikov=tan}.

%%%%%%%%%%%%%%%%%%%%%%%%%%%%%%%%%%%%%%%%%%%%%%%%%%
\parbf{\ref{Surrounded area}.}
\textit{Surrounded area.}
Denote by $C_1$ and $C_2$ the compact regions bounded by $\gamma_1$ and $\gamma_2$ correspondingly.

By Kirszbraun theorem, 
any short map $X\to \RR^2$ defined on $X\subset \RR^2$
can be extended to a short map on whole $\RR^2$.
In particular there is a short map $f\:\RR^2\to\RR^2$ 
such that $f(\gamma_2(v))=f(\gamma_1(v))$ for any $v\in\mathbb S^1$.

Note that $f(C_2)\supset C_1$.
Whence the statement follows.



%%%%%%%%%%%%%%%%%%%%%%%%%%%%%%%%%%%%%%%%%%%%%%%%%%
\parbf{\ref{Asymptotic geodesic}.}
\textit{Periodic asymptote.}
Assume contrary.
Passing to a finite cover, we can assume that the asymptote has no self intersections.
In this case 
the restriction $\gamma|_{[a,\infty)}$  
has no self-intersections if $a$ is large large enough.

Cut $\Sigma$ along $\gamma([a,\infty))$ and then cut from the obtained surface an infinite triangle $\triangle$ with two sides formed by both sides of cuts along $\gamma$; let us denote these sides of $\triangle$ by $\gamma_-$ and $\gamma_+$.
Note that 
\[\area\triangle<\area \Sigma<\infty\leqno(*)\]
and both sides $\gamma_\pm$ 
form infinite minimizing geodesics in $\triangle$.

Consider the Buseman function $f$ for $\gamma_+$;
denote by $\ell(t)$ the length of the level curve $f^{-1}(t)$.
Let $-\kappa(t)$  be the total curvature of the suplevel set $f^{-1}([t,\infty))$.  
Note that for all large $t$ we have
\[\ell'(t)=\kappa(t)
\ \ \text{and}\ \ 
\kappa'(t)\le C\cdot \ell(t)^2\] 
where $C$ is a fixed constant.
The later implies that there is $\eps>0$ such that
\[\ell(t)\ge \frac\eps{t-a}\]
for any large $t$.
In particular,
\[\int\limits_a^\infty\ell(t)=\infty.\]
By coarea formula we get 
\[\area\triangle=\infty;\]
the late contradicts $(*)$.

\parit{Comment.}
I've learned the problem from Dmitri Burago and Sergei
Ivanov, it is originated from a discussions between
Keith Burns, Michael Brin and Yakov Pesin.
 


%%%%%%%%%%%%%%%%%%%%%%%%%%%%%%%%%%%%%%%%%%%%%%%%%%
\parbf{\ref{Immersed surface}.}
\textit{Immersed surface.}
Let $\ell$ be a linear function which vanish on $\Pi$ and positive in $\Sigma$.

Let $z$ be a point of maximum of $\ell$ on $\Sigma$;
set $s_0=\ell(z)$.
Given $s<s_0$, denote by $\Sigma_s$ the connected component of $z$ in $\Sigma\cap\ell^{-1}([s,s_0])$.
Note that for all $s$ sufficiently close to $s_0$
we have
\begin{itemize}
\item $\Sigma_s$ is an embeded disc;
\item $\partial\Sigma_s$ is convex plane curve.
\end{itemize}

Applying open-close argument, we get that the same holds for all $s\in[0,s_0)$.

Since $\Sigma$ is connected, $\Sigma_0=\Sigma$.
Hence the result follows.

\parit{Comments.}
This problem is a discussed in Gromov's lectures \cite[\S$\tfrac12$]{gromov-SGMC}.



%%%%%%%%%%%%%%%%%%%%%%%%%%%%%%%%%%%%%%%%%%%%%%%%%%
\parbf{\ref{Two discs}.}
\textit{Two discs.}
Choose a continuous map $h\:\Sigma_1\to \Sigma_2$
which is identical on $\gamma$.
Let us prove that for some $p_1\in \Sigma_1$ and $p_2=h(p_1)\in \Sigma_2$
the tangent plane $\mathrm{T}_{p_1} \Sigma_1$ is parallel to the tangent plane $\mathrm{T}_{p_2} \Sigma_2$;
this is stronger than required.

Arguing by contradiction,
assume that such point does not exist.
Then for each $p\in\Sigma_1$
there is unique line $\ell_p\ni p$ 
which is parallel to each of the the tangent planes $\mathrm{T}_{p} \Sigma_1$ and $\mathrm{T}_{h(p)} \Sigma_2$.

Note that the lines $\ell_p$ form a tangent line distribution over $\Sigma_1$
and $\ell_p$ is tangent to $\gamma$ at any $p\in\gamma$.

Let $D$ be the disc in $\Sigma_1$ bounded by $\gamma$.
Consider the doubling of $D$ in $\gamma$;
it is diffeomorphic to $\mathbb S^2$.
The line distribution $\ell$ lifts to a line distribution on the doubling;
the later contradicts the hairy ball theorem.

\parit{Comments.} This proof was suggested nearly simultaneously by Steven Sivek and user damiano, see \cite{two-discs}.

Note that the same proof works in case if $\Sigma_i$ are oriented open surfaces such that $\gamma$ cuts a compact domain in each $\Sigma_i$.

There are examples of three disks $\Sigma_1$, $\Sigma_2$ and $\Sigma_3$
with common closed curve $\gamma$ such that there no triple of points $p_i\in\Sigma_i$ with parallel tangent plane.
Such examples can be found among ruled surfaces, see \cite{three-discs}.



%%%%%%%%%%%%%%%%%%%%%%%%%%%%%%%%%%%%%%%%%%%%%%%%%%
\parbf{\ref{Simple geodesic}.} 
\textit{Simple geodesic.}
Let $\gamma$ be a two-sided infinite geodesic in $\Sigma$.
The following is the key statement in the proof.

\parbf{Claim.}
{\it The geodesic $\gamma$ contains at most one simple loop.}
\medskip

To prove the claim use the following observations.
\begin{itemize}
\item The total curvature $\omega$ of $\Sigma$ can not exceed $2\cdot\pi$.
\item If $\phi$ is the angle at the base of a simple geodesic loop then the total curvature surrounded by the loop equals to $\pi+\phi$.
\end{itemize}

Once the claim is proved, note that if a geodesic $\gamma$ has a self-intersection
then it contains a simple loop.
From above there is only one such loop;
it cuts a disc from $\Sigma$ 
and can go around it either clockwise or counterclockwise.
This way we divide all the self-intersecting geodesics 
into two sets which we will call \emph{clockwise} and \emph{counterclockwise}.

Note that the geodesic $t\mapsto \gamma(t)$ is clockwise 
if and only if 
$t\mapsto \gamma(-t)$
is counterclockwise.
The sets of clockwise and counterclockwise are open and the space of geodesics is connected. 
It follows that there are geodesics which neither clockwise nor counterclockwise;
by the definition, these geodesics have no self-intersections.

\parit{Comment.}
The idea in the proof is due to Bangert, see \cite[Cor. 2]{bangert}.

%%%%%%%%%%%%%%%%%%%%%%%%%%%%%%%%%%%%%%%%%%%%%%%%%%
\parbf{\ref{Long geodesic}.}
\textit{Long geodesic.}
Denote by $a$ the area of the surface.

Cut the surface along a long closed simple geodesic $\gamma$.
We get two discs with nonnegative curvature and large perimeter, 
say $\ell$.
Note that the area of each disc is bounded above by $a$.

Choose one of the discs $D$ and equip it with intrinsic metric.
Note that $D$ is non-negatively curved in the sense of Alexandrov.
Denote by $p$ and $q$ be the points in $D$ which lie on the maximal distance from each other.

Fix $\eps>0$.
Fix a geodesic $[pq]$ in $D$.
Show that if $\ell$ is large enough in terms of $\eps$ 
the distance from any point in $D$ to $[pq]$ is at most $\eps$
and the curvature of $\eps$-neighborhood of $p$ in $D$
is at least $\pi-\eps$.

By Gauss--Bonnet formula the total curvature of $\Sigma$ is $4\cdot\pi$.
Since $\eps>0$ is arbitrary, we get that there are 4 point in $\Sigma$, each with curvature $\pi$
and remaining part of $\Sigma$ is flat.

\begin{wrapfigure}{o}{21mm}
\begin{lpic}[t(-7mm),b(-2mm),r(0mm),l(0mm)]{pics/akopyan(1)}
\end{lpic}
\end{wrapfigure}

It remains to show that any surface with this property is isometric to the surface of tetrahedron with equal opposite edges.
To do this cut $\Sigma$ along three geodesics which connect one singular point to the remaining three,
develop the obtained flat surface on the plane and think (also look at the diagram).

\parit{Comments.}
The problem was suggested by Arseniy Akopyan.


\parbf{\ref{Corkscrew geodesic}.}
\textit{Corkscrew geodesic.}
An example can be found among the surfaces of convex polyhedrons 
such that the nondegenerate intersections with horizontal planes are triangles with parallel sides.

The polyhedron $K$ should look like a vertical needle. 
On the surface of $K$ there three broken lines, formed by the corresponding vertices of triangle.
The polyhedron can be made in such a way that any minimizing geodesic from the top of $K$ to its bottom
has to cross these lines in the cyclic order at nearly each edge, and the number of edges can be made arbitrary large. 

\parit{Comments.} This construction is due to Imre, Kuiperberg and Zamfirescu,
see
\cite{imre-kuiperberg-zamfirescu}.



%%%%%%%%%%%%%%%%%%%%%%%%%%%%%%%%%%%%%%%%%%%%%%%%%%
%%%%%%%%%%%%%%%%%%%%%%%%%%%%%%%%%%%%%%%%%%%%%%%%%%
%%%%%%%%%%%%%%%%%%%%%%%%%%%%%%%%%%%%%%%%%%%%%%%%%%
%%%%%%%%%%%%%%%%%%%%%%%%%%%%%%%%%%%%%%%%%%%%%%%%%%
\section*{Comparison geometry}



%%%%%%%%%%%%%%%%%%%%%%%%%%%%%%%%%%%%%%%%%%%%%%%%%%
\parbf{\ref{Totally geodesic hypersurface}.} 
\textit{Totally geodesic hypersurface.}
Assume $\Sigma$ is a totally geodesic embedded hypersurface in $M$.
Without loss of generality, we can assume that $\Sigma$ is connected.

The complement $M\backslash\Sigma$ has one or two connected components.
First let us show that if the number of connected components is two, then $M$ is homeomorphic to sphere.

Cut $M$ along $\Sigma$,
you get two manifolds $M_1$ and $M_2$
with geodesic boundaries. 
Prove that the distance functions to the boundary 
$f_1\:M_1\to\mathbb{R}$ and $f_2\:M_2\to\mathbb{R}$ are siricly convex in the interiors of the manifolds.

Smooth the functions $f_i$ keeping them convex, this can be done by applying Greene--Wu Theorem (\cite[Theorem 2]{greene-wu}).
In particular each $f_i$ has singe critical point which is its maximum.

Applying Morse lemma, we get that each manifold $M_i$ is homeomorphic to a ball; 
hence $M$ 
is homeomorphic to the sphere.

If $M\backslash\Sigma$ is connected,
passing to a double cover of $M$ 
we reduce the problem to the case which already have been considered.

\parit{Comments.}
The problem was suggested by Peter Petersen.



%%%%%%%%%%%%%%%%%%%%%%%%%%%%%%%%%%%%%%%%%%%%%%%%%%
\parbf{\ref{Immersed convex hypersurface I}.} 
\textit{Immersed convex hypersurface {\rm I}.}
Observe first that any closed embedded locally convex hypersurface in a non-positively curved simply connected complete manifold bounds a convex region.


Let $\Sigma$ be an immersed locally convex hypersurface in $M$.
Set 
\[m=\dim \Sigma=\dim M-1\]

Given a point in $p$ on $\Sigma$ 
denote by $p_r$ the point on distance $r$ from $p$
which lies on the geodesic starting from $p$ in the outer normal direction to $\Sigma$.
For fixed $r\ge 0$,
the points $p_r$ swap an immersed locally convex hypersurface which we denote by $\Sigma_r$.

Fix $z\in \Sigma$.
Denote by $S_r$ the sphere of radius $r$ centered at $z$.
Note that $S_r$ is diffeomorphic to $m$-dimensional sphere.

Denote by $d$ the diameter of $\Sigma$.
Note that for all $r>0$
any point on $\Sigma_r$
lies on the distance at most $d$ from $S_r$.
Conclude that for large $r$ the closest point projection $\phi_r\:\Sigma_r\to S_r$ is an immersion.


Since $\Sigma$ is connected
and $m\ge 2$, it follows that $\phi_r$ is a diffeomorphism for all large $r$.

By the observation above, $\Sigma_r$ bounds a convex region for all large $r$.
By open-close argument, the same holds for all $r\ge 0$.
Hence the result follows.

\parit{Comments.}
The problem was considered by Alexander in \cite{alexander}.



%%%%%%%%%%%%%%%%%%%%%%%%%%%%%%%%%%%%%%%%%%%%%%%%%%
\parbf{\ref{Immersed convex hypersurface II}.} 
\textit{Immersed convex hypersurface {\rm II}.}
Equip $\Sigma$ with the induced intrinsic metric.
Denote by $\kappa$ the lower bound for principle curvatures of $\Sigma$.
Note that we can assume that $\kappa>0$.

Fix sufficiently small $\eps=\eps(M,\kappa)>0$.
Given $p\in \Sigma$ consider the lift $\tilde h_p\:B(p,\eps)\to \mathrm{T}_{h(p)}$ along the exponential map $\exp_{h(p)}\:\mathrm{T}_{h(p)}\to M$.
More precisely:
\begin{enumerate}
\item Connect each point $q\in B(p,\eps)\subset \Sigma$ to $p$
by a the minimizing geodesic  path $\gamma_q\:[0,1]\to \Sigma$
\item Consider the lifting $\tilde\gamma_q$ in $\mathrm{T}_{h(p)}$; 
that is the curve such that $\tilde\gamma_q(0)=0$ and $\exp_{h(p)}\circ\tilde\gamma_q(t)=\gamma_q(t)$ for any $t\in[0,1]$.
 \item Set $\tilde h(q)=\tilde\gamma_q(1)$.
\end{enumerate}

Show any the hypersurface $\tilde h_p(B(p,\eps))\subset \mathrm{T}_{h(p)}$ has principle curvatures at least $\tfrac\kappa2$.

Use the same idea as in problem~\ref{Immersed surface} to show that 
one can fix $\delta\z=\delta(M,\kappa)>0$ such that the restriction of $\tilde h_p|_{B(p,\delta)}$ is injective.
Conclude that the restriction $h|_{B(p,\delta)}$ is injective for any $p\in\Sigma$.

Now consider locally equidistant surfaces $\Sigma_t$ in the inward direction for small $t$. 
The principle curvatures of $\Sigma_t$ remain at least $\kappa$ in the barrier sense.
By the same argument as above, any $\delta$-ball in $\Sigma_t$
is embedded.

Applying open-close argument we get a one parameter family of locally convex locally equidistant surfaces $\Sigma_t$
for defined in a maximal interval $[0,a)$
and 
the surface $\Sigma_a$ degenerates to a point, say $p$. 

To construct the immersion $\partial \bar B^m\looparrowright M$,
take the point $p$ as the image of the center $\bar B^m$ 
and take the surfaces $\Sigma_t$ as the restrictions of the  embedding to the spheres;
the existance of the immersion follows from the Morse lemma.

\begin{wrapfigure}[5]{r}{23mm}
%\begin{center}
\begin{lpic}[t(-8mm),b(0mm),r(0mm),l(0mm)]{pics/ass(1)}
%\lbl[b]{24,41;x}
\end{lpic}
%\end{center}
\end{wrapfigure}

\textit{Comments.}
As you see on the picture, the analogous statement does not hold in the two-dimensional case.

The proof presented above was indicated in Gromov's lectures \cite{gromov-SGMC};
it was written rigorously by Eschenburg in \cite{eschenburg}.
A variation of this proof was obtained independently by Andrews in \cite{andrews}.
Instead of equidistant deformation, 
Andrews use so called \emph{inverse mean curvature flow};
this way he has to perform some calculations, but does not have to worry about non-smoothness of the hypersurfaces. 




%%%%%%%%%%%%%%%%%%%%%%%%%%%%%%%%%%%%%%%%%%%%%%%%%%
\parbf{\ref{almgren}.} 
\textit{Almgren's inequalities.}
Fix a  geodesic $m$-dimensional sphere $\mathbb{S}^m$ in $\mathbb{S}^n$.

Given $r\in (0,\tfrac\pi2]$,
denote by $U_r$ and $\tilde U_r$ the tubular $r$-neighbohood 
of $\Sigma$ and $\mathbb{S}^m$ in $\mathbb{S}^n$ correspondingly.

Prove that $U_{\frac\pi2}\supset\mathbb{S}^n$.
Then it follows that
\[U_{\frac\pi2}=\tilde U_{\frac\pi2}=\mathbb{S}^n.\leqno({*})\]

Prove that for any $x\in \partial U_r$ we have
\[H_r(x)\ge \tilde H_r,\] 
where $H_r(x)$ denotes the mean curvature of $\partial U_r$  at point $x$
and $\tilde H_r$ its mean curvature of $\partial\tilde U_r$.

Set 
\begin{align*}
a(r)&=\vol_{n-1} \partial U_r,
&
\tilde a(r)&=\vol_{n-1} \partial\tilde U_r,
\\
v(r)&=\vol_n U_r,
&
\tilde v(r)&=\vol_n \tilde U_r.
\intertext{by coarea formula,}
\tfrac d{dr} v(r)&= a(r),
&
\tfrac d{dr}\tilde v(r)&=\tilde a(r).
\end{align*}
for almost all $r$.
Note that
\begin{align*}\tfrac d{dr}a(r)&\le \int\limits_{\partial U_r} H_r(x)\cdot d_x\vol_{n-1}\le
\\
&\le a(r)\cdot \tilde H_r
\end{align*}
and
\begin{align*}
\tfrac d{dr}\tilde a(r)
&= \tilde a(r)\cdot \tilde H_r.
\intertext{It follows that}
\frac {v''(r)}{v(r)}&\le \frac {\tilde v''(r)}{\tilde v(r)}
\end{align*}
for almost all $r$. Therefore
\[v(r)\le\frac{\area\Sigma}{\area \mathbb{S}^m}\cdot \tilde v(r)\]

for any $r>0$.

According to $({*})$,
\[v(\tfrac\pi2)=\tilde v(\tfrac\pi2)=\vol\mathbb{S}^n.\]
Whence the result follows.

\parit{Comments.}
This problem is the most geometric part in the proof of Almgren's isoperimetric inequality \cite{almgren}.
The argument presented here is very similar to 
the proof of Gromov--Levy isometric inequality given in the Gromov's appendix to \cite{gromov-apendix}.

%%%%%%%%%%%%%%%%%%%%%%%%%%%%%%%%%%%%%%%%%%%%%%%%%%
\parbf{\ref{codim=2}.} 
\textit{Hypercurve.}
Fix $p\in M$.
Denote by $s$ 
the second fundamental form of $M$ at $p$;
it is a symmetric bi-linear form on the tangent space $\mathrm{T}_pM$ of $M$ with values in the normal space $\mathrm{N}_pM$ to $M$, see page~\pageref{Second fundamental form}.
Note that the normal space $\mathrm{N}_pM$ is two-dimensional.

Prove that if sectional curvature of $M$ is positive, 
then
\[\<s(X,X),s(Y,Y)\> > 0\leqno({*})\]
for any pair of nonzero vectors $X,Y\in\mathrm{T}_pM$.

Show that $({*})$ implies that there is an orthonormal basis $e_1,e_2$ in $\mathrm{N}_pM$ 
such that the real-valued quadratic forms 
\begin{align*}
s_1(X,X)&=\<s(X,X),e_1\>,
&
s_2(X,X)&=\<s(X,X),e_2\>
\end{align*}
are positive definite.

Note that the curvature operators $R_1$ and $R_2$ defined by the following identity
\[R_{i}(X\wedge Y), V\wedge W\rangle 
=s_i(X,W)\cdot s_i(Y,V)-s_i(X,V)\cdot s_i(Y,W)\]
 are positive.
Finally, note that $R_{1}+R_{2}$ is the curvature operator of $M$ at $p$.

\parit{Comments.}
The problem appears in Weinstein's paper \cite{weinstein}.

Note that follows from \cite{micallef-moore}/\cite{boehm-wilking} it follows that
that the universal cover of $M$ is homeomorphic/diffeomorphic to a standard sphere.



%%%%%%%%%%%%%%%%%%%%%%%%%%%%%%%%%%%%%%%%%%%%%%%%%%
\parbf{\ref{Horosphere}.} 
\textit{Horosphere.}
Set 
$m=\dim \Sigma=\dim M-1$.

Let $b\:M\to\RR$ be the Busemann function such that $\Sigma=b^{-1}(\{0\})$.
Set  $\Sigma_r=b^{-1}(\{r\})$, so $\Sigma_0=\Sigma$.

Let us equip each $\Sigma_r$ with induced Riemannian metric.
Note that all $\Sigma_r$ have bounded curvature.
In particular, unit ball in $\Sigma_r$ has volume bounded above by universal constant, say $v_0$.
 

Given $x\in \Sigma$ denote by $\gamma_x$ 
the (necessary unique) unit-speed geodesic
such that $\gamma_x(0)=x$ and $b(\gamma_x(t))=t$ for any $t$.
Consider the map $\phi_{r}\:\Sigma\to\Sigma_r$ defined as
$\phi_r\:x\mapsto \gamma_x(r)$.

Notice that $\phi_r$ is a bi-Lipschitz map with the Lipschitz constants $e^{a\cdot r}$ and $e^{b\cdot r}$.
In particular, the ball of radius $R$ in $\Sigma$ is mapped by $\phi_r$
to a ball of radius $e^{a\cdot r}\cdot R$ in $\Sigma_r$.
Therefore
\[\vol_m B(x,R)_\Sigma\le e^{m\cdot b\cdot r}\cdot \vol_m B(x,e^{a\cdot r}\cdot R)_{\Sigma_r}\]
for any $R,r>0$.
Applying this formula in case $e^{a\cdot r}\cdot R=1$ implies that
\[\vol_m B(x,R)_\Sigma\le v_0\cdot R^{m\cdot \frac ba}.\]

\parit{Comment.}
The problem was suggested by Vitali Kapovitch.

There are examples of horospheres as above with degree of polynomial growth higher than $m$.
For example, consider the horosphere $\Sigma$ in the
the complex hyperbolic space 
of real dimension $4$.
Clearly $m=\dim \Sigma=3$ but the degree of its volume growth is $4$.
The later follows since $\Sigma$ admits has left-invariant metric on the \hyperref[Heisenberg group]{\emph{Heisenberg group}}.


                                                      



%%%%%%%%%%%%%%%%%%%%%%%%%%%%%%%%%%%%%%%%%%%%%%%%%%
\parbf{\ref{Minimal spheres}.} 
\textit{Minimal spheres.}
Choose a pair of sufficiently close minimal spheres $\Sigma$ and $\Sigma'$,
say assume that the distance $a$ between $\Sigma$ and $\Sigma'$ is strictly smaller than the injectivity radius of the manifold.
Note that in this case there is a bijection $\Sigma\to \Sigma'$, which will be denoted by $p\mapsto p'$ such that the distance $|p-p'|=a$ for any $p\in\Sigma$.

Let $\iota_p\:\mathrm{T}_p\to\mathrm{T}_{p'}$ be the parallel translation along the (necessary unique) minimizing geodesic from $p$ to $p'$.
Use hairy ball theorem 
to show that there is a pair $(p,p')$ such that $\iota_p(\mathrm{T}_p\Sigma)=\mathrm{T}_{p'}\Sigma'$.

Consider pairs of unit-speed geodesics $\alpha$ and $\alpha'$ 
in $\Sigma$ and $\Sigma'$  
which start at $p$ and $p'$ correspondingly
and go in the parallel directions, say $\nu$ and $\nu'$. 
Set $\ell_\nu(t)=|\alpha(t)-\alpha'(t)|$.

Use the second variation formula to show that $\ell_\nu''(0)$ has negative average for all tangent directions $\nu$ to $\Sigma$ at $p$. 
In particular $\ell_\nu''(0)<0$ for a pair $\alpha$ and $\alpha'$ as above.
It follows that there are points $v\in\Sigma$ near $p$ 
and $v'\in\Sigma'$ near $p'$
such that 
\[|v-v'|<|p-p'|;\]
the later leads to a contradiction.

\parit{Comments.}
The problem was suggested by D. Burago,
it is related to Frankel's theorem on minimal surfaces; 
see \cite{frankel}.

It seems pleasurable that a 
compact 
positively curved 
4-dimensional manifold
can not contain a pair of equidistant spheres.
The argument above implies that the distance between such a pair has to exceed the injectivity radius of the manifold.

Here is a short list of classical problems with similar solutions:
\begin{itemize}
\item (Synge's problem \cite{synge})
{\it Any compact even-dimensional orientable manifold with strictly positive sectional curvature is
simply connected.}
\item (Frankel's problem \cite{frankel})
{\it Show that any two compact \hyperref[Minimal surface]{\emph{minimal hypersurfaces}} in a Riemannian manifold with positive Ricci curvature must intersect.}
\item (Bochner's problem \cite{bochner}.)
{\it  Let $(M,g)$ be a closed Riemannian manifold with negative Ricci curvature.
Prove that $(M,g)$ does not admit an isometric $\mathbb{S}^1$-action.}
\end{itemize}
The problem \ref{Totally geodesic immersion} can be considered as further development of this idea.




%%%%%%%%%%%%%%%%%%%%%%%%%%%%%%%%%%%%%%%%%%%%%%%%%%
\parbf{\ref{Totally geodesic immersion}.} 
\textit{Totally geodesic immersion.}
Set $n=\dim N$ and $m=\dim M$.

Fix a smooth increasing concave function $\phi$.
Consider the function $f=\phi\circ\dist_N$.
Note that if $f$ is smooth at $x$ 
the the Hessian, $\Hess_xf$, has at least $n+1$ negative eigenvalues.

Moreover, at any point $x\notin \iota(N)$ the same holds in the barrier sense;
that is, there is a smooth function $h\ge f$ defined on $M$ of $x$
such that $h(x)=f(x)$ and $\Hess_xf$ has at least $n+1$ negative eigenvalues.

Use that $m< 2\cdot n$ and the property to prove the following
analog of Morse lemma for $f$.

\parbf{Claim.}
{\it Given $x\notin \iota(N)$ there is a neighborhood $U\ni x$ such that the set 
\[U_-=\set{z\in U}{f(z)<x}\] is simply connected.}

\medskip

Since $M$ is simply connected,
any closed curve in $\iota(N)$
can be contracted by a disc, say $f_0\:\mathbb D\to M$.
According to the claim, 
there is a homotopy $f_t\:\mathbb D\to M$, $t\in [0,1]$ 
such that $f_t(\partial \mathbb D)\subset \iota(N)$ for any $t$ and $f_1(\mathbb D)\subset \iota(N)$.
It follows that $\iota(N)$ is simply connected.

Finally note that if $\iota\:N\to M$ has a self-intersection
then the image
$\iota(N)$ is not simply connected.
Hence the result follows.

\parit{Comments.}
The statement was proved by Fang, Mendon{\c{c}}a and Rong in \cite{FMR}.
The main idea in this proof was discovered by Wilking, 
see \cite{wilking-2003}.

%%%%%%%%%%%%%%%%%%%%%%%%%%%%%%%%%%%%%%%%%%%%%%%%%%
\parbf{\ref{kleiner-hopf}.} 
\textit{Positive curvature and symmetry.}
Let $M$ be a 4-dimensional Riemannian manifold with isometric $\mathbb{S}^1$-action.
Consider the quotient space $X=M/\mathbb{S}^1$.

Note that $X$ is a positively curved 3-dimensional Alexandrov space;
see \cite{akp} if in doubt.
In particular the angle $\measuredangle\hinge xyz$ between any two geodesics $[xy]$ and $[xz]$ is defined
and 
\[\measuredangle\hinge xyz+\measuredangle\hinge yzx+\measuredangle\hinge zxy> \pi.\leqno({*})\]
for any non-degenerate triangle $[xyz]$ formed by the minimizing geodesics $[xy]$, $[yz]$ and $[zx]$ in $X$.

Assume $p\in X$ corresponds to a fixed point of $\mathbb{S}^1$-action.
Show that 
for any three geodesics $[px]$, $[py]$ and $[pz]$ in $X$ we have
\[\measuredangle\hinge pxy+\measuredangle\hinge pyz+\measuredangle\hinge pzx\le \pi.\leqno({*}{*})\]
and
\[\measuredangle\hinge pxy, \measuredangle\hinge pyz, \measuredangle\hinge pzx\le \tfrac\pi2.\leqno({*}{*}{*})\]

Arguing by contradiction,
assume that there are 4 fixed points $q_1$, $q_2$, $q_3$ and $q_4$.
Connect each pair $q_i\ne q_j$ by a minimizing geodesic $[q_iq_j]$.

Denote by $\omega$ the sum of all 12 angles of the type  $\measuredangle\hinge{q_i}{q_j}{q_k}$.
By $({*}{*}{*})$, each triangle $\triangle q_iq_jq_k$ is non-degenerate.
Therefore by $({*})$, we have
\[\omega>4\cdot\pi.\]
Applying $({*}{*})$ at each vertex $q_i$, we have 
\[\omega\le 4\cdot\pi,\]
a contradiction.

\parit{Comment.}
The problem appears in the paper of Hsiang and Kleiner \cite{hsiang-kleiner}.
The connection of this proof to Alexandrov geometry was noticed by Grove in \cite{grove}.
An interesting development of this idea is given by Grove and Wilking in  \cite{grove-wilking}.



%%%%%%%%%%%%%%%%%%%%%%%%%%%%%%%%%%%%%%%%%%%%%%%%%%
\parbf{\ref{scalar-curv}.} 
\textit{Curvature vs. injectivity radius.}
We will show that if the injectivity radius of the manifold $(M,g)$ is at least $\pi$
then the average of sectional curvatures on $(M,g)$ is at most $1$.
This is equivalent to the problem.

Fix a point $p\in M$ and two orthonormal vectors $U,V\in\mathrm{T}_p M$.
Consider the geodesic $\gamma$ in $M$ such that $\dot\gamma(0)=U$.

Set $U_t=\dot\gamma(t)\in \mathrm{T}_{\gamma(t)}$ 
and let $V_t\in \mathrm{T}_{\gamma(t)}$ be the parallel translation of $V=V_0$ along $\gamma$.

Consider the field $W_t=\sin t\cdot V_t$ on $\gamma$.
Set 
\begin{align*}
\gamma_\tau(t)&=\exp_{\gamma(t)} (\tau\cdot W_t),
&
\ell(\tau)&=\length(\gamma_\tau|_{[0,\pi]}),
&
q(U,V)&=\ell''(0).
\end{align*}
Note that
\[q(U,V)=\int\limits_{0}^\pi [(\cos t)^2-K(U_t,V_t)\cdot (\sin t)^2]\cdot dt,\leqno({*})\]
where $K(U,V)$ denotes the curvature 
in the sectional direction spanned by $U$ and $V$. 

Since any geodesics of length $\pi$ is minimizing,
we get $q(U,V)\ge0$ for any pair of orthonormal vectors $U$ and $V$.
It follows that average value of the right hand side in $({*})$ is nonnegative.

By Liouville's theorem, while taking the average of $({*})$, we can switch the order of integrals;
therefore  
\[0\le \tfrac\pi2\cdot(1-\bar{K}),\]
where $\bar{K}$ denotes the average of sectional curvatures on $(M,g)$.
Hence the result follows.

\parit{Comments.} Here is a short list of problems which can be solved the same way; that is, by switch the order of integrals using Liouville's theorem.

%???+LIST-OF-PROBLEMS







%%%%%%%%%%%%%%%%%%%%%%%%%%%%%%%%%%%%%%%%%%%%%%%%%%
\parbf{\ref{almost-flat}.} 
\textit{Almost flat manifold.}
First prove that for given $\eps>0$, 
there is big enough $m$ and $m\times m$ integer matrix 
$A$ such that all its eigenvalues are $\eps$-close to $1$. 

Consider $(m+1)$-dimensional manifold $S$ obtained from $\TT^m\times [0,1]$ by gluing $\TT^m\times 0$ to $\TT^m\times 1$ along the map given by $A$.

Assuming that $\eps$ is small,
show that $S$ admits a metric with curvature and diameter sufficiently small.

\parit{Comment.} 
This example was constructed by Guzhvina in \cite{guzhvina}.

The main theorem of Gromov in \cite{gromov-almost-flat}, 
states that there are no such examples of fixed dimension;
a more detailed proof can be found in \cite{buser-karcher}
and a more precise statement can be found in \cite{ruh}.

It is expected that for small enough $\eps>0$,
a manifold of any dimension 
with diameter $\le 1$ and sectional
curvature at most $\eps$ 
has nontrivial fundamental group.


%%%%%%%%%%%%%%%%%%%%%%%%%%%%%%%%%%%%%%%%%%%%%%%%%%
\parbf{\wrenches\ref{lie-nonneg}.} 
\textit{Lie group.} 
Consider the product $(G,g_0)\times (G,g_1)$;
it is the Lie group $G\times G$ with a left invariant metric.

Consider the diagonal subgroup 
\[\Delta= \{(g,g)\in G\times G\}.\]
The quotient space $(G,g_0)\times (G,g_1)/\Delta$
is isometric to $(G,???g_0+g_1)$.
Hence the result follows.

\parit{Comment.}
This trick was used in \cite{GKM} to show that Berger's spheres have positive curvature.
This is the earliest case I was able to find. 
Most of the examples of positively and non-negatively curved manifolds are constructed using this trick,
see \cite{aloff-wallach}, \cite{gromoll-meyer}, \cite{eschenburg-spaces} and \cite{bazajkin}.




%%%%%%%%%%%%%%%%%%%%%%%%%%%%%%%%%%%%%%%%%%%%%%%%%%
\parbf{\ref{milka-polar}.} 
\textit{Polar points.}
Fix a unit-speed geodesic $\gamma$ such that $\gamma(0)=p$.
Set $p^*=\gamma(\pi)$.

Prove that $p^*$ is a solution.

\parit{Alterntive proof.} 
Assume contrary;
that is, for any $x\in M$ there is a point $x'$ such that 
\[|x-x'|_g+|p-x'|_g>\pi.\]

Show that there is a continuous map $x\mapsto x'$
such that the above inequaliy holds for any $x$.

Fix sufficiently small $\eps>0$.
Prove that the set $W_\eps=M\backslash B(p,\eps)$ 
is homeomorphic to a ball 
and the map $x\mapsto x'$ sends $W_\eps$ into itself.

By Brouwer's fixed-point theorem, $x=x'$ for some $x$.
In this case 
\[|x-x'|_g+|p-x'|_g\le \pi,\]
a contradiction.
 
\parit{Comments.}
The problem was considered by Milka's in \cite{milka-poly}, he gave there the first proof.

%%%%%%%%%%%%%%%%%%%%%%%%%%%%%%%%%%%%%%%%%%%%%%%%%%
\parbf{\wrenches\ref{Deformation to a product}.} 
\textit{Deformation to a product.} 
Denote by $\Gamma$ the fundamental group of $M$;
if $\Gamma$ is finite then the universal cover will do the trick.

Let $(\tilde M,\tilde g)$ be universal cover of $(M,g)$ with induced Riemannian metric.
The space $(\tilde M,\tilde g)$ is isometric to a product $\RR^k\times K$, where $K$ is a compact Riemannian manifold.

Denote by $G$ the isometry group of $K$.
Given a continuous one parameter family of homomorphisms $\phi_t\:\RR^k\to G$,
consider the the  one parameter family of diffeomorphisms of $\RR^k\times K$ to itself defined as
\[\Phi_t\:(x,k)\mapsto (x,\phi_t(x)\cdot k).\]
Denote by 
 $\tilde g_t$ pullback 
of $\tilde g$ via $\Phi_t$,
so 
\[\Phi_t\:(\RR^k\times K,\tilde g_t)\to (\RR^k\times K,\tilde g)\]
is an isometry.

It remains to find the one parameter family $\phi_t$ such that 
 $\tilde g_t$ is $\Gamma$-invariant for all $t$.
and $(M,g_1)=(\tilde M,\tilde g)/\Gamma$ admits a finite Riemannian cover by the product of a flat torus and $K$.


In terms of $\phi_t$, it can be formulated the following way.
There is a normal subgroup of finite index $\Gamma_0\vartriangleleft\Gamma$ such that 
\begin{itemize}
\item $\Gamma_0$ acts on $\RR^k$ by parallel translations; 
in particular $\Gamma_0$ can be identified with with a lattice in $\RR^k$.
\item the action of $\Gamma_0$ on $K$ is given by $\phi_1(\Gamma_0)$.
\end{itemize}


\parit{Comment.}
The problem appears in Wilking's paper \cite{wilking-2000}.



%%%%%%%%%%%%%%%%%%%%%%%%%%%%%%%%%%%%%%%%%%%%%%%%%%
\parbf{\ref{Isometric section}.} 
\textit{Isometric section.}
Arguing by contradiction, 
assume $\iota\: M\z\to W$ is an isometric section.
It makes possible to treat $M$ as a submanifold in $W$.

Given $p\in M$,
denote by $\nu_pM$ the sphere of unit normal vectors to $M$ at $p$.
Given $v\in \nu_p$ and real value $k$,
set 
\[p^{k\cdot v}=s\circ\exp_{\iota(p)} (k\cdot v).\]
Note that 
\[p^0=p\ \ \text{for any}\ \  p\in M.\leqno({*})\]

Fix sufficiently small $\delta>0$.
By Rauch comparison, if $w\in \nu_q$ 
is the parallel translation of $v\in \nu_q$ 
along the minimizing geodesic from $p$ to $q$ in $M$
then 
\[|p^{k\cdot v}-q^{k\cdot w}|_M<|p-q|_M
\leqno({*}{*})\]
assuming $|k|\le \delta$.
The same comparison implies that 
\[|p^{k\cdot v}-q^{k'\cdot w}|_M^2<|p-q|_M^2+ (k-k')^2
\leqno({*}{*}{*})\]
assuming $|k|,|k'|\le \delta$.

Choose $p$ and $v \in \nu_p$ so that $r=|p-p^{\delta\cdot v}|$ 
takes the maximal possible value.
From $({*}{*})$ it follows that $r>0$.

Let $\gamma$ be the extension of unit-speed minimizing geodesic from $p_v$ to $p$;
denote by $v_t$ the parallel translation of $v$ to $\gamma(t)$ along $\gamma$. 

We can choose the parameter of $\gamma$ so that $p=\gamma(0)$, $p^v=\gamma(-r)$.
Set $p_n=\gamma(n\cdot r)$, so $p=p_0$ and $p^v=p_{-1}$. 
Fix large integer $N$ and set $w_n=(1-\tfrac nN)\cdot v_{n\cdot r}$
and $q_n=p_n^{w_n}$.

%%%???+PIC

From $({*}{*}{*})$, there is a constant $C$ independent of $N$ such that
\[|q_k-q_{k+1}|<r+\tfrac C{N^2}\cdot\delta^2.\]
Therefore 
\[|q_{k+1}-p_{k+1}|>|q_k-p_k|-\tfrac C{N^2}\cdot\delta^2.\]
By induction, we get 
\[|q_N-p_N|>r-\tfrac C{N}\cdot\delta^2.\]
Since $N$ is large we get
\[|q_N-p_N|>0.\]
By $({*})$ we get $q_N=p_N^0=p_N$, a contradiction.

\parit{Comment.} This proof is the core of Perelman's proof of Soul conjecture, 
see \cite{perelman}.

%%%%%%%%%%%%%%%%%%%%%%%%%%%%%%%%%%%%%%%%%%%%%%%%%%
\parbf{\ref{pr:Minkowski space}.} 
\textit{Minkowski space.}
Fix an increasing function $\phi\:(0,r)\to \RR$
such that 
\[\phi''+(n-1)\cdot(\phi')^2+C=0.\]

Note that if $\Ric_{g_n}\ge C$ then the function 
$x\mapsto\phi(|p-x|_{g_n})$ is subharmonic.
In follows that, 
for arbitrary array of points $p_i$ 
and positive reals $\lambda_i$ the function $f_n\:M_n\to \RR$
defined by the formula
$$f(x)=\sum_i\lambda_i\cdot\phi(|p_i-x|_M)$$
is subharmonic.
In particular $f_n$ can not admit a local minima in $M_n$.

Passing the limit as $n\to \infty$, we get that any function $f\:\mathbb{R}^m\to\mathbb{R}$
of the form 
$$f(x)=\sum_i\lambda_i\cdot\phi(|p_i-x|_{\ell_p})$$
does not admit a local minima in $\mathbb{R}^m$.

It remains to arrive to a contradiction
by showing that if $p\ne 2$ then there is an array
points $p_i$ and positive reals $\lambda_i$
such that the function 
$$f(x)=\sum_i\lambda_i\cdot\phi(|p_i-x|_{\ell_p})$$
has strict local minimum.

\parit{Comment.} The argument given here is very close to the proof of Abresch--Gromoll inequality, see \cite{abresch-gromoll}.
An alternative solution of this problem can be build on almost splitting theorem of Cheeger and Colding, see \cite{cheeger-colding}.


%%%%%%%%%%%%%%%%%%%%%%%%%%%%%%%%%%%%%%%%%%%%%%%%%%
\parbf{\ref{Curvature hollow}.} 
\textit{Curvature hollow.}
Construct a metric that the connected sum
$M=\RR^3\#\mathbb{S}^2\times\mathbb{S}^1$ admits a metric which is flat outside a compact set and has non positive scalar curvature.
Further, note that such metric can be constructed in such a way that it has a closed geodesic $\gamma$ with trivial holonomy and with constant negative curvature in its a tubular neighborhood.

Cut the tubular neigbourhood $D^2\times \mathbb{S}^1$ of $\gamma$, 
prepare a metric $g$ on $\mathbb{S}^1\times D^2$ with negative scalar curvature which 
is identical to the original metric near the boundary.
The needed patch $(\mathbb{S}^1\times D^2,g)$ can be found among wrap products $\mathbb{S}^1\times_f D^2$.

Note that after the surgery we get a manifold diffeomorphic to $\RR^3$ with the required metric.

\parit{Comments.}
This construction was given by Lohkamp in \cite{lohkamp},
he descrives there yet an other equally simple construction.
In fact Lohkamp  constructs the hollows with negative Ricci curvature.

On the other hand there are no hollows with positive scalar curvature;
the later follows from Positive Mass Conjecture.

%%%%%%%%%%%%%%%%%%%%%%%%%%%%%%%%%%%%%%%%%%%%%%%%%%
\parbf{\wrenches\ref{hS=>S}.} 
\textit{If hemisphere then sphere.}
Denote by $q$ a point in $M$ which lies on the maximal distance from $p$.

Consider the function $f=\cos \dist_p+\cos\dist_q$.
Note that 
\[\triangle f+m\cdot f\le 0\] 
in the sense of distributions.
If follows that $f\ge 0$, in particular 
\[B(p,\tfrac\pi2)\cup B(q,\tfrac\pi2)=M.\]

Set \[a(r)=\area(\partial [B(q,r)\backslash B(p,\tfrac\pi2)]).\]
Prove that the function
\[r \mapsto \tfrac{a(r)}{(\sin r)^{m-1}}\]
is nonincreasing 
and 
\[\tfrac{a(r)}{(\sin r)^{m-1}}\le\area\mathbb{S}^{m-1}.\]
Moreover if equality holds for some $r$ then $B(q,r)\backslash B(p,\tfrac\pi2)$ is isometric to an $r$-ball in the unit sphere.
This statement is analogous to the Bishop--Gromov inequality and can be proved the same way.

Finally note that $a(\tfrac\pi2)=\area\mathbb{S}^{m-1}$,
hence the result follows.
 

\parit{Comments.}
The problem appears in paper of Hang %Fengbo Hang fengbo@cims.nyu.edu
and Wang %Xiaodong Wang xwang@math.msu.edu 
\cite{hang-wang};
their proof is different.
The problem is still nontrivial 
even if instead of the first condition one has that sectional curvature $\ge 1$.
If instead of first condition one only has that scalar curvature $\ge m\cdot(m-1)$, then the conclusion does not hold; 
it was conjectured by Min-Oo (1995) and disproved in \cite{brendle-marques-neve}



%%%%%%%%%%%%%%%%%%%%%%%%%%%%%%%%%%%%%%%%%%%%%%%%%%
\parbf{\ref{Flat coordinate planes}.} 
\textit{Flat coordinate planes.}
Fix $\eps>0$ such that there is unique geodesic between any two points on distance $<\eps$ from the origin of $\RR^3$.

Consider three points $a$, $b$ and $c$ 
on the coordinate lines which are $\eps$-close 
to the origin.

Prove that the angles of the triangle $\triangle abc$
coincide with its model angles.
It follows that there is a flat geodesic triangle in $(\RR^3,g)$ with vertex at $a$, $b$ and $c$.

Use the family of constructed flat triangles 
to show that at any $x$ point in the $\tfrac\eps{10}$-neighborhood of the origin
the sectional curvature 
vanish in an open set of sectional directions.
The later implies that the curvature is identically zero 
in this neighborhood.

Moving the origin and apply the same argument we get that the curvature is identically zero everywhere.
Hence the result follows. 

\parit{Comment.} This problem appears in the paper of Panov and me \cite{panov-petrunin}; it is based on a lemma discovered by Buyalo in \cite{buyalo}.

%%%%%%%%%%%%%%%%%%%%%%%%%%%%%%%%%%%%%%%%%%%%%%%%%%
\parbf{\ref{Two-convexity}.}
\textit{Two-convexity.}
Assume $W$ is a hypersurface of needed type.

\parit{Morse-style solution.}
Let us equip $\RR^4$ with $(x,y,z,t)$-coordinates.

Consider a generic linear function $\ell\:\RR^4\to\RR$
which is close to the sum of coordinates $x+y+z+t$.
Note that $\ell$
has non-degenerate critical points on $W$ and all its critical values are different.

Consider the sets 
$$W_s=\set{w\in \RR^4\backslash K}{\ell(w)<s}.$$
Note that $W_{-1000}$ contains a closed curve, say $\alpha$, 
which is contactable in $\RR^4\backslash K$, 
but not constructable in the set $W_{-1000}$.

Set $s_0$ to be the infimum of the values $s$ such that
the $\alpha$ is contactable in $W_s$.

Note that $s_0$ is a critical value of $\ell$ on $W$;
denote by $p_0$ the corresponding critical point.
By 2-convexity of $\RR^4\backslash K$,
the index of $p_0$ has to be at most $1$.
On the other hand, since the disc hangs at this point,
its index has to be at least $2$,
 a contradiction.

\parit{Alexandrov-style proof.}
Fix a constant Riemannian metric $g$ on $\RR^4$.
According to the main result of Alexander Bishop and Berg in \cite{ABB}, $X_g=(\RR^4\backslash (\Int K),g)$ has nonpositive curvature in the sense of Alexandrov.
In particular the universal cover of $\tilde X_g$ of $X_g$ is a $\CAT[0]$ space.

By rescaling $g$ and passing to the limit we obtain that universal Riemannian cover $Z_g$ of $(\RR^4,g)$ branching in the coordinate planes is a $\CAT[0]$ space.
Show that $Z_g$ is $\CAT[0]$ space if and only if the two planes are orthogonal with respect to $g$;
the later leads to a contradiction.

\parit{Comments.}
Note that the closed $1$-neighborhood of these two planes has two-convex complement, but the boundary of this neighborhood is not smooth.

The Morse-style is closely related 
to Gromov's lectures \cite[\S$\sfrac{1}{2}$]{gromov-SGMC}.

\section*{Curvature free differential geometry}

%%%%%%%%%%%%%%%%%%%%%%%%%%%%%%%%%%%%%%%%%%%%%%%%%%
\parbf{\ref{gromomorphic-curves}.} 
\textit{Minimal foliation.}
First show that there is a self-dual harmonic 2-form on $(\mathbb{S}^2\times\mathbb{S}^2,g)$;
that is, a 2-form $\omega$ such that $d\omega=0$ and $\star\omega=\omega$,
where $\star$ denotes the Hodge star operator.

Fix $p\in \mathbb{S}^2\times\mathbb{S}^2$.
Use the identity $\star\omega_p=\omega_p$
to show that
there is a real number $\lambda_p$ and the isometry $\mathrm{J}_p\:\mathrm{T}_p\to\mathrm{T}_p$ 
such that
$\mathrm{J}_p\circ\mathrm{J}_p =-\id$ 
and 
$\omega(X,Y)=\lambda_p\cdot g(X,\mathrm{J}_pY)$ for any $X,Y\in \mathrm{T}_p$.

Consider canonical symplectic form $\omega_0$ on $\mathbb{S}^2\times\mathbb{S}^2$;
that is sum of pullbacks of volume forms on $\mathbb{S}^2$  
for the two projections $\mathbb{S}^2\times\mathbb{S}^2\to \mathbb{S}^2$.
Note that for the canonical metric on $\mathbb{S}^2\times\mathbb{S}^2$,
the form $\omega_0$ is harmonic and self-dual. 
Since $g$ is close to the standard metric,
we can assume that $\omega$ is close to $\omega_0$.
In particular $\lambda_p\ne0$ for any $p\in \mathbb{S}^2\times\mathbb{S}^2$.

It follows that $\omega$ defines simplectic structure on $\mathbb{S}^2\times\mathbb{S}^2$
and $\mathrm{J}$ is its psudocomplex structure.
It remains to take the reparametrization of $\mathbb{S}^2\times \mathbb{S}^2$
so that vertical and horizontal spheres will form pseudoholomorphic curves in the homology classes of $\mathbb{S}^2\times x$ and $x\times \mathbb{S}^2$.
 
\parit{Comments.} The pseudoholomorphic curves (sometimes called \emph{gromomorphic curves}) 
were introduced by Gromov in \cite{gromov-pseudoholomorphic}.
For general metric the form $\omega$ might vanish at some points;
if the metric is generic then it happens on disjoint circles,
see \cite{honda}.





%%%%%%%%%%%%%%%%%%%%%%%%%%%%%%%%%%%%%%%%%%%%%%%%%%
\parbf{\ref{Loewner's theorem}.} 
\textit{Loewner's theorem.}
Denote by $\lambda$ the conformal factor of $g$;
i.e, $g=\lambda^2\cdot g_{\mathrm{can}}$.

Denote by $s$ the average of $g$-lengths of the lines in $\RP^n$.
Prove that 
\[\ell \le s=\pi\cdot\oint\limits_{\RP^n}\lambda\cdot d\vol_{\mathrm{can}},\]
where $\vol_{\mathrm{can}}$ denotes the volume for $g_{\mathrm{can}}$ and $\oint$ denoted the avarage value.

Note that
\[\vol(\RP^n,g)=\vol(\RP^n,g_{\mathrm{can}})\cdot\oint\limits_{\RP^n}\lambda^n\cdot d\vol_{\mathrm{can}}.\]
By H\"older's inequality, we have
\[\left(\,\,\oint\limits_{\RP^n}\lambda\cdot d\vol_{\mathrm{can}}\right)^n
\le \oint\limits_{\RP^n}\lambda^n\cdot d\vol_{\mathrm{can}}.\]
Hence the result follows.

%%%%%%%%%%%%%%%%%%%%%%%%%%%%%%%%%%%%%%%%%%%%%%%%%%
\parbf{\ref{convex-infinite}.} 
\textit{Convex function vs. finite volume.}
Assume contrary; that is, there is a complete Riemannian manifold $M$
with finite volume which admits a convex function $f$.

Denote by $\tau\:\mathrm{T}^1 M\to M$ the unit tangent bundle over $M$. 
Clearly $\vol T^1M$ is finite.

Note that 
there is a nonempty bounded open set $U\subset \mathrm{T}^1 M$
such that $df(u)>\eps$ for any $u\in U$ and some fixed $\eps>0$.

Denote by $\phi^t$ the geodesic flow on $\mathrm{T}^1 M$.
Given $u\in U$,
consider the function $h\:t\mapsto f\circ\tau\circ\phi^t(u)$.
Note that $h'(t)>\eps$ for any $t\ge 0$.

Prove that there is an infinite sequence of positive reals $t_1,t_2,\dots$
such that 
$$\phi^{t_i}(U)\cap\phi^{t_j}(U)=\emptyset$$ 
if $i\ne j$.
The later implies that $\vol T^1M=\infty$,
a contradiction.

\parit{Comment.} The problem appears in Yau's paper \cite{yau}.

%%%%%%%%%%%%%%%%%%%%%%%%%%%%%%%%%%%%%%%%%%%%%%%%%%
\parbf{\ref{Besikovitch inequality}.} 
\textit{Besikovitch inequality.}
Set 
\[A_i=\set{(x_1,x_2,\dots,x_n)\in[0,1]^n}{x_i=0}.\]

Consider functions $f_i\:[0,1]^n\to\RR$ defined by
$f_i(x)=\dist_{A_i}x$.
Note that 
the map $\bm{f}\:([0,1]^n,g)\to\RR^n$
defined as
\[\bm{f}\:x\mapsto(f_1(x),f_2(x),\dots,f_n(x))\]
is Lipschitz.

Prove that Jacobian of  $\bm{f}$
is at most $1$
and $\bm{f}([0,1]^n)\supset [0,1]^n$.
Hence the result follows.

It remains to do the equality case.

\parit{Comments.} 
This inequality was proved by Besicovitch in \cite{besicovitch}.
%???It has number of deep applications in Riemannian geometry, for example assume $W$ be a convex body with smooth boundary in $\RR^m$ and $g$ be a Riemannian metric on $W$


%%%%%%%%%%%%%%%%%%%%%%%%%%%%%%%%%%%%%%%%%%%%%%%%%%
\begin{wrapfigure}{r}{32mm}
%\begin{center}
\begin{lpic}[t(-2mm),b(3mm),r(0mm),l(0mm)]{pics/tripod}
%\lbl[b]{24,41;x}
\end{lpic}
%\end{center}
\end{wrapfigure}

\parbf{\ref{Distant involution}.} 
\textit{Distant involution.}
Given $\eps>0$, construct a disc $D$ in the plane with 
$$\length\partial D<10\ \ \text{and}\ \ \area D<\eps$$
which admits an continuous involution $\iota$ such that 
$$|\iota(x)-x|\ge 1$$ 
for any $x\in\partial D$.
An example of $D$ can be guessed from the picture. 

Take the product $D\times D\subset \RR^4$;
it is homeomorphic to a 4-dimensional ball.
Note that 
$$\vol_3[\partial(D\times D)]=2\cdot\area D\cdot\length \partial D<20\cdot\eps.$$
The boundary $\partial(D\times D)$ homeomorphic to $\mathbb{S}^3$
and the restriction of the involution $(x,y)\mapsto (\iota(x),\iota(y))$ has the needed property.

It remains to smooth $\partial(D\times D)$.

\parit{Comments.}
This example was discovered by Croke in \cite{croke}.

It is instructive to show that for $\mathbb{S}^2$ such thing is not possible.

Note that according to Gromov's systolic inequality, 
the involution $\iota$ above can not be made isometric (see \cite {gromov-filling}).

%%%%%%%%%%%%%%%%%%%%%%%%%%%%%%%%%%%%%%%%%%%%%%%%%%
\parbf{\ref{Normal exponential map}.} 
\textit{Normal exponential map.}
Assume contrary; that is, there is a point $p\in M$ 
such that the image of normal exponential map to $N$
 does not touch $\eps$-neighborhood of $p$.

Show that given $R>0$ there is $\delta>0$ such that 
if $x\in N$ and $|p-x|_M<R$ 
then there is a unit speed curve in $N$
which moves to $p$ with velocity at least $\delta$.
(In fact, the value $\delta$ depends on $\eps$, $R$ and the curvature bounds in $B(p,R)$.)

\begin{wrapfigure}{r}{25mm}
\begin{lpic}[t(-0mm),b(0mm),r(0mm),l(0mm)]{pics/spiral}
\end{lpic}
\end{wrapfigure}

Following this curve for sufficient time brings us to $p$;
that is, $p\in N$, a contradiction.

\parit{Comments.} 
The problem was suggested by Alexander Lytchak.

From the picture, you should guess an example of immersion 
$\iota\:\RR\looparrowright\RR^2$ 
such that one point does not lie in the image of the corresponding normal exponential.
It might be interesting to see a more detailed picture for the sets which can appear as the complement to the image of normal exponential map.


%%%%%%%%%%%%%%%%%%%%%%%%%%%%%%%%%%%%%%%%%%%%%%%%%%
\parbf{\ref{Symplectic squeezing in the torus}.} 
\textit{Symplectic squeezing in the torus.}
Equip $\RR^4$ with $(x_1,y_1,x_2,y_2)$-coordinates
so that 
\[\omega=dx_1\wedge dy_1+dx_2\wedge dy_2\]
is the symplectic form. 

The embedding will be given as a composition of a linear symplectomorphism $\lambda$ 
with the quotient map $\phi\:\RR^4\to \TT^2\times\RR^2$ by the integer $(x_1,y_1)$-lattice.
Clearly $\phi\circ\lambda$ preserves the symplectic structure,
it remains to find $\lambda$ such that the restriction $\phi\circ\lambda|_\Omega$
is injective.

Without loss of generality,
we can assume that $\Omega$ is a ball centered at the origin.
Choose an oriented 2-dimensional subspace $V$ subspace of $\RR^4$ 
such that the integral of $\omega$ over 
$\Omega\cap V$ is small positive number, say $\tfrac\pi4$. 

Note that there is a linear symplectomorphism $\lambda$
 which maps planes parallel to $V$ to planes
parallel to the $(x_1,y_1)$-plane, 
and that maps the disk $V\cap\Omega$ to a disk.
It follows that the the intersection of $\lambda(\Omega)$ 
with any plane parallel to the $(x_1,y_1)$-plane is a disk of radius at most $\tfrac 12$.
In particular $\phi\circ\lambda|_\Omega$
is injective.

\parit{Comments.}
This construction is given by Guth in \cite{guth-symplectic}
and attributed to Leonid Polterovich.

Note that according to Gromov's non-squeezing theorem \cite{gromov-pseudoholomorphic}, 
an analogous statement with $\CC\times \DD$ as the target does not hold, here $\DD\subset \CC$ is the open disc with incuced symplectic structure.
In particular, it shows that
the projection of $\lambda(\Omega)$ as above 
to $(x_1,y_1)$-plane
can not be made arbitrary small.

%%%%%%%%%%%%%%%%%%%%%%%%%%%%%%%%%%%%%%%%%%%%%%%%%%
\parbf{\ref{Diffeomorphism test}.} 
\textit{Diffeomorphism test.}
Since $N$ is simply connected, 
it is sufficient to show that $f\:M\to N$ is a covering map.

Note that $f$ is an open immersion.
Let $h$ be the pullback metric on $M$ for $f\:M\to N$.
Clearly $h\ge g$.
In particular $(M,h)$ is complete and the map $f\:(M,h)\to N$ is a local isometry. 

It remains to prove that any local isometry between complete connected Riemannian manifolls of the same dimension if a covering map.  

%%%%%%%%%%%%%%%%%%%%%%%%%%%%%%%%%%%%%%%%%%%%%%%%%%
\parbf{\ref{Volume of tubular neighborhoods}.} 
\textit{Volume of tubular neighborhoods.}
Let us denote by $\mathrm{N} M$ and $\mathrm{T} M$ the normal and tangent bundle of $M$ in $\RR^n$.

Consider the the normal exponential map $\exp_M\:\mathrm{N} M\to\RR^n$
and denote by $J_V$ its Jacobian at $V\in \mathrm{N}_pM$.
Note that for all small $\eps>0$, we have
\[\vol B_\eps(M)=\int\limits_M d_p\vol_m\cdot\int\limits_{B(0,r)_{\mathrm{N}_pM}}J_V\cdot d_V\vol_{n-m}.\leqno{({*})}\]

Set $m=\dim M$.
Given $p\in M$, 
denote by $s_p\:\mathrm{T}_p\times \mathrm{T}_p\to \mathrm{N}_p$
the \hyperref[Second fundamental form]{second fundamental form} of $M$.
Recall that the curvature tensor of $M$ at $p$ can be expressed the following way
\[R_p(X\wedge Y), V\wedge W\rangle 
=\langle s_p(X,W), s_p(Y,V)\rangle-\langle s_p(X,V), s_p(Y,W)\rangle.\]

Given $V\in \mathrm{N}_p M$,
express $J_V$ in terms of $\<s_p(X,Y),V\>$.
Show that for small $r$ the integral
\[v(r)=\int\limits_{B(0,r)_{\mathrm{N}_pM}}J_V\cdot d_V\vol_{n-m}\]
is a polynomial 
of $r$ and its coefficients can be expressed in terms of the curvature tensor $R_p$.

It follows that the right hand side in $({*})$ can be expressed in terms of curvature tensor of $M$.
The problem follows since the curvature tensor can be expressed in terms of metric tensor of $M$.

\parit{Comments.} The formula for volume of tubular neighborhood of submanifolds
used in the proof was described by Weyl in \cite{weyl}.



%%%%%%%%%%%%%%%%%%%%%%%%%%%%%%%%%%%%%%%%%%%%%%%%%%

\begin{wrapfigure}{r}{42mm}
%\begin{center}
\begin{lpic}[t(-5mm),b(-3mm),r(0mm),l(0mm)]{pics/tree(1)}
%\lbl[b]{24,41;x}
\end{lpic}
%\end{center}
\end{wrapfigure}

\parbf{\ref{Disc}.} 
\textit{Disc.}
Show that given a positive integer $n$ one can construct a tree $T$ embedded into the disc such that any homotopy of the boundary of the disc to a point pass through a curve which intersects $n$ different edges.
(For the tree on the diagram $n=3$.)


Fix small $\eps>0$, say $\eps=\tfrac1{10}$.
Consider the disc with embedded tree $T$ as above.
We will construct a metric on the disc 
with diameter and length of its boundary below $1$
such that 
the distance between any two edges of $T$ of without common vertex 
is at least $\eps$ .

To construct such a metric, first fix a metric on the cylinder $\mathbb S^1\times [0,1]$ such that 
\begin{itemize}
\item The $\eps$-neighborhoods of the boundary components are product metrics.
\item Any vertical sigment $x\times[0,1]$ has length $\tfrac 12$.
\item One of the boundary component has length $\eps$.
\item The other boundary component has length $2\cdot m\cdot \eps$, 
where $m$ is the number of edges in $T$.
\end{itemize}
Equip $T$ with a metric so that each edge has length $\eps$
and glue the long boundary component of the cylinder to $T$ by piecewise isometry so that the resulting space is homeomorphic to disc and the tree corresponds to it-self.

According to the first construction,
for any null-homotopy of the boundary 
the least length is at least $n\cdot\tfrac{\eps}{10}$.
The obtained metric is not Riemannian, but is is easy to smooth.
Since $n$ is arbitrary the result follows.

\parit{Comments.} 
This example was constructed by Frankel and Katz \cite{frankel-katz}.
 

%%%%%%%%%%%%%%%%%%%%%%%%%%%%%%%%%%%%%%%%%%%%%%%%%%
\parbf{\ref{short-homotopy}.} 
\textit{Shortening homotopy.}
Set 
\[p=\gamma_0(0)\ \ \text{and}\ \  \ell_0=\length\gamma_0.\]

By compactness argument,
there exists $\delta>0$ 
such that no geodesic loops based at $p$ with has length in the interval $(L-D, L+D+\delta]$. 

Assume $\ell_0\ge L+\delta$.
Choose $t_0\in [0,1]$ such that
\[\length\left(\gamma_0|_{[0,t_0]}\right)=L+\delta\]
Let $\sigma$ be a the minimizing geodesic from $\gamma(t_0)$
to $p$.
Note that $\gamma_0$ is homotopic to the joint 
\[\gamma_0'=\gamma_0|_{[0,t_0]}*\sigma*\bar\sigma*\gamma|_{[t_0,1]},\]
where $\bar\sigma$ denotes the backward parametrization of $\sigma$.

Consider the loop $\lambda_0$ at $p$
formed by joint of $\gamma|_{[0,t_0]}$ and $\sigma$.
Applying a curve shortening process to $\lambda_0$, 
we get a curve shortening homotopy $\lambda_t$
rel. its ends 
from the loop $\lambda_0$ to a geodesic loop $\lambda_1$ at $p$.
From above, 
\[\length\lambda_1\le L-D.\]

The joint $\gamma_t=\lambda_t*\bar\sigma*\gamma|_{[t_0,1]}$
is a homotopy
from $\gamma_0'$ to an other curve $\gamma_1$.
From the construction it is clear that 
\begin{align*}
 \length \gamma_t&\le \length \gamma_0+2\cdot \length\sigma\le
 \\
 &\le \length \gamma_0+2\cdot D
\end{align*}
for any $t\in[0,1]$
and 
\begin{align*}
 \length \gamma_1&=\length\lambda_1+\length\sigma+\length\gamma|_{[t_0,1]}\le
\\ &\le L-D+D+\length\gamma-(L+\delta)=
\\ &=\ell_0 -\delta.
\end{align*}
Repeating the procedure few times we get we get curves $\gamma_2$, $\gamma_3,\dots,\gamma_n$
joint by the needed homotopies so that 
$\ell_{i+1}\le\ell_i-\delta$ and $\ell_n< L+\delta$,
where $\ell_i=\length\gamma_i$.

If $\ell_n\le L$, we are done.
Otherwise repeat the argument once more for $\delta'=\ell_n-L$.

\parit{Comments.}
The problem discussed by Nabutovsky and Rotman in \cite{nabutovsky-rotman}.

It is not at all easy to find an example of a manifold  which satisfy the above condition for some $L$;
they are found among the Zoll spheres
by Balachev, Croke and Katz, 
see \cite{balacheff-croke-katz}.

%%%%%%%%%%%%%%%%%%%%%%%%%%%%%%%%%%%%%%%%%%%%%%%%%%
\parbf{\ref{Geodescic hypersurface}.} 
\textit{Geodescic hypersurface.}
Let $h$ be the maximal distance from points in $W$ to $M$.

Fix a fine triangulation of $W$ 
so that $M$ becomes a subcomplex.
Say, let us assume that the diameter of each simplex in $\tau$ is less than 
$\eps$.
We can assume that $\tau$ is a barycentric subdivision of an other triangulation, so all the vertices of $\tau$ can be colored into colors $(0,\dots, m+1)$
in such a way that the vertices of each simplex 
get different colors.
Denote by $\tau_i$ the maximal $i$-dimensional subcomplex of $\tau$ 
with all the vertices colored by $0,\dots, i$.

For each vertex $v$ in $\tau$ 
choose a point $v'\in M$ on the distance $\le h$.
Note that if $v$ and $w$ are the vertices of one simplex then
\[|v'-w'|_M<2\cdot h+\eps.\]

If $\tfrac{r}{2\cdot(m+1)}>h$, take $\eps<\tfrac{r}{2\cdot(m+1)}-h$.
Let us extend the map $v\mapsto v'$ 
to a continuous 
map $W\to M$.
The map is already defined on $\tau_0$.
Using the cone construction we can extend it to $\tau_1$;
we can do this since the distance between vertices in one simplex are below injectivity radius of $M$.
Repeat the cone construction recursively, to extend the map to $\tau_2,\dots,\tau_{m+1}=\tau$;
some distance estimates are needed here.

It follows that fundamental class of $M$ vanish in the homology ring of $M$, 
a contradiction.  

\parit{Comment.}
This problem is a stripped version of Gromov's bound on filling radius given in \cite{gromov-filling}.  



\section*{Metric geometry}



%%%%%%%%%%%%%%%%%%%%%%%%%%%%%%%%%%%%%%%%%%%%%%%%%%
\parbf{\ref{Noncontracting map}.} 
\textit{Noncontracting map.}
Given any pair of point $x_0,y_0\in K$, 
consider two sequences $x_0,x_1,\dots$ and $y_0,y_1,\dots$
such that 
and $x_{n+1}=f(x_n)$ and $y_{n+1}=f(y_n)$ for each $n$.

Since $K$ is compact, 
we can choose an increasing sequence of integers $n_k$
such that both sequences $(x_{n_i})_{i=1}^\infty$ and $(y_{n_i})_{i=1}^\infty$
converge.
In particular, both of these sequences  are Cauchy;
that is,
\[
|x_{n_i}-x_{n_j}|_K, |y_{n_i}-y_{n_j}|_K\to 0
\ \ 
\text{as}
\ \ \min\{i,j\}\to\infty.
\]


Since $f$ is noncontracting, we get
\[
|x_0-x_{|n_i-n_j|}|
\le 
|x_{n_i}-x_{n_j}|.
\]

It follows that  
there is a sequence $m_i\to\infty$ such that
\[
x_{m_i}\to x\ \ \text{and}\ \ y_{m_i}\to y\ \ \text{as}\ \ i\to\infty.
\leqno({*})\]

Set \[\ell_n=|x_n-y_n|_K,\]
where $|{*}-{*}|_K$ denotes the distance between points in $K$.
Since $f$ is noncontracting, $(\ell_n)$ is a nondecreasing sequence.

By $({*})$, it follows that $\ell_{m_i}\to\ell_0$ as $m_i\to\infty$.
It follows that $(\ell_n)$ is a constant sequence.

In particular 
\[|x_0-y_0|_K=\ell_0=\ell_1=|f(x_0)-f(y_0)|_K\]
for any pair of points $(x_0,y_0)$ in $K$.
I.e., $f$ is distance preserving, in particular injective.

From $({*})$, we also get that $f(K)$ is everywhere dense.
Since $K$ is compact $f\:K\to K$ is surjective. Hence the result follows.

\parit{Comment.}
This is a basic lemma in the introduction to Gromov--Hausdorff distance;
see for example \cite[7.3.30]{bbi}.
The proof presented here is not quite standard;
it was given by Travis Morrison, when he was a students at MASS program at Penn State (Fall 2011).



%%%%%%%%%%%%%%%%%%%%%%%%%%%%%%%%%%%%%%%%%%%%%%%%%%
\parbf{\ref{compact}.} 
\textit{Embedding of a compact.}
Let $K$ be a compact metric space.
Denote by $B(K)$ the space of bounded functions on $K$
equipped with sup norm; 
that is, 
\[|f|=\sup_{x\in K}|f(x)|.\]

Note that the map $\phi\:K\to B(K)$, defied by $x\mapsto \dist_x$
is a distance preserving embedding.

Denote by $W$ the linear convex hull of the image $\phi(K)\subset B(K)$ with the metric induced from $B(K)$.
It remains to show that $W$ forms a compact length space.

\parit{Comment.}
The map $\phi$ is called \emph{Kuratowski embedding},
although it was essentially discovered by Fr\'echet in the same paper he introduced metric spaces.



%%%%%%%%%%%%%%%%%%%%%%%%%%%%%%%%%%%%%%%%%%%%%%%%%%
\parbf{\ref{2-sphere is far from a ball}.} 
\textit{Disc and 2-sphere.}
Assume contrary, let $(\mathbb{S}^2,g)$ is sufficiently close to $B^2$.

Choose a closed simple curve $\gamma$ in $\mathbb{S}^2$ which is close to the boundary of $B^2$.
Choose two points $p_1$ and $p_2$ in $\mathbb{S}^2$ 
on the opposite sides of $\gamma$ which are sufficiently close to the center of $B^2$.

On one had $p_1$ and $p_2$ have to be close in $\mathbb{S}^2$.
On the other hang, to get from $p_1$ to $p_2$ in $\mathbb{S}^2$,
one has to cross $\gamma$.
Hence the distance from $p_1$ to $p_2$ in $\mathbb{S}^2$ has to be about $2$,
a contradiction.

\parit{Comment.}
In fact if $X$ is a Gromov--Hausdorff limit of $(\mathbb{S}^2,g_n)$
then any point $x_0\in X$ either admits a neighborhood homeomorphic to $\RR^2$ or it is a cut point;
that is $X\backslash\{x_0\}$ is disconnected; see \cite[3.32]{gromov-MetStr}.

%%%%%%%%%%%%%%%%%%%%%%%%%%%%%%%%%%%%%%%%%%%%%%%%%%
\parbf{\ref{3-sphere is close to a ball}.} 
\textit{Ball and 3-sphere.}
Make fine burrows in the standard 3-ball which do not change its topology,
but at the same time a come sufficiently close to any point in the ball.

Consider the doubling of obtained ball in its boundary.
Clearly the obtained space is homeomorphic to $\mathbb{S}^3$.
Prove that the burrows can be made 
so that it is sufficiently close to the original ball 
in the Gromov--Hausdorff metric.

It remains to smooth the obtained space slightly 
to get a genuine Riemannian metric with needed property.

\parit{Comment.}
This construction is a stripped version of Ferry--Okun theorem in \cite{ferry-okun},
which states that Riemannian metrics on a smooth closed manifold $M$ with $\dim M\ge 3$ 
can approximate given compact length-metric space $X$ 
if and only if 
there is a continuous map $M\to X$
which is surjective on the fundamental groups. 
%??? Closely related constructions are discussed in Gromov's book ??? 

%%%%%%%%%%%%%%%%%%%%%%%%%%%%%%%%%%%%%%%%%%%%%%%%%%
\parbf{\ref{macrodimension}.} 
\textit{Macrodimension.}
Choose a point $p\in M$,
denote by $f$ the distance function from $p$.

Let us cover $M$ by the connected components of the preimages 
$f^{-1}((n-1,n+1))$.
Clearly any point in $M$ is covered by at most two such components.
It remains to show that each of these components has diameter less than $100$.

Assume contrary; let $x$ and $y$ be two points in such connected component 
and $|x-y|_M\ge 100$.
Connect $x$ to $y$ by a curve $\tau$ in the component.
Consider the closed curve $\sigma$ formed by two geodesics $[px]$, $[py]$ and $\tau$.

Prove that $\sigma$ can be divided into 4 arcs $\alpha$, $\beta$, $\gamma$ and $\delta$
in such a way that the minimal distance from $\alpha$ to $\gamma$ as well as the minimal distance from $\beta$ to $\delta$ is at least $10$.

Use the last statement to show that $\sigma$ 
can not be shrank 
by a disc it its $1$-neighborhood;
the later contradicts the assumption.

\parit{Comment.}
The problem was discussed it a talk by Nikita Zinoviev around 2004.

%%%%%%%%%%%%%%%%%%%%%%%%%%%%%%%%%%%%%%%%%%%%%%%%%%
\parbf{\wrenches\ref{anti-collaps}.} 
\textit{Anti-collapse.}
Fix a decreasing sequence $\eps_0,\eps_1,\dots$ of positive numbers converging to $0$ as $n\to \infty$.

Let $T$ be the infinite binary tree
and $T_n\subset T$ be the subtree up to level $n$.
Let us equip $T$ with the length-metric such that the edges coming from $(n-1)$-th level to $n$-th level have length $\eps_{n-1}-\eps_n$.

Denote by $\bar T$ the completion of $T$.
Note that $C=\bar T\backslash T$ is a Cantor set;
The set $C$ can be identified with the set of $\{0,1\}$-sequences 
with the distance between two sequences $\bm{x}=(x_0,x_1,\dots)$ and $\bm{y}=(y_0,y_1,\dots)$ defined as $\eps_n$, where $n$ is the least number such that $x_n\ne y_n$.

Choosing $\eps_n$ one can make $C$ to have arbitrary large Hausdorff dimension.

Now choose $\delta_n\ll\eps_n$ and prepare for each edge of $T$ a cilinder with hight ... and radius of the base $\delta_n$.
 

Note that there is a natural embedding $T_{n}\to T_{n+1}$ for all $n$.


Assume $S$ be a surface with a flat disc $D_0\subset S$ of radius $r_0$.
Let us cut from $D_0$ two discs $D_1$ and $D_1'$ 
of radii $r_1=r_0/10$ and glue instead a cylinder with high $\eps_0$ 
with discs on the top.
Now repeat the operation for each $D_1$ and $D_1'$ cutting 
from each two discs of radius $r_2=r_1/10$
and glue instead cylinder with high $\eps_0$ 
with discs on the top.
Continue the process, we get an increasing sequence of Riemannian metrics on $S$.
%???PIC

\parit{Comments.}
The problem appears in the paper \cite{BIS} by Burago, Ivanov and Shoenthal.

%%%%%%%%%%%%%%%%%%%%%%%%%%%%%%%%%%%%%%%%%%%%%%%%%%
\parbf{\ref{weird-metric}.} 
\textit{No short embedding.}
Consider a chain of disjoint circles $c_0,c_1,\dots,c_n$ in $\RR^3$;
that is, $c_i$ and $c_{i-1}$ are linked for each $i$. 

%\begin{wrapfigure}{r}{50mm}
\begin{center}
\begin{lpic}[t(-0mm),b(0mm),r(0mm),l(0mm)]{pics/chain(1)}
\lbl[t]{5,0;$c_0$}
\lbl[t]{11,0;$c_1$}
\lbl[t]{31,0;$\cdots$}
\lbl[t]{54,0;$c_n$}
\end{lpic}
\end{center}
%\end{wrapfigure}

Assume that $\RR^3$ is equipped with a length-metric $d$,
such that the total length of the circles is $\ell$
and $U$ is an open set containing all the circles $c_i$.
Note that for any short homeomorphism $f\:(U,d)\to\RR^3$ the distance from $f(c_0)$ to $f(c_n)$ is less than $\ell$.

Fix a line segment $[ab]$ in $\RR^3$.
Modify 
the length metric on $\RR^3$ in arbitrary small neighbohood of $[ab]$
so that there is a chain $(c_i)$ of circles as above,
which goes from $a$ to $b$ 
such that
(1) the total length, say $\ell$, 
of $(c_i)$ is arbitrary small,
but 
(2) the obtained metric $d$ 
is arbitrary close to the canonical, say
\[\bigl|d(x,y)-|x-y|\bigr|<\eps\]
for any two points $x,y\in\RR^3$
and fixed in advanced small $\eps>0$.
The construction of $d$ 
is done by shrinking the length of each circle
and expanding the length in the normal diections  
to the circles in their small neigbourhood.
The later is made in order to make impossible to use the circles $c_i$ as a shortcut;
that is, one spends more time to go from one circle to an other 
than saves on going along the circle.

Set $a_n=(0,\tfrac1n,0)$ and $b_n=(1,\tfrac1n,0)$.
Note that the line segmments $[a_nb_n]$ are disjoint and converging
to $[a_\infty b_\infty]$
where $a_\infty=(0,0,0)$ and $b_\infty=(1,0,0)$.

Apply the above construction in nonoverlaping convex neighborhoods of $[a_nb_n]$ 
and for a sequences 
$\eps_n$ and $\ell_n$ 
which converge to zero very fast.

The obtainded lenght metric $d$ is still close to the canonical,
but for any open set $U$ containing $[a_\infty b_\infty]$
the space $(U,d)$ does not admit 
a short homeomorphism to $\RR^3$.
Indeed, if such homeomorphism $h$
exists then 
from the above construction,
we get 
\begin{align*}
|h(a_\infty)-f(b_\infty)|
&\le 
|h(a_n)-f(b_n)|
+
\\
&\ \ \ \ \ +
|h(a_\infty)-f(a_n)|
+
|h(b_n)-f(b_\infty)|
\le
\\
&\le
\ell_n+\tfrac2n+100\cdot\eps_n.
\end{align*}
The right hand side converges to $0$ as $n\to\infty$.
Therefore 
\[h(a_\infty)=f(b_\infty),\] 
a contradiction.

It remains to performs similar construction countably many times so a bad segment as $[a_\infty b_\infty]$ above
appears in any open set of $\RR^3$.


\parit{Comments.}
The problem is discussed 
by Burago, Ivanov and Shoenthal 
in \cite{BIS}.

%%%%%%%%%%%%%%%%%%%%%%%%%%%%%%%%%%%%%%%%%%%%%%%%%%
\parbf{\ref{sub-Riemannian}.} 
\textit{Sub-Riemannian sphere.}
Prove that there is a nondecreasing sequence of Riemannian metric tensors
$g_0\le g_1\le ...$ such that the induced metrics converge to the given sub-Riemannian metrics.
The metric $g_0$ can be assumed to be a metric on round sphere.

Applying the construction as in Nash--Kuiper theorem,
one can produce a sequence of smooth embedings $h_n\:\mathbb{S}^m\to \RR^{m+1}$ with the induced metrics $g_n'$
such that $|g_n'-g_n|\to 0$.

Moreover, assume we assign a positive real number $\eps(h)$ for any smooth embedding $h\:\mathbb{S}^m\to\RR^{m+1}$.
Then we can assume that 
\[|h_{n+1}(x)-h_n(x)|<\eps(h_n)\] for any $x\in \mathbb{S}^m$ and $n$.

Show that for a right choice of function $\eps(h_n)$,
the sequence $h_n$ converges, say to $h_\infty$, 
and the metric induced by $h_\infty$ coincides with the given sub-Riemannian metric.

\parit{Comments.} 
The original papers of Nash \cite{nash} and Kuiper \cite{kuiper} are very readable.

The problem appeared 
on this list first it was rediscovered later by Le Donne in \cite{le-donne}.
Similar construction described in the our lecture notes \cite{petrunin-yashinsky} 
aimed for undergraduate students. 
Yet my paper \cite{petrunin-paths} is closely relevant.

%%%%%%%%%%%%%%%%%%%%%%%%%%%%%%%%%%%%%%%%%%%%%%%%%%
\parbf{\ref{two2one}.} 
\textit{Length preserving map.}
Assume there is a length-preserving map $f\:\RR^2\to \RR$.

Note that $f$ is Lipschitz.
Therefore by Rademacher's theorem, $f$ is differentiable almost everywhere.

Fix a unit vector $u$.
Prove that, for almost all $x$, the length of curve 
$\alpha\:t\mapsto x+t\cdot u$, $t\in[0,1]$ can be expressed as the integral
\[\int\limits_0^1 (d_{\alpha(t)}f)(u) \cdot dt.\]

It follows that $|d_xf(v)|=|v|$ for almost all $x,v\in\RR^2$;
in particular $d_xf$ is defined and has rank 2 at some point $x$, a contradiction.  

\parit{Comment.} The idea above can be also used to show the following.

\textit{Let $\mathbb M^2$ be a \emph{Minkowski plane} which is not isometric to the Euclidean plane.
Show that $\mathbb M^2$ does not admit a \hyperref[Length preserving map]{\emph{length preserving map}} to $\RR^3$.}



%%%%%%%%%%%%%%%%%%%%%%%%%%%%%%%%%%%%%%%%%%%%%%%%%%
\parbf{\ref{Hyperbolic space}.} 
\textit{Hyperbolic space.}
Note that $2$-dimensional hyperbolic space 
can be viewed as $(\RR^2,g)$, where 
\[g(x,y)=\left(\begin{matrix}
     1&0
     \\
     0&e^{x}
    \end{matrix}\right).\]
The same way $3$-dimensional hyperbolic space 
can be viewed as $(\RR^3,h)$, where 
where 
\[h(x,y,z)=\left(\begin{matrix}
     1&0&0
     \\
     0&e^{x}&0
     \\
     0&0&e^{x}
    \end{matrix}\right).\]

Prove that the map $\RR^3\to \RR^4$ defined as
$$(x,y,z)\mapsto (x,y,x,z)$$
is a quasi-isometry from $(\RR^3,h)$ to its image in $(\RR^2,g)\times (\RR^2,g)$.

\parit{Comments.}
In the proof we used that horosphere in the hyperbolic space is isometric to the Euclidean plane.
This observation appears already in the book of Lobachevsky, see \cite[34]{lobachevsky}.



%%%%%%%%%%%%%%%%%%%%%%%%%%%%%%%%%%%%%%%%%%%%%%%%%%
\parbf{\ref{Fixed line}.} 
\textit{Fixed line.}
Note that it is sufficient to show that if 
\[f(a)=a\ \ \text{and}\ \ f(b)=b\]
for some $a,b\in\RR^m$
then 
\[f(\tfrac{a+b}2)=\tfrac12\cdot(f(a)+f(b)).\]

(This statement is not trivial since in general
metric midpoint of $a$ and $b$ in $(\RR^m,d)$ 
are not defined uniquely.)

Without loss of generality, we can assume that $b+a=0$.

Consider the sequence of isometries $f_0$, $f_1,\dots$ defined recursively
\[f_{n+1}(x)= -f_n^{-1}(-f_n(x))\]
with $f_0=f$.
Note that $f_n(a)=a$ and $f_n(b)=b$ for any $n$ and 
$$|f_{n+1}(0)|=2\cdot|f_n(0)|.$$
The later condition implies that if $f(0)\ne 0$
then $|f_n(0)|\to\infty$ as $n\to\infty$.
On the other hand, since $f_n$ is isometry and $f(a)=a$,
we get $|f_n(0)|\le 2\cdot |a|$, a contradiction.

\parit{Comment.}
The solution above
is the main step in the V\"ais\"al\"a's proof of Mazur--Ulam theorem;
it states that any isometry $(\RR^m,d)\to (\RR^m,d)$ has to be affine.
See \cite{vaisala}) and \cite{mazur-ulam}.

%%%%%%%%%%%%%%%%%%%%%%%%%%%%%%%%%%%%%%%%%%%%%%%%%%

\parbf{\ref{Pogorelov's construction}.} 
\textit{Pogorelov's construction.}
Positivity and symmetry of $\rho$ is evident.

The triangle inequality follows since
\[[B(x,\tfrac \pi2)\backslash B(y,\tfrac\pi2)]
\cup 
[B(y,\tfrac\pi2)\backslash B(z,\tfrac\pi2)]
\supset 
B(x,\tfrac \pi2) \backslash B(z,\tfrac\pi2).
\leqno(*)\]

Note that we get equality in $(*)$ if and only if $y$ lies on the great circle arc from $x$ to $z$.
Therefore the second statement follows.

\parit{Comments.} This construction was given by Pogorelov in \cite{pogorelov}.
It is closely related to the construction given by Hilbert in \cite{hilbert}
which was the motivating example of his 4-th problem \cite{hilbert-problems}.


%%%%%%%%%%%%%%%%%%%%%%%%%%%%%%%%%%%%%%%%%%%%%%%%%%
\parbf{\ref{Straight geodesics}.} 
\textit{Straight geodesics.}
From uniqueness of straight segment between given points in $\RR^m$,
it follows that any straight line in $\RR^m$ forms a geodesic in $(\RR^m,\rho)$.

Set 
\[\|\bm{v}\|_{\bm{x}}=\rho(\bm{x},(\bm{x}+\bm{v})).\]
Note that 
\[ \|\lambda\cdot\bm{v}\|_{\bm{x}}
=
|\lambda|\cdot\|\bm{v}\|_{\bm{x}}\]
for any $\bm{x},\bm{v}\in\RR^m$ and $\lambda\in\RR$.

Prove that 
\[
\|\lambda\cdot\bm{v}\|_{\bm{x}}
-
\|\lambda\cdot\bm{v}\|_{\bm{x}'}
\le 
\Const\cdot |\bm{x}-\bm{x'}|\]
for any $\bm{x},\bm{x'},\bm{v}\in\RR^m$, 
$\lambda\in\RR$
and some fixed $\Const\in\RR$.

Passing to the limit as $\lambda\to\infty$, 
we get
$\|\bm{v}\|_{\bm{x}}$ does not depend on $\bm{x}$;
hence the result follows.



%%%%%%%%%%%%%%%%%%%%%%%%%%%%%%%%%%%%%%%%%%%%%%%%%%
\parbf{\ref{hom-near-QI}.} 
\textit{A homeomorphism near quasi-isometry.}
Let $M\ge 1$ and $A\ge 0$.
Define $(M,A)$-quasi-isometry
as a map $f\:X\to Y$ between metric spaces $X$ and $Y$ such that for any $x,y\in X$,
 we have
\[\tfrac1M\cdot |x-y|-A\le |f(x)-f(y)|\le M\cdot |x-y|+A\]
and any point in $Y$ lies on the distance at most $A$ from a point in the immage $f(X)$.

Note that $(M,0)$-quasi-isometry is a $[\tfrac1M,M]$-bi-Lipschitz map.
Moreover,
if $f_n\:\RR^m\to\RR^m$ is a  $(M,\frac1n)$-quasi-isometry 
for each $n$ then any partial limit of $f_n$ as $n\to\infty$
is a $[\tfrac1M,M]$-bi-Lipschitz map.

It follows that given $M\ge 1$ and $\eps>0$ there is $\delta>0$ such that 
for any $(M,\delta)$-quasi-isometry $f\:\RR^m\to\RR^m$ and any $p\in \RR^m$
there is an $[\tfrac1M,M]$-bi-Lipschitz map $h\:B(p,1)\to \RR^m$
such that
\[|f(x)-h(x)|<\eps\]
for any $x\in B(p,1)$.

Applying recaling, we can get the following equivalent formulation. 
Given $M\ge 1$, $A\ge 0$ and $\eps>0$
there is big enough $R>0$ such that for any $(M,A)$-quasi-isometry 
$f\:\RR^m\to\RR^m$ and any $p\in\RR^m$ there is a $[\tfrac1M,M]$-bi-Lipschitz map $h\:B(p,R)\to \RR^m$
such that 
\[|f(x)-h(x)|<\eps\cdot R\]
for any $x\in B(p,R)$.

Now cover $\RR^m$ by balls
$B(p_n,R)$, construct a $[\tfrac1M,M]$-bi-Lipschitz map $h_n\:B(p_n,R)\to \RR^m$ for each $n$.

The maps $h_n$ are $2\cdot \eps\cdot R$ close to each other on the overlaps of their domains of definition.
This makes possible to deform slightly each $h_n$ so that they agree on the overlaps.
This can be done by Siebenmann' Theorem, see \cite{siebenmann}.
If instead you apply Sulivan's theorem, you get a bi-Lipschitz homeomorphism $h\:\RR^m\to\RR^m$,
see \cite{sulivan} or \cite{tukia-vaisala}.


\parit{Comments.}
The problem was suggested by Dmiti Burago.





%%%%%%%%%%%%%%%%%%%%%%%%%%%%%%%%%%%%%%%%%%%%%%%%%%
\parbf{\ref{hausdorff-section}.} 
\textit{A family of sets with no section.}
Identify $\mathbb{S}^1$ with $[0,1]/(0\sim 1)$.
Consider one parameter family of Cantor sets $K_t$
formed by all possible sums $\sum_{n=1}^\infty a_n\cdot t^n$,
where $a_i$ is $0$ or $1$ and $t\in[0,\tfrac12]$.

Note that $K_{\frac12}=\mathbb{S}^1$.

Denote by $\rho_\alpha\:\mathbb{S}^1\to\mathbb{S}^1$ 
the rotation by angle $\alpha$.
Set $Z_t=\rho_{\frac1{1-2\cdot t}}(K_t)$ for $t\in[0,\tfrac12)$ and $Z_{\frac12}=\mathbb{S}^1$.

Prove that the family of sets $Z_t$
is a continuous in Hausdorff topology and it does not have a section.

\parit{Comments.} The problem is suggested by Stephan Stadler.

It is instructive to check that any Hausdorff continuous family of closed sets in $\RR$ admits a continuous section.



%%%%%%%%%%%%%%%%%%%%%%%%%%%%%%%%%%%%%%%%%%%%%%%%%%
\parbf{\ref{pr:Sasaki metric}.} 
\textit{Sasaki metric.}
Show that there is a constant $\ell$
such that for any two unit tangent vectors $v\in\mathrm{T}_p\mathbb{S}^2$ 
and $w\in T_q\mathbb{S}^2$
there is a path 
$\gamma\:[0,1]\to\mathbb{S}^2$ from $p$ to $q$
such that 
\[\length \gamma\le \ell\] 
and
$w$ is the parallel transformation of $v$ along $\gamma$.

Note that once it is proved, 
it follows that diameter of the set of all vectors of fixed length in $\mathrm{T} \mathbb{S}^2$ has diameter at most $\ell$;
in particular the map $\mathrm{T}\mathbb{S}^2\to[0,\infty)$ defined as $v\mapsto |v|$ 
preserves the distance with the maximal error $\ell$.
Hence the result follows.

\section*{Actions and coverings}




%%%%%%%%%%%%%%%%%%%%%%%%%%%%%%%%%%%%%%%%%%%%%%%%%%
\parbf{\ref{Bounded orbit}.} 
\textit{Bounded orbit.}
Note that we can assume that the orbit $x_n=\iota^n(x)$ is dense in $X$;
otherwise pass to the closure of this orbit.
In particular, we can choose a finite number of positive integers values $n_1$, $n_2,\dots,n_k$
such that the points $x_{n_1}$, $x_{n_2},\dots,x_{n_k}$ form a $\tfrac1{10}$-net in $B(x_0,10)$.

Prove that that if $x_m\in B(x_0,1)$ then $x_{m+n_i}\in B(x_0,1)$ for some $i\in\{1,\dots,k\}$.

Set $N=\max_i\{n_i\}$.
It follows 
that among any $N$ elements in a row $x_{i+1},\dots x_{i+N}$
there is at least one in $B(x_0,1)$.
In particular, $N$ isometric copies of $B(x_0,1)$ cover whole $X$.
Hence the result follows.

\parit{Comments.}
The problem appears in the Ca{\l}ka's paper \cite{calka}.

%%%%%%%%%%%%%%%%%%%%%%%%%%%%%%%%%%%%%%%%%%%%%%%%%%

\parbf{\ref{figure-eight-1}.} 
\textit{Covers of figure eight.}
First show that any compact metric space can be presented as a limit of a sequence of finite metric graphs $\Gamma_n$.
Moreover, show that one can assume  each vertex of $\Gamma_n$ has degree 3 
and the length of each edge in $\Gamma_n$ is multiple of $\tfrac 1n$.

It remains to approximate $\Gamma_n$ by finite coverings of $(\Phi,d/n)$.
Guess this part  
from the following picture; it shows the needed approximation of the doted graph.

\begin{center}
\begin{lpic}[t(-0mm),b(0mm),r(0mm),l(0mm)]{pics/fig8(1)}
%\lbl[b]{24,41;x}
\end{lpic}
\end{center}

\parit{Comments.} 
The problem appears in thesis of Sahovic, see \cite{sahovic}.

The same idea works if instead of figure eight, we have any compact length-metric space $X$ such that $\pi_1X$ admits an epimorphism onto a free group with two generators.
In particular, since in any dimension starting from 2, there are compact hyperbolic manifolds with large fundamental group, any compact metric space can be approximated by space forms.

A similar idea was used later to show that any group can appear as a fundamental group of underlying space of 3-dimensional hyperbolic orbifold, see \cite{panov-petrunin-telescopic}.





%%%%%%%%%%%%%%%%%%%%%%%%%%%%%%%%%%%%%%%%%%%%%%%%%%
\parbf{\ref{m-fold-cover}.} 
\textit{Diameter of \textit{m}-fold cover.}
Fix points $\tilde p,\tilde q\in\tilde M$.
Let  
$\tilde\gamma\:[0,1]\to \tilde M$ be a minimizing geodesic from $\tilde p$ to $\tilde q$. 

We need to show that 
$$\length \tilde\gamma\le m\cdot diam(M).$$ 
Suppose the contrary.

Denote by $p,q$ and $\gamma$ the projections of $\tilde p,\tilde q$ and $\tilde \gamma$ in $M$. 
Represent $\gamma$
as joint of $m$ paths of equal length,
\[\gamma=\gamma_1{*}\dots{*}\gamma_m,\] 
so
\[\length(\gamma_i)=\length(\gamma)/m>\diam(M).\] 

Let $\sigma_i$ be a minimizing geodesic in $M$ connecting the endpoints of $\gamma_i$. 
Note that 
$$\length\sigma_i\le \diam M< \length\gamma_i.$$ 

Consider $m+1$ paths $\alpha_0,\dots,\alpha_m$ defined as 
\[\alpha_i=\sigma_1{*}\dots{*}\sigma_i{*}\gamma_{i+1}{*}\dots{*}\gamma_m.\]

Consider their liftings $\tilde\alpha_0,\dots,\tilde\alpha_m$ 
with $\tilde q$ as the endpoint.
Note that two curves, say $\alpha_i$ and $\alpha_j$ for $i<j$, 
have the same starting point in $\tilde M$.

Consider the path
\[\beta=\gamma_1{*}\dots{*}\gamma_i{*}\sigma_{i+1}{*}\dots{*}\sigma_j{*}\gamma_{j+1}{*}\dots{*}\gamma_m.\]
Prove that there is lift $\tilde\beta$ of $\beta$ 
which connects $\tilde p$ to $\tilde q$ in $\tilde M$.
Clearly $\length \beta<\length \gamma$, a contradiction.

\parit{Comments.}
The question was asked by Alex Nabutovsky;
it was answered by Sergei Ivanov, see \cite{ivanov}.



%%%%%%%%%%%%%%%%%%%%%%%%%%%%%%%%%%%%%%%%%%%%%%%%%%
\parbf{\ref{Symmetric square}.} 
\textit{Symmetric square.}
Let $\Gamma=\pi_1 X$ and $\Delta=\pi_1((X\times X)/\ZZ_2)$.
Consider the homomorphism $\phi\:\Gamma\times \Gamma\to \Delta$
induced by the projection $X\times X\to (X\times X)/\ZZ_2$.

Prove that the restrictions $\phi|_{\Gamma\times \{1\}}$ and $\phi|_{\{1\}\times\Gamma}$
are onto.

It remains to note that 
$$\phi(\alpha,1)\phi(1,\beta)=\phi(1,\beta)\phi(\alpha,1)$$
for any $\alpha$ and $\beta$ in $\Gamma$.

\parit{Comments.} The problem was suggested by Rostislav Matveyev.



%%%%%%%%%%%%%%%%%%%%%%%%%%%%%%%%%%%%%%%%%%%%%%%%%%
\parbf{\ref{Sierpinski triangle}.} 
\textit{Sierpinski triangle.}
Denote the Sierpinski triangle by $\triangle$.

Let us show that any homeomorphism of $\triangle$ is also its isometry.
Therefore the group homeomorphisms is the symmetric group $S_3$. 

Let $\{x,y,z\}$ be a 3-point set in $\triangle$ such that $\triangle \backslash\{x,y,z\}$ has 3 connected components.
Prove that there is unique choice for the set $\{x,y,z\}$ and 
it is formed by the midpoints of its big sides.

It follows that any homeomorphism of $\triangle$ permutes the set $\{x,y,z\}$.

A similar argument shows that this permutation  uniquely describes the homeomorphism.

\parit{Comments.}
The problem was suggested by Bruce Kliener.



%%%%%%%%%%%%%%%%%%%%%%%%%%%%%%%%%%%%%%%%%%%%%%%%%%
\parbf{\ref{Boys and girls in a Lie group}.} 
\textit{Latices in a Lie group.}
Denote by $V_\ell$ and $W_m$
the Voronoi domain of for each $\ell\in L$ and $m\in M$ correspondingly;
that is,
\[V_\ell=\set{g\in G}{|g-\ell|_G\le|g-\ell'|\ \text{for any}\ \ell'\in L}\]
\[W_m=\set{g\in G}{|g-m|_G\le|g-m'|\ \text{for any}\ m'\in M}\]

Note that for any $\ell\in L$ and $m \in M$ we have
\[\begin{aligned}
\vol V_\ell&=\vol(L\backslash (G,h))=
\\
&=\vol(M\backslash (G,h))=
\\
&=\vol W_m.
\end{aligned}
\leqno({*})
\]

Consider the bipartite graph $\Gamma$ with vertices formed by the elements of $L$ and $M$
such that $\ell\in L$ is adjacent  to $m \in M$ if and only if $V_\ell\cap W_m\ne\emptyset$.

By $({*})$ the graph $\Gamma$ satisfies the condition in the Hall's marriage theorem.
Therefore there is a bijection $f\: L\to M$ such that 
\[V_\ell\cap W_{f(\ell)}\ne\emptyset\] for any $\ell\in L$. 

It remains to notice that $f$ is bi-Lipschitz.

\parit{Comments.} The problem is discussed in the paper \cite{burago-kleiner} by Burago and Kleiner. 
For a finitely generated group $G$  
it is not known if $G$ and $G\times \ZZ_2$ can fail to be bi-Lipscitz.
(The groups are assumed to be equipped with word metric.)
 



%%%%%%%%%%%%%%%%%%%%%%%%%%%%%%%%%%%%%%%%%%%%%%%%%%
\parbf{\ref{Piecewise Euclidean quotient}.} 
\textit{Piecewise Euclidean quotient.}
Note that the group $\Gamma$ serves as holonomy group of the quotient space $P=\RR^n/\Gamma$ with the induced polyhedral metric.
More precisely, one can identify $\RR^n$ with the tangent space of a regular point $x_0$ of $P$ in such a way that
for any $\gamma\in\Gamma$ there is a loop $\ell$ in $P$ which pass only through regular points and has the holonomy $\gamma$.

Fix $\gamma\in\Gamma$. 
Let $\ell$ be the corresponding loop.
Since $P$ is simply connected, we can shrink $\ell$ by a disc.
By general position argument we can assume that the disc 
only pass through simplices of codimension $0$, $1$ and $2$
and intersect the simplices of codimension $2$ transversely.

In other words, $\ell$ can be presented as a product of 
loops such that each loop goes around a single simplex of codimension $2$ and comes back.
The holonomy for each of these loops is a rotation around a hyperplane.
Hence the result follows.

\parit{Comments.}
The converse to the problem also holds;
it was proved by Lange in \cite{lange},
his proof based ealier results of Mikhailova, see \cite{mikhailova}.

Note that the cone over spherical suspension over Poincar\'e sphere is homeomorphic to $\RR^5$ and it is quotient of $\RR^5$ by a finite subgroup of $\SO(5)$. 
I.e., if in this problem one exchanges ``PL-homeomorphism'' to ``homeomorphism'' then the answer is different; 
a complete classification of such actions was also obtained in \cite{lange}.

%%%%%%%%%%%%%%%%%%%%%%%%%%%%%%%%%%%%%%%%%%%%%%%%%%
\parbf{\ref{Subgroups of free group}.} 
\textit{Subgroups of free group.}
Let $G$ be a finitely generated subgroup of free group with $m$ generators, further denoted by $F_m$.

Let $W$ be the wedge sum of $n$ circles, 
so  $\pi_1(W,p)=F_m$.
Equip $W$ with length metric so that 
that each circle has unit length.

Pass to the metric cover $\tilde W$ of $W$ 
such that  $\pi_1(\tilde W,\tilde p)=G$ 
for a lift $\tilde p$ of $p$.

Fix sufficiently large integer $n$ and consider doubling of the closed ball $\bar B(\tilde p,n+\frac12)$ in its boundary.
Let us denote the obtained doubling by $Z_n$ and set $G_n=\pi(Z_n,\tilde p)$.

Prove that $Z_n$ is a metric covering of $W$;
it makes possible to consider $G_n$ as a subgroup of $F_m$.
By construction, $Z_n$ is compact;
therefore $G_n$ has finite order in $F_m$.


It remains to show that that 
\[G=\bigcap_{n>k} G_n,\]
where $k$ is the maximal length of word in the generating set of $G$.

%???+PIC
 
\parit{Comments.} 
Originally the problem was solved by Hall in \cite{hall}.
The proof presented here is close to the solution of Stalings in \cite{stallings};
see also \cite{wilton}.

The same idea can be used to prove the following statements:
\begin{itemize}
\item Subgroups of free groups are free.
\item Two elements of the free groups $u$ and $v$ commute 
if and only if they are powers of
some third element.
\end{itemize}



%%%%%%%%%%%%%%%%%%%%%%%%%%%%%%%%%%%%%%%%%%%%%%%%%%
\parbf{\ref{Lengths of generators of the fundamental group}.}
\textit{Lengths of generators of the fundamental group.}
Choose a length-minimizing geodesic loop $\gamma$ 
which represents a given element $a\in\pi_1M$.

Fix $\eps>0$.
Represent $\gamma$ 
as a joint 
\[\gamma=\gamma_1{*}\dots{*}\gamma_n\]
of paths with $\length\gamma_i<\eps$ for each $i$.
 
Denote by $p=p_0,p_1,\dots, p_n=p$ the endpoints of these arcs.
Connect $p$ to $p_i$ by a length minimizing geodesic $\sigma_i$.
Note that $\gamma$ is homotopic to a product of loops
\[\alpha_i=\sigma_{i-1}{*}\gamma_i{*}\sigma_{i-1}\]
and $\length \alpha_i<2\cdot\diam M+\eps$ for each $i$.

It remains to show that for sufficiently small $\eps>0$
any loop with length less than $2\cdot\diam M+\eps$ 
is homotopic to a loop with length at most $2\cdot\diam M$.

\parit{Comments.} The statement was observed by Gromov, see \cite[Proposition 3.22]{gromov-MetStr}.

%%%%%%%%%%%%%%%%%%%%%%%%%%%%%%%%%%%%%%%%%%%%%%%%%%
\parbf{\ref{Short basis}.}
\textit{Short basis.}
Consider universal Riemannian cover $\tilde M$ of $M$.
Note that $\tilde M$ is nonnegatively curved and
$\pi_1M$ acts by isometries on $\tilde M$.

Fix $p\in \tilde M$.
Given  $a\in \pi_1M$,
set 
\[|a|=|p- a\cdot p|_{\tilde M}.\]
Construct a sequence of elements $a_1,a_2,\dots\in \pi_1M$ the following way:
\begin{enumerate}[(i)]
\item Choose $a_1\in\pi_1M$ so that $|a_1|$ takes the minimal value.
\item Choose $a_2\in\pi_1M\backslash\langle a_1 \rangle$ so that $|a_2|$ takes the minimal value.
\item Choose $a_3\in\pi_1M\backslash\langle a_1,a_2 \rangle$ so that $|a_2|$ takes the minimal value.
\item and so on.
\end{enumerate}

Note that the sequence terminates at $n$-th step if the 
$(a_1,a_2,\dots,a_n)$ form a generating system.
By construction, we have
\begin{align*}
|a_j\cdot a_i^{-1}|&\ge |a_j|\ge |a_i|
\intertext{for any $j>i$. 
Set $p_i=a_i\cdot p$.
Note that}
|p_j-p_i|_{\tilde M}
&=|a_j\cdot a_i^{-1}|\ge
\\
&\ge |a_j|=
\\
&=|p_j-p|_{\tilde M}\ge
\\
&\ge|a_i|=
\\
&=|p_i-p|_{\tilde M}.
\intertext{By Toponogov comparison theorem we get}
\tilde\measuredangle \hinge p{p_i}{p_j}&\ge \tfrac\pi3
\end{align*}
for any $i\ne j$.
Hence the result follows.

\parit{Comments.} This construction introduced by Gromov 
in the paper on almost flat manifolds, 
see \cite{gromov-almost-flat}.

\section*{Topology}

%%%%%%%%%%%%%%%%%%%%%%%%%%%%%%%%%%%%%%%%%%%%%%%%%%
\parbf{\wrenches\ref{Milnor's disks}.} 
\textit{Milnor's disks.}
The immersed circle below is called Bennequin’s curve.
It is a good exercise to count the essentially different immersed discs 
bounded by the given immersed circle. 

%\begin{wrapfigure}{r}{52mm}
\begin{center}
\begin{lpic}[t(-0mm),b(0mm),r(0mm),l(0mm)]{pics/Bennequin-curve(.5)}
%\lbl[b]{24,41;x}
\end{lpic}
\end{center}
%\end{wrapfigure}

%???PIC

\cite{gromov-PDR}


%%%%%%%%%%%%%%%%%%%%%%%%%%%%%%%%%%%%%%%%%%%%%%%%%%
\parbf{\ref{Positive Dehn twist}.} 
\textit{Positive Dehn twist.}
Consider the universal covering 
$\tilde\Sigma\to\Sigma$.
The surface $\tilde \Sigma$ comes with the orientation induced from $\Sigma$.

Note that we may assume that $\tilde\Sigma$ has infinite number of boundary components.

Fix a point $x_0$ on the boundary of $\tilde \Sigma$.
Given other points $y$ and $z$ we will write
$y\prec z$ if $z$ lies on the left side from one (and therefore any) simple curve from $x_0$ to $y$ in $\tilde\Sigma$.
Note that  $\prec$ defines a linear order on $\partial\tilde\Sigma\backslash\{x_0\}$.
We will write $y\preceq z$ 
if $y\prec z$ or $y=z$.

Note that any homeomorphism $h\:\Sigma\to\Sigma$ which is identity on the boundary
lifts to unique homeomorphism $\tilde h\:\tilde \Sigma\to\tilde\Sigma$ 
is such a way that $\tilde h(x_0)=x_0$.

Assume $h$ is positive Dehn twist.
Show that 
$y\preceq \tilde h(y)$ for any  $y\in\partial\tilde\Sigma\backslash\{x_0\}$
and there is a point $y_0\in\partial\tilde\Sigma\backslash\{x_0\}$
such that $y_0\prec \tilde h(y_0)$.

Finally note that the later property is a homotopy invariant 
and it survives under compositions of maps.
Hence the statement follows.

\parit{Comments.} The problem was suggested by Rostislav Matveyev.

%%%%%%%%%%%%%%%%%%%%%%%%%%%%%%%%%%%%%%%%%%%%%%%%%%
\parbf{\ref{Function with no critical points}.} 
\textit{Function with no critical points.}
Construct an immersion $\psi\:B^n\z\to\RR^n$ such that 
\[\ell\circ\phi\ne\ell\circ\psi\]
for any embedding  $\phi\:B^n\to\RR^n$.

It remains to note that the composition $f=\ell\circ\psi$ has no critical points.

%%%%%%%%%%%%%%%%%%%%%%%%%%%%%%%%%%%%%%%%%%%%%%%%%%
\parbf{\ref{Conic neighborhood}.} 
\textit{Conic neighborhood.}
Let $V$ and $W$ be two conic neighborhoods of $p$.
Without loss of generality, we may assume that $V\subset W$.

We will need to construct a sequence of embeddings $f_n\:V\to W$
such that 
\begin{enumerate}[(i)]
\item 
For any compact set $K\subset V$ 
there is a postive ineteger $n=n_K$ such that 
$f_n(k)=f_m(k)$ for any $k\in K$ and $m\ge n$.
\item For any point $w\in W$ there is a point $v\in V$ such that $f_n(v)=w$ for all large $n$.
\end{enumerate}

Note that once such sequence is constructed, $f\:V\to W$ defined as $f(v)=f_n(v)$ for all large values of $n$ gives the needed homeomorphism.

The sequence $f_n$ can be constructed recursevely, setting
\[f_{n+1}=\Psi_n\circ f_n\circ \Phi_n,\]
where $\Phi_n\:V\to V$ 
and $\Psi_n\:W\to W$ 
are homeomorphisms
of the form 
\[\Phi_n(x)=\phi_n(x)\cdot x\quad \Phi_n(x)=\psi_n(x)\cdot x,\]
where $\phi_n\:V\to \RR_+$, $\psi_n\:W\to \RR_+$ are suitable continuous functions 
and 
``$\cdot$'' denotes the ``multiplication'' in the cone structures of $V$ and $W$ correspondingly.

\parit{Comments.}
The problem is discussed by Kwun in \cite{kwun}.

Note that for two cones $\mathop{\rm Cone}(\Sigma_1)$ and $\mathop{\rm Cone}(\Sigma_2)$ might be homeomorphic while $\Sigma_1$ and $\Sigma_2$ are not.



%%%%%%%%%%%%%%%%%%%%%%%%%%%%%%%%%%%%%%%%%%%%%%%%%%
\parbf{\wrenches\ref{No knots}.} 
\textit{No $C^0$-knots.}

%???+PIC

\parit{Comment.}
This problem was suggested by Greg Kuperberg, see \cite{One-step problems in geometry}.


%%%%%%%%%%%%%%%%%%%%%%%%%%%%%%%%%%%%%%%%%%%%%%%%%%
\parbf{\ref{Simple stabilization}.} 
\textit{Simple stabilization.}
The example can be guessed from the diagram.

\begin{center}
\begin{lpic}[t(-0mm),b(0mm),r(0mm),l(0mm)]{pics/Simple-stabilization(1)}
\end{lpic}
\end{center}

\parit{Caomments.} 
It was one of the special problems in my analysis class taught by Galuzina.
Likely this is not the original source.


%%%%%%%%%%%%%%%%%%%%%%%%%%%%%%%%%%%%%%%%%%%%%%%%%%
\parbf{\ref{Isotropy}.}
\textit{Isotropy.}
Fix a hoemeomrphism $\phi\:K_1\to K_2$.

By Tietze extension theorem,
the hoemeomrphisms $\phi\:K_1\to K_2$ and $\phi^{-1}\:K_2\to K_1$ can be extended to a continuous maps,
say $f\:\RR^n\to \RR^n$ and $g\:\RR^n\to \RR^n$ correspondingly.

Consider the homeomorphisms
$h_1, h_2, h_3\:\RR^n\times\RR^n\to\RR^n\times\RR^n$ defined the following way
\begin{align*}
h_1(x,y)&=(x,y+f(x)),
\\
h_2(x,y)&=(x-g(y),y),
\\ 
h_3(x,y)&=(y,-x).
\end{align*}

It remains to prove that each homeomorphism $h_i$ is isotopic to the indentity map and
\[h=h_3\circ h_2\circ h_1.\] 

\parit{Comments.}
The problem had been discussed by Klee in
\cite{klee}.


%%%%%%%%%%%%%%%%%%%%%%%%%%%%%%%%%%%%%%%%%%%%%%%%%%
\parbf{\wrenches\ref{Knaster's circle}.} 
\textit{Knaster's circle.}
A map $f\:\mathbb S^1\to\mathbb S^1$ will be called $\eps$-crooked 
if for any arc $\II\subset \mathbb S^1$ with end points $a$ and $b$ there are points $x,y\in \II$ such that the points $a,y,x,b$ appear on $\II$ in the same order and
\[|f(x)-f(a)|_{\mathbb S^1},|f(y)-f(b)|_{\mathbb S^1}<\eps.\]

Show that for any $\eps>0$ there is an $\eps$-crooked map $f\:\mathbb S^1\to\mathbb S^1$ of degree $1$.

Take a sequence of $\eps_n$-crooked maps for a sequence $\eps_n$ which converge fast ot $0$
and use this map to construct a nested sequence of embedding of annuli in the plane.
Each annulus bounds a disc and the intersection 
of all these annuli bound a disc which is the union of all thse discs.

It remains to show that the boundary of the obtained disc does not contain a simple curve.
%???PIC
\parit{Comments.}
\cite{wayne}.

%%%%%%%%%%%%%%%%%%%%%%%%%%%%%%%%%%%%%%%%%%%%%%%%%%
\parbf{\ref{Boundary in R}.} 
\textit{Boundary in $\RR$.}
Prove that the Cantor's set forms a boundary of three disjoint open set in $\RR$.

\parit{Comments.} In the plane 
one can assume in addition that each set is connected.
This examples are called \emph{lakes of Wada};
these are three disjoint open discs in the plane which share the same boundary.   
This example described by Yoneyama in \cite{yoneyama}.
It is easy to see that the boundary of each lake contains no simple nontrivial curves, therefore it also solves problem  \ref{Knaster's circle}.

\begin{center}
\begin{lpic}[t(-0mm),b(0mm),r(0mm),l(0mm)]{pics/Antoine's_Necklace_Iteration_1(.1)}
\end{lpic}
\begin{lpic}[t(-0mm),b(0mm),r(0mm),l(0mm)]{pics/Antoine's_Necklace_Iteration_2(.1)}
\end{lpic}
%???https://en.wikipedia.org/wiki/User:Blacklemon67
\end{center}

An other related problem is to construct a set in $\RR^3$ homeomorphic to the Cantor's set with non simply connected complement.
This example was constructed by Louis Antoine in \cite{antoine}.
The construction can be guessed from the first and second itaration on the shown on the pictures%
\footnote{These are black-and-white versions of the pictures 
made by \href{http://en.wikipedia.org/wiki/User:Blacklemon67}{Blacklemon67} 
for the article on Antoine's Necklace in Wikipedia.} above.

%%%%%%%%%%%%%%%%%%%%%%%%%%%%%%%%%%%%%%%%%%%%%%%%%%
\parbf{\wrenches\ref{Deformation of homeomorphism}.} 
\textit{Deformation of homeomorphism.}
Fix a smooth increasing concave function $\phi\:\RR\to\RR$ such that
$\phi(r)=r$ for any $r\le 1$ and $\sup_r\phi(r)=2$.

Consider $\RR^n$ with polar coordinates $(u,r)$, where $u\in\mathbb{S}^{n-1}$ and $r\ge 0$.
Let $\Phi\:\RR^n\to\RR^n$
is defined by $\Phi(u,r)=(u,\phi(r))$.

Set $h(x)=\Phi\circ f \circ \Phi^{-1}(x)$ is $|x|<2$
and $h(x)=x$ otherwise.

Prove that $h\:\RR^n\to\RR^n$ is a solution.

\parit{Comments.}
???

%%%%%%%%%%%%%%%%%%%%%%%%%%%%%%%%%%%%%%%%%%%%%%%%%%
\parbf{\wrenches\ref{Finite topological space}.} 
\textit{Finite topological space.}
Let $F$ be a finite topological space.
Given two points $p,q\in F$ we will write $p\preccurlyeq q$ if $p$ lies in any closed set containing $q$.

Prepare a cell for each point in $F$


Consider a finite CW complex $W$.

Denote by $S$ the set of all cells of $W$
and equip $S$ with the topology  such that ???


\section*{Piecewise linear geometry}


%%%%%%%%%%%%%%%%%%%%%%%%%%%%%%%%%%%%%%%%%%%%%%%%%%
\parbf{\ref{4-poly}.} 
\textit{Triangulation of 3-sphere.}
Choose 100 distinct points $x_1,x_2,\z\dots,x_{100}$
on the curve 
\[\gamma\:t\mapsto (t,t^2,t^3,t^4)\] 
in $\RR^4$.
Let $P$ be the convex hull of $\{x_1,x_2,\z\dots,x_{100}\}$.

Prove that for any two points $x_i$ and $x_j$ there is a hyperplane $H$ in $\RR^4$ which pass through $x_i$ and $x_i$ and leaves $\gamma$ on one side.
The later statement implies that any two vertices $x_i$ and $x_j$
of $P$ are connected by an edge.

The statement follows
since the surface of $P$ is homeomorphic to $\mathbb{S}^2$.

\parit{Comments.} The polyhedron $P$ above is an example of so called \emph{cyclic polytopes}.

%%%%%%%%%%%%%%%%%%%%%%%%%%%%%%%%%%%%%%%%%%%%%%%%%%
\parbf{\ref{Spherical arm lemma}.} 
\textit{Spherical arm lemma.}
Let us cut the polygon $A$ from the sphere and glue instead the polygon $B$.
Denote by $\Sigma$ the obtained spherical polyhedral space.
Note that 
\begin{itemize}
\item $\Sigma$ is homeomorphic $\mathbb S^2$.
\item $\Sigma$ has curvature $\ge 1$ in the sense of Alexandrov; that is, the total angle around each singular point is less than $2\cdot \pi$.
\item All the singular points of $\Sigma$ 
lie outside of an isometric copy of a hemisphere $\mathbb{S}^2_+\subset \Sigma$
\end{itemize}

It is sufficient to show that $\Sigma$ is isometric to the standard sphere.
Assume contrary.
If $n$ denotes be the number of singular points in $\Sigma$, 
it means that $n>0$.

We will arrive to a contradiction applying induction on $n$.
The base case $n=1$ is trivial; 
that is, $\Sigma$ can not have single singular point.

Now assume $\Sigma$ has $n>1$ singular points.
Choose two singular points $p, q$,
cut $\Sigma$ along a geodesic $[pq]$
and patch the hole so that the obtained new polyhedron $\Sigma'$ has curvature $\ge 1$.
The patch is obtained by doubling a
spherical triangle in two sides.
For the right choice of the triangle,
the points $p$ and $q$ become regular in $\Sigma'$
and exactly one new singular point appears in the patch.

This way, constructed a  spherical polyhedral space $\Sigma'$
with $n-1$ singular points which satisfy the same conditions as $\Sigma$ 

By induction hypothesis $\Sigma'$ does not exist. Hence the result follow.

\parit{Alternative end of proof.} 
By Alexandrov embedding theorem, $\Sigma$ is isometric to the surface of convex polyhedron $P$ in the unit 3-dimensional sphere $\mathbb S^3$. 
The center of hemisphere has to lie in a facet, say $F$ of $P$.
It remains to note that $F$ contains the equator and therefore $P$ has to be hemisphere in $\mathbb S^3$ or intersection of two hemispheres.
In both cases its surface is isometric to $\mathbb S^2$.

\parit{Comments.}
The problem appear in Zalgaller's paper \cite{zalgaller-shperical-polygon}.

The patch construction above was introduced by Alexandrov
in his proof of convex embeddabilty of polyhedrons;
the earliest reference we have found is
\cite[VI, \S7]{alexandrov1948}.

An alternative proof can be build of Alexandrov's embedding theorem,
see \cite{panov-petrunin}.



%%%%%%%%%%%%%%%%%%%%%%%%%%%%%%%%%%%%%%%%%%%%%%%%%%
\parbf{\ref{Piecewise linear isometry I}.} 
\textit{Piecewise linear isometry {\rm I}.}
Given a triangulation of $P$
consider the Voronoi domain for each vertex.
Prove that the triangulation can be subdivided if necessary
so that Voronoi domain of each vertex is isometric to a convex subset in the cone with vertex corresponding to the tip.

Note that the boundaries of all the Voronoi domains form a graph with straight edges.
One can triangulate $P$ so that each triangle has such edge as the base 
and the opposite vertex is the center of an adjusted Voronoi domain; such a vertex will be called \emph{main} vertex of the triangle.

Fix a triangle $\triangle vab$ in the constructed triangulation; 
let $v$ be its main vertex.
Given a point 
$x\in  \triangle$, set 
\[\rho(x)=|x-v|\ \ \text{and}\ \  \theta(x)=\min \{\measuredangle \hinge vax,\measuredangle\hinge vbx\}.\]
Map $x$ to the plane the point with polar coordinates $(\rho(x),\theta(x))$.

It is easy to see that the constructed map $\triangle\to\RR^2$ is piecewise distance preserving.
It remains to check that the constructed maps on all triangles agree on common sides.


\parit{Comments.}
This construction was given by Zalgaller in \cite{zalgaller-polyhedra}, see also \cite{petrunin-yashinsky}.
It admits a straightforward generalization to the higher dimensions, see \cite{krat}.



%%%%%%%%%%%%%%%%%%%%%%%%%%%%%%%%%%%%%%%%%%%%%%%%%%
\parbf{\ref{iso-kirzhbraun}.} 
\textit{Piecewise linear isometry {\rm II}.}
Let $a_1,a_2,\dots,a_n$
and $b_1$, $b_2,\z\dots,b_n$
be two collections of points in $\RR^2$
such that $|a_i-a_j|\ge |b_i-b_j|$ for all pairs $i$, $j$.
We need to construct a piecewise liner length-preserving map $f\:\RR^2\to\RR^2$
such that $f(a_i)=b_i$ for each $i$.

Assume that the problem is already solved if $n<m$;
let us do the case $n=m$.
By assumption, 
there is a piecewise liner length-preserving map $f\:\RR^2\to\RR^2$
such that $f(a_i)=b_i$ for each $i>1$.
Consider the set 
\[\Omega=\set{x\in\RR^2}{|f(x)-b_1|>|x-a_1|}.\]
If $\Omega=\emptyset$ then $f(a_1)=b_1$; 
that is, the problem is solved.

Prove that $\Omega$ is the interior of a polygon
which is star-shaped with respect to $a_1$.
Redefine the map $f$ inside $\Omega$ so that it remains piecewise liner length-preserving and $f(a_1)=b_1$.

\parit{Comments.}
The same proof works in all dimensions;
it was given by Brehm in \cite{brehm}.
The same proof was rediscovered by Akopyan and Tarasov in \cite{akopyan-tarasov}.
See also our lectures \cite{petrunin-yashinsky} aimed for undergraduate students.

The problem is closely related to Kirszbraun's theorem \cite{kirszbraun},
which was reproved by Valentine in \cite{valentine};
the proof of Brehm is very close to the one given by Valentine.






%%%%%%%%%%%%%%%%%%%%%%%%%%%%%%%%%%%%%%%%%%%%%%%%%%
\parbf{\ref{Minimal polyhedron}.} 
\textit{Minimal polyhedron.}
Arguing by contradiction, assume $T$ is a minimal polyhedral surface which is not saddle.

Prove that 
one can move one of the vertices of $T$ in such a way that the lengths of all edges starting at this vertex decrease.

Prove that if, 
by this deformation, 
the area does not decease 
then there are two adjusted triangles in the triangulation, 
say $\triangle pxy$ and $\triangle qxy$
such that 
\[\measuredangle \hinge pxy+\measuredangle \hinge qxy> \pi.\]

Finally show that in this case exchanging triangles $\triangle pxy$ and $\triangle qxy$
to the triangles $\triangle pxq$ and $\triangle pyq$
leads to a polyhedral surface with smaller area.
I.e., $T$ was is not minimal, a contradiction.

\begin{wrapfigure}{r}{21mm}
\begin{lpic}[t(-2mm),b(-6mm),r(0mm),l(0mm)]{pics/tent}
\end{lpic}
\end{wrapfigure}

\parit{Comments.}
This problem is discussed in my paper \cite{petrunin-monthly}.

For general polyhedral surface, the deformation which decrease the legths of all edges may not decrease the area.
Moreover, the surface which minimize the area among all surfaces with fixed  triagulation may be not saddle. 
An example of such surface can be seen on the picture. %???+PIC


%%%%%%%%%%%%%%%%%%%%%%%%%%%%%%%%%%%%%%%%%%%%%%%%%%
\parbf{\ref{Coherent  triangulation}.} 
\textit{Coherent triangulation.} Look at the diagram and think.
%\begin{wrapfigure}[2]{r}{15mm}
\begin{center}
\begin{lpic}[t(-0mm),b(0mm),r(0mm),l(0mm)]{pics/Convex-triangulation(1)}
\end{lpic}
\end{center}
%\end{wrapfigure}

\parit{Comments.}
The problem discussed in section 7C of the book \cite{GKZ} by Gelfand, Kapranov and Zelevinsky.



%%%%%%%%%%%%%%%%%%%%%%%%%%%%%%%%%%%%%%%%%%%%%%%%%%
\parbf{\ref{conic neighborhoods}.} 
\textit{Characterization of polytope.}
Arguing by contradiction, let us assume that $P\subset \RR^m$
is a counterexample and $m$ takes minimal possible value.

Choose a finite cover $B_1,B_2,\dots B_n$ of $K$,
where $B_i=B(z_i,\eps_i)$ 
and $B_i\cap P=B_i\cap K_i$, 
where $K_i$ is a cone with the tip at $z_i$.

For each $i$, consider function $f_i(x)=|z_i-x|^2-\eps_i^2$.
Note that
\[W_{i,j}=\set{x\in\RR^n}{f_i(x)=f_j(x)}\]
is a hyperplane for any pair $i\ne j$.

The subset $P_{i,j}=P\cap W_{i,j}$ satisfies the same asumtions as $P$, but lies in a hyperplane.
Since $m$ is minimal, we get that $P_{i,j}$ is a polytope for any pair $i,j$.

Consider Voronoi domains 
\[V_{i}=\set{x\in\RR^n}{f_i(x)\ge f_j(x) \ \text{for any}\ j}.\]
Note that $P\cap V_i$ is formed by the points which lie on the segments from $z_i$ to a point in  $P\cap \partial V_i$.

The statement follows since $\partial V_i$ is covered by the hyperplanes $W_{i,j}$.

\parit{Comments.}
The problem is mentioned in the paper \cite{lebedeva-petrunin} by Lebedeva and Petrunin.



%%%%%%%%%%%%%%%%%%%%%%%%%%%%%%%%%%%%%%%%%%%%%%%%%%
\parbf{\ref{panov-S^3}.} 
\textit{A sphere with one edge.}
Such example $P$ can be found among the spherical polyhedral spaces which admit
an isometric $\mathbb{S}^1$-action with geodesic orbits.

Fix large relatively prime integers $p>q$. 
Consider the triangle $\Delta$ with angles $\tfrac\pi p$, $\tfrac\pi q$ and say $\pi\cdot(1-\tfrac1 p)$ in the sphere of radius $\tfrac12$.
Denote by $\hat \Delta$ the  doubling of $\Delta$ in its boundary.
Note that $\hat \Delta$ is homeomorphic to $\mathbb S^2$,
it has 3 singular points with total angles $2\cdot\tfrac\pi p$,
$2\cdot\tfrac\pi q$ and $2\cdot\pi\cdot(1-\tfrac1 p)$.

Consider $\mathbb S^1$-action on $\mathbb S^3\subset\CC^2$ by the diagonal matrices $\left(\begin{smallmatrix}z^p&0\\0&z^q\end{smallmatrix}\right)$, $z\in\mathbb S^1\subset\CC$.
Construct a spherical polyhedral metric $d$ on  $\mathbb S^3$
such that the $\mathbb S^1$-orbits become geodesics 
and the quotient $(\mathbb S^3,d)/\mathbb S^1$
is isometric to $\hat \Delta$.

In the constructed example 
the singular points with total angles $2\cdot\tfrac\pi p$ and
$2\cdot\tfrac\pi q$
should correspond to the points with isotropy groups $\ZZ/p$ and $\ZZ/q$ of the action.
The points in $P=(\mathbb{S}^3,d)$ on the orbits over these points will be regular points of $P$.
The singular locus $P^\star$
of $P$ will be formed by the orbit corresponding to the remaining singular point of  $\hat \Delta$.
By construction,
\begin{itemize}
\item $P^\star$ is a closed geodesic with angle $2\cdot\pi\cdot(1-\tfrac1p)$ around it.
\item $P^\star$ forms a $(p,q)$-torus knot in the ambient $\mathbb{S}^3$.
\end{itemize}


\parit{Comments.}
Likely only the torus knots can appear this way.

The construction appears in the paper \cite{panov-Kaeler} by Panov.
The cone $K$ over $P$ is a polyhedral space with natural complex structure;
that is, one can cut simplices from $\CC^2$ and the glue the cone from them in such a way that complex structures will agree along the gluings.
Moreover the cone $K$ can be holomorphically parametrized by $\CC^2$ in such a way that the cone over $P^\star$ becomes an algebraic curve $z^p=w^q$ in $(z,w)$-coordinates of $\CC^2$.



%%%%%%%%%%%%%%%%%%%%%%%%%%%%%%%%%%%%%%%%%%%%%%%%%%
\parbf{\ref{Triangulation of a torus}.} 
\textit{Triangulation of a torus.}
Let us equip the torus with the flat metric such that each triangle is equilateral.
The metric will have two singular cone points,
the first corresponds to the vertex $v_5$ with 5 triangles,
the total angle around this point is $\tfrac53\cdot\pi$
and the second corresponds to the vertex $v_7$ with 7 triangles,
the total angle around this point is $\tfrac73\cdot\pi$.

Prove the following.

\parbf{Observation} \textit{The holonomy group of this metric is generated by rotation by $\tfrac\pi3$.}

\medskip

Consider a closed geodesic $\gamma_1$ which minimize the length of all circles which are not null-homotopic.
Let $\gamma_2$ be an other closed geodesic which minimize the length and is not homotopic to any power of $\gamma_1$.

Show that $\gamma_1$ and $\gamma_2$ intersect at a single point.

Show that $\gamma_i$ can not pass $v_5$.

Apply the observation above 
to show that if $\gamma_i$ pass through $v_7$ then the measure  
of one of two angles which $\gamma_i$ cuts at $v_7$ equals to $\pi$.
Use the later statement to show that  
one can push $\gamma_i$ aside so it does not longer pass through $v_7$, but remains a closed geodesic.

Cut $\TT^2$ along $\gamma_1$ and $\gamma_2$.
In the obtained quadrilateral, connect $v_5$ to $v_7$ by a minimizing geodesic and cut along it.
This way we obtain an annulus with flat metric.
Look at the neighborhood of the boundary components and show that the anulus can and can not be isometrically immersed into the plane;
this is a contradiction.

\begin{wrapfigure}{r}{20mm}
%\begin{center}
\begin{lpic}[t(-7mm),b(-4mm),r(0mm),l(0mm)]{pics/57-triangulation(1)}
%\lbl[b]{24,41;x}
\end{lpic}
%\end{center}
\end{wrapfigure}

\parit{Comments.}
There are flat metrics on the torus with two singular points which have the total angles $\tfrac53\cdot\pi$ and $\tfrac73\cdot\pi$.
Such example can be obtained by identifying the the hexagon on the picture  according to the arrows.
But the holonomy group of the obtained torus is generated by the rotation by angle $\tfrac\pi6$. 
In particular, the observation is necessary in the proof.

The same argument shows that 
holonomy group of flat torus with exactly two singular points with total angle $2\cdot(1\pm \tfrac1n)\cdot\pi$ has more than $n$ elements.
In the solution we did the case $n=6$.

The problem was originally discovered and solved in \cite{jendrol-jucovich},
their proof is combinatorial.
The solution described above was given by Rostislav Matveyev
in his lectures \cite{matveyev}.
A complex-analytic proof was later found in the paper \cite{izmestiev-rote-springborn-kusner} by Izmestiev, Rote, Springborn and Kusner.

%%%%%%%%%%%%%%%%%%%%%%%%%%%%%%%%%%%%%%%%%%%%%%%%%%
\parbf{\ref{Unique geodesics imply CAT}.} 
\textit{Unique geodesics imply $\mathrm{CAT}(0)$.}
Uniqueness of geodesics implies that $P$ is contractable.
In particular, $P$ is simply connected.

It remains to prove that $P$ is locally $\mathrm{CAT}(0)$;
equivalently, the space of directions $\Sigma_p$
at any point $p\in P$ is  a $\mathrm{CAT}(1)$ space.

We can assume that the statement holds in all dimensions less than $\dim P$. 
In particular, $\Sigma_p$ is locally $\mathrm{CAT}(1)$.
If $\Sigma_p$ is not $\mathrm{CAT}(1)$ then it contains a periodic geodesic $\gamma$ of length $\ell<2\cdot\pi$,
such that any arc of $\gamma$ of length $\tfrac\ell2$ is length minimizing.

Consider two points $x$ and $y$
in the tangent cone of $p$
in directions $\gamma(0)$ and $\gamma(\tfrac\ell2)$.
Show that there are two distinct minimizing geodesics between $x$ and $y$.
The later leads to a contradiction.

\parit{Comments.}
The existence of geodesic $\gamma$ seem to be proved first by Bowditch in \cite{bowditch};
a simpler proof can be found in \cite{akp}.



%%%%%%%%%%%%%%%%%%%%%%%%%%%%%%%%%%%%%%%%%%%%%%%%%%
\parbf{\ref{No simple geodesic}.} 
\textit{No simple geodesic.}
The curvature of a vertex on the surface of a convex polyhedron
is defined as the $2\cdot\pi-\theta$, where $\theta$ is the total angle around the vertex.

Notice that a simple closed geodesic cuts the surface into two discs with total curvature $2\cdot\pi$ each.
Therefore it is sufficient to construct a convex polyhedron with curvatures of the vertices $\omega_1,\omega_2,\dots,\omega_n$ such that
$2\cdot\pi$ can not be obtained as sum of some of $\omega_i$.
An example of that type can be found among 3-simplexes.
 
\parit{Comments.} The problem discussed in the Galperin's paper \cite{galperin}.

\section*{Discrete geometry}
%%%%%%%%%%%%%%%%%%%%%%%%%%%%%%%%%%%%%%%%%%%%%%%%%%
\parbf{\ref{box-in-box}.} 
\textit{Box in a box.}
Let $\Pi$ be a parallelepiped
with dimensions $a$, $b$ and $c$.
Denote by $v(r)$ the volume of  $r$-neighborhoodsof $\Pi$,
 
Note that for all positive $r$ we have
\[v(r)=w_3+w_2\cdot r+w_1\cdot r^2+w_0\cdot r^3,\leqno({*})\]
where 
\begin{itemize}
\item $w_0=\tfrac43\cdot \pi$ is the volume of unit ball,
\item $w_1=\pi\cdot (a+b+c)$,
\item $w_2=2\cdot(a\cdot b+b\cdot c+c\cdot a)$ is the surface area of $\Pi$,
\item $w_3=a\cdot b\cdot c$ is the volume of $\Pi$,
\end{itemize}

Assume $\Pi'$ be an other parallelepiped
with dimensions $a'$, $b'$ and $c'$.
For the volume $v'(r)$ the volume of  $r$-neighborhoods of $\Pi'$ we have a formula similar $({*})$.

Note that if $\Pi\subset \Pi'$ then $v(r)\le v'(r)$ for any $r$.
Checking this inequality for $r\to\infty$,
we get 
\[a+b+c\le a'+b'+c'.\]

\parit{Comments.}
The problem was discussed by Shen in \cite{shen}.

A formula analogous to $({*})$
holds for arbitrary convex body $B$ in arbitrary dimension $m$.
The coefficient $w_i(B)$ in the polynomial with different normalization constants 
uppear under different names most commonly
\emph{intrinsic volume} and
\emph{quermassintegral}.
The later can be also defined as the average 
of area of projections of $B$ to the $i$-dimensional planes.
In particular if $B'$ and $B$ are convex bodies such that $B'\subset B$
then $w_i(B')\le w_i(B)$ for any $i$.
This generalize our problem quite a bit.
Further generalizations lead to so called \emph{mixed volumes},
see \cite{burago-zalgaller} for more on the subject.



%%%%%%%%%%%%%%%%%%%%%%%%%%%%%%%%%%%%%%%%%%%%%%%%%%
\parbf{\ref{Round circles}.} 
\textit{Round circles in $\mathbb{S}^3$.}
For each circle consider the containing it plane in $\RR^4$.
Note that the circles are linked 
if and only if 
the corresponding planes intersect at a single point inside $\mathbb{S}^3$.

Take the intersection of the planes with the sphere of radius $R\ge 1$,
rescale and pass to the limit as $R\to\infty$.  
This way we get needed isotopy.

\parit{Comments.} 
The problem was discussed in the thesis of Walsh, see \cite{walsh}.

%%%%%%%%%%%%%%%%%%%%%%%%%%%%%%%%%%%%%%%%%%%%%%%%%%
\parbf{\ref{Harnack}.} 
\textit{Harnack's circles.}
Let $\sigma\subset \RP^2$ be a smooth algebraic curve of degree $d$.
Consider the complexification $\Sigma\subset \CP^2$ of $\sigma$.
Without loss of generality, we may assume that $\Sigma$ is regular.

Prove that all regular complex algebraic curves of degree $d$ in $\RP^2$
are homeomorphic to each other.
Straightforward calculation show that $\Sigma$ has genus $n\z=\tfrac12\cdot(d^2-3\cdot d+4)$.

The real curve $\sigma$ forms the fixed point set of $\Sigma$ by complex conjugation. 
Prove that each connected component of $\sigma$ adds $1$ to the genus of $\Sigma$.
Hence the result follows.

\parit{Comment.}
This problem was suggested by Greg Kuperberg, see \cite{One-step problems in geometry}.



%%%%%%%%%%%%%%%%%%%%%%%%%%%%%%%%%%%%%%%%%%%%%%%%%%
\parbf{\ref{2pts-on-line}.} 
\textit{Two points on each line.}
Take any complete ordering of the set of all lines 
so that each beginning interval has cardinality less than continuum.

Assume we have a set of points $X$ such that each line intersects $X$ at at most $2$ points and cardinality of $X$ is less than continuum.

Choose the least line $\ell$ in the ordering which intersect $X$ 
by $0$ or $1$ point.
Note that the set of all lines intersecting $X$ at two points has cardinality less than continuum.
Therefore we can choose a point on $\ell$ and add it to $X$ so that the remaining lines are not overloaded.

It remains to apply well ordering principle.



%%%%%%%%%%%%%%%%%%%%%%%%%%%%%%%%%%%%%%%%%%%%%%%%%%
\parbf{\ref{Bodies with the same of shadows}.} 
\textit{Bodies with the same of shadows.}
Let $B$ be the unit ball in $\RR^3$ centered at the origin.

Fix small $\eps>0$.
Consider two bodies 
\begin{align*}
B''&=\set{(x,y,z)\in B}{x\le 1-\eps,\  y\le 1-\eps},
\\ 
B'''&=\set{(x,y,z)\in B}{x\le 1-\eps,\  y\le 1-\eps,\  z\le 1-\eps}.
\end{align*}
Prove that $B''$ and $B'''$ have the same shadows.

\parit{Comments.} The problem based on the question of Hamkins \cite{hamkins} answered by Ivanov.



%%%%%%%%%%%%%%%%%%%%%%%%%%%%%%%%%%%%%%%%%%%%%%%%%%
\parbf{\ref{pr:Kissing number}.} 
\textit{Kissing number.}
Let $m=\mathop{\rm kiss} B$
and $B_1,B_2,\dots, B_m$ the the copies of $B$ 
which touch $B$ and have no common interior points.
For each $B_i$ consider the vector $v_i$ from the center of $B$ to the center of $B_i$.
Note that $\measuredangle(v_i,v_j)\ge \tfrac\pi3$ if $i\ne j$.

For each $i$,
consider supporting hyperplane $\Pi_i$
to $W$
with outer normal vector $v_i$.
Denote by $W_i$ the reflection of $W$ in $\Pi_i$.

Prove that $W_i$ and $W_j$ have no common interior points if $i\ne j$;
the later gives the needed inequality.

\parit{Comments.}
The proof is given by Halberg, Levin and Straus in \cite{halberg-levin-straus}.

It expected that the same inequality holds for the orientation-preserving version of kissing number.



%%%%%%%%%%%%%%%%%%%%%%%%%%%%%%%%%%%%%%%%%%%%%%%%%%
\parbf{\ref{mono-homotopy}.} 
\textit{Monotonic homotopy.}
Note that we can assume
that $h_0(F)$ and $h_1(F)$ both lie in the coordinate $m$-spaces of $\RR^{2\cdot m}=\RR^m\times \RR^m$;
that is,
$h_0(F)\z\subset\RR^m\times\{0\}$
and $h_1(F)\subset  \{0\}\times\RR^m$.

Show that the following homotopy is monotonic
\[h_t(x)=\bigl(h_0(x)\cdot \cos\tfrac{\pi\cdot t}2
\,,\,
 h_1(x)\cdot\sin\tfrac{\pi\cdot t}{2}\bigr).\] 


\parit{Comment.}
This homotopy was discovered by Ralph Alexander in \cite{ralexander}.
It has number of applications, 
one of the most beautiful is the given by Bezdek and Connelly \cite{bezdek-connelly} in their proof of Kneser--Poulsen  and Klee--Wagon conjectures in dimension $2$.



%%%%%%%%%%%%%%%%%%%%%%%%%%%%%%%%%%%%%%%%%%%%%%%%%%
\parbf{\ref{Cube}.} 
\textit{Cube.}
Consider the cube $[-1,1]^n\subset \RR^n$.
Any vertex this cube has the form $\bm{q}=(q_1,q_2,\dots,q_n)$,
where  $q_i=\pm1$.

For each vertex $\bm{q}$,
consider the intersection of the corresponding octant with the unit sphere;
that is, the set
\[V_{\bm{q}}=\set{(x_1,x_2,\dots,x_n)\in\mathbb{S}^{n-1}}{q_i\cdot x_i\ge 0\ \text{for each}\ i}.\]

Consider the set $\mathcal{A}\subset\mathbb{S}^{n-1}$
formed by the union of all the sets $V_{\bm{q}}$ for $\bm{q}\in A$.
Note that 
\[\vol_{n-1}\mathcal{A}=\tfrac12\cdot\vol_{n-1}\mathbb{S}^{n-1}\]
and 
\[\vol_{n-2}\partial\mathcal{A}
=
\tfrac k{2^{n-1}}\cdot\vol_{n-2}\mathbb{S}^{n-2},\]
where $k$ is the number of edges of the cube with one end in $A$ and the other in $B$.

It remains to  show that 
\[\vol_{n-2}\partial\mathcal{A}
\ge \vol_{n-2}\mathbb{S}^{n-2}.\]
The later follows from the isoperimetric inequality for $\mathbb{S}^n$. 

\parit{Comment.}
The problem was suggested by Greg Kuperberg, 
see \cite{One-step problems in geometry}.



%%%%%%%%%%%%%%%%%%%%%%%%%%%%%%%%%%%%%%%%%%%%%%%%%%
\parbf{\ref{Right-angled polyhedron}.} 
\textit{Right-angled polyhedron.}
Before coming into proof read 
about \hyperref[Dehn--Sommerville equations]{\emph{Dehn--Sommerville equations}}
on page \pageref{Dehn--Sommerville equations}.

Let $P$ be a right-angled hyperbolic polyhedron of dimension $m$.
Note that $P$ is simple; that is, exactly $m$ facets meet at each vertex of $P$.

From the projective model of hyperbolic plane, 
one can see that for any simple compact hyperbolic polyhedron there is a simple Euclidean polyhedron with the same combinatorics. 
In particular Dehn--Sommerville equations hold for $P$.

Denote by $(f_0,f_1,\dots f_m)$ and $(h_0,h_1,\dots h_m)$ the $f$- and $h$-vectors of $P$.
Recall that $h_i\ge 0$ for any $i$ and $h_0=h_m=1$.
By Dehn--Sommerville equations, we get
\[f_2> \tfrac{m-2}4\cdot f_1.
\leqno({*})\]

Since $P$ is hyperbolic, each 2-dimensional face of $P$ has at least 5 sides.
It follows that
\[f_2\le \tfrac{m-1}5\cdot f_1.\]
The later contradicts $({*})$ for $m\ge 6$.

\parit{Comments.} 
The proof above 
is the core of Vinberg's proof of nonexistance of compact hyperbolic Coxeter's polyhedra of large dimensions given in \cite{vinberg}, see also \cite{vinberg-strong}.

Playing a bit more with the same inequalities, 
one gets nonexistance of  right-angled hyperbolic polyhedra,
in all dimensions starting from $5$.
In 4-dimensional case, an example of a bonded right-angled hyperbolic polyhedron
can be found among regular \emph{120-cells}.







\chapter{Dictionary}

\begin{description}

\item{\bf Asymptotic line}\refstepcounter{thm}\label{Asymptotic line} on the surface $\Sigma\subset \RR^3$
is a curve always tangent to an \emph{asymptotic direction} of $\Sigma$; 
that is, the direction in which the normal curvature of $\Sigma$ is zero.

\item{\bf Busemann function.}\refstepcounter{thm}\label{Busemann function} 
Let $X$ be a metric space
and $\gamma$ is a ray in $X$; 
that is, $\gamma\:[0, \infty)\to X$ is a \hyperref[Geodesic]{\emph{minimizing unit-speed geodesic}}.

The Busemann function $b_\gamma\:X\to\RR$ is defined by
$$b_\gamma(p)=\lim_{t\to\infty}\left(|p-\gamma(t)|_X-t\right).$$
From the triangle inequality, 
the expression under the limit is non-increasing in $t$; 
therefore  the limit above is defined for any $p$.

\item{\bf Curvature operator.}\refstepcounter{thm}\label{Curvature operator}
The Riemannian curvature tensor $R$
can be viewed as an operator $\text{\bf R}$ on the space of tangent bi-vectors $\bigwedge^2 \T$;
it is uniquely defined by identity
$$\langle\mathbf{R}(X\wedge Y),V\wedge W\rangle
=
\langle R(X,Y)V,W\rangle,$$
The operator $\mathbf{R}\:\bigwedge^2 \T\to \bigwedge^2 \T$ is called \emph{curvature operator} and it is said to be \emph{positive definite} if
$\langle\mathbf{R}(\phi),\phi\rangle>0$ for all non zero
bi-vector $\phi\in\bigwedge^2 \T$.

\item{\bf Dehn twist.}\refstepcounter{thm}\label{Dehn twist}
Let $\Sigma$ be a surface and $\gamma\:\RR/\ZZ\to\Sigma$ be non-contractible closed \hyperref[Simple curve]{\emph{simple curve}}.
Let $U_\gamma$ be a neighborhood of $\gamma$ which admits a homeomorphism $h\:U_\gamma\to \RR/\ZZ\times (0,1)$.
Dehn twist along $\gamma$ is a homeomorphism $f\:\Sigma\to\Sigma$
which is identity outside of $U_\gamma$ and 
such that
$$h\circ f\circ h^{-1}\:(x,y)\mapsto(x+y,y).$$

If $\Sigma$ is oriented 
and $h$ is orientation preserving,
then the Dehn twist described above is called \emph{positive}.

\item{\bf Dehn--Sommerville equations}\refstepcounter{thm}\label{Dehn--Sommerville equations}.
Assume $P$ is a \emph{simple} Euclidean $m$-dimensional polyhedron;
that is, every vertex of $P$ exactly $m$ facets are meeting.
Denote by $f_k$ the number of $k$-dimensional faces of $P$;
the array of integers $(f_0,f_1,\dots f_m)$ is called $f$-vector of $P$.

Fix an order of the vertices $v_1,v_2,\dots v_{f_0}$
of $P$ so that for some linear function $\ell$, we have $\ell(v_i)>\ell(v_j)\ \Leftrightarrow\ i<j$.
The \emph{index} of the vertex $v_i$ 
is defined as the number of edges $[v_iv_j]$ such that $j<i$. 
The number of vertices of given index $k$ will be denoted as $h_k$.
The array of integers $(h_0,h_1,\dots h_m)$ is called $h$-vector of $P$.
Clearly $h_0=h_m=1$ and $h_k\ge 0$ for all $k$.

Each $k$-face of $P$ contains unique vertex which maximize $\ell$;
if the vertex has index $i$,
then $i\ge k$ and
then it is the maximal vertex for exactly $\tfrac{i!}{k!\cdot (i-k)!}$
faces of dimension $k$.
This observation can be packed in the following polynomial identity 
\[\sum_k h_k\cdot (t+1)^k=\sum_k f_k\cdot t^k.\]

Note that the identity above implies that $h$-vector does not depend on the choice of order of the vertices.
In particular, the $h$ vector is the same for the reversed order;
that is
\[h_k=h_{m-k}\]
for any $k$.
These identities are called Dehn--Sommerville equations.
It gives the complete list of linear equations for $h$-vectors (and therefore $f$-vectors) of simple polyhedrons.

\item{\bf Doubling}\refstepcounter{thm}\label{Doubling} 
of a metric space $V$ in a closed subset $A\subset V$
is the metric space $W$ which obtained by gluing two copies of $V$ along the corresponding points of $A$.

More precisely, consider the minimal equivalence relation $\sim$ on the set $V\times\{1,2\}$,  such that $(a,1)\sim (a,2)$ for any $a\in A$.
Then  $W$ 
is the set $(V\times\{1,2\})/\sim$, 
equipped with the metric such that 
\begin{align*}
|(x,i)-(y,i)|_W&=|x-y|_V
\intertext{and}
|(x,1)-(y,2)|_W&=\inf\set{|x-a|_V+|y-a|_V}{a\in A}
\end{align*}

for any $x,y\in V$.

For a manifold with boundary,
the doubling is usually assumed to be taken in its boundary;
in this case the resulting space is a manifold without boundary.

\item{\bf Euclidean cone.}\refstepcounter{thm}\label{Euclidean cone} 
Let $\Sigma$ be a metric space with diameter $\le \pi$. 
A metric space $K$ is called Euclidean cone over $\Sigma$
if its underling set 
coincides with the quotient 
$\Sigma\times [0,\infty)/{\sim}$
by the minimal equivalence relation $\sim$ such that $(x,0)\sim(y,0)$ for any $x,y\in \Sigma$
and the metric is defined by cosine rule;
that is,
$$|(x,a)-(y,b)|^2_K=a^2+b^2-2\cdot a\cdot b\cdot\!\cos|x-y|_\Sigma.$$

\item{\bf Energy functional.}\refstepcounter{thm}\label{Energy functional} Let $F$ be a smooth map from a closed Riemannian manifold $M$ to a Riemannian manifold $N$.
Then energy functional of $F$ is defined as
\[E(F)=\int\limits_M |d_xF|^2\cdot d_x\vol_M.\]
If $(a_{i,j})$ denote the components 
of the differential $d_xF$ 
written in the orthonormal bases in $\T_xM$ and $\T_{F(x)}N$,
then 
\[|d_xF|^2=\sum_{i,j}a_{i,j}^2.\]

\item{\bf Equidistant subsets.}\refstepcounter{thm}\label{Equidistant subsets} 
Two subsets $A$ and $B$ in a metric space $X$ are called equidistant if the distance function $\dist_A\:X\to\RR$ is constant on $B$ and $\dist_B$ is constant on $A$.

\item{\bf Exponential map.}\refstepcounter{thm}\label{Exponential map} 
Let $(M,g)$ be a Riemannian manifold;
denote by $\T M$ the tangent bundle over $M$ and by $\T_p=\T_pM$ the tangent space at point $p\in M$.

Given a vector $v\in\T_pM$ denote by $\gamma_v$ the geodesic in $(M,g)$
such that $\gamma(0)=p$ and $\gamma'(0)=v$.
The map $\exp\:\T M\to M$ defined by $v\mapsto \gamma_v(1)$ is called exponential map.

The restriction of $\exp$ to the $\T_p$ is called \emph{exponential map at} $p$ and denoted as $\exp_p$.

Given a smooth submanifold $S\subset M$;
denote by $\mathrm{N} S$ the normal bundle over $S$.
The restriction of $\exp$ to $\mathrm{N} S$ is called \emph{normal exponential map} of $S$ and denoted as $\exp_S$. 

\item{\bf G-delta set}\refstepcounter{thm}\label{G-delta set} is a countable intersection of open sets.

Clearly dense G-delta set has to be a countable intersection of dense open sets.
According to Baire category theorem, the converse also holds in complete metric spaces and  locally compact Hausdorff space.
That is a countable intersection of dense open set is dense.

\item{\bf Geodesic.}\refstepcounter{thm}\label{Geodesic}  
Let $X$ be a metric space and $\II$ be a real interval.
A locally isometric immersion $\gamma\:\II\looparrowright X$ is called \emph{unit-speed geodesic}.
In other words, $\gamma$ is a unit-speed geodesic
if for any $t_0\in\II$ we have 
$$|\gamma(t)-\gamma(t')|_X=|t-t'|$$ 
for all $t,t'\in\II$ sufficiently close to $t_0$.

If the condition holds for any $t,t'\in\II$, then $\gamma$ is called \emph{minimizing}.
A minimizing geodesic from point $p$ to point $q$ usually denoted $[pq]$.

Any linear re-parametrization of $\gamma$ is called \emph{geodesic}.

\item{\bf Heisenberg group}\refstepcounter{thm}\label{Heisenberg group}
is the group of $3\times3$ upper triangular matrices of the form
\[\begin{pmatrix}
 1 & a & c\\
 0 & 1 & b\\
 0 & 0 & 1\\
\end{pmatrix}\]
under the operation of matrix multiplication. The elements $a$, $b$ and $c$ usually assumed to be real,
but they can be taken from any commutative ring with identity.

\item{\bf Kissing number.}\refstepcounter{thm}\label{Kissing number}
Let  $W_0$ be a convex body in $\RR^m$.
The kissing number of $W_0$ is the maximal integer $k$ such that there are $k$ bodies $W_1,W_2,\dots,W_k$ such that each $W_i$ is congruent to $W_0$,
for each $i$ we have $W_i\cap W_0\not=\emptyset$ and have no pair $W_i,W_j$ has common interior points.

\item{\bf  Length-metric space.}\refstepcounter{thm}\label{Length-metric space} 
A complete metric space $X$ is called {\it length-metric space}  if the distance between any pair of points in $X$ is equal to the infimum of lengths of curves connecting these points. 

\item{\bf Macro-dimension.}\refstepcounter{thm}\label{Macro-dimension}
Let $X$ be a locally compact metric space $a>0$ and $m$ is an integer.
We say that the macro-dimension  of $X$ at the scale $a$ is at most $m$
if there is a continuous map $f$ from $X$ to an $m$-dimensional simplicial complex $K$
such that for any $k\in K$ the inverse image $f^{-1}(\{k\})$ has diameter less than $a$.

If macro-dimension of $X$ at the scale $a$ is at most $m$,
but not at most $m-1$, 
we say that $m$ is the macro-dimension of $X$ at the scale $a$.

Equivalently, the macro-dimension of $X$ on scale $a$ can be defined as 
the least integer $m$ such that $X$ admits an open covering with diameter of each set less than $a$ 
and such that each point in $X$ is covered by at most $m+1$ sets in the cover.


\item{\bf Length-preserving map.}\refstepcounter{thm}\label{Length-preserving map} 
A continuous map $f\:X\to Y$ between 
\hyperref[Length-metric space]{\emph{length-metric spaces}} 
$X$ and $Y$ is a length-preserving map if for any path $\alpha\:[0,1]\to X$, we have 
$$\length(\alpha)=\length(f\circ\alpha).$$

\item{\bf Minimal surface.}\refstepcounter{thm}\label{Minimal surface} 
Let $\Sigma$ be a $k$-dimensional smooth surface in
a Riemannian manifold $M$
and $\T=\T\,\Sigma$ and $\mathrm{N}=\mathrm{N}\,\Sigma$ correspondingly tangent and normal bundle.
Let $s\:\T\otimes \T\to \mathrm{N}$ denotes the \hyperref[Second fundamental form]{\emph{second fundamental form}} of $\Sigma$.
Let  $e_i$ is an orthonormal basis for $\T_x$, 
set $H_x=\sum_i s(e_i,e_i)\in \mathrm{N}_x$; 
it is the mean curvature vector at $x\in \Sigma$. 

We say that $\Sigma$ is \emph{minimal} if $H\equiv 0$.

\item{\bf Nil-manifolds}\refstepcounter{thm}\label{Nil-manifolds} form the minimal class of manifolds which includes a point, and has the following property:  
the total space of any principle $\mathbb{S}^1$-bundle over a nil-manifold is a nil-manifold. 

Any nil-manifold is diffeomorphic to the quotient of a connected nilpotent Lie group by a lattice.

The celebrated Gromov's theorem states that almost flat manifolds admit a finite cover by a nil-manifold.



\item{\bf Polyhedral space}\refstepcounter{thm}\label{Polyhedral space}
is a complete length-metric space which admits a finite triangulation 
such that each simplex is globally isometric to a simplex in a Euclidean space.

A point in a polyhedral space is called \emph{regular} if it has a neighborhood isometric to an open set in a Euclidean space;
otherwise it called \emph{singular}.

Often finiteness of the triangulation is relaxed to \emph{local finiteness}.

If one exchange Euclidean space to sphere or hyperbolic space,
one gets definition of \emph{spherical} and correspondingly \emph{hyperbolic polyhedral spaces}.
To define regular/singular points in spherical or hyperbolic space,
one has to exchange in the above definition Euclidean space to unit sphere or hyperbolic space with curvature $-1$.


\item{\bf Polynomial volume growth.}\refstepcounter{thm}\label{Polynomial volume growth} A Riemannian manifold $M$ has polynomial volume growth if for some (and therefore any) $p\in M$, 
we have 
$$\vol B(p,r)\z\le C\cdot (r^k+1),$$ 
where $B(p,r)$ is the ball in $M$ and  $C$, $k$ are real constants.

\item{\bf Proper metric space.}
\refstepcounter{thm}\label{Proper metric space} 
A metric space $X$ is called \emph{proper} if any closed bounded set in $X$ is compact.

\item{\bf Piecewise distance preserving map.}%
\refstepcounter{thm}%
\label{Piecewise distance preserving map} 
Let $P$ and $Q$ be polyhedral spaces, a map $f\:P\to Q$ is called piecewise linear isometry if there is a triangulation $\mathcal{T}$ of $P$ such that at any simplex $\Delta\in \mathcal{T}$ the restriction $f|_\Delta$ is distance preserving.

\item{\bf  PL-homeomorphism}\refstepcounter{thm}\label{PL-homeomorphism} or piecewise linear homeomorphism.
A map $h\:P\to Q$ between polyhedral spaces $P$ and $Q$ is called PL-homeomorphism if it is a homeomorphism and both spaces $P$ and $Q$ admit triangulations such that each simplex of $P$ is mapped to a simplex of $Q$ by an affine map.

\item{\bf  Quasi-isometry.}\refstepcounter{thm}\label{Quasi-isometry} A map $f\:X\to Y$ is called a quasi-isometry if there is a positive real constant $C$ such that $f(X)$ is a $C$-net in $Y$ and
$$\tfrac{1}{C}\cdot|x-y|_X-C
\le 
|f(x)-f(y)|_Y\le C\cdot|x-y|_X+C.$$

Note that a quasi-isometry is not assumed to be continuous, for example any map between compact metric spaces is a quasi-isometry.

\item{\bf Saddle surface.}\refstepcounter{thm}\label{Saddle surface} A smooth surface $\Sigma$ in $\RR^3$ is saddle 
(correspondingly strictly saddle) 
if  the product of the principle curvatures at each point is $\le 0$ (correspondingly $<0$).

It admits the following generalization to non-smooth case and arbitrary dimension of the ambient space:
A surface $\Sigma$ in $\RR^m$ is saddle if the restriction $\ell|_\Sigma$ of any linear function $\ell\:\RR^m\to\RR$ has no strict local minima at interior points of $\Sigma$.

One can generalize it further to an arbitrary ambient space, using  convex functions instead of linear functions in the above definition.


\item{\bf Sasaki metric.}\label{Sasaki metric}
Let $(M,g)$ be a Riemannian manifold.
The Sasaki metric is the most natural choice of metric on the tangent space $\T M$.
It is uniquely defined by the following properties:
\begin{enumerate}[(i)]
\item The natural projection $\tau\:\T M\to M$ is a Riemannian submersion.
\item The metric on each tangent space $\T_p\subset \T M$ is the Euclidean metric induced by $g$.
\item Assume $\gamma(t)$ is a curve in $M$ and $v(t)\in\T_{\gamma(t)}$ is a parallel vector field along $\gamma$. 
Note that $v(t)$ forms a curve in $\T M$ 
and $\T_{\gamma(t)}M$ forms a submanifold in $\T M$.
For the Sasaki metric, we have $\dot v(t)\perp \T_{\gamma(t)}M$ for any $t$.
\end{enumerate}

A more constructive way to describe Sasaki metric is given by identifying 
$\T_u[\T M]$ for any $u\in \T_p M$ with the direct sum of so called vertical and horizontal vectors $\T_p M\oplus \T_p M$.
The projection of this splitting defined by the differential of $\tau$
and the Levi-Civita connection.
Then $\T_u[\T M]$ is equipped with the metric  defined as 
\[\hat g(X,Y)=g(X^V,Y^V)+g(X^H,Y^H),\]
where $X^V,X^H\in\T_pM$ denotes the vertical and horizontal components of $X\in\T_u[\T M]$.

\item{\bf Second fundamental form.}\refstepcounter{thm}\label{Second fundamental form} 
Assume $f\:M\looparrowright \RR^m$ be an immersion of smooth manifold $M$.
Given a point $p\in M$ denote by $\T_p$ and $\mathrm{N}_p=\T_p^\bot$
the tangent and normal spaces of $L$ at $p$.
The second fundamental form for $f$ at $p\in M$ is defined as $$s(v,w)=(\nabla_v w)^\bot,\eqno({*})$$ 
where $(\nabla_v w)^\bot$ denotes the orthogonal projection of covariant derivative $\nabla_v w$ onto the normal bundle.

Assume $\gamma_v\:\RR\to M$ is a geodesic with tangent vector $v\in \T_p$;
that is, such that $\gamma_v(0)=p$ and $\gamma'(0)=v$.
Then 
\[s(v,v)=(f\circ\gamma_v)''(0).\]

This property can be also used to define second fundamental form via the identity
$$s(v,w)=\tfrac 12\cdot[s(v+w,v+w)-s(v,v)-s(w,w)].$$

The formula $({*})$ can be used to define the second fundamental form for smooth immersions from into Riemannian manifold.

\item{\bf Short map}\refstepcounter{thm}\label{Short map} --- the same as 1-Lipschitz 
or distance non-expanding map.

\item{\bf Simple curve}\refstepcounter{thm}\label{Simple curve} --- an image of a continuous injective map of a real segment or a circle in a topological space.

\item{\bf Sub-Riemannian metric.}\refstepcounter{thm}\label{Sub-Riemannian metric}
Let $(M,g)$ is a Riemannian manifold.

Assume that in the tangent bundle $\T M$ 
a choice of sub-bundle $H$ is given;
the sub-bundle $H$ which will be called  \emph{horizontal distribution}.
The tangent vectors which lie in $H$ will be called \emph{horizontal}.
A piecewise smooth curve will be called \emph{horizontal}
if all its tangent vectors are horizontal.

The sub-Riemannian distance between points $x$ and $y$ is defined as infimum of lengths of horizontal curves connecting $x$ to $y$.

Alternatively, the distance can be defined as a limit of Riemannian distances 
for the metrics 
\[g_\lambda(X,Y)=g(X^H,Y^H)+\lambda\cdot g(X^V,Y^V)\] 
as $\lambda\to \infty$,
where $X^H$ denotes the horizontal part of $X$;
that is, the orthogonal projection of $X$ to $H$
and $X^V$ denotes the vertical part of $X$;
so, $X^V+X^H=X$.

One usually adds a condition which ensure that any curve in $M$ can be arbitrary well approximated by a horizontal curve with the same endpoints.
(In particular this ensures that the distance will not take infinite values.)
The most common condition is so called  \emph{complete non-integrability};
it means that for any $x\in M$, 
one can choose a basis in $T_xM$
from the vectors of the following type:
$A(x)$, $[A,B](x)$, $[A,[B,C]](x)$, $[A,[B,[C,D]]](x),\dots$ where all vector fields $A,B,C,D, \dots$ are horizontal.


\item{\bf Total curvature.}
\refstepcounter{thm}\label{Total curvature} 
Let $\gamma\:[a,b]\to\RR^m$ be a curve.
The total curvature of $\gamma$ is defined as supremum of sum of external angles for broken lines inscribed in $\gamma$. 
Namely, 
\[\sup\set{\sum_{i=1}^{n-1}\alpha_i}{a=t_0<t_1<\dots<t_n=b},\]
where $\alpha_i=\pi-\measuredangle \hinge{\gamma(t_{i})}{\gamma(t_{i-1})}{\gamma(t_{i+1})}$.

If $\gamma$ is smooth and parametrized by the arc length, 
then its total curvature equals to
\[\int\limits_a^b|\gamma''(t)|\cdot dt.\]

\item{\bf Warped product.}
\refstepcounter{thm}
\label{def:Warped product} 
Let $(M,g)$ and $(N,h)$ be Riemannian manifolds 
and $f$ be a smooth positive function defined on $M$.
Consider the product manifold $W=M\times N$.
Given a tangent vector 
$X\z\in \T_{(p,q)} W
=\T_p M\times \T_p N$ denote by 
$X_M\z\in \T M$ and $X_N\z\in \T N$ its projections.
Let us equip $W$ with the Riemannian metric defined as
\[s(X,Y)=g(X_M,Y_M)+f^2\cdot h(X_N,Y_N).\]
The obtained Riemannian manifold $(W,s)$ is called warped product of $M$ and $N$ with respect to $f\:M\to \RR$;
it can be written as  $(W,g)=(N,h)\times_f(M,g)$.

\end{description}
\begin{bibdiv}
\begin{biblist}

%\bibitem[Lawlor]{lawlor} Lawlor G., \textit{A new Area-Maximization Proof for the Circle,} Math. Intelligencer 20 (1998) no.1.

\bib{rox}{article}{
   title={Geometric group theory, hyperbolic dynamics and symplectic
   geometry},
   %note={Abstracts from the workshop held July 15--21, 2012;
   %Organized by Gerhard Knieper, Leonid Polterovich and Leonid %Potyagailo},
   journal={Oberwolfach Rep.},
   volume={9},
   date={2012},
   number={3},
   pages={2139--2203},
   %issn={1660-8933},
   %review={\MR{3156727}},
   %doi={10.4171/OWR/2012/35},
}


\bib{abresch-gromoll}{article}{
   author={Abresch, Uwe},
   author={Gromoll, Detlef},
   title={On complete manifolds with nonnegative Ricci curvature},
   journal={J. Amer. Math. Soc.},
   volume={3},
   date={1990},
   number={2},
   pages={355--374},
   %issn={0894-0347},
   %review={\MR{1030656 (91a:53071)}},
   %doi={10.2307/1990957},
}

\bib{akopyan-tarasov}{article}{
   author={Akopyan, A. V.},
   author={Tarasov, A. S.},
   title={A constructive proof of Kirszbraun's theorem},
   %language={Russian},
   %journal={Mat. Zametki},
   %volume={84},
   %date={2008},
   %number={5},
   %pages={781--784},
   %issn={0025-567X},
   %translation={
      journal={Math. Notes},
      volume={84},
      date={2008},
      number={5-6},
      pages={725--728},
      %issn={0001-4346},
   %},
   %review={\MR{2500644 (2010c:47123)}},
   %doi={10.1134/S000143460811014X},
}

\bib{alexander-osserman}{article}{
   author={Alexander, H.},
   author={Osserman, R.},
   title={Area bounds for various classes of surfaces},
   journal={Amer. J. Math.},
   volume={97},
   date={1975},
   number={3},
   pages={753--769},
   %issn={0002-9327},
   %review={\MR{0380596 (52 \#1495)}},
}

\bib{alexander-hoffman-osserman}{article}{
   author={Alexander, H.},
   author={Hoffman, D.},
   author={Osserman, R.},
   title={Area estimates for submanifolds of Euclidean space},
   conference={ title={Symposia Mathematica, Vol. XIV},address={Convegno di Teoria Geometrica dell'Integrazione e Variet\`a Minimali, INDAM, Rome},date={1973},},
   book={
      publisher={Academic Press, London},
   },
   date={1974},
   pages={445--455},
   %review={\MR{0388253 (52 \#9090)}},
}

\bib{ralexander}{article}{
   author={Alexander, Ralph},
   title={Lipschitzian mappings and total mean curvature of polyhedral
   surfaces. I},
   journal={Trans. Amer. Math. Soc.},
   volume={288},
   date={1985},
   number={2},
   pages={661--678},
   %issn={0002-9947},
   %review={\MR{776397 (86c:52004)}},
   %doi={10.2307/1999957},
}

\bib{alexander}{article}{
   author={Alexander, S.},
   title={Locally convex hypersurfaces of negatively curved spaces},
   journal={Proc. Amer. Math. Soc.},
   volume={64},
   date={1977},
   number={2},
   pages={321--325},
   %issn={0002-9939},
   %review={\MR{0448262 (56 \#6571)}},
}

\bib{ABB}{article}{
   author={Alexander, Stephanie B.},
   author={Berg, I. David},
   author={Bishop, Richard L.},
   title={Geometric curvature bounds in Riemannian manifolds with boundary},
   journal={Trans. Amer. Math. Soc.},
   volume={339},
   date={1993},
   number={2},
   pages={703--716},
   %issn={0002-9947},
   %review={\MR{1113693 (93m:53034)}},
   %doi={10.2307/2154294},
} 

\bib{akp}{book}{
   author={Alexander, S.},
   author={Kapovitch, V.},
   author={Petrunin A.},
   title={Alexandrov geometry},
   status={in preparation}
}

\bib{alexandrov1948}{book}{
   author={Aleksandrov, A. D.},
   title={Vnutrennyaya Geometriya Vypuklyh Poverhnoste\u\i},
   %language={Russian},
   publisher={OGIZ, Moscow-Leningrad},
   date={1948},
   %pages={387},
   %review={\MR{0029518 (10,619c)}},
}

\bib{almgren}{article}{
   author={Almgren, F.},
   title={Optimal isoperimetric inequalities},
   journal={Indiana Univ. Math. J.},
   volume={35},
   date={1986},
   number={3},
   pages={451--547},
   %issn={0022-2518},
   %review={\MR{855173 (88c:49032)}},
   %doi={10.1512/iumj.1986.35.35028},
}


\bib{aloff-wallach}{article}{
   author={Aloff, Simon},
   author={Wallach, Nolan R.},
   title={An infinite family of distinct $7$-manifolds admitting positively
   curved Riemannian structures},
   journal={Bull. Amer. Math. Soc.},
   volume={81},
   date={1975},
   pages={93--97},
   %issn={0002-9904},
   %review={\MR{0370624 (51 \#6851)}},
}

\bib{andrews}{article}{
   author={Andrews, Ben},
   title={Contraction of convex hypersurfaces in Riemannian spaces},
   journal={J. Differential Geom.},
   volume={39},
   date={1994},
   number={2},
   pages={407--431},
   %issn={0022-040X},
   %review={\MR{1267897 (95b:53044)}},
}

\bib{antoine}{article}{
   author={Antoine, Louis},
   title={Sur l'homeomorphisme de deux figures et leurs voisinages},
   journal={J. Math. Pures Appl.},
   volume={4},
   date={1921},
   number={},
   pages={221--325},
}

\bib{balacheff-croke-katz}{article}{
   author={Balacheff, Florent},
   author={Croke, Christopher},
   author={Katz, Mikhail G.},
   title={A Zoll counterexample to a geodesic length conjecture},
   journal={Geom. Funct. Anal.},
   volume={19},
   date={2009},
   number={1},
   pages={1--10},
   %issn={1016-443X},
   %review={\MR{2507217 (2010k:53062)}},
   %doi={10.1007/s00039-009-0708-9},
}


\bib{bangert}{article}{
   author={Bangert, Victor},
   title={Geodesics and totally convex sets on surfaces},
   journal={Invent. Math.},
   volume={63},
   date={1981},
   number={3},
   pages={507--517},
   %issn={0020-9910},
   %review={\MR{620682 (82g:53050)}},
   %doi={10.1007/BF01389067},
}

\bib{bazajkin}{article}{
   author={Baza{\u\i}kin, Ya. V.},
   title={On a family of $13$-dimensional closed Riemannian manifolds of
   positive curvature},
   %language={Russian, with Russian summary},
   journal={Sibirsk. Mat. Zh.},
   volume={37},
   date={1996},
   number={6},
   pages={1219--1237, ii},
   issn={0037-4474},
   translation={
      journal={Siberian Math. J.},
      volume={37},
      date={1996},
      number={6},
      %pages={1068--1085},
      %issn={0037-4466},
   },
   %review={\MR{1440379 (98c:53045)}},
   %doi={10.1007/BF02106732},
}

\bib{bennequin}{article}{
   author={Bennequin, Daniel},
   title={Exemples d'immersions du disque dans le plan qui ne sont pas
   projections de plongements dans l'espace},
   %language={French, with English summary},
   journal={C. R. Acad. Sci. Paris S\'er. A-B},
   volume={281},
   date={1975},
   number={2-3},
   pages={Aii, A81--A84},
   %review={\MR{0380806 (52 \#1703)}},
}

\bib{berg}{article}{
   author={Berg, I. D.},
   title={An estimate on the total curvature of a geodesic in Euclidean
   $3$-space-with-boundary},
   journal={Geom. Dedicata},
   volume={13},
   date={1982},
   number={1},
   pages={1--6},
   %issn={0046-5755},
   %review={\MR{679213 (84d:53049)}},
   %doi={10.1007/BF00149423},
}

\bib{besicovitch}{article}{
   author={Besicovitch, A. S.},
   title={On two problems of Loewner},
   journal={J. London Math. Soc.},
   volume={27},
   date={1952},
   pages={141--144},
   %issn={0024-6107},
   %review={\MR{0047126 (13,831d)}},
}

\bib{bezdek-connelly}{article}{
   author={Bezdek, K{\'a}roly},
   author={Connelly, Robert},
   title={Pushing disks apart---the Kneser-Poulsen conjecture in the plane},
   journal={J. Reine Angew. Math.},
   volume={553},
   date={2002},
   pages={221--236},
   %issn={0075-4102},
   %review={\MR{1944813 (2003m:52001)}},
   %doi={10.1515/crll.2002.101},
}

\bib{bing}{article}{
   author={Bing, R. H.},
   title={Some aspects of the topology of $3$-manifolds related to the
   Poincar\'e conjecture},
   conference={
      title={Lectures on modern mathematics, Vol. II},
   },
   book={
      publisher={Wiley, New York},
   },
   date={1964},
   pages={93--128},
   %review={\MR{0172254 (30 \#2474)}},
}

\bib{birkhoff}{article}{
   author={Birkhoff, George D.},
   title={Proof of Poincar\'e's geometric theorem},
   journal={Trans. Amer. Math. Soc.},
   volume={14},
   date={1913},
   number={1},
   pages={14--22},
   issn={0002-9947},
   review={\MR{1500933}},
   doi={10.2307/1988766},
}

\bib{bochner}{article}{
   author={Bochner, S.},
   title={Vector fields and Ricci curvature},
   journal={Bull. Amer. Math. Soc.},
   volume={52},
   date={1946},
   pages={776--797},
   %issn={0002-9904},
   %review={\MR{0018022 (8,230a)}},
}

\bib{boehm-wilking}{article}{
   author={B{\"o}hm, Christoph},
   author={Wilking, Burkhard},
   title={Manifolds with positive curvature operators are space forms},
   journal={Ann. of Math. (2)},
   volume={167},
   date={2008},
   number={3},
   pages={1079--1097},
   %issn={0003-486X},
   %review={\MR{2415394 (2009h:53146)}},
   %doi={10.4007/annals.2008.167.1079},
}

\bib{bowditch}{article}{
   author={Bowditch, B. H.},
   title={Notes on locally ${\rm CAT}(1)$ spaces},
   conference={
      title={Geometric group theory},
      address={Columbus, OH},
      date={1992},
   },
   book={
      series={Ohio State Univ. Math. Res. Inst. Publ.},
      volume={3},
      publisher={de Gruyter, Berlin},
   },
   date={1995},
   pages={1--48},
   %review={\MR{1355101 (97e:53070)}},
}

\bib{brehm}{article}{
   author={Brehm, Ulrich},
   title={Extensions of distance reducing mappings to piecewise congruent
   mappings on ${\bf R}^{m}$},
   journal={J. Geom.},
   volume={16},
   date={1981},
   number={2},
   pages={187--193},
   %issn={0047-2468},
   %review={\MR{642266 (83b:51020)}},
   %doi={10.1007/BF01917587},
}

\bib{brendle-marques-neve}{article}{
   author={Brendle, Simon},
   author={Marques, Fernando C.},
   author={Neves, Andre},
   title={Deformations of the hemisphere that increase scalar curvature},
   journal={Invent. Math.},
   volume={185},
   date={2011},
   number={1},
   pages={175--197},
   %issn={0020-9910},
   %review={\MR{2810799 (2012h:53094)}},
   %doi={10.1007/s00222-010-0305-4},
}

\bib{buser-gromoll}{article}{
   author={Buser, Peter},
   author={Gromoll, Detlef},
   title={On the almost negatively curved $3$-sphere},
   conference={
      title={Geometry and analysis on manifolds},
      address={Katata/Kyoto},
      date={1987},
   },
   book={
      series={Lecture Notes in Math.},
      volume={1339},
      publisher={Springer, Berlin},
   },
   date={1988},
   pages={78--85},
   review={\MR{961474 (90a:53042)}},
   doi={10.1007/BFb0083048},
}

\bib{buser-karcher}{book}{
   author={Buser, Peter},
   author={Karcher, Hermann},
   title={Gromov's almost flat manifolds},
   series={Ast\'erisque},
   volume={81},
   publisher={Soci\'et\'e Math\'ematique de France, Paris},
   date={1981},
   pages={148},
   %review={\MR{619537 (83m:53070)}},
}


\bib{bbi}{book}{
   author={Burago, Dmitri},
   author={Burago, Yuri},
   author={Ivanov, Sergei},
   title={A course in metric geometry},
   series={Graduate Studies in Mathematics},
   volume={33},
   publisher={American Mathematical Society, Providence, RI},
   date={2001},
   pages={xiv+415},
   %isbn={0-8218-2129-6},
   %review={\MR{1835418 (2002e:53053)}},
}

\bib{burago-zalgaller}{book}{
   author={Burago, Yu. D.},
   author={Zalgaller, V. A.},
   title={Geometric inequalities},
   series={Grundlehren der Mathematischen Wissenschaften [Fundamental
   Principles of Mathematical Sciences]},
   volume={285},
   %note={Translated from the Russian by A. B. Sosinski\u\i;
   %Springer Series in Soviet Mathematics},
   publisher={Springer-Verlag, Berlin},
   date={1988},
   %pages={xiv+331},
   %isbn={3-540-13615-0},
   %review={\MR{936419 (89b:52020)}},
   %doi={10.1007/978-3-662-07441-1},
}

\bib{BIS}{article}{
   author={Burago, D.},
   author={Ivanov, S.},
   author={Shoenthal, D.},
   title={Two counterexamples in low-dimensional length geometry},
   journal={Algebra i Analiz},
   volume={19},
   date={2007},
   number={1},
   pages={46--59},
   %issn={0234-0852},
   %translation={
      %journal={St. Petersburg Math. J.},
      %volume={19},
      %date={2008},
      %number={1},
      %pages={33--43},
      %issn={1061-0022},
  % },
   %review={\MR{2319509 (2008b:53101)}},
   %doi={10.1090/S1061-0022-07-00984-3},
}

\bib{burago-kleiner}{article}{
   author={Burago, D.},
   author={Kleiner, B.},
   title={Rectifying separated nets},
   journal={Geom. Funct. Anal.},
   volume={12},
   date={2002},
   number={1},
   pages={80--92},
   %issn={1016-443X},
   %review={\MR{1904558 (2003h:26020)}},
   %doi={10.1007/s00039-002-8238-8},
}

\bib{buyalo}{article}{
   author={Buyalo, S. V.},
   title={Volume and fundamental group of a manifold of nonpositive
   curvature},
   %language={Russian},
   journal={Mat. Sb. (N.S.)},
   volume={122(164)},
   date={1983},
   number={2},
   pages={142--156},
   %issn={0368-8666},
   %review={\MR{717671 (86a:53043)}},
}

\bib{zeb}{misc}{    
    title={Is it possible to capture a sphere in a knot?},    
    author={Zarathustra Elessar Brady},    
    note={Question 8091},    
    eprint={http://mathoverflow.net/q/8091},    
    organization={MathOverflow}  
}

\bib{calka}{article}{
   author={Ca{\l}ka, Aleksander},
   title={On conditions under which isometries have bounded orbits},
   journal={Colloq. Math.},
   volume={48},
   date={1984},
   number={2},
   pages={219--227},
   %issn={0010-1354},
   %review={\MR{758530 (85m:54027)}},
}

\bib{cheeger}{article}{
   author={Cheeger, Jeff},
   title={Some examples of manifolds of nonnegative curvature},
   journal={J. Differential Geometry},
   volume={8},
   date={1973},
   pages={623--628},
   %issn={0022-040X},
   %review={\MR{0341334 (49 \#6085)}},
}

\bib{cheeger-colding}{article}{
   author={Cheeger, Jeff},
   author={Colding, Tobias H.},
   title={Lower bounds on Ricci curvature and the almost rigidity of warped
   products},
   journal={Ann. of Math. (2)},
   volume={144},
   date={1996},
   number={1},
   pages={189--237},
   %issn={0003-486X},
   %review={\MR{1405949 (97h:53038)}},
   %doi={10.2307/2118589},
}

\bib{croke}{article}{
   author={Croke, Christopher B.},
   title={Small volume on big $n$-spheres},
   journal={Proc. Amer. Math. Soc.},
   volume={136},
   date={2008},
   number={2},
   pages={715--717 %(electronic)
   },
   %issn={0002-9939},
   %review={\MR{2358513 (2008h:53044)}},
   %doi={10.1090/S0002-9939-07-09079-X},
}

\bib{edelstein-schwatz}{article}{
   author={Edelstein, Michael},
   author={Schwarz, Binyamin},
   title={On the length of linked curves},
   journal={Israel J. Math.},
   volume={23},
   date={1976},
   number={1},
   pages={94--95},
   %issn={0021-2172},
   %review={\MR{0397558 (53 \#1417)}},
}

\bib{efimov}{article}{
   author={Efimov, N. V.},
   title={Qualitative problems of the theory of deformation of surfaces},
   %language={Russian},
   journal={Uspehi Matem. Nauk (N.S.)},
   volume={3},
   date={1948},
   number={2(24)},
   pages={47--158},
   %issn={0042-1316},
   %review={\MR{0027567 (10,324a)}},
}

\bib{eschenburg-spaces}{article}{
   author={Eschenburg, J.-H.},
   title={New examples of manifolds with strictly positive curvature},
   journal={Invent. Math.},
   volume={66},
   date={1982},
   number={3},
   pages={469--480},
   %issn={0020-9910},
   %review={\MR{662603 (83i:53061)}},
   %doi={10.1007/BF01389224},
}

\bib{eschenburg}{article}{
   author={Eschenburg, J.-H.},
   title={Local convexity and nonnegative curvature---Gromov's proof of the
   sphere theorem},
   journal={Invent. Math.},
   volume={84},
   date={1986},
   number={3},
   pages={507--522},
   %issn={0020-9910},
   %review={\MR{837525 (87j:53080)}},
   %doi={10.1007/BF01388744},
}


\bib{falconer}{book}{
   author={Falconer, K. J.},
   title={The geometry of fractal sets},
   series={Cambridge Tracts in Mathematics},
   volume={85},
   publisher={Cambridge University Press, Cambridge},
   date={1986},
   %pages={xiv+162},
   %isbn={0-521-25694-1},
   %isbn={0-521-33705-4},
   %review={\MR{867284 (88d:28001)}},
}

\bib{FMR}{article}{
   author={Fang, Fuquan},
   author={Mendon{\c{c}}a, S{\'e}rgio},
   author={Rong, Xiaochun},
   title={A connectedness principle in the geometry of positive curvature},
   journal={Comm. Anal. Geom.},
   volume={13},
   date={2005},
   number={4},
   pages={671--695},
   %issn={1019-8385},
   %review={\MR{2191903 (2007h:53049)}},
}

\bib{fary}{article}{
   author={F{\'a}ry, Istv{\'a}n},
   title={Sur certaines in\'egalites g\'eom\'etriques},
   %language={French},
   journal={Acta Sci. Math. Szeged},
   volume={12},
   date={1950},
   number={Leopoldo Fejer et Frederico Riesz LXX annos natis dedicatus, Pars
   A},
   pages={117--124},
   %issn={0001-6969},
   %review={\MR{0038090 (12,353b)}},
}

\bib{ferry-okun}{article}{
   author={Ferry, Steven C.},
   author={Okun, Boris L.},
   title={Approximating topological metrics by Riemannian metrics},
   journal={Proc. Amer. Math. Soc.},
   volume={123},
   date={1995},
   number={6},
   pages={1865--1872},
   %issn={0002-9939},
   %review={\MR{1246524 (95g:53050)}},
   %doi={10.2307/2161004},
}

\bib{frankel-katz}{article}{
   author={Frankel, S.},
   author={Katz, M.},
   title={The Morse landscape of a Riemannian disk},
   %language={English, with English and French summaries},
   journal={Ann. Inst. Fourier (Grenoble)},
   volume={43},
   date={1993},
   number={2},
   pages={503--507},
   %issn={0373-0956},
   %review={\MR{1220281 (94c:53057)}},
}

\bib{frankel}{article}{
   author={Frankel, T.},
   title={On the fundamental group of a compact minimal submanifold},
   journal={Ann. of Math. (2)},
   volume={83},
   date={1966},
   pages={68--73},
   %issn={0003-486X},
   %review={\MR{0187183 (32 \#4637)}},
}

\bib{frechet}{article}{
Author = {M. Fr\'echet},
Title = {Les ensembles abstraits et le calcul fonctionnel.},
Journal = {Rendiconti del Circolo Matematico di Palermo},
   % Journal = {{Rend. Circ. Mat. Palermo}},
    %ISSN = {0009-725X; 1973-4409/e},
    Volume = {30},
    Pages = {1--26},
    Year = {1910},
    Publisher = {Circolo Matematico, Palermo},
    %Language = {French},
    %DOI = {10.1007/BF03014860},
    %Zbl = {41.0103.01}
}

\bib{galperin}{article}{
   author={Galperin, G.},
   title={Convex polyhedra without simple closed geodesics},
   journal={Regul. Chaotic Dyn.},
   volume={8},
   date={2003},
   number={1},
   pages={45--58},
   %issn={1560-3547},
   %review={\MR{1963967 (2004c:37085)}},
   %doi={10.1070/RD2003v008n01ABEH000231},
}

\bib{GKZ}{book}{
   author={Gelfand, I. M.},
   author={Kapranov, M. M.},
   author={Zelevinsky, A. V.},
   title={Discriminants, resultants and multidimensional determinants},
   series={Modern Birkh\"auser Classics},
   %note={Reprint of the 1994 edition},
   %publisher={Birkh\"auser Boston, Inc., Boston, MA},
   date={2008},
   %pages={x+523},
   %isbn={978-0-8176-4770-4},
   %review={\MR{2394437 (2009a:14065)}},
}

\bib{ginzburg}{article}{
   author={Ginzburg, Viktor L.},
   title={On the existence and non-existence of closed trajectories for some
   Hamiltonian flows},
   journal={Math. Z.},
   volume={223},
   date={1996},
   number={3},
   pages={397--409},
   %issn={0025-5874},
   %review={\MR{1417851 (97i:58144)}},
   %doi={10.1007/PL00004565},
}

\bib{greene-wu}{article}{
   author={Greene, R. E.},
   author={Wu, H.},
   title={On the subharmonicity and plurisubharmonicity of geodesically
   convex functions},
   journal={Indiana Univ. Math. J.},
   volume={22},
   date={1972/73},
   pages={641--653},
   %issn={0022-2518},
   %review={\MR{0422686 (54 \#10672)}},
}

\bib{GKM}{book}{
   author={Gromoll, D.},
   author={Klingenberg, W.},
   author={Meyer, W.},
   title={Riemannsche Geometrie im Grossen},
   %language={German},
   series={Lecture Notes in Mathematics, No. 55},
   publisher={Springer-Verlag, Berlin-New York},
   date={1968},
   pages={vi+287},
   %review={\MR{0229177 (37 \#4751)}},
}

\bib{gromoll-meyer}{article}{
   author={Gromoll, Detlef},
   author={Meyer, Wolfgang},
   title={An exotic sphere with nonnegative sectional curvature},
   journal={Ann. of Math. (2)},
   volume={100},
   date={1974},
   pages={401--406},
   %issn={0003-486X},
   %review={\MR{0375151 (51 \#11347)}},
}

\bib{gromov-almost-flat}{article}{
   author={Gromov, Mikhael},
   title={Almost flat manifolds},
   journal={J. Differential Geom.},
   volume={13},
   date={1978},
   number={2},
   pages={231--241},
   %issn={0022-040X},
   %review={\MR{540942 (80h:53041)}},
}

\bib{gromov-hyperbolic}{article}{
   author={Gromov, Mikhael},
   title={Hyperbolic manifolds, groups and actions},
   conference={
      title={Riemann surfaces and related topics: Proceedings of the 1978
      Stony Brook Conference},
      address={State Univ. New York, Stony Brook, N.Y.},
      date={1978},
   },
   book={
      series={Ann. of Math. Stud.},
      volume={97},
      publisher={Princeton Univ. Press, Princeton, N.J.},
   },
   date={1981},
   pages={183--213},
  % review={\MR{624814 (82m:53035)}},
}

\bib{gromov-filling}{article}{
   author={Gromov, Mikhael},
   title={Filling Riemannian manifolds},
   journal={J. Differential Geom.},
   volume={18},
   date={1983},
   number={1},
   pages={1--147},
   %issn={0022-040X},
   %review={\MR{697984 (85h:53029)}},
}

\bib{gromov-pseudoholomorphic}{article}{
   author={Gromov, Mikhael},
   title={Pseudoholomorphic curves in symplectic manifolds},
   journal={Invent. Math.},
   volume={82},
   date={1985},
   number={2},
   pages={307--347},
   %issn={0020-9910},
   %review={\MR{809718 (87j:53053)}},
   %doi={10.1007/BF01388806},
}

\bib{gromov-PDR}{book}{
   author={Gromov, Mikhael},
   title={Partial differential relations},
   series={Ergebnisse der Mathematik und ihrer Grenzgebiete (3) [Results in
   Mathematics and Related Areas (3)]},
   volume={9},
   %publisher={Springer-Verlag, Berlin},
   date={1986},
   pages={x+363},
   %isbn={3-540-12177-3},
   %review={\MR{864505 (90a:58201)}},
   %doi={10.1007/978-3-662-02267-2},
}

\bib{gromov-SGMC}{article}{
   author={Gromov, Mikhael},
   title={Sign and geometric meaning of curvature},
   %language={English, with English and Italian summaries},
   journal={Rend. Sem. Mat. Fis. Milano},
   volume={61},
   date={1991},
   pages={9--123},
   %issn={0370-7377},
   %review={\MR{1297501 (95j:53055)}},
   %doi={10.1007/BF02925201},
}

\bib{gromov-MetStr}{book}{
   author={Gromov, Mikhael},
   title={Metric structures for Riemannian and non-Riemannian spaces},
   series={Modern Birkh\"auser Classics},
   %edition={Reprint of the 2001 English edition},
   %note={Based on the 1981 French original; With appendices by M. Katz, P. Pansu and S. Semmes; Translated from the French by Sean Michael Bates},
   %publisher={Birkh\"auser Boston, Inc., Boston, MA},
   date={2007},
   %pages={xx+585},
   %isbn={978-0-8176-4582-3},
   %isbn={0-8176-4582-9},
   %review={\MR{2307192 (2007k:53049)}},
}

\bib{grove}{article}{
   author={Grove, Karsten},
   title={Geometry of, and via, symmetries},
   conference={
      title={Conformal, Riemannian and Lagrangian geometry (Knoxville, TN,
      2000)},
   },
   book={
      series={Univ. Lecture Ser.},
      volume={27},
      %publisher={Amer. Math. Soc., Providence, RI},
   },
   date={2002},
   pages={31--53},
   %review={\MR{1922721 (2003g:53046)}},
}

\bib{grove-wilking}{article}{
   author={Grove, Karsten},
   author={Wilking, Burkhard},
   title={A knot characterization and 1--connected nonnegatively curved
   4--manifolds with circle symmetry},
   journal={Geom. Topol.},
   volume={18},
   date={2014},
   number={5},
   pages={3091--3110},
   %issn={1465-3060},
   %review={\MR{3285230}},
   %doi={10.2140/gt.2014.18.3091},
}

\bib{guth-symplectic}{article}{
   author={Guth, Larry},
   title={Symplectic embeddings of polydisks},
   journal={Invent. Math.},
   volume={172},
   date={2008},
   number={3},
   pages={477--489},
   %issn={0020-9910},
   %review={\MR{2393077 (2008k:53195)}},
   %doi={10.1007/s00222-007-0103-9},
}

\bib{guzhvina}{article}{
   author={G.~Guzhvina},
   title={Gromov's pinching constant},
   journal={\tt arXiv:0804.0201 [math.DG]},
}

\bib{hang-wang}{article}{
   author={Hang, Fengbo},
   author={Wang, Xiaodong},
   title={Rigidity theorems for compact manifolds with boundary and positive
   Ricci curvature},
   journal={J. Geom. Anal.},
   volume={19},
   date={2009},
   number={3},
   pages={628--642},
   %issn={1050-6926},
   %review={\MR{2496569 (2010k:53065)}},
   %doi={10.1007/s12220-009-9074-y},
}

\bib{halberg-levin-straus}{article}{
   author={Halberg, Charles J. A., Jr.},
   author={Levin, Eugene},
   author={Straus, E. G.},
   title={On contiguous congruent sets in Euclidean space},
   journal={Proc. Amer. Math. Soc.},
   volume={10},
   date={1959},
   pages={335--344},
   %issn={0002-9939},
   %review={\MR{0106438 (21 \#5170)}},
}

\bib{hamkins}{misc}{    
    title={Is the sphere the only surface all of whose projections are circles? Or: Can we deduce a spherical Earth by observing that its shadows on the Moon are always circular?},    
    author={Joel David Hamkins},    
    note={Question 39127},    
    eprint={http://mathoverflow.net/q/39127},    
    organization={MathOverflow}  
}

\bib{hilbert}{article}{
   author={Hilbert, David},
   title={Ueber die gerade Linie als k\"urzeste Verbindung zweier Punkte},
   %language={German},
   journal={Math. Ann.},
   volume={46},
   date={1895},
   number={1},
   pages={91--96},
   %issn={1432-1807},
   %doi={10.1007/BF02096204},
}

\bib{hilbert-problems}{article}{
   author={Hilbert, David},
   title={Mathematical problems},
   journal={Bull. Amer. Math. Soc.},
   volume={8},
   date={1902},
   number={10},
   pages={437--479},
   %issn={0002-9904},
   %review={\MR{1557926}},
   %doi={10.1090/S0002-9904-1902-00923-3},
}

\bib{honda}{article}{
   author={Honda, Ko},
   title={Transversality theorems for harmonic forms},
   journal={Rocky Mountain J. Math.},
   volume={34},
   date={2004},
   number={2},
   pages={629--664},
   %issn={0035-7596},
   %review={\MR{2072799 (2005f:58052)}},
   %doi={10.1216/rmjm/1181069872},
}

\bib{hsiang-kleiner}{article}{
   author={Hsiang, Wu-Yi},
   author={Kleiner, Bruce},
   title={On the topology of positively curved $4$-manifolds with symmetry},
   journal={J. Differential Geom.},
   volume={29},
   date={1989},
   number={3},
   pages={615--621},
   %issn={0022-040X},
   %review={\MR{992332 (90e:53053)}},
}

\bib{imre-kuiperberg-zamfirescu}{article}{
   author={B{\'a}r{\'a}ny, Imre},
   author={Kuperberg, Krystyna},
   author={Zamfirescu, Tudor},
   title={Total curvature and spiralling shortest paths},
   %note={U.S.-Hungarian Workshops on Discrete Geometry and Convexity
   %(Budapest, 1999/Auburn, AL, 2000)},
   journal={Discrete Comput. Geom.},
   volume={30},
   date={2003},
   number={2},
   pages={167--176},
   %issn={0179-5376},
   %review={\MR{2007957 (2004h:52009)}},
   %doi={10.1007/s00454-003-0001-z},
}

\bib{izmestiev-rote-springborn-kusner}{article}{
   author={Izmestiev, Ivan},
   author={Kusner, Robert B.},
   author={Rote, G{\"u}nter},
   author={Springborn, Boris},
   author={Sullivan, John M.},
   title={There is no triangulation of the torus with vertex degrees
   $5,6,\dots,6,7$ and related results: geometric proofs for combinatorial
   theorems},
   journal={Geom. Dedicata},
   volume={166},
   date={2013},
   pages={15--29},
   %issn={0046-5755},
   %review={\MR{3101158}},
   %doi={10.1007/s10711-012-9782-5},
}

\bib{jendrol-jucovich}{article}{
   author={Jendrol{\soft{l}}, S.},
   author={Jucovi{\v{c}}, E.},
   title={On the toroidal analogue of Eberhard's theorem},
   journal={Proc. London Math. Soc. (3)},
   volume={25},
   date={1972},
   pages={385--398},
   %issn={0024-6115},
   %review={\MR{0307968 (46 \#7083)}},
}

\bib{kirby}{article}{
   author={Kirby, Robion C.},
   title={Stable homeomorphisms and the annulus conjecture},
   journal={Ann. of Math. (2)},
   volume={89},
   date={1969},
   pages={575--582},
   %issn={0003-486X},
   %review={\MR{0242165 (39 \#3499)}},
}

\bib{kirszbraun}{article}{
    author = {M. D. Kirszbraun},
    title = {\"Uber die zusammenziehenden und Lipschitzschen Transformationen.},
    journal = {Fundam. Math.},
    %ISSN = {0016-2736; 1730-6329/e},
    volume = {22},
    pages = {77--108},
    date = {1934},
    publisher = {Polish Academy of Sciences (Polska Akademia Nauk - PAN), Institute of Mathematics (Instytut Matematyczny), Warsaw},
    %language = {German},
    %Zbl = {0009.03904}
}

\bib{klee}{article}{
   author={Klee, V. L., Jr.},
   title={Some topological properties of convex sets},
   journal={Trans. Amer. Math. Soc.},
   volume={78},
   date={1955},
   pages={30--45},
   %issn={0002-9947},
   %review={\MR{0069388 (16,1030c)}},
}

\bib{knaster}{article}{
author = {B. Knaster},
title = {Un continu dont tout sous-continu est ind\'ecomposable.},
journal = {Fundamenta Mathematicae},
volume = {3},
pages = {247--286},
year = {1922},
publisher = {Polish Academy of Sciences (Polska Akademia Nauk - PAN), Institute of Mathematics (Instytut Matematyczny), Warsaw}
    %Zbl = {48.0212.01}
}

\bib{kneser}{article}{
   author={Kneser, A.},
   title={Bemerkungen \"uber die Anzahl der Extreme der Kr\"ummung
auf geschlossenen Kurven und \"uber vertwandte Fragen in einer nichteuklidischen
Geometrie},
   journal={Festschrift H. Weber},
   date={1912},
   pages={170--180},
}

\bib{krat}{thesis}
{author={S.~Krat},
title={Approximation Problems in Length Geometry},
address={Pennsylvania State University},
date={2005},
type={Ph.D. thesis},
}

\bib{kuiper}{article}{
   author={Kuiper, Nicolaas H.},
   title={On $C^1$-isometric imbeddings. I, II},
   journal={Nederl. Akad. Wetensch. Proc. Ser. A. {\bf 58} = Indag. Math.},
   volume={17},
   date={1955},
   pages={545--556, 683--689},
   %review={\MR{0075640 (17,782c)}},
}

\bib{kuratowski}{article}{
    Author = {Kuratowski, Casimir},
    Title = {Quelques probl\`emes concernant les espaces m\'etriques nonseparables.},
    Journal = {Fundamenta Mathematicae},
    %Journal = {{Fundam. Math.}},
    %ISSN = {0016-2736; 1730-6329/e},
    Volume = {25},
    Pages = {534--545},
    Year = {1935},
    Publisher = {Polish Academy of Sciences (Polska Akademia Nauk - PAN), Institute of Mathematics (Instytut Matematyczny), Warsaw}
    %Language = {French},
   % Zbl = {0012.32103}
}

\bib{kwun}{article}{
   author={Kwun, Kyung Whan},
   title={Uniqueness of the open cone neighborhood},
   journal={Proc. Amer. Math. Soc.},
   volume={15},
   date={1964},
   pages={476--479},
   %issn={0002-9939},
   %review={\MR{0161319 (28 \#4527)}},
}

\bib{montgomery}{article}{
   author={Montgomery, Deane},
   title={Pointwise Periodic Homeomorphisms},
   journal={Amer. J. Math.},
   volume={59},
   date={1937},
   number={1},
   pages={118--120},
   issn={0002-9327},
   review={\MR{1507223}},
   doi={10.2307/2371565},
}


\bib{newman}{article}{
author = {M. H. A. Newman},
title = {A theorem on periodic transformations of spaces.},
journal = {The Quarterly Journal of Mathematics. Oxford Series},
 %   Journal = {{Q. J. Math., Oxf. Ser.}},
    %ISSN = {0033-5606},
    Volume = {2},
    Pages = {1--8},
    Year = {1931},
    Publisher = {Oxford University Press, Oxford}
    %Language = {English},
    %Zbl = {57.0496.02}
}

\bib{lagarias-richardson}{article}{
   author={Lagarias, Jeffrey C.},
   author={Richardson, Thomas J.},
   title={Convexity and the average curvature of plane curves},
   journal={Geom. Dedicata},
   volume={67},
   date={1997},
   number={1},
   pages={1--30},
   %issn={0046-5755},
   %review={\MR{1468858 (98f:52007)}},
   %doi={10.1023/A:1004912521664},
}

\bib{lange}{article}{
author={Lange, Christian},
title={When is the underlying space of an orbifold a topological manifold},
journal={\tt arXiv:1307.4875 [math.GN]},
}

\bib{lebedeva-petrunin}{article}
{author={Lebedeva, N.},
author={Petrunin, A.},
title={Local characterization of polyhedral spaces},
journal={Geometriae Dedicata (to appear) \tt arXiv:1402.6670 [math.DG]}
} 

\bib{le-donne}{article}{
   author={Le Donne, Enrico},
   title={Lipschitz and path isometric embeddings of metric spaces},
   journal={Geom. Dedicata},
   volume={166},
   date={2013},
   pages={47--66},
   %issn={0046-5755},
   %review={\MR{3101160}},
   %doi={10.1007/s10711-012-9785-2},
}


\bib{liberman}{article}{
   author={Liberman, J.},
   title={Geodesic lines on convex surfaces},
   journal={C. R. (Doklady) Acad. Sci. URSS (N.S.)},
   volume={32},
   date={1941},
   pages={310--313},
   %review={\MR{0010994 (6,100g)}},
}

\bib{lobachevsky}{book}{
author={Lobachevsky, N. I.}, 
title={Geometrische Untersuchungen zur Theorie der Parallellinien},
publisher={Berlin: F. Fincke},
date={1840}
}

\bib{matveyev}{article}
{author={Rostislav Matveev},
title={Surfaces with polyhedral metrics},
conference={
      title={International Mathematical Summer School for Students 2011},
      address={Jacobs University, Bremen},
   },
}

\bib{micallef-moore}{article}{
   author={Micallef, Mario J.},
   author={Moore, John Douglas},
   title={Minimal two-spheres and the topology of manifolds with positive
   curvature on totally isotropic two-planes},
   journal={Ann. of Math. (2)},
   volume={127},
   date={1988},
   number={1},
   pages={199--227},
   %issn={0003-486X},
   %review={\MR{924677 (89e:53088)}},
   %doi={10.2307/1971420},
}

\bib{mikhailova}{article}{
   author={Mikha{\u\i}lova, M. A.},
   title={A factor space with respect to the action of a finite group
   generated by pseudoreflections},
   %language={Russian},
   journal={Izv. Akad. Nauk SSSR Ser. Mat.},
   volume={48},
   date={1984},
   number={1},
   pages={104--126},
   %issn={0373-2436},
   %review={\MR{733360 (85k:20127)}},
}

\bib{milka-poly}{article}{
   author={Milka, A. D.},
   title={Multidimensional spaces with polyhedral metric of nonnegative
   curvature. I},
   %language={Russian},
   journal={Ukrain. Geometr. Sb. Vyp.},
   volume={5--6},
   date={1968},
   pages={103--114},
   %review={\MR{0286057 (44 \#3273)}},
}

\bib{milka-geod}{article}{
   author={Milka, A. D.},
   title={Shortest lines on convex surfaces},
   %language={Russian},
   journal={Dokl. Akad. Nauk SSSR},
   volume={248},
   date={1979},
   number={1},
   pages={34--36},
   %issn={0002-3264},
   %review={\MR{549365 (81c:53056)}},
}

\bib{gromov-apendix}{book}{
   author={Milman, Vitali D.},
   author={Schechtman, Gideon},
   title={Asymptotic theory of finite-dimensional normed spaces},
   series={Lecture Notes in Mathematics},
   volume={1200},
   note={With an appendix by M. Gromov},
   publisher={Springer-Verlag, Berlin},
   date={1986},
   pages={viii+156},
   %isbn={3-540-16769-2},
   %review={\MR{856576 (87m:46038)}},
}

\bib{hall}{article}{
   author={Hall, Marshall, Jr.},
   title={Subgroups of finite index in free groups},
   journal={Canadian J. Math.},
   volume={1},
   date={1949},
   pages={187--190},
   %issn={0008-414X},
   %review={\MR{0028836 (10,506a)}},
}

\bib{wayne}{article}{
   author={Lewis, Wayne},
   title={The pseudo-arc},
   journal={Bol. Soc. Mat. Mexicana (3)},
   volume={5},
   date={1999},
   number={1},
   pages={25--77},
   %issn={1405-213X},
   %review={\MR{1692467 (2000f:54029)}},
}

\bib{lohkamp}{article}{
   author={Lohkamp, Joachim},
   title={Metrics of negative Ricci curvature},
   journal={Ann. of Math. (2)},
   volume={140},
   date={1994},
   number={3},
   pages={655--683},
   %issn={0003-486X},
   %review={\MR{1307899 (95i:53042)}},
   %doi={10.2307/2118620},
}

\bib{mazur-ulam}{article}{
   author={Mazur, S.},
   author={Ulam, S.},
   title={Sur les transformations isom\'etriques d'espaces vectoriels norm\'es},
   %language={French},
   journal={C. R. Acad. Sci., Paris},
   volume={194},
   date={1932},
   number={1},
   pages={946--948},
   %issn={0026-9255},
   %review={\MR{1550413}},
   %doi={10.1007/BF01733278},
}

\bib{monsky}{article}{
   author={Monsky, Paul},
   title={On dividing a square into triangles},
   journal={Amer. Math. Monthly},
   volume={77},
   date={1970},
   pages={161--164},
   issn={0002-9890},
   review={\MR{0252233 (40 \#5454)}},
}

\bib{nabutovsky-rotman}{article}{
   author={Nabutovsky, Alexander},
   author={Rotman, Regina},
   title={Length of geodesics and quantitative Morse theory on loop spaces},
   journal={Geom. Funct. Anal.},
   volume={23},
   date={2013},
   number={1},
   pages={367--414},
   %issn={1016-443X},
   %review={\MR{3037903}},
   %doi={10.1007/s00039-012-0207-2},
}

\bib{nash}{article}{
   author={Nash, John},
   title={$C^1$ isometric imbeddings},
   journal={Ann. of Math. (2)},
   volume={60},
   date={1954},
   pages={383--396},
   %issn={0003-486X},
   %review={\MR{0065993 (16,515e)}},
}

\bib{osgood}{article}{
   author={Osgood, William F.},
   title={A Jordan curve of positive area},
   journal={Trans. Amer. Math. Soc.},
   volume={4},
   date={1903},
   number={1},
   pages={107--112},
   issn={0002-9947},
   review={\MR{1500628}},
   doi={10.2307/1986455},
}

\bib{ovsienko-tabachnikov}{book}{
   author={Ovsienko, V.},
   author={Tabachnikov, S.},
   title={Projective differential geometry old and new},
   series={Cambridge Tracts in Mathematics},
   volume={165},
   %note={From the Schwarzian derivative to the cohomology of diffeomorphism groups},
   publisher={Cambridge University Press, Cambridge},
   date={2005},
   %pages={xii+249},
   %isbn={0-521-83186-5},
   %review={\MR{2177471 (2007b:53017)}},
}

\bib{rourke}{misc}{    
    title={Why is the half-torus rigid?},    
    author={Joseph O'Rourke},    
    note={Question 77760},    
    eprint={http://mathoverflow.net/q/77760},    
    organization={MathOverflow}  
}

\bib{panov-Kaeler}{article}{
   author={Panov, Dmitri},
   title={Polyhedral K\"ahler manifolds},
   journal={Geom. Topol.},
   volume={13},
   date={2009},
   number={4},
   pages={2205--2252},
   %issn={1465-3060},
   %review={\MR{2507118 (2010f:53129)}},
   %doi={10.2140/gt.2009.13.2205},
}

\bib{panov-torus}{article}{
   author={Panov, Dmitri},
   title={Foliations with unbounded deviation on $\mathbb{T}^2$},
   journal={J. Mod. Dyn.},
   volume={3},
   date={2009},
   number={4},
   pages={589--594},
   %issn={1930-5311},
   %review={\MR{2587087 (2011d:37046)}},
   %doi={10.3934/jmd.2009.3.589},
}

\bib{panov-curves}{article}{
   author={Panov, Dmitri},
   title={Parabolic curves and gradient mappings},
      journal={Proc. Steklov Inst. Math.},
      date={1998},
      number={2 (221)},
      pages={261--278},
      issn={0081-5438},
   %review={\MR{1683700 (2000c:53085)}},
}

\bib{panov-petrunin-telescopic}{article}{
   author={Panov, Dmitri},
   author={Petrunin, Anton},
   title={Telescopic actions},
   journal={Geom. Funct. Anal.},
   volume={22},
   date={2012},
   number={6},
   pages={1814--1831},
   %issn={1016-443X},
   %review={\MR{3000502}},
   %doi={10.1007/s00039-012-0194-3},
}


\bib{panov-petrunin}{article}{
   author={Panov, Dmitri},
   author={Petrunin, Anton},
   title={Sweeping out sectional curvature},
   journal={Geom. Topol.},
   volume={18},
   date={2014},
   number={2},
   pages={617--631},
   %issn={1465-3060},
   %review={\MR{3159972}},
   %doi={10.2140/gt.2014.18.617},
}

\bib{panov-petrunin-ramification}{article}{
   author={Panov, Dmitri},
   author={Petrunin, Anton},
   title={Ramification conjecture},
   journal={{\tt arXiv:1312.6856 [math.GT]}},
}

\bib{perelman}{article}{
   author={Perelman, G.},
   title={Proof of the soul conjecture of Cheeger and Gromoll},
   journal={J. Differential Geom.},
   volume={40},
   date={1994},
   number={1},
   pages={209--212},
   %issn={0022-040X},
   %review={\MR{1285534 (95d:53037)}},
}

\bib{two-discs}{misc}{    
title={Two discs with no parallel tangent planes},    
author={Petrunin, Anton},    
note={Question 17486}, 
eprint={http://mathoverflow.net/q/17486},    
organization={MathOverflow}
}

\bib{One-step problems in geometry}{misc}{    
    title={One-step problems in geometry},    
    author={Petrunin, Anton},    
    note={Question 8247},    
    eprint={http://mathoverflow.net/q/8247},    
    organization={MathOverflow}  
}

\bib{ivanov}{misc}{    
    title={Diameter of m-fold cover},    
    author={Petrunin, Anton},    
    note={Question 7732},    
    eprint={http://mathoverflow.net/q/7732},    
    organization={MathOverflow}  
}

\bib{petrunin-paths}{article}{
   author={Petrunin, Anton},
   title={Intrinsic isometries in Euclidean space},
      journal={St. Petersburg Math. J.},
      volume={22},
      date={2011},
      number={5},
      pages={803--812},
      %issn={1061-0022},
   %review={\MR{2828830 (2012e:54037)}},
   %doi={10.1090/S1061-0022-2011-01169-0},
}

\bib{petrunin-monthly}{article}{
   author={Petrunin, Anton},
   title={Area Minimizing Polyhedral Surfaces are Saddle},
   journal={Amer. Math. Monthly},
   volume={122},
   date={2015},
   number={3},
   pages={264--267},
   %issn={0002-9890},
   %review={\MR{3327716}},
   %doi={10.4169/amer.math.monthly.122.03.264},
}

\bib{petrunin-yashinsky}{article}{
   author={Petrunin, Anton},
   author={Yashinski, Allan},
   title={Piecewise distance preserving maps},
   journal={to appear in St. Petersburg Mathematical Journal.},
}

\bib{pogorelov}{book}{
   author={Pogorelov, Aleksei},
   title={Hilbert's fourth problem},
   %note={Translated by Richard A. Silverman;
   %Scripta Series in Mathematics},
   publisher={V. H. Winston \& Sons, Washington, D.C.; A Halsted Press Book,
   John Wiley \& Sons, New York-Toronto, Ont.-London},
   date={1979},
   pages={vi+97},
   %isbn={0-470-26735-6},
   %review={\MR{550440 (80j:53066)}},
}

\bib{poincare}{article}{
author={H. Poincar\'e},
title={Sur un th\'{e}or\`{e}me de g\'{e}ometrie}, 
journal={Rend. Circ. Mat. Palermo}, 
volume={33},
date={1912},
pages={75--407}
}

\bib{three-discs}{misc}{    
    title={A generalization of Cauchy's mean value theorem},    
    author={Petya Pushkar},    
    note={Question 16335},    
    eprint={http://mathoverflow.net/q/16335},    
    organization={MathOverflow}  
}

\bib{rademacher}{article}{
   author={Rademacher, Hans},
   title={\"Uber partielle und totale differenzierbarkeit von Funktionen
   mehrerer Variabeln und \"uber die Transformation der Doppelintegrale},
   language={German},
   journal={Math. Ann.},
   volume={79},
   date={1919},
   number={4},
   pages={340--359},
   issn={0025-5831},
   review={\MR{1511935}},
   doi={10.1007/BF01498415},
}

\bib{rembs}{article}{
    author = {Eduard Rembs},
    title = {Verbiegungen h\"oherer Ordnung und ebene Fl\"achenrinnen.},
    %FJournal = {Mathematische Zeitschrift},
    journal = {Math. Z.},
    %ISSN = {0025-5874; 1432-1823/e},
    volume = {36},
    pages = {110--121},
    date = {1932},
    publisher = {Springer, Berlin/Heidelberg},
    %Language = {German},
    %DOI = {10.1007/BF01188611},
    %Zbl = {0005.02401}
}

\bib{ruh}{article}{
   author={Ruh, Ernst A.},
   title={Almost flat manifolds},
   journal={J. Differential Geom.},
   volume={17},
   date={1982},
   number={1},
   pages={1--14},
   %issn={0022-040X},
   %review={\MR{658470 (84a:53047)}},
}

\bib{sabitov}{article}{
   author={Sabitov, I. H.},
   title={Infinitesimal bendings of troughs of revolution. I},
   %language={Russian},
   journal={Mat. Sb. (N.S.)},
   volume={98(140)},
   date={1975},
   number={1 (9)},
   pages={113--129, 159},
   %review={\MR{0405299 (53 \#9093)}},
}

\bib{sahovic}{thesis}
{author={V.~Sahovic},
title={Approximations of Riemannian Manifolds with Linear Curvature Constraints},
address={Westf\"alische Wilhelms-Universit\"at M\"unster},
date={2009},
type={Dissertation},
}

\bib{siebenmann}{article}{
   author={Siebenmann, L. C.},
   title={Deformation of homeomorphisms on stratified sets. I, II},
   journal={Comment. Math. Helv.},
   volume={47},
   date={1972},
   pages={123--136; 137--163},
   %issn={0010-2571},
   %review={\MR{0319207 (47 \#7752)}},
}

\bib{shen}{article}{
author={Shen, A.},
title={Unexpected proofs},
subtitle={Boxes in a Train},
pages={48--50},
journal={Math. Intelligencer},
volume={21},
date={1999},
number={3}
}

\bib{shor-van wyk}{article}{
   author={Shor, Peter W.},
   author={Van Wyk, Christopher J.},
   title={Detecting and decomposing self-overlapping curves},
   journal={Comput. Geom.},
   volume={2},
   date={1992},
   number={1},
   pages={31--50},
   %issn={0925-7721},
   %review={\MR{1186078 (93i:68183)}},
   %doi={10.1016/0925-7721(92)90019-O},
}

\bib{stallings}{article}{
   author={Stallings, John R.},
   title={Topology of finite graphs},
   journal={Invent. Math.},
   volume={71},
   date={1983},
   number={3},
   pages={551--565},
   %issn={0020-9910},
   %review={\MR{695906 (85m:05037a)}},
   %doi={10.1007/BF02095993},
}

\bib{sullivan}{article}{
   author={Sullivan, Dennis},
   title={Hyperbolic geometry and homeomorphisms},
   conference={
      title={Geometric topology (Proc. Georgia Topology Conf., Athens, Ga.,
      1977)},
   },
   book={
      %publisher={Academic Press, New York-London},
   },
   date={1979},
   pages={543--555},
   %review={\MR{537749 (81m:57012)}},
}

\bib{synge}{article}{
Author = {J. L. Synge},
Title = {On the connectivity of spaces of positive curvature},
Year = {1937},
%Language = {English},
journal = {C. R. Congr. internat. Math., Oslo 1936,},
volume={2},
date={1937},
pages={138--139}
    %Zbl = {63.0710.05}
}

\bib{tabachnikov}{article}{
   author={Tabachnikov, Serge},
   title={The tale of a geometric inequality},
   conference={
      title={MASS selecta},
   },
   book={
      publisher={Amer. Math. Soc., Providence, RI},
   },
   date={2003},
   pages={257--262},
   %review={\MR{2027183}},
}

\bib{tabachnikob-mi}{article}
{author={Tabachnikov, Serge},
title={Supporting cords of convex sets},
subtitle={Problem 91-2 in Mathematical Entertainments},
journal={Mathematical Intelligencer},
volume={13},
number={1},
pages={33},
date={Winter 1991} %???
}


\bib{tabacnikov=tan}{article}{
   author={Tabachnikov, Serge},
   journal={The American Mathematical Monthly},
   title={The (Un)equal Tangents Problem},
   volume={119},
   number={5},
   date={May 2012},
   pages={398--405},
   %review={\MR{2027183}},
}

\bib{tait}{article}{
   author={Tait, P.},%Peter Guthrie Tait
   title={Note on the circles of curvature of a plane curve},
   journal={Proc. Edinburgh Math. Soc.},
   volume={14},
   date={February 1895},
   pages={26},
}

\bib{toponogov}{article}{
   author={Toponogov, V. A.},
   title={Evaluation of the length of a closed geodesic on a convex surface},
   %language={Russian},
   journal={Dokl. Akad. Nauk SSSR},
   volume={124},
   date={1959},
   pages={282--284},
   %issn={0002-3264},
   %review={\MR{0102055 (21 \#850)}},
}

%\bib{weinstein-Fixed-pnt}{article}{
   %author={Weinstein, Alan},
   %title={A fixed point theorem for positively curved manifolds},
   %journal={J. Math. Mech.},
   %volume={18},
   %date={1968/1969},
   %pages={149--153},
   %review={\MR{0227894 (37 \#3478)}},
%}

\bib{vaisala}{article}{
   author={V{\"a}is{\"a}l{\"a}, Jussi},
   title={A proof of the Mazur-Ulam theorem},
   journal={Amer. Math. Monthly},
   volume={110},
   date={2003},
   number={7},
   pages={633--635},
   %issn={0002-9890},
   %review={\MR{2001155 (2004d:46021)}},
   %doi={10.2307/3647749},
}

\bib{valentine}{article}{
   author={Valentine, F. A.},
   title={On the extension of a vector function so as to preserve a
   Lipschitz condition},
   journal={Bull. Amer. Math. Soc.},
   volume={49},
   date={1943},
   %pages={100--108},
   %issn={0002-9904},
   %review={\MR{0008251 (4,269d)}},
}

\bib{vinberg-strong}{article}{
   author={Vinberg, {\`E}. B.},
   title={The nonexistence of crystallographic reflection groups in
   Lobachevski\u\i\ spaces of large dimension},
   %language={Russian},
   journal={Funktsional. Anal. i Prilozhen.},
   volume={15},
   date={1981},
   number={2},
   pages={67--68},
   %issn={0374-1990},
   %review={\MR{617472 (83d:51026)}},
}

\bib{vinberg}{article}{
   author={Vinberg, {\`E}. B.},
   title={Discrete groups of reflections in Lobachevski\u\i\ spaces of large
   dimensions},
   %language={Russian},
   note={Modules and algebraic groups, 2},
   journal={Zap. Nauchn. Sem. Leningrad. Otdel. Mat. Inst. Steklov. (LOMI)},
   volume={132},
   date={1983},
   pages={62--68},
   %issn={0373-2703},
   %review={\MR{717573 (85c:51028)}},
}

\bib{walsh}{thesis}{
author={Genevieve Walsh},
title={Great circle links in the three-sphere},
journal={\tt arXiv:math/0308048 [math.GT]},
date={2003},
type={Ph.D. thesis}
}

\bib{weyl}{article}{
author={H. Weyl}, 
title={On the volume of tubes}, 
journal={Amer. J. Math.},
volume={61},
date={1939}, 
pages={461--472},
}

\bib{weinstein}{article}{
   author={Weinstein, Alan},
   title={Positively curved $n$-manifolds in $\RR^{n+2}$},
   journal={J. Differential Geometry},
   volume={4},
   date={1970},
   pages={1--4},
   %issn={0022-040X},
   %review={\MR{0264562 (41 \#9154)}},
}

\bib{whitehead}{article}{
    Author = {Whitehead, J.H.C.},
    Title = {Certain theorems about three-dimensional manifolds. I.},
    Journal = {The Quarterly Journal of Mathematics. Oxford Series},
    %Journal = {{Q. J. Math., Oxf. Ser.}},
    %ISSN = {0033-5606},
    Volume = {5},
    Pages = {308--320},
    Year = {1934},
    Publisher = {Oxford University Press, Oxford},
    %Language = {English},
    %Zbl = {0010.27504}
}

\bib{wilking-2000}{article}{
   author={Wilking, Burkhard},
   title={On fundamental groups of manifolds of nonnegative curvature},
   journal={Differential Geom. Appl.},
   volume={13},
   date={2000},
   number={2},
   pages={129--165},
   %issn={0926-2245},
   %review={\MR{1783960 (2001g:53076)}},
   %doi={10.1016/S0926-2245(00)00030-9},
}

\bib{wilking-2003}{article}{
   author={Wilking, Burkhard},
   title={Torus actions on manifolds of positive sectional curvature},
   journal={Acta Math.},
   volume={191},
   date={2003},
   number={2},
   pages={259--297},
   %issn={0001-5962},
   %review={\MR{2051400 (2005g:53063)}},
   %doi={10.1007/BF02392966},
}

\bib{wilton}{webpage}
{author={Henry Wilton},
title={In Memoriam J. R. Stallings --- Topology of Finite Graphs},
date={2008},
url={https://ldtopology.wordpress.com/2008/12/01/}
}

\bib{zalgaller-shperical-polygon}{article}{
   author={Zalgaller, V. A.},
   title={On deformations of a polygon on a sphere},
   %language={Russian},
   journal={Uspehi Mat. Nauk (N.S.)},
   volume={11},
   date={1956},
   number={5(71)},
   pages={177--178},
   %issn={0042-1316},
   %review={\MR{0083768 (18,758e)}},
}

\bib{zalgaller-polyhedra}{article}{
   author={Zalgaller, V. A.},
   title={Isometric imbedding of polyhedra},
   %language={Russian},
   journal={Dokl. Akad. Nauk SSSR},
   volume={123},
   date={1958},
   pages={599--601},
   %issn={0002-3264},
   %review={\MR{0103511 (21 \#2279)}},
}

\bib{zamfirescu}{article}{
   author={Zamfirescu, Tudor},
   title={Baire categories in convexity},
   journal={Atti Sem. Mat. Fis. Univ. Modena},
   volume={39},
   date={1991},
   number={1},
   pages={139--164},
   %issn={0041-8986},
   %review={\MR{1111764 (92c:52002)}},
}

\bib{yau}{article}{
   author={Yau, Shing Tung},
   title={Non-existence of continuous convex functions on certain Riemannian
   manifolds},
   journal={Math. Ann.},
   volume={207},
   date={1974},
   pages={269--270},
   %issn={0025-5831},
   %review={\MR{0339008 (49 \#3771)}},
}

\bib{yoneyama}{article}{
author={Yoneyama, Kuniz\^{o}},
title={Theory of Continuous Set of Points},
journal={Tohoku Mathematical Journal, First Series},
volume={12},
date={1917},
pages={43--158}
}


\end{biblist}
\end{bibdiv}



\end{document}