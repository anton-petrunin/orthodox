\appendix
\chapter{Semisolutions}

\section*{Curves and surfaces}

%%%%%%%%%%%%%%%%%%%%%%%%%%%%%%%%%%%%%%%%%%%%%%%%%%
\parbf{\ref{liberman}.} 
\textit{Geodesic for birds.}
Consider a geodesic 
\[t\mapsto(x(t),y(t),z(t))\] 
in $W$;
assume it is defined in the interval $\II\subset \RR$.
Let us denote by $\phi$ the variation of turn;
it is a measure on $\II$.
We need to estimate $\phi(\II)$.

Denote by $s=s(t)$ the natural parameter of the plane curve \[t\mapsto (x(t),y(t)).\]

Prove that the function $f\:s\mapsto z$ is concave.

Given a semiopen interval $\mathbb J=(a,b]\subset \II$,
set
$\mu(\mathbb J)=f^+(a)-f^+(b)$,
where $f^+$ denotes right derivatives.
The function $\mu$ extends to a measure which could be also written as
\[\mu=\tfrac{dz^2}{d^2s}\cdot ds.\]
if $\tfrac{dz^2}{d^2s}$ understood in the sense of distribution.
 
Note that $|\tfrac{dz}{ds}|\le \ell$.
In particular $\mu(\II)\le 2\cdot\ell$.

Further note that $\phi\le \sqrt{1+\ell^2}\cdot\mu$.
In particular, 
$$\phi(\II)\le 2\cdot\ell\cdot\sqrt{1+\ell^2}.$$

A straightforward improvement of these estimates gives 
$$\phi(\II)\le 2\cdot\ell.$$
This bound is optimal, check for example $f(x,y)=-\ell\cdot\sqrt{x^2+y^2}$.

\parit{Comments.}
The problem is due to David Berg, \cite{berg}.
The main observation (the concavity of the function $s\mapsto z$)
is called \emph{Libeman's lemma}; 
it was used earlier 
to bound on the variation of turn 
of a geodesic on a convex surface,
see \cite{liberman}.



%%%%%%%%%%%%%%%%%%%%%%%%%%%%%%%%%%%%%%%%%%%%%%%%%%
\parbf{\ref{spiral}.}
\textit{Spiral.}
Without loss of generality we may assume that the curvature of $\gamma$ decreases in $t$.

\begin{wrapfigure}{o}{50mm}
%\begin{center}
\begin{lpic}[t(-3mm),b(0mm),r(0mm),l(0mm)]{pics/kneser-log(.5)}
%\lbl[b]{24,41;x}
\end{lpic}
%\end{center}
\end{wrapfigure}

Let $z(t)$ be the center of osculating circle at $\gamma(t)$
and $r(t)$ is its radius.
Prove that 
$$|z'(t)|\le r'(t).$$

Conclude that the osculating discs are nested;
that is, $D_{t_1}\supset D_{t_0}$ for $t_1>t_0$.
Hence the result follows.

\parit{Comments.}
The problem can be considered as a continuous analog of the Leibniz's test for alternating series.

It seems that the problem first discovered by Peter Tait in \cite{tait}
and later rediscovered by Adolf Kneser in \cite{kneser};
see also \cite{ovsienko-tabachnikov}.


%%%%%%%%%%%%%%%%%%%%%%%%%%%%%%%%%%%%%%%%%%%%%%%%%%
%???+PIC
\parbf{\ref{moon-in-puddle}.}
\textit{The moon in the puddle.}
Consider the {\it cut locus} $W$
of $F$ with respect to $\partial F$;
it is defined as the closure
of the set of points $x\in F$ 
such that there are two or more points in $\partial F$ which minimize distance to $x$.

Note that after a small perturbation
of $\partial F$ we may assume that
$W$ is a graph embedded in
$F$ with finite number of edges.

Note that $W$ is a
deformation retract of $F$.
The retraction can be obtained by moving each point $y\in F\setminus W$ to $W$
along the geodesic from the closest point to $y$ on $\partial F$ which pass through $y$.

In particular, $W$ is a tree.
Therefore $W$  has
at least two end vertices;
Denote one of them by $z$.

Prove that the disc of radius $1$ centered at $z$ lies completely in $F$.

\parit{Comments.} 
The statement still holds if the curve fails to be smooth at one point.
A spherical version of this statement 
was used by Dmitri Panov and me 
in \cite{panov-petrunin-ramification}.
 


%%%%%%%%%%%%%%%%%%%%%%%%%%%%%%%%%%%%%%%%%%%%%%%%%%
\parbf{\ref{3D-moon-in-puddle}.} 
\textit{Closed surface.}
The solution should be guessed from the picture.

%???CHANGE PIC
\begin{lpic}[t(-0mm),b(0mm),r(0mm),l(-5mm)]{pics/bing-1(.2)}
\end{lpic}
\begin{lpic}[t(-0mm),b(0mm),r(0mm),l(-5mm)]{pics/bing-2(.2)}
\end{lpic}
\begin{lpic}[t(-0mm),b(0mm),r(0mm),l(-5mm)]{pics/bing-3(.2)}
\end{lpic}

\begin{lpic}[t(-0mm),b(0mm),r(0mm),l(-5mm)]{pics/bing-6(.2)}
\end{lpic}
\begin{lpic}[t(-0mm),b(0mm),r(0mm),l(-5mm)]{pics/bing-5(.2)}
\end{lpic}
\begin{lpic}[t(-0mm),b(0mm),r(0mm),l(-5mm)]{pics/bing-4(.2)}
\end{lpic}

\parit{Comments.}
This solution is a fattening of \emph{Bing's House}, 
a nontrivial example of contractible 2-complex in $\RR^3$, 
see \cite{bing}.

%%%%%%%%%%%%%%%%%%%%%%%%%%%%%%%%%%%%%%%%%%%%%%%%%%
\parbf{\ref{curve-in-S^2}.}
\textit{A curve in a sphere.}
Let $\alpha$ be a closed curve in $\mathbb{S}^2$ of length $2\cdot\ell$ which intesects each equator.

\parit{A solution with Crofton formula.}
Note that we can assume that $\alpha$ is a broken line.

Given a unit vector $u$ denote by $e_u$ the equator with pole at $u$.
Let $k(u)$ the number of intersections
of the $\alpha$ and $e_u$.

Note that for almost all $u\in \mathbb{S}^2$, the value $k(u)$ is even.
Since each equator intersects $\alpha$, we get $k(u)\ge 2$ for almost all $u$.

Then we get
\begin{align*}
2\cdot\ell&=\tfrac14\cdot\int\limits_{\mathbb{S}^2}k(u)\cdot\, d_u\area\ge 
\\
&\ge\tfrac12\cdot\area\mathbb{S}^2=
\\
&=2\cdot\pi.
\end{align*}
The first identity above is called \emph{Crofton formula};
prove it first for one geodesic segment in $\alpha$ and then sum it up for all segments in $\alpha$.

\parit{Solution with symmetry.}
Let $\check\alpha$ be a subarc of $\alpha$ of length $\ell$, with endpoints $p$ and $q$.  
Let $z$ be the midpoint of a minimizing geodesic $[pq]$ in $\mathbb{S}^2$.  

Let $r$ be a point of intersection of $\alpha$ with the equator with pole at $z$.  
Without loss of generality we may assume that $r\in\check\alpha$. 

The arc $\check\alpha$ together with its reflection in $z$ 
form a closed curve of length $2\cdot \ell$ 
that passes through $r$ and its antipodal point $r'$.
Therefore 
\[\ell=\length \check\alpha\ge |r-r'|_{\mathbb S^2}=\pi.\]

\parit{Comments.} 
The problem was suggested by Nikolai Nadirashvili;
it is a the first step 
in the proof of Reshetnyak's majorization theorem for $\CAT[1]$ spaces, see \cite{akp}.
 


%%%%%%%%%%%%%%%%%%%%%%%%%%%%%%%%%%%%%%%%%%%%%%%%%%
\parbf{\ref{A spring in a tin}.} \textit{A spring in a tin.}
Let $\alpha$ be a closed curve in the unit disc;
denote by $\ell$ its length.

Let us equip the plane with complex coordinates so that $0$ is the center of the unit disc.
We can assume that $\alpha$ equipped with $\ell$-periodic parametrization by length.

Consider the curve $\beta(t)=t-\tfrac{\alpha(t)}{\alpha'(t)}$.
Note that 
\[\beta(t+\ell)=\beta(t)+\ell\] 
for any $t$.
In particular 
\[\length (\beta|_{[0,\ell]}) 
\ge 
|\beta(\ell)-\beta(0)|
=
\ell.\]

Note that 
\begin{align*}
|\beta'(t)|&=|\tfrac{\alpha(t)\cdot\alpha''(t)}{\alpha'(t)^2}|\le
\\
&\le|\alpha''(t)|.
\end{align*}
Since $|\alpha''(t)|$ is the curvature of $\alpha$ at $t$,
we get the result.

\parit{Comment.}
The statement was originally proved 
by Istv{\'a}n F{\'a}ry in \cite{fary};
number of different proofs are discussed by Serge Tabachnikov in \cite{tabachnikov}.

If instead of a disc we have a region bounded by closed convex curve $\gamma$ then it is still true that the average curvature of $\alpha$ is at least as big as average curvature of $\gamma$. 
The proof was given by Jeffrey Lagarias
and Thomas Richardson in \cite{lagarias-richardson}.


%%%%%%%%%%%%%%%%%%%%%%%%%%%%%%%%%%%%%%%%%%%%%%%%%%
\parbf{\ref{Convex hat}.}
\textit{Convex hat.}
Let $\gamma$ be a minimizing geodesic with the ends in $\Delta$.

Assume $\gamma\backslash\Delta\ne\emptyset$.
Denote by $\gamma'$ the curve formed by $\gamma\cap \Delta$ 
and the reflection on $\gamma\backslash\Delta$ in $\Pi$.
Note 
\[\length\gamma'=\length\gamma\]
and $\gamma'$ runs partly in and partly outside of the surface, but does not get inside of $\Sigma$.

Denote by $\gamma''$ the closest point projection of $\gamma'$ on $\Sigma$.
The curve $\gamma''$ lies in $\Sigma$ has the same ends as $\gamma$.

It remains to note that 
\[\length\gamma''<\length\gamma;\]
the later leads to a contradiction.

 



%%%%%%%%%%%%%%%%%%%%%%%%%%%%%%%%%%%%%%%%%%%%%%%%%%
\parbf{\ref{Unbended geodesic}.} 
\textit{Unbended geodesic.}
Let $W$ be the closed unbounded set formed by $\Sigma$ and its exterior points.

Prove that for any $x\in\Sigma$ the distance $|x - p_t|$ is nondecreasing in $t$.

Use the later statement, to prove the same for $|x - p_t|_W$,
where $|x - p_t|_W$ stays for the intrinsic distance from $x$ to $p_t$ in $W$.

Prove that 
\[|q - p|_W=|q - p|_\Sigma\] 
for any $p,q\in\Sigma$.


Conclude that the distance $|q - p_t|_W=|q - p|_\Sigma$
for any $t$.
It follows that the curve 
$$\gamma_t(\tau)=\left[
\begin{aligned}
&(\tau-t)\cdot\gamma'(\tau)&&\text{if}&&\tau< t;
\\
&\gamma(\tau)&&\text{if}&&\tau> t.
\end{aligned}
\right.$$
is a minimizing geodesic from $p_t$ to $q$ in the intrinsic metric of $W$. 

If $q$ is visible from $p_t$ for some $t$ then the line segment $[qp_t]$ intersects $\Sigma$ only at $q$.
From above, 
$\gamma_t$  coinsides with the line segment $[qp_t]$ which is impossible.

\parit{Comment.}
This  observation was used by Anatoliy Milka
to generalize Alexandrov's comparison theorem for convex surfaces, see \cite{milka-geod}.

%%%%%%%%%%%%%%%%%%%%%%%%%%%%%%%%%%%%%%%%%%%%%%%%%%
\parbf{\ref{min-surf}.} 
\textit{A minimal surface.}
Without loss of generality we may assume that the sphere is centered at $0\in\RR^3$.

Consider the restriction $h$ of the function $x\mapsto |x|^2$ to the surface $\Sigma$.
Prove that $\Delta_\Sigma h\le 2$ and apply apply the divergence theorem for $\nabla_\Sigma h$.
It follows that the function
\[f\:r\mapsto \frac{\area(\Sigma\cap B(0,r))}{r^2}
\]
is non-decreasing in the interval $(0,1)$.
Hence the result follows.

\parit{Comments.}
We described a partial case of so called \emph{monotonicity formula}.

Note that if we assume in addition that the surface is a disc,
then the statement holds for any saddle surface. 
Indeed, denote by $S_r$ the sphere of radius $r$ concentrated with the unit sphere. 
Then according to Problem~\ref{curve-in-S^2}, 
$\length \Sigma\cap S_r\ge 2\cdot\pi\cdot r$.
Then coarea formula leads to the solution.

On the other hand there are saddle surfaces homeomorphic to the cylinder
may have have arbitrary small area in the ball. 

If $\Sigma$ does not pass through the center 
and we only know the distance, say $r$, 
from center to $\Sigma$ 
then optimal bound is expected to be $\pi\cdot(1-r^2)$.
This is known if $\Sigma$ is homeomorphic to  disc.
This is also known for the area minimizing surfaces;
an analogous statement
holds in all dimensions and codimensions.
These results were proved by 
Herbert Alexander, 
David Hoffman
and Robert Osserman in \cite{alexander-osserman} and \cite{alexander-hoffman-osserman} correspondingly.






%%%%%%%%%%%%%%%%%%%%%%%%%%%%%%%%%%%%%%%%%%%%%%%%%%
\parbf{\ref{half-torus}.} 
\textit{Half-torus.}
Let $K$ be the convex hull of $\Omega'$.
Consider the boundary curve $\gamma'$ of $\partial K\cap \Omega'$ in $\Omega'$.

First note that the Gauss curvature of $\Omega'$ has to vanish at the points of $\gamma'$;
in other words, $\gamma'$ is the image of $\gamma$ 
under the length-preserving map.
Indeed since $\gamma'$ lies on convex part, 
the Gauss curvature at the points of $\gamma'$ has to be nonnegative. 
On the other hand $\gamma'$ bounds a flat disc in $\partial K$;
therefore its integral intrinsic curvature has to be $2{\cdot}\pi$.
If the Gauss curvature is positive at some point of $\gamma'$ then total intrinsic curvature of $\gamma'$ has to be $<2{\cdot}\pi$, a contradiction.

Now prove that $\gamma'$ is an asymptotic line.
(Assume that the asymptotic direction goes transversely to $\gamma'(t)$ and conclude $\gamma(t)\notin\partial K$.)

Without loss of generality, we can assume that the length of $\gamma$ is $2{\cdot}\pi$ and its intrinsic curvature is $\equiv 1$.
Therefore, as the space curve,
$\gamma'$ has to be a curve with constant curvature $1$ and it should be closed.
Any such curve is congruent to a flat circle.

\parit{Comments.} It is not known if $\Omega'$ is congruent to $\Omega$.

The solution presented above is based on my answer 
to the question of Joseph O'Rourke, see \cite{rourke}.
Here are some related statements.
\begin{itemize}
\item Half-torus is second order rigid;
this was proved by Eduard Rembs in
\cite{rembs}, see also \cite{efimov}*{p. 135}.
\item Any second order rigid surface does not admit analytic deformation (proved by Nikolay Efimov in \cite{efimov}*{p. 121})
and for the surfaces of revolution, the assumption of analyticity can be removed (proved by Idzhad Sabitov in \cite{sabitov}).
\end{itemize}






%%%%%%%%%%%%%%%%%%%%%%%%%%%%%%%%%%%%%%%%%%%%%%%%%%
\parbf{\ref{asymptotic-line}.} 
\textit{Asymptotic line.}
Arguing by contradiction, assume that the projection $\bar\gamma$
of $\gamma$ on $x y$-plane is star shaped with respect to the origin.

Consider the function 
$$h(t)=(d_{\bar\gamma(t)}f)(\gamma(t)).$$
Prove that $h'(t)\ne 0$.
In particular $h(t)$ is a strictly monotonic function of $\mathbb{S}^1$, a contradiction.

\parit{Comments.}
The problem is discussed by Dmitri Panov in \cite{panov-curves}.


%%%%%%%%%%%%%%%%%%%%%%%%%%%%%%%%%%%%%%%%%%%%%%%%%%
\parbf{\ref{torus}.}
\textit{Non-contractible geodesics.}
Take a torus of revolution $T$;
the rotations of the circle produce a family closed geodesics which we will call \emph{meridians}.

Note that a geodesic on $T$ is either a meridian
or it is transversal to all the meridians.
No closed curve of these types can be contractible. 

\parit{Comments.} I know this problem 
from the book of Mikhael Gromov \cite{gromov-MetStr},
where it is attributed to Y. Colin de Verdi\`ere.
I am not aware of any solutions 
which do not admit a foliation by geodesics.



%%%%%%%%%%%%%%%%%%%%%%%%%%%%%%%%%%%%%%%%%%%%%%%%%%
\parbf{\ref{Convex figures}.}
\textit{Convex figures.}
Consider the set $\Omega_n$ of all convex figures $F\subset\RR^2$ 
such that for any $x\z\in\partial F$ there are $y,z\z\in F$ such that
$\measuredangle \hinge xyz>\pi-\tfrac1n$.

Prove that $\Omega_n$ 
is open and dense in $\mathfrak{C}$.
Finally note that the intersection
$\bigcap_n\Omega_n$
forms the subset of all smooth figures in $\mathfrak{C}$.  

\parit{Comments.} 
Number of similar problems surveyed by Tudor Zamfirescu in \cite{zamfirescu}.

%%%%%%%%%%%%%%%%%%%%%%%%%%%%%%%%%%%%%%%%%%%%%%%%%%
%???CHANGE PIC
\begin{wrapfigure}{o}{29mm}
\begin{lpic}[t(-0mm),b(-3mm),r(0mm),l(0mm)]{pics/hilbertcurve5(.25)}
\end{lpic}
\end{wrapfigure}
\parbf{\ref{Fat curve}.}
\textit{Fat curve.} 
Modify your favorite space filling curve 
to keep area nearly the same and removing self-intersections.

Say, you can modify the Hilbert curve which can be constructed as a limit of recursively defined sequence of curve;
see the 5-th iteration on the diagram. 

\parit{Comments.} The existence of such curves was observed by William Osgood in \cite{osgood}.

%%%%%%%%%%%%%%%%%%%%%%%%%%%%%%%%%%%%%%%%%%%%%%%%%%
\parbf{\ref{Rectifiable curve}.}
\textit{Rectifiable curve.}
The 1-dimensional Hausdorff measure will be denoted as $\mathcal{H}_1$. 

Set $L=\mathcal{H}_1(K)$.
Without loss of generality, we may assume that $K$ has diametr $1$.

Assume that $0<\eps<\tfrac12$.
Prove that 
\[\mathcal{H}_1(B(x,\eps)\cap K)\ge\eps\leqno(*)\]
for any $x\in K$.

Let $x_1,\dots, x_n$ be a maximal set of points in $K$ such that 
\[|x_i-x_j|\z\ge\eps\] for all $i\ne j$. 
From $(*)$ we have $n\le2\cdot L/\eps$.

Construct a curve $\gamma_\eps$ such that (1) $\gamma_\eps$ is passing through all $x_i$, (2) $\length\gamma_\eps\le10\cdot L$ and (3) $\gamma_\eps$ lies in $\eps$-neighborhood of $K$.
We can assume that $\gamma_\eps$ is parametrized by length.

The needed curve can be obtained by passing to 
a partial limit of $\gamma_\eps$
 as $\eps\to 0$. 

\parit{Comments.}
This problem was given as an exercise 
in the book of Kenneth Falconer,
see \cite{falconer}*{Ex. 3.5}.



%%%%%%%%%%%%%%%%%%%%%%%%%%%%%%%%%%%%%%%%%%%%%%%%%%
\parbf{\ref{Capture a sphere in a knot}.}
\textit{Capture a sphere in a knot.}
We can assume that the knot is given by a diagram on the the sphere.

Fix a M\"obius transformation $\mathbb{S}^2\to\mathbb{S}^2$ which is not an isometry.
Denote by $u$ its conformal factor. 
Since the M\"obius transformation preservs total area, 
we get 
$$\frac1{\area \mathbb{S}^2}\cdot\int\limits_{\mathbb{S}^2} u^2=1.$$ 
Therefore, 
$$\frac1{\area \mathbb{S}^2}\cdot\int\limits_{\mathbb{S}^2} u<1.$$ 
It follows that after a suitable rotation of $\mathbb{S}^2$, 
the length of the knot decreases.

Similar argument gives a continuous one parameter family of M\"obius transformations which moves the knot in a hemisphere 
and allows the ball to escape. 

\parit{Comments.}
This is a problem of Zarathustra Brady, 
the idea in the solution is due to David Eppstein, 
see \cite{zeb}.



%%%%%%%%%%%%%%%%%%%%%%%%%%%%%%%%%%%%%%%%%%%%%%%%%%
\parbf{\ref{linked-circles}.}
\textit{Linked circles.} 
Fix a point $x\in\alpha$. 
Note that one can find another point $x'\in\alpha$ such that the interval 
$[xx']$ intersects $\beta$, say at the point $z$. 
Otherwise we can move each point of $\alpha$ along the line segment to $x$.
This deformation of $\alpha$ will not cross $\beta$;
the later contradicts that $\alpha$ and $\beta$ are linked. 


Consider the curve $\alpha'$ which is the central projection of $\alpha$ 
from $z$ onto the unit sphere around $z$;
clearly
$$\length \alpha\ge \length\alpha'.$$

Note that $\alpha'$ passes through two antipodal points of the sphere;
therefore 
$$\length \alpha'\ge 2\cdot\pi.$$
Hence the result follows.

\parit{Comments.}
This is the simplest case of so called \emph{Gehring's problem}. 
The solution above was given by Michael Edelstein and Binyamin Schwatz in \cite{edelstein-schwatz};
later the same solution was rediscovered few times.




%%%%%%%%%%%%%%%%%%%%%%%%%%%%%%%%%%%%%%%%%%%%%%%%%%
\parbf{\ref{Oval in oval}.}
\textit{Oval in oval.}
Show that the chord which minimize (or maximize) the ratio in which it divides the bigger oval solves the problem.

\begin{wrapfigure}{o}{51mm}
\begin{lpic}[t(-1mm),b(-0mm),r(0mm),l(0mm)]{pics/tangent-eq-sol(1)}
\lbl[tr]{20,5;$u$}
\lbl[rt]{18,15;$r$}
\lbl[rt]{26,6;$l$}
\lbl[bl]{25,10;$x_u$}
\end{lpic}
\end{wrapfigure}

\parit{Alternative solution.}
Given a unit vector $u$, denote by $x_u$ the point on the inner curve
with outer normal vector $u$.
Draw a chord of outer curve which is tangent to the inner curve at $x_u$;
denote by $r=r(u)$ and $l=l(u)$ the lengths of this chord at the right and left from $x_u$.


Arguing by contradiction, assume $r(u)\ne l(u)$ for any $u\in\mathbb{S}^1$.
Since the functions $r$ and $l$ are continuous,
we can assume that 
$$r(u)>l(u)\ \ \text{for any}\ \ u\in\mathbb{S}^1.\leqno{({*})}$$

Prove that
each of the following two integrals 
\begin{align*}
\tfrac12\cdot\int\limits_{\mathbb{S}^1}r^2(u)\cdot du
\quad\text{and}\quad
\tfrac12\cdot\int\limits_{\mathbb{S}^1}l^2(u)\cdot du
\end{align*}
give 
the area between the curves.
In particular 
the integrals are equal to eachother. 
The later contradicts $({*})$.

\parit{Comments.} This is a problem of Serge Tabachnikov, see \cite{tabachnikob-mi}.
A closely related, so called \emph{equal tangents problem} is discussed by the same author in \cite{tabacnikov=tan}.

%%%%%%%%%%%%%%%%%%%%%%%%%%%%%%%%%%%%%%%%%%%%%%%%%%

\begin{wrapfigure}{o}{44mm}
\begin{lpic}[t(-0mm),b(-0mm),r(0mm),l(0mm)]{pics/birkhoff(1)}
\lbl[r]{33,44.5;{\small $\hat\gamma(0)$}}
\lbl[r]{31,57.5;{\small $\hat\gamma(1)$}}
\lbl[rb]{6.5,63;{\small $\hat\gamma(n)$}}
\lbl[l]{10.3,28.3;{\small $\check\gamma(0)$}}
\lbl[l]{12.4,18.6;{\small $\check\gamma(1)$}}
\lbl[l]{37,19;{\small $\check\gamma(n)$}}
\lbl[b]{21.5,35.5;$0$}
\lbl[br]{33.5,35.5;$1$}
\lbl{40,3;$H_+$}
\lbl{3,3;$H_-$}
\end{lpic}
\end{wrapfigure}

\parbf{\ref{Poincare's last theorem}.} 
\textit{Poincar\'e's last theorem.}
Set 
\begin{align*}
H_+&=\set{z\in\CC}{\Re(z)\ge 1},
\\
H_-&=\set{z\in\CC}{\Re(z)\le -1}.
\end{align*}

Assume $f$ has no fixed points;
in other words the image of the map 
\[\phi\:z\mapsto f(z)-z\] 
lies in $\CC^*=\CC\backslash\{0\}$.

Fix $\eps>0$ such that $|f(z)-z|>\eps$ for any $z\in\CC$.
Note that the map 
\[\check f\:z\mapsto f(z)+\eps\]
is area preserving and has no fixed points.

Prove that for some positive integer $n$,
there is a curve 
\[\check \gamma\:[0,n]\to \CC\]
which starts in $H_-$, ends in $H_+$
and 
$\check f\circ\check\gamma(t)=\check\gamma(t+1)$
for any $t\z\in [0,n-1]$.

Repeat the same construction for the function $\hat f(z)=f(z)-\eps$ and obtain a curve $\hat \gamma\:[0,n]\to \CC$ starting in $H_+$ and ending in $H_-$.

Connect $\check\gamma(n)$ to $\hat \gamma(0)$ by a curve in $H_+$ 
and 
$\hat\gamma(n)$ to  $\check\gamma(0)$ by a curve in $H_-$.
Denote by $\sigma$ the obtained loop.

Prove that
\begin{itemize}
\item The loop $\phi\circ\sigma$ has to be null-homotopic in $\CC^*$.
\item The loop $\phi\circ\sigma$ is a generator of $\pi_1\CC^*$.
\end{itemize}
These two statements contradict each other. 

\parit{Comments.}
The problem was proposed by Henri Poincar\'e in \cite{poincare}
and solved by George Birkhoff in \cite{birkhoff}.

%%%%%%%%%%%%%%%%%%%%%%%%%%%%%%%%%%%%%%%%%%%%%%%%%%
\parbf{\ref{Surrounded area}.}
\textit{Surrounded area.}
Denote by $C_1$ and $C_2$ the compact regions bounded by $\gamma_1$ and $\gamma_2$ correspondingly.

By Kirszbraun theorem, 
any short map $X\to \RR^2$ defined on $X\subset \RR^2$
can be extended to a short map on whole $\RR^2$.
In particular there is a short map $f\:\RR^2\to\RR^2$ 
such that $f(\gamma_2(v))=f(\gamma_1(v))$ for any $v\in\mathbb S^1$.

Note that $f(C_2)\supset C_1$.
Whence the statement follows.

\parit{Comments.}
The Kirszbraun theorem appears in his thesis (\cite{kirszbraun}) 
and reproved later by Frederick Valentine in \cite{valentine}.



%%%%%%%%%%%%%%%%%%%%%%%%%%%%%%%%%%%%%%%%%%%%%%%%%%
\parbf{\ref{Asymptotic geodesic}.}
\textit{Periodic asymptote.}
Assume contrary.
Passing to a finite cover, we can assume that the asymptote has no self intersections.
In this case 
the restriction $\gamma|_{[a,\infty)}$  
has no self-intersections if $a$ is large large enough.

Cut $\Sigma$ along $\gamma([a,\infty))$ and then cut from the obtained surface an infinite triangle $\triangle$ with two sides formed by both sides of cuts along $\gamma$; let us denote these sides of $\triangle$ by $\gamma_-$ and $\gamma_+$.
Note that 
\[\area\triangle<\area \Sigma<\infty\leqno(*)\]
and both sides $\gamma_\pm$ 
form infinite minimizing geodesics in $\triangle$.

Consider the Buseman function $f$ for $\gamma_+$;
denote by $\ell(t)$ the length of the level curve $f^{-1}(t)$.
Let $-\kappa(t)$  be the total curvature of the suplevel set $f^{-1}([t,\infty))$.  
Note that for all large $t$ we have
\[\ell'(t)=\kappa(t)
\ \ \text{and}\ \ 
\kappa'(t)\le C\cdot \ell(t)^2\] 
where $C$ is a fixed constant.
The later implies that there is $\eps>0$ such that
\[\ell(t)\ge \frac\eps{t-a}\]
for any large $t$.
In particular,
\[\int\limits_a^\infty\ell(t)=\infty.\]
By coarea formula we get 
\[\area\triangle=\infty;\]
the late contradicts $(*)$.

\parit{Comment.}
I've learned the problem from Dmitri Burago and Sergei
Ivanov, it is originated from a discussions between
Keith Burns, Michael Brin and Yakov Pesin.
 


%%%%%%%%%%%%%%%%%%%%%%%%%%%%%%%%%%%%%%%%%%%%%%%%%%
\parbf{\ref{Immersed surface}.}
\textit{Immersed surface.}
Let $\ell$ be a linear function which vanish on $\Pi$ and positive in $\Sigma$.

Let $z$ be a point of maximum of $\ell$ on $\Sigma$;
set $s_0=\ell(z)$.
Given $s<s_0$, denote by $\Sigma_s$ the connected component of $z$ in $\Sigma\cap\ell^{-1}([s,s_0])$.
Note that for all $s$ sufficiently close to $s_0$
we have
\begin{itemize}
\item $\Sigma_s$ is an embeded disc;
\item $\partial\Sigma_s$ is convex plane curve.
\end{itemize}

Applying open-close argument, we get that the same holds for all $s\in[0,s_0)$.

Since $\Sigma$ is connected, $\Sigma_0=\Sigma$.
Hence the result follows.

\parit{Comments.}
This problem is a discussed in the lectures of Mikhael Gromov, 
see \cite{gromov-SGMC}*{\S\textonehalf}.



%%%%%%%%%%%%%%%%%%%%%%%%%%%%%%%%%%%%%%%%%%%%%%%%%%
\parbf{\ref{Two discs}.}
\textit{Two discs.}
Choose a continuous map $h\:\Sigma_1\to \Sigma_2$
which is identical on $\gamma$.
Let us prove that for some $p_1\in \Sigma_1$ and $p_2=h(p_1)\in \Sigma_2$
the tangent plane $\mathrm{T}_{p_1} \Sigma_1$ is parallel to the tangent plane $\mathrm{T}_{p_2} \Sigma_2$;
this is stronger than required.

Arguing by contradiction,
assume that such point does not exist.
Then for each $p\in\Sigma_1$
there is unique line $\ell_p\ni p$ 
which is parallel to each of the the tangent planes $\mathrm{T}_{p} \Sigma_1$ and $\mathrm{T}_{h(p)} \Sigma_2$.

Note that the lines $\ell_p$ form a tangent line distribution over $\Sigma_1$
and $\ell_p$ is tangent to $\gamma$ at any $p\in\gamma$.

Let $D$ be the disc in $\Sigma_1$ bounded by $\gamma$.
Consider the doubling of $D$ in $\gamma$;
it is diffeomorphic to $\mathbb S^2$.
The line distribution $\ell$ lifts to a line distribution on the doubling;
the later contradicts the hairy ball theorem.

\parit{Comments.} This proof was suggested nearly simultaneously by Steven Sivek and user damiano, see \cite{two-discs}.

Note that the same proof works in case if $\Sigma_i$ are oriented open surfaces such that $\gamma$ cuts a compact domain in each $\Sigma_i$.

There are examples of three disks $\Sigma_1$, $\Sigma_2$ and $\Sigma_3$
with common closed curve $\gamma$ such that there no triple of points $p_i\in\Sigma_i$ with parallel tangent plane.
Such examples can be found among ruled surfaces, see \cite{three-discs}.



%%%%%%%%%%%%%%%%%%%%%%%%%%%%%%%%%%%%%%%%%%%%%%%%%%
\parbf{\ref{Simple geodesic}.} 
\textit{Simple geodesic.}
Let $\gamma$ be a two-sided infinite geodesic in $\Sigma$.
The following is the key statement in the proof.

\parbf{Claim.}
{\it The geodesic $\gamma$ contains at most one simple loop.}
\medskip

To prove the claim use the following observations.
\begin{itemize}
\item The total curvature $\omega$ of $\Sigma$ can not exceed $2\cdot\pi$.
\item If $\phi$ is the angle at the base of a simple geodesic loop then the total curvature surrounded by the loop equals to $\pi+\phi$.
\end{itemize}

Once the claim is proved, note that if a geodesic $\gamma$ has a self-intersection
then it contains a simple loop.
From above there is only one such loop;
it cuts a disc from $\Sigma$ 
and can go around it either clockwise or counterclockwise.
This way we divide all the self-intersecting geodesics 
into two sets which we will call \emph{clockwise} and \emph{counterclockwise}.

Note that the geodesic $t\mapsto \gamma(t)$ is clockwise 
if and only if 
$t\mapsto \gamma(-t)$
is counterclockwise.
The sets of clockwise and counterclockwise are open and the space of geodesics is connected. 
It follows that there are geodesics which neither clockwise nor counterclockwise;
by the definition, these geodesics have no self-intersections.

\parit{Comment.}
This idea is due to 
Victor Bangert, 
see \cite{bangert}*{Cor. 2}.

%%%%%%%%%%%%%%%%%%%%%%%%%%%%%%%%%%%%%%%%%%%%%%%%%%
\parbf{\ref{Long geodesic}.}
\textit{Long geodesic.}
Denote by $a$ the area of the surface.

Cut the surface along a long closed simple geodesic $\gamma$.
We get two discs with nonnegative curvature and large perimeter, 
say $\ell$.
Note that the area of each disc is bounded above by $a$.

Choose one of the discs $D$ and equip it with intrinsic metric.
Note that $D$ is non-negatively curved in the sense of Alexandrov.
Denote by $p$ and $q$ be the points in $D$ which lie on the maximal distance from each other.

Fix $\eps>0$.
Fix a geodesic $[pq]$ in $D$.
Show that if $\ell$ is large enough in terms of $\eps$ 
the distance from any point in $D$ to $[pq]$ is at most $\eps$
and the curvature of $\eps$-neighborhood of $p$ in $D$
is at least $\pi-\eps$.

By Gauss--Bonnet formula the total curvature of $\Sigma$ is $4\cdot\pi$.
Since $\eps>0$ is arbitrary, we get that there are 4 point in $\Sigma$, each with curvature $\pi$
and remaining part of $\Sigma$ is flat.

\begin{wrapfigure}{o}{21mm}
\begin{lpic}[t(-3mm),b(-2mm),r(0mm),l(0mm)]{pics/akopyan(1)}
\end{lpic}
\end{wrapfigure}

It remains to show that any surface with this property is isometric to the surface of tetrahedron with equal opposite edges.
To do this cut $\Sigma$ along three geodesics which connect one singular point to the remaining three,
develop the obtained flat surface on the plane and think (also look at the diagram).

\parit{Comments.}
The problem was suggested by Arseniy Akopyan.

%%%%%%%%%%%%%%%%%%%%%%%%%%%%%%%%%%%%%%%%%%%%%%%%%%
\parbf{\ref{Corkscrew geodesic}.}
\textit{Corkscrew geodesic.}
An example can be found among the surfaces of convex polyhedrons 
such that the nondegenerate intersections with horizontal planes are triangles with parallel sides.

The polyhedron $K$ should look like a vertical needle. 
On the surface of $K$ there three broken lines, formed by the corresponding vertices of triangle.
The polyhedron can be made in such a way that any minimizing geodesic from the top of $K$ to its bottom
has to cross these lines in the cyclic order at nearly each edge, and the number of edges can be made arbitrary large. 

\parit{Comments.} This construction is due to 
Imre B{\'a}r{\'a}ny, 
Krystyna Kuiperberg 
and Tudor Zamfirescu,
see
\cite{imre-kuiperberg-zamfirescu}.

%%%%%%%%%%%%%%%%%%%%%%%%%%%%%%%%%%%%%%%%%%%%%%%%%%
\parbf{\ref{Dense homeomorphism}.} 
\textit{Dense homeomorphism.}
Note that there is countable set of homeomorphisms $h_1,h_2,\dots$ which is dense in $\mathcal{H}$
such that
each $h_n$ fix all the points outside an open round disc, say $D_n$.

Choose a countable disjoint collection of round discs $D_n'$
and consider the homeomorphism $h\:\mathbb S^2\to \mathbb S^2$
which fix all the poins ouside of $\bigcup_nD'_n$ and
for each $n$,
the restriction $h|_{D_n'}$ is conjugate to $h_n|_{D_n}$. 

Show that $h$ solves the problem.

\parit{Comments.} The problem was mentioned by Frederic Le Rox, see \cite{rox}*{Problem section}.

%%%%%%%%%%%%%%%%%%%%%%%%%%%%%%%%%%%%%%%%%%%%%%%%%%
%%%%%%%%%%%%%%%%%%%%%%%%%%%%%%%%%%%%%%%%%%%%%%%%%%
%%%%%%%%%%%%%%%%%%%%%%%%%%%%%%%%%%%%%%%%%%%%%%%%%%
%%%%%%%%%%%%%%%%%%%%%%%%%%%%%%%%%%%%%%%%%%%%%%%%%%
\section*{Comparison geometry}



%%%%%%%%%%%%%%%%%%%%%%%%%%%%%%%%%%%%%%%%%%%%%%%%%%
\parbf{\ref{Geodesic hypersurface}.} 
\textit{Geodesic hypersurface.}
Assume $\Sigma$ is a totally geodesic embedded hypersurface in $M$.
Without loss of generality, we can assume that $\Sigma$ is connected.

The complement $M\backslash\Sigma$ has one or two connected components.
First let us show that if the number of connected components is two, then $M$ is homeomorphic to sphere.

Cut $M$ along $\Sigma$,
you get two manifolds $M_1$ and $M_2$
with geodesic boundaries. 
Prove that the distance functions to the boundary 
$f_1\:M_1\to\mathbb{R}$ and $f_2\:M_2\to\mathbb{R}$ are siricly convex in the interiors of the manifolds.

Smooth the functions $f_i$ keeping them convex, this can be done by applying Greene--Wu Theorem (\cite{greene-wu}*{Thm. 2}).
In particular each $f_i$ has singe critical point which is its maximum.

Applying Morse lemma, we get that each manifold $M_i$ is homeomorphic to a ball; 
hence $M$ 
is homeomorphic to the sphere.

If $M\backslash\Sigma$ is connected,
passing to a double cover of $M$ 
we reduce the problem to the case which already have been considered.

\parit{Comments.}
The problem was suggested by Peter Petersen.



%%%%%%%%%%%%%%%%%%%%%%%%%%%%%%%%%%%%%%%%%%%%%%%%%%
\parbf{\ref{If convex then embedded}.} 
\textit{If convex then embedded.}
Observe first that any closed embedded locally convex hypersurface in a non-positively curved simply connected complete manifold bounds a convex region.


Let $\Sigma$ be an immersed locally convex hypersurface in $M$.
Set 
\[m=\dim \Sigma=\dim M-1\]

Given a point in $p$ on $\Sigma$ 
denote by $p_r$ the point on distance $r$ from $p$
which lies on the geodesic starting from $p$ in the outer normal direction to $\Sigma$.
For fixed $r\ge 0$,
the points $p_r$ swap an immersed locally convex hypersurface which we denote by $\Sigma_r$.

Fix $z\in \Sigma$.
Denote by $S_r$ the sphere of radius $r$ centered at $z$.
Note that $S_r$ is diffeomorphic to $m$-dimensional sphere.

Denote by $d$ the diameter of $\Sigma$.
Note that for all $r>0$
any point on $\Sigma_r$
lies on the distance at most $d$ from $S_r$.
Conclude that for large $r$ the closest point projection $\phi_r\:\Sigma_r\to S_r$ is an immersion.


Since $\Sigma$ is connected
and $m\ge 2$, it follows that $\phi_r$ is a diffeomorphism for all large $r$.

By the observation above, $\Sigma_r$ bounds a convex region for all large $r$.
By open-close argument, the same holds for all $r\ge 0$.
Hence the result follows.

\parit{Comments.}
The problem is due to Stephanie Alexander, see \cite{alexander}.



%%%%%%%%%%%%%%%%%%%%%%%%%%%%%%%%%%%%%%%%%%%%%%%%%%
\parbf{\ref{Immersed ball}.} 
\textit{Immersed ball.}
Equip $\Sigma$ with the induced intrinsic metric.
Denote by $\kappa$ the lower bound for principle curvatures of $\Sigma$.
Note that we can assume that $\kappa>0$.

Fix sufficiently small $\eps=\eps(M,\kappa)>0$.
Given $p\in \Sigma$ consider the lift $\tilde h_p\:B(p,\eps)\to \mathrm{T}_{h(p)}$ along the exponential map $\exp_{h(p)}\:\mathrm{T}_{h(p)}\to M$.
More precisely:
\begin{enumerate}
\item Connect each point $q\in B(p,\eps)\subset \Sigma$ to $p$
by a the minimizing geodesic  path $\gamma_q\:[0,1]\to \Sigma$
\item Consider the lifting $\tilde\gamma_q$ in $\mathrm{T}_{h(p)}$; 
that is the curve such that $\tilde\gamma_q(0)=0$ and $\exp_{h(p)}\circ\tilde\gamma_q(t)=\gamma_q(t)$ for any $t\in[0,1]$.
 \item Set $\tilde h(q)=\tilde\gamma_q(1)$.
\end{enumerate}

Show any the hypersurface $\tilde h_p(B(p,\eps))\subset \mathrm{T}_{h(p)}$ has principle curvatures at least $\tfrac\kappa2$.

Use the same idea as in problem~\ref{Immersed surface} to show that 
one can fix $\delta\z=\delta(M,\kappa)>0$ such that the restriction of $\tilde h_p|_{B(p,\delta)}$ is injective.
Conclude that the restriction $h|_{B(p,\delta)}$ is injective for any $p\in\Sigma$.

Now consider locally equidistant surfaces $\Sigma_t$ in the inward direction for small $t$. 
The principle curvatures of $\Sigma_t$ remain at least $\kappa$ in the barrier sense.
By the same argument as above, any $\delta$-ball in $\Sigma_t$
is embedded.

Applying open-close argument we get a one parameter family of locally convex locally equidistant surfaces $\Sigma_t$
for defined in a maximal interval $[0,a)$
and 
the surface $\Sigma_a$ degenerates to a point, say $p$. 

To construct the immersion $\partial \bar B^m\looparrowright M$,
take the point $p$ as the image of the center $\bar B^m$ 
and take the surfaces $\Sigma_t$ as the restrictions of the  embedding to the spheres;
the existance of the immersion follows from the Morse lemma.

\begin{wrapfigure}{o}{23mm}
%\begin{center}
\begin{lpic}[t(-2mm),b(0mm),r(0mm),l(0mm)]{pics/ass(1)}
%\lbl[b]{24,41;x}
\end{lpic}
%\end{center}
\end{wrapfigure}

\textit{Comments.}
As you see on the picture, the analogous statement does not hold in the two-dimensional case.

The proof presented above was indicated in the lectures of Mikhael Gromov, see \cite{gromov-SGMC};
it was written rigorously by Jost Eschenburg in \cite{eschenburg}.

A variation of Gromov's proof 
was obtained independently by Ben Andrews in \cite{andrews}.
Instead of equidistant deformation, 
he uses so called \emph{inverse mean curvature flow};
this way he has to perform some calculations, but does not have to worry about non-smoothness of the hypersurfaces. 




%%%%%%%%%%%%%%%%%%%%%%%%%%%%%%%%%%%%%%%%%%%%%%%%%%
\parbf{\ref{almgren}.} 
\textit{Almgren's inequalities.}
Fix a  geodesic $n$-dimensional sphere $\mathbb{S}^n$ in $\mathbb{S}^m$.

Given $r\in (0,\tfrac\pi2]$,
denote by $U_r$ and $\tilde U_r$ the tubular $r$-neighbohood 
of $\Sigma$ and $\mathbb{S}^n$ in $\mathbb{S}^m$ correspondingly.

Prove that $U_{\frac\pi2}\supset\mathbb{S}^m$.
Then it follows that
\[U_{\frac\pi2}=\tilde U_{\frac\pi2}=\mathbb{S}^m.
\leqno({*})\]

Prove that for any $x\in \partial U_r$ we have
\[H_r(x)\ge \tilde H_r,\] 
where $H_r(x)$ denotes the mean curvature of $\partial U_r$  at point $x$
and $\tilde H_r$ its mean curvature of $\partial\tilde U_r$.

Set 
\begin{align*}
a(r)&=\vol_{m-1} \partial U_r,
&
\tilde a(r)&=\vol_{m-1} \partial\tilde U_r,
\\
v(r)&=\vol_m U_r,
&
\tilde v(r)&=\vol_m \tilde U_r.
\intertext{by coarea formula,}
\tfrac d{dr} v(r)&= a(r),
&
\tfrac d{dr}\tilde v(r)&=\tilde a(r).
\end{align*}
for almost all $r$.
Note that
\begin{align*}\tfrac d{dr}a(r)&\le \int\limits_{\partial U_r} H_r(x)\cdot d_x\vol_{m-1}\le
\\
&\le a(r)\cdot \tilde H_r
\end{align*}
and
\begin{align*}
\tfrac d{dr}\tilde a(r)
&= \tilde a(r)\cdot \tilde H_r.
\intertext{It follows that}
\frac {v''(r)}{v(r)}&\le \frac {\tilde v''(r)}{\tilde v(r)}
\end{align*}
for almost all $r$. 
Therefore
\[v(r)\le\frac{\area\Sigma}{\area \mathbb{S}^n}\cdot \tilde v(r)\]

for any $r>0$.

According to $({*})$,
\[v(\tfrac\pi2)=\tilde v(\tfrac\pi2)=\vol\mathbb{S}^m.\]
Whence the result follows.

\parit{Comments.}
This problem is the most geometric part of the isoperimetric inequality proved by Frederick Almgren in \cite{almgren}.
The argument presented here is very similar to 
the proof of Gromov--Levy isometric inequality given in the Gromov's appendix to \cite{gromov-apendix}.

%%%%%%%%%%%%%%%%%%%%%%%%%%%%%%%%%%%%%%%%%%%%%%%%%%
\parbf{\ref{codim=2}.} 
\textit{Hypercurve.}
Fix $p\in M$.
Denote by $s$ 
the second fundamental form of $M$ at $p$;
it is a symmetric bi-linear form on the tangent space $\mathrm{T}_pM$ of $M$ with values in the normal space $\mathrm{N}_pM$ to $M$, see page~\pageref{Second fundamental form}.
Note that the normal space $\mathrm{N}_pM$ is two-dimensional.

Prove that if sectional curvature of $M$ is positive, 
then
\[\<s(X,X),s(Y,Y)\> > 0\leqno({*})\]
for any pair of nonzero vectors $X,Y\in\mathrm{T}_pM$.

Show that $({*})$ implies that there is an orthonormal basis $e_1,e_2$ in $\mathrm{N}_pM$ 
such that the real-valued quadratic forms 
\begin{align*}
s_1(X,X)&=\<s(X,X),e_1\>,
&
s_2(X,X)&=\<s(X,X),e_2\>
\end{align*}
are positive definite.

Note that the curvature operators $R_1$ and $R_2$ defined by the following identity
\[R_{i}(X\wedge Y), V\wedge W\rangle 
=s_i(X,W)\cdot s_i(Y,V)-s_i(X,V)\cdot s_i(Y,W)\]
 are positive.
Finally, note that $R_{1}+R_{2}$ is the curvature operator of $M$ at $p$.

\parit{Comments.}
The problem is due to Alan Weinstein, see  \cite{weinstein}.
Note that from \cite{micallef-moore}/\cite{boehm-wilking} it follows that
that the universal cover of $M$ is homeomorphic/diffeomorphic to a standard sphere.



%%%%%%%%%%%%%%%%%%%%%%%%%%%%%%%%%%%%%%%%%%%%%%%%%%
\parbf{\ref{Horosphere}.} 
\textit{Horosphere.}
Set 
$m=\dim \Sigma=\dim M-1$.

Let $b\:M\to\RR$ be the Busemann function such that $\Sigma=b^{-1}(\{0\})$.
Set  $\Sigma_r=b^{-1}(\{r\})$, so $\Sigma_0=\Sigma$.

Let us equip each $\Sigma_r$ with induced Riemannian metric.
Note that all $\Sigma_r$ have bounded curvature.
In particular, unit ball in $\Sigma_r$ has volume bounded above by universal constant, say $v_0$.
 

Given $x\in \Sigma$ denote by $\gamma_x$ 
the (necessary unique) unit-speed geodesic
such that $\gamma_x(0)=x$ and $b(\gamma_x(t))=t$ for any $t$.
Consider the map $\phi_{r}\:\Sigma\to\Sigma_r$ defined as
$\phi_r\:x\mapsto \gamma_x(r)$.

Notice that $\phi_r$ is a bi-Lipschitz map with the Lipschitz constants $e^{a\cdot r}$ and $e^{b\cdot r}$.
In particular, the ball of radius $R$ in $\Sigma$ is mapped by $\phi_r$
to a ball of radius $e^{a\cdot r}\cdot R$ in $\Sigma_r$.
Therefore
\[\vol_m B(x,R)_\Sigma\le e^{m\cdot b\cdot r}\cdot \vol_m B(x,e^{a\cdot r}\cdot R)_{\Sigma_r}\]
for any $R,r>0$.
Applying this formula in case $e^{a\cdot r}\cdot R=1$ implies that
\[\vol_m B(x,R)_\Sigma\le v_0\cdot R^{m\cdot \frac ba}.\]

\parit{Comment.}
The problem was suggested by Vitali Kapovitch.

There are examples of horospheres as above with degree of polynomial growth higher than $m$.
For example, consider the horosphere $\Sigma$ in the
the complex hyperbolic space 
of real dimension $4$.
Clearly $m=\dim \Sigma=3$ but the degree of its volume growth is $4$.
The later follows since $\Sigma$ comes with a left-invariant metric on the \hyperref[Heisenberg group]{\emph{Heisenberg group}}.


                                                      



%%%%%%%%%%%%%%%%%%%%%%%%%%%%%%%%%%%%%%%%%%%%%%%%%%
\parbf{\ref{Minimal spheres}.} 
\textit{Minimal spheres.}
Choose a pair of sufficiently close minimal spheres $\Sigma$ and $\Sigma'$,
say assume that the distance $a$ between $\Sigma$ and $\Sigma'$ is strictly smaller than the injectivity radius of the manifold.
Note that in this case there is a bijection $\Sigma\to \Sigma'$, which will be denoted by $p\mapsto p'$ such that the distance $|p-p'|=a$ for any $p\in\Sigma$.

Let $\iota_p\:\mathrm{T}_p\to\mathrm{T}_{p'}$ be the parallel translation along the (necessary unique) minimizing geodesic from $p$ to $p'$.
Use hairy ball theorem 
to show that there is a pair $(p,p')$ such that $\iota_p(\mathrm{T}_p\Sigma)=\mathrm{T}_{p'}\Sigma'$.

Consider pairs of unit-speed geodesics $\alpha$ and $\alpha'$ 
in $\Sigma$ and $\Sigma'$  
which start at $p$ and $p'$ correspondingly
and go in the parallel directions, say $\nu$ and $\nu'$. 
Set $\ell_\nu(t)=|\alpha(t)-\alpha'(t)|$.

Use the second variation formula to show that $\ell_\nu''(0)$ has negative average for all tangent directions $\nu$ to $\Sigma$ at $p$. 
In particular $\ell_\nu''(0)<0$ for a pair $\alpha$ and $\alpha'$ as above.
It follows that there are points $v\in\Sigma$ near $p$ 
and $v'\in\Sigma'$ near $p'$
such that 
\[|v-v'|<|p-p'|;\]
the later leads to a contradiction.

\parit{Comments.}
It seems pleasurable that a 
compact 
positively curved 
4-dimensional manifold
can not contain a pair of equidistant spheres.
The argument above implies that the distance between such a pair has to exceed the injectivity radius of the manifold.

The problem was suggested by Dima Burago.
Here is a short list of classical problems with similar solutions:
\begin{itemize}
\item Synge's problem, see \cite{synge}.
\begin{itemize}
 \item {\it Any compact even-dimensional orientable manifold with positive sectional curvature is
simply connected.}
\end{itemize}
\item Frankel's problems, see \cite{frankel}.
\begin{itemize}
\item {\it Any two compact \hyperref[Minimal surface]{\emph{minimal hypersurfaces}} in a Riemannian manifold with positive Ricci curvature must intersect.}
\item {\it Assume $\Sigma_1$ and $\Sigma_2$ be two compact geodesic submanifolds in a manifold with positive sectional curvature $M$ and \[\dim \Sigma_1+\dim \Sigma_2\ge \dim M.\] 'Show that $\Sigma_1\cap\Sigma_2\ne\emptyset$.}
\end{itemize}
\item Bochner's problem, see \cite{bochner}.
\begin{itemize}
\item{\it  Let $(M,g)$ be a closed Riemannian manifold with negative Ricci curvature.
Prove that $(M,g)$ does not admit an isometric $\mathbb{S}^1$-action.}
\end{itemize}
\end{itemize}
The problem \ref{Geodesic immersion} can be considered as further development of the same idea.




%%%%%%%%%%%%%%%%%%%%%%%%%%%%%%%%%%%%%%%%%%%%%%%%%%
\parbf{\ref{Geodesic immersion}.} 
\textit{Geodesic immersion.}
Set $n=\dim N$ and $m=\dim M$.

Fix a smooth increasing concave function $\phi$.
Consider the function $f=\phi\circ\dist_N$.
Note that if $f$ is smooth at $x$ 
the the Hessian, $\Hess_xf$, has at least $n+1$ negative eigenvalues.

Moreover, at any point $x\notin \iota(N)$ the same holds in the barrier sense;
that is, there is a smooth function $h\ge f$ defined on $M$ of $x$
such that $h(x)=f(x)$ and $\Hess_xf$ has at least $n+1$ negative eigenvalues.

Use that $m< 2\cdot n$ and the property to prove the following
analog of Morse lemma for $f$.

\parbf{Claim.}
{\it Given $x\notin \iota(N)$ there is a neighborhood $U\ni x$ such that the set 
\[U_-=\set{z\in U}{f(z)<x}\] is simply connected.}

\medskip

Since $M$ is simply connected,
any closed curve in $\iota(N)$
can be contracted by a disc, say $f_0\:\mathbb D\to M$.
According to the claim, 
there is a homotopy $f_t\:\mathbb D\to M$, $t\in [0,1]$ 
such that $f_t(\partial \mathbb D)\subset \iota(N)$ for any $t$ and $f_1(\mathbb D)\subset \iota(N)$.
It follows that $\iota(N)$ is simply connected.

Finally note that if $\iota\:N\to M$ has a self-intersection
then the image
$\iota(N)$ is not simply connected.
Hence the result follows.

\parit{Comments.}
The statement was proved by 
Fuquan Fang, 
S{\'e}rgio Mendon{\c{c}}a 
and Xiaochun Rong in \cite{FMR}.
The main idea was discovered by 
Burkhard Wilking, 
see \cite{wilking-2003}.

%%%%%%%%%%%%%%%%%%%%%%%%%%%%%%%%%%%%%%%%%%%%%%%%%%
\parbf{\ref{kleiner-hopf}.} 
\textit{Positive curvature and symmetry.}
Let $M$ be a 4-dimensional Riemannian manifold with isometric $\mathbb{S}^1$-action.
Consider the quotient space $X=M/\mathbb{S}^1$.

Note that $X$ is a positively curved 3-dimensional Alexandrov space;
see \cite{akp} if in doubt.
In particular the angle $\measuredangle\hinge xyz$ between any two geodesics $[xy]$ and $[xz]$ is defined
and 
\[\measuredangle\hinge xyz+\measuredangle\hinge yzx+\measuredangle\hinge zxy> \pi.\leqno({*})\]
for any non-degenerate triangle $[xyz]$ formed by the minimizing geodesics $[xy]$, $[yz]$ and $[zx]$ in $X$.

Assume $p\in X$ corresponds to a fixed point of $\mathbb{S}^1$-action.
Show that 
for any three geodesics $[px]$, $[py]$ and $[pz]$ in $X$ we have
\[\measuredangle\hinge pxy+\measuredangle\hinge pyz+\measuredangle\hinge pzx\le \pi.\leqno({*}{*})\]
and
\[\measuredangle\hinge pxy, \measuredangle\hinge pyz, \measuredangle\hinge pzx\le \tfrac\pi2.\leqno({*}{*}{*})\]

Arguing by contradiction,
assume that there are 4 fixed points $q_1$, $q_2$, $q_3$ and $q_4$.
Connect each pair $q_i\ne q_j$ by a minimizing geodesic $[q_iq_j]$.

Denote by $\omega$ the sum of all 12 angles of the type  $\measuredangle\hinge{q_i}{q_j}{q_k}$.
By $({*}{*}{*})$, each triangle $\triangle q_iq_jq_k$ is non-degenerate.
Therefore by $({*})$, we have
\[\omega>4\cdot\pi.\]
Applying $({*}{*})$ at each vertex $q_i$, we have 
\[\omega\le 4\cdot\pi,\]
a contradiction.

\parit{Comment.}
The problem is due to 
Wu-Yi Hsiang 
and Bruce Kleiner, see \cite{hsiang-kleiner}.
The connection of this proof to Alexandrov geometry was noticed by Karsten Grove in \cite{grove}.
An interesting new twist of this idea 
is given by 
Karsten Grove 
and Burkhard Wilking 
in  \cite{grove-wilking}.



%%%%%%%%%%%%%%%%%%%%%%%%%%%%%%%%%%%%%%%%%%%%%%%%%%
\parbf{\ref{scalar-curv}.} 
\textit{Curvature vs. injectivity radius.}
We will show that if the injectivity radius of the manifold $(M,g)$ is at least $\pi$
then the average of sectional curvatures on $(M,g)$ is at most $1$.
This is equivalent to the problem.

Fix a point $p\in M$ and two orthonormal vectors $U,V\in\mathrm{T}_p M$.
Consider the geodesic $\gamma$ in $M$ such that $\dot\gamma(0)=U$.

Set $U_t=\dot\gamma(t)\in \mathrm{T}_{\gamma(t)}$ 
and let $V_t\in \mathrm{T}_{\gamma(t)}$ be the parallel translation of $V=V_0$ along $\gamma$.

Consider the field $W_t=\sin t\cdot V_t$ on $\gamma$.
Set 
\begin{align*}
\gamma_\tau(t)&=\exp_{\gamma(t)} (\tau\cdot W_t),
&
\ell(\tau)&=\length(\gamma_\tau|_{[0,\pi]}),
&
q(U,V)&=\ell''(0).
\end{align*}
Note that
\[q(U,V)=\int\limits_{0}^\pi [(\cos t)^2-K(U_t,V_t)\cdot (\sin t)^2]\cdot dt,\leqno({*})\]
where $K(U,V)$ denotes the curvature 
in the sectional direction spanned by $U$ and $V$. 

Since any geodesics of length $\pi$ is minimizing,
we get $q(U,V)\ge0$ for any pair of orthonormal vectors $U$ and $V$.
It follows that average value of the right hand side in $({*})$ is nonnegative.

By Liouville's theorem, while taking the average of $({*})$, we can switch the order of integrals;
therefore  
\[0\le \tfrac\pi2\cdot(1-\bar{K}),\]
where $\bar{K}$ denotes the average of sectional curvatures on $(M,g)$.
Hence the result follows.

\parit{Comments.} Liouville's theorem has a number of similar applications,
one of the most beautiful is sharp isoperimetric inequality for 
4-dimensional Hadamard manifolds;
it was proved by Cristopher Croke in \cite{croke-4d},
see also \cite{croke-eigenvalue}.






%%%%%%%%%%%%%%%%%%%%%%%%%%%%%%%%%%%%%%%%%%%%%%%%%%
\parbf{\ref{almost-flat}.} 
\textit{Almost flat manifold.}
First prove that for given $\eps>0$, 
there is big enough $m$ and $m\times m$ integer matrix 
$A$ such that all its eigenvalues are $\eps$-close to $1$. 

Consider $(m+1)$-dimensional manifold $S$ obtained from $\TT^m\times [0,1]$ by gluing $\TT^m\times 0$ to $\TT^m\times 1$ along the map given by $A$.

Assuming that $\eps$ is small,
show that $S$ admits a metric with curvature and diameter sufficiently small.

\parit{Comment.} 
This example was constructed by Guzhvina in \cite{guzhvina}.

The main theorem of Gromov in \cite{gromov-almost-flat}, 
states that there are no such examples of fixed dimension;
a more detailed proof can be found in \cite{buser-karcher}
and a more precise statement can be found in \cite{ruh}.

It is expected that for small enough $\eps>0$,
a Riemannian manifold $(M,g)$ of any dimension 
with  $\diam(M,g)\le 1$ and $|K_g|\le \eps$ can not be simply connected,
here $K_g$ denotes the sectional curvature of $g$.
It is not true if instead one asks only have $K_g\le \eps$;
in fact for any $\eps>0$,
there are metrics $g$ on $\mathbb{S}^3$ 
with $K_g\le \eps$ and $\diam(\mathbb{S}^3,g)\le 1$; 
this example was originally constructed by Mikhael Gromov in \cite{gromov-almost-flat} and a simplified proof was given by 
Peter Buser
and Detlef Gromoll in \cite{buser-gromoll}.


%%%%%%%%%%%%%%%%%%%%%%%%%%%%%%%%%%%%%%%%%%%%%%%%%%
\parbf{\ref{lie-nonneg}.} 
\textit{Lie group.} We will write the metrics in contrvariant form, 
so metric $g$ is considered to be a linear transformation $\mathrm T\to \mathrm T^*$ and 
\[|v|_g=\sqrt{[g^{-1}(v)](v)}\]
for any tangent vector $v$.


Let $G$ be the compact Lie group and $e\in G$ is the identity element.

Denote by $K$ the set of all contrvariant metrics $g\:\mathrm{T}^*_e\to \mathrm{T}_e $ 
which extend to non-negatively curved left invariant metrics $\tilde g$ on $G$. 

First recall that the bi-invariant metric on a compact Lie groups is nonnegatively curved. 
Therefore $K$ is non empty.

We will show that $K$ is convex, this is stronger than required.
That is, we need to show that if $g_0, g_1\in K$
then 
\[g_t=(1-t)\cdot g_0+t\cdot g_1\in K\] 
for any $t\in [0,1]$.

Denote by $\Delta$ the diagonal subgroup of $G\times G$.

For each $t\in (0,1)$, consider the product 
\[(G\times G,\tilde h_t)
=
(G,(1-t)\cdot \tilde g_0)\times (G,t\cdot \tilde g_1).\]
Note that $h_t$ is nonnegatively curved left invariant metric on $G\times G$.

The quotient map 
\[
 (G\times G,\tilde h_t)\to (G\times G,\tilde h_t)/\Delta
\]
is a Riemannian submersion.
By O'Nail's formula,
the target has nonneggative curvature.

Fianally note that $(G,\tilde g_t)$
is isometric to
$(G\times G,\tilde h_t)/\Delta$ for any $t\in (0,1)$.
Therfore $g_t\in K$ for any $t\in [0,1]$ as required.

\parit{Comment.}
The earliest use of this construction 
I found  in \cite{GKM}.
It was used to show that Berger's spheres have positive curvature;
it was done by showing that
Berger's spheres are isometric to the quotient space $(\mathbb S^3\times \RR)/\RR$ for certain $\RR$-actions which shifts $\RR$ and rotates $\mathbb{S}^3$ at the same time.

This construction some times is called \emph{Cheeger's trick}.
It is used to construct most of the examples of positively and non-negatively curved manifolds,
see \cite{cheeger}, \cite{aloff-wallach}, \cite{gromoll-meyer}, \cite{eschenburg-spaces} and \cite{bazajkin}.





%%%%%%%%%%%%%%%%%%%%%%%%%%%%%%%%%%%%%%%%%%%%%%%%%%
\parbf{\ref{milka-polar}.} 
\textit{Polar points.}
Fix a unit-speed geodesic $\gamma$ such that $\gamma(0)=p$.
Set $p^*=\gamma(\pi)$.

Prove that $p^*$ is a solution.

\parit{Alterntive proof.} 
Assume contrary;
that is, for any $x\in M$ there is a point $x'$ such that 
\[|x-x'|_g+|p-x'|_g>\pi.\]

Show that there is a continuous map $x\mapsto x'$
such that the above inequaliy holds for any $x$.

Fix sufficiently small $\eps>0$.
Prove that the set $W_\eps=M\backslash B(p,\eps)$ 
is homeomorphic to a ball 
and the map $x\mapsto x'$ sends $W_\eps$ into itself.

By Brouwer's fixed-point theorem, $x=x'$ for some $x$.
In this case 
\[|x-x'|_g+|p-x'|_g\le \pi,\]
a contradiction.
 
\parit{Comments.}
The problem is due to Anatoliy Milka, see \cite{milka-poly}.

%%%%%%%%%%%%%%%%%%%%%%%%%%%%%%%%%%%%%%%%%%%%%%%%%%
\parbf{\wrenches\ref{Deformation to a product}.} 
\textit{Deformation to a product.} 
Denote by $\Gamma$ the fundamental group of $M$.

Let $(\tilde M,\tilde g)$ be universal cover of $(M,g)$ with induced Riemannian metric.
The space $(\tilde M,\tilde g)$ is isometric to a product $\RR^k\times K$, 
where $k$ is nonnegative integer and $K$ is a compact Riemannian manifold.

Denote by $G$ the isometry group of $K$.
Given a continuous one parameter family of homomorphisms $\phi_t\:\RR^k\to G$,
consider the the  one parameter family of diffeomorphisms of $\RR^k\times K$ to itself defined as
\[\Phi_t\:(x,k)\mapsto (x,\phi_t(x)\cdot k).\]
Denote by 
 $\tilde g_t$ pullback 
of $\tilde g$ via $\Phi_t$,
so 
\[\Phi_t\:(\RR^k\times K,\tilde g_t)\to (\RR^k\times K,\tilde g)\]
is an isometry.

It remains to find the one parameter family $\phi_t$ such that 
 $\tilde g_t$ is $\Gamma$-invariant for all $t$.
and $(M,g_1)=(\tilde M,\tilde g)/\Gamma$ admits a finite Riemannian cover by the product of a flat torus and $K$.


In terms of $\phi_t$, it can be formulated the following way.
There is a normal subgroup of finite index $\Gamma_0\vartriangleleft\Gamma$ such that 
\begin{itemize}
\item $\Gamma_0$ acts on $\RR^k$ by parallel translations; 
in particular $\Gamma_0$ can be identified with with a lattice in $\RR^k$.
\item the action of $\Gamma_0$ on $K$ is given by $\phi_1(\Gamma_0)$.
\end{itemize}


\parit{Comment.}
The problem is due to Burkhard Wilking, see \cite{wilking-2000}.



%%%%%%%%%%%%%%%%%%%%%%%%%%%%%%%%%%%%%%%%%%%%%%%%%%
\parbf{\ref{Isometric section}.} 
\textit{Isometric section.}
Arguing by contradiction, 
assume $\iota\: M\z\to W$ is an isometric section.
It makes possible to treat $M$ as a submanifold in $W$.

Given $p\in M$,
denote by $\nu_pM$ the sphere of unit normal vectors to $M$ at $p$.
Given $v\in \nu_p$ and real value $k$,
set 
\[p^{k\cdot v}=s\circ\exp_{p} (k\cdot v).\]
Note that 
\[p^{0\cdot v}=p\ \ \text{for any}\ \  p\in M\ \ \text{and}\ \ v\in \nu_p.\leqno({*})\]

Fix sufficiently small $\delta>0$.
By Rauch comparison, if $w\in \nu_q$ 
is the parallel translation of $v\in \nu_q$ 
along a minimizing geodesic from $p$ to $q$ in $M$
then 
\[|p^{k\cdot v}-q^{k\cdot w}|_M<|p-q|_M
\leqno({*}{*})\]
assuming $|k|\le \delta$.
The same comparison implies that 
\[|p^{k\cdot v}-q^{k'\cdot w}|_M^2<|p-q|_M^2+ (k-k')^2
\leqno({*}{*}{*})\]
assuming $|k|,|k'|\le \delta$.

Choose $p$ and $v \in \nu_p$ so that $r=|p-p^{\delta\cdot v}|$ 
takes the maximal possible value.
From $({*}{*})$ it follows that $r>0$.

Let $\gamma$ be the extension of the unit-speed minimizing geodesic from $p_v$ to $p$;
denote by $v_t$ the parallel translation of $v$ to $\gamma(t)$ along $\gamma$. 

We can choose the parameter of $\gamma$ so that $p=\gamma(0)$, $p^v=\gamma(-r)$.
Set $p_n=\gamma(n\cdot r)$, so $p=p_0$ and $p^v=p_{-1}$. 
Fix large integer $N$ and set $w_n=(1-\tfrac nN)\cdot v_{n\cdot r}$
and $q_n=p_n^{w_n}$.

%%%???+PIC%\begin{wrapfigure}{r}{50mm}
\begin{center}
\begin{lpic}[t(-0mm),b(0mm),r(0mm),l(0mm)]{pics/perelman(1)}
\lbl[trw]{8,5;$p_{-1}{=}q_0$}
\lbl[tr]{14,5;$p_0$}
\lbl[tl]{15,5;$q_1$}
\lbl[tr]{24,5;$p_1$}
\lbl[tl]{26,5;$q_2$}
\lbl[tr]{34.5,5;$p_2$}
\lbl[tl]{36.5,5;$q_3$}
\lbl[tr]{45,5;$p_3$}
\lbl[tl]{47.2,5;$q_4$}
\lbl[tr]{55.5,5;$p_4$}
\lbl[b]{64,8;$M$}
\lbl[t]{64,5;$\dots$}
\lbl[b]{35.2,26,90;{\small $\exp_{p_3} (w_3)$}}
\end{lpic}
\end{center}
%\end{wrapfigure}

From $({*}{*}{*})$, there is a constant $C$ independent of $N$ such that
\[|q_k-q_{k+1}|<r+\tfrac C{N^2}\cdot\delta^2.\]
Therefore 
\[|q_{k+1}-p_{k+1}|>|q_k-p_k|-\tfrac C{N^2}\cdot\delta^2.\]
By induction, we get 
\[|q_N-p_N|>r-\tfrac C{N}\cdot\delta^2.\]
Since $N$ is large we get
\[|q_N-p_N|>0.\]
By $({*})$ we get $q_N=p_N^0=p_N$, a contradiction.

\parit{Comment.} 
This proof is the core of the solution of Soul conjecture
by Grigori Perelman, 
see \cite{perelman}.

%%%%%%%%%%%%%%%%%%%%%%%%%%%%%%%%%%%%%%%%%%%%%%%%%%
\parbf{\ref{pr:Minkowski space}.} 
\textit{Minkowski space.}
Fix an increasing function $\phi\:(0,r)\to \RR$
such that 
\[\phi''+(n-1)\cdot(\phi')^2+C=0.\]

Note that if $\Ric_{g_n}\ge C$ then the function 
$x\mapsto\phi(|p-x|_{g_n})$ is subharmonic.
In follows that, 
for arbitrary array of points $p_i$ 
and positive reals $\lambda_i$ the function $f_n\:M_n\to \RR$
defined by the formula
$$f(x)=\sum_i\lambda_i\cdot\phi(|p_i-x|_M)$$
is subharmonic.
In particular $f_n$ can not admit a local minima in $M_n$.

Passing the limit as $n\to \infty$, we get that any function $f\:\mathbb{R}^m\to\mathbb{R}$
of the form 
$$f(x)=\sum_i\lambda_i\cdot\phi(|p_i-x|_{\ell_p})$$
does not admit a local minima in $\mathbb{R}^m$.

It remains to arrive to a contradiction
by showing that if $p\ne 2$ then there is an array
points $p_i$ and positive reals $\lambda_i$
such that the function 
$$f(x)=\sum_i\lambda_i\cdot\phi(|p_i-x|_{\ell_p})$$
has strict local minimum.

\parit{Comment.} The argument given here is very close to the proof of Abresch--Gromoll inequality in \cite{abresch-gromoll}.
An alternative solution of this problem can be build on almost splitting theorem proved by  Jeff Cheeger and Tobias Colding in \cite{cheeger-colding}.


%%%%%%%%%%%%%%%%%%%%%%%%%%%%%%%%%%%%%%%%%%%%%%%%%%
\parbf{\ref{Curvature hollow}.} 
\textit{Curvature hollow.}
Construct a metric that the connected sum
$M=\RR^3\#\mathbb{S}^2\times\mathbb{S}^1$ admits a metric which is flat outside a compact set and has non positive scalar curvature.
Further, note that such metric can be constructed in such a way that it has a closed geodesic $\gamma$ with trivial holonomy and with constant negative curvature in its a tubular neighborhood.

Cut the tubular neigbourhood $D^2\times \mathbb{S}^1$ of $\gamma$, 
prepare a metric $g$ on $\mathbb{S}^1\times D^2$ with negative scalar curvature which 
is identical to the original metric near the boundary.
The needed patch $(\mathbb{S}^1\times D^2,g)$ can be found among wrap products $\mathbb{S}^1\times_f D^2$.

Note that after the surgery we get a manifold diffeomorphic to $\RR^3$ with the required metric.

\parit{Comments.}
This construction was given by Joachim Lohkamp in \cite{lohkamp},
he describes there yet an other equally simple construction.
In fact this  constructions produce 
$\mathbb{S}^1$-invariant hollows 
with negative Ricci curvature.

On the other hand there are no hollows with positive scalar curvature;
the later equivalent to the Positive Mass Conjecture.

%%%%%%%%%%%%%%%%%%%%%%%%%%%%%%%%%%%%%%%%%%%%%%%%%%
\parbf{\wrenches\ref{hS=>S}.} 
\textit{If hemisphere then sphere.}
Denote by $q$ a point in $M$ which lies on the maximal distance from $p$.

Consider the function $f=\cos \dist_p+\cos\dist_q$.
Note that 
\[\triangle f+m\cdot f\le 0\] 
in the sense of distributions.
If follows that $f\ge 0$, in particular 
\[B(p,\tfrac\pi2)\cup B(q,\tfrac\pi2)=M.\]

Set \[a(r)=\area(\partial [B(q,r)\backslash B(p,\tfrac\pi2)]).\]
Prove that the function
\[r \mapsto \tfrac{a(r)}{(\sin r)^{m-1}}\]
is nonincreasing 
and 
\[\tfrac{a(r)}{(\sin r)^{m-1}}\le\area\mathbb{S}^{m-1}.\]
Moreover if equality holds for some $r$ then $B(q,r)\backslash B(p,\tfrac\pi2)$ is isometric to an $r$-ball in the unit sphere.
This statement is analogous to the Bishop--Gromov inequality and can be proved the same way.

Finally note that $a(\tfrac\pi2)=\area\mathbb{S}^{m-1}$,
hence the result follows.
 

\parit{Comments.}
The problem is due to Fengbo Hang %Fengbo Hang fengbo@cims.nyu.edu
and Xiaodong Wang %Xiaodong Wang xwang@math.msu.edu 
\cite{hang-wang};
their proof is different.

The problem is still nontrivial 
even if instead of the first condition one has that sectional curvature $\ge 1$;
In dimension 2, 
it was proved by Victor Toponogov in \cite{toponogov},
but it also follow easily from the proof given by Victor Zalgaller in \cite{zalgaller-shperical-polygon}.


If instead of first condition one only has that scalar curvature $\ge m\cdot(m-1)$, then the conclusion does not hold; 
it was conjectured by Maung Min-Oo in 1995 
and disproved by
Simon Brendle,
Fernando Marques
and Andre Neves in \cite{brendle-marques-neve}.



%%%%%%%%%%%%%%%%%%%%%%%%%%%%%%%%%%%%%%%%%%%%%%%%%%
\parbf{\ref{Flat coordinate planes}.} 
\textit{Flat coordinate planes.}
Fix $\eps>0$ such that there is unique geodesic between any two points on distance $<\eps$ from the origin of $\RR^3$.

Consider three points $a$, $b$ and $c$ 
on the coordinate lines which are $\eps$-close 
to the origin.

Prove that the angles of the triangle $\triangle abc$
coincide with its model angles.
It follows that there is a flat geodesic triangle in $(\RR^3,g)$ with vertex at $a$, $b$ and $c$.

Use the family of constructed flat triangles 
to show that at any $x$ point in the $\tfrac\eps{10}$-neighborhood of the origin
the sectional curvature 
vanish in an open set of sectional directions.
The later implies that the curvature is identically zero 
in this neighborhood.

Moving the origin and apply the same argument we get that the curvature is identically zero everywhere.
Hence the result follows. 

\parit{Comment.} 
This problem appears in the paper of Dmitri Panov and me \cite{panov-petrunin}; 
it is based on a lemma discovered by Sergei Buyalo in \cite{buyalo}.

%%%%%%%%%%%%%%%%%%%%%%%%%%%%%%%%%%%%%%%%%%%%%%%%%%
\parbf{\ref{Two-convexity}.}
\textit{Two-convexity.}
Assume $W$ is a hypersurface of needed type.

\parit{Morse-style solution.}
Let us equip $\RR^4$ with $(x,y,z,t)$-coordinates.

Consider a generic linear function $\ell\:\RR^4\to\RR$
which is close to the sum of coordinates $x+y+z+t$.
Note that $\ell$
has non-degenerate critical points on $W$ and all its critical values are different.

Consider the sets 
$$W_s=\set{w\in \RR^4\backslash K}{\ell(w)<s}.$$
Note that $W_{-1000}$ contains a closed curve, say $\alpha$, 
which is contactable in $\RR^4\backslash K$, 
but not constructable in the set $W_{-1000}$.

Set $s_0$ to be the infimum of the values $s$ such that
the $\alpha$ is contactable in $W_s$.

Note that $s_0$ is a critical value of $\ell$ on $W$;
denote by $p_0$ the corresponding critical point.
By 2-convexity of $\RR^4\backslash K$,
the index of $p_0$ has to be at most $1$.
On the other hand, since the disc hangs at this point,
its index has to be at least $2$,
 a contradiction.

\parit{Alexandrov-style proof.}
Fix a constant Riemannian metric $g$ on $\RR^4$.
According to the main result of Alexander Bishop and Berg in \cite{ABB}, $X_g=(\RR^4\backslash (\Int K),g)$ has nonpositive curvature in the sense of Alexandrov.
In particular the universal cover of $\tilde X_g$ of $X_g$ is a $\CAT[0]$ space.

By rescaling $g$ and passing to the limit we obtain that universal Riemannian cover $Z_g$ of $(\RR^4,g)$ branching in the coordinate planes is a $\CAT[0]$ space.
Show that $Z_g$ is $\CAT[0]$ space if and only if the two planes are orthogonal with respect to $g$;
the later leads to a contradiction.

\parit{Comments.}
Note that the closed $1$-neighborhood of these two planes has two-convex complement, but the boundary of this neighborhood is not smooth.

The Morse-style is based on the idea of Mikhael Gromov, see \cite{gromov-SGMC}*{\S\textonehalf}.

\section*{Curvature free differential geometry}

%%%%%%%%%%%%%%%%%%%%%%%%%%%%%%%%%%%%%%%%%%%%%%%%%%
\parbf{\ref{gromomorphic-curves}.} 
\textit{Minimal foliation.}
First show that there is a self-dual harmonic 2-form on $(\mathbb{S}^2\times\mathbb{S}^2,g)$;
that is, a 2-form $\omega$ such that $d\omega=0$ and $\star\omega=\omega$,
where $\star$ denotes the Hodge star operator.

Fix $p\in \mathbb{S}^2\times\mathbb{S}^2$.
Use the identity $\star\omega_p=\omega_p$
to show that
there is a real number $\lambda_p$ and the isometry $\mathrm{J}_p\:\mathrm{T}_p\to\mathrm{T}_p$ 
such that
$\mathrm{J}_p\circ\mathrm{J}_p =-\id$ 
and 
$\omega(X,Y)=\lambda_p\cdot g(X,\mathrm{J}_pY)$ for any $X,Y\in \mathrm{T}_p$.

Consider canonical symplectic form $\omega_0$ on $\mathbb{S}^2\times\mathbb{S}^2$;
that is sum of pullbacks of volume forms on $\mathbb{S}^2$  
for the two projections $\mathbb{S}^2\times\mathbb{S}^2\to \mathbb{S}^2$.
Note that for the canonical metric on $\mathbb{S}^2\times\mathbb{S}^2$,
the form $\omega_0$ is harmonic and self-dual. 
Since $g$ is close to the standard metric,
we can assume that $\omega$ is close to $\omega_0$.
In particular $\lambda_p\ne0$ for any $p\in \mathbb{S}^2\times\mathbb{S}^2$.

It follows that $\omega$ defines simplectic structure on $\mathbb{S}^2\times\mathbb{S}^2$
and $\mathrm{J}$ is its psudocomplex structure.
It remains to take the reparametrization of $\mathbb{S}^2\times \mathbb{S}^2$
so that vertical and horizontal spheres will form pseudoholomorphic curves in the homology classes of $\mathbb{S}^2\times x$ and $x\times \mathbb{S}^2$.
 
\parit{Comments.} The pseudoholomorphic curves (sometimes called \emph{gromomorphic curves}) 
were introduced by Mikhael Gromov in \cite{gromov-pseudoholomorphic}.
For general metric the form $\omega$ might vanish at some points;
if the metric is generic then it happens on disjoint circles,
see \cite{honda}.



%%%%%%%%%%%%%%%%%%%%%%%%%%%%%%%%%%%%%%%%%%%%%%%%%%
\parbf{\ref{Energy minimizer}.} 
\textit{Energy minimizer.}
Denote by $\mathrm{U}$ the unit tangent bundle over $\RP^m$
and $\mathrm{L}^{2\cdot(m-1)}$ the space of projective lines in $\ell\:\RP^1\to \RP^m$.
The spaces $\mathrm{U}$ and $\mathrm{L}$ 
have dimesions $2\cdot m-1$ 
and $2\cdot(m-1)$
correspondingly.


According to Liouville's theorem, the identity
\[\int\limits_{\mathrm{U}}f(v)\cdot d_v\vol_{2\cdot m-1}
=
\int\limits_{\mathrm{L}}d_\ell\vol_{2\cdot(m-1)}\cdot\int\limits_{\RP^1} f(\ell'(t))\cdot dt\]
holds for any integrable function $f\:\mathrm{U}\to\RR$.

Let $F\:\RP^m\to\RP^m$ be a smooth map.
Note that up to multiplicative constant,
the energy of $F$ can be expressed the following way
\[\int\limits_{\mathrm{U}} |dF(v)|^2\cdot d_v\vol_{2m-1}
=
\int\limits_{\mathrm{L}}d_\ell\vol_{2\cdot(m-1)}\cdot\int\limits_{\RP^1} |(F\circ \ell)(t)|^2\cdot dt\].

The result follows since
\[\int\limits_{\RP^1} |(F\circ \ell)(t)|^2\cdot dt\ge \pi\]
for any line $\ell\:\RP^1\to \RP^m$.

\parit{Comments.} 
The same idea is used to prove Loewner's inequality on the volume in of $\RP^m$ with metric conformally equivalent to the canonical one, see \cite{gromov-filling}.

The problem is due to Cristopher Croke, see \cite{croke-energy}. 
He use the same idea to show that the identity map on $\CP^m$ is energy minimizing in its homotopy class.
For $\mathbb S^m$, an analogous statement does not hold if $m\ge 3$.
In fact if a closed Riemannian manifold $M$ 
has dimension at least 3 
and $\pi_1M=\pi_2M=0$ 
then the identity map on $M$ is homotopic 
to a map with arbitrary small energy;
the later was shown by Brian White in \cite{white}.



%%%%%%%%%%%%%%%%%%%%%%%%%%%%%%%%%%%%%%%%%%%%%%%%%%
\parbf{\ref{Volume and convexity}.} 
\textit{Volume and convexity.}
Assume contrary; that is, there is a complete Riemannian manifold $M$
with finite volume which admits a convex function $f$.

Denote by $\tau\:\mathrm{T}^1 M\to M$ the unit tangent bundle over $M$. 
Clearly $\vol T^1M$ is finite.

Note that 
there is a nonempty bounded open set $U\subset \mathrm{T}^1 M$
such that $df(u)>\eps$ for any $u\in U$ and some fixed $\eps>0$.

Denote by $\phi^t$ the geodesic flow on $\mathrm{T}^1 M$.
Given $u\in U$,
consider the function $h\:t\mapsto f\circ\tau\circ\phi^t(u)$.
Note that $h'(t)>\eps$ for any $t\ge 0$.

Prove that there is an infinite sequence of positive reals $t_1,t_2,\dots$
such that 
$$\phi^{t_i}(U)\cap\phi^{t_j}(U)=\emptyset$$ 
if $i\ne j$.
The later implies that $\vol T^1M=\infty$,
a contradiction.

\parit{Comment.} The idea in the proof is essentially 
Poincar\'e recurrence theorem.

The problem is due to Shing-Tung Yau, see \cite{yau};
for smooth functions the statement was proved by Richard Bishop and Barrett O'Neill in \cite{bishop-o'neill}.


%%%%%%%%%%%%%%%%%%%%%%%%%%%%%%%%%%%%%%%%%%%%%%%%%%
\parbf{\ref{Besikovitch inequality}.} 
\textit{Besikovitch inequality.}
Set 
\[A_i=\set{(x_1,x_2,\dots,x_m)\in[0,1]^m}{x_i=0}.\]

Consider functions $f_i\:[0,1]^m\to\RR$ defined by
$f_i(x)=\dist_{A_i}x$.
Note that 
the map $\bm{f}\:([0,1]^m,g)\to\RR^m$
defined as
\[\bm{f}\:x\mapsto(f_1(x),f_2(x),\dots,f_m(x))\]
is Lipschitz.

Prove that Jacobian of  $\bm{f}$
is at most $1$
and $\bm{f}([0,1]^m)\supset [0,1]^m$.
Hence the result follows.

It remains to do the equality case.

\parit{Comments.} 
The inequality was proved by Abram Besicovitch in \cite{besicovitch}.
It has number applications in Riemannian geometry.
For example using this inequality it is easy to solve the following problem.

Assume a metric $g$ on $\RR^m$ coincides with Euclidean outside of a bounded set $K$;
assume further that any geodesic which comes into $K$ goes out from $K$ the same way as if the metric would be Euclidean everwhere. 
Show that $g$ is flat.


%%%%%%%%%%%%%%%%%%%%%%%%%%%%%%%%%%%%%%%%%%%%%%%%%%
\begin{wrapfigure}{o}{32mm}
%\begin{center}
\begin{lpic}[t(-5mm),b(3mm),r(0mm),l(0mm)]{pics/tripod}
%\lbl[b]{24,41;x}
\end{lpic}
%\end{center}
\end{wrapfigure}

\parbf{\ref{Distant involution}.} 
\textit{Distant involution.}
Given $\eps>0$, construct a disc $D$ in the plane with 
$$\length\partial D<10\ \ \text{and}\ \ \area D<\eps$$
which admits an continuous involution $\iota$ such that 
$$|\iota(x)-x|\ge 1$$ 
for any $x\in\partial D$.
An example of $D$ can be guessed from the picture. 

Take the product $D\times D\subset \RR^4$;
it is homeomorphic to a 4-dimensional ball.
Note that 
$$\vol_3[\partial(D\times D)]=2\cdot\area D\cdot\length \partial D<20\cdot\eps.$$
The boundary $\partial(D\times D)$ homeomorphic to $\mathbb{S}^3$
and the restriction of the involution $(x,y)\mapsto (\iota(x),\iota(y))$ has the needed property.

It remains to smooth $\partial(D\times D)$.

\parit{Comments.}
This example is given by Cristopher Croke in \cite{croke}.

It is instructive to show that for $\mathbb{S}^2$ such thing is not possible.

Note that according to systolic inequality, 
the involution $\iota$ above can not be made isometric (see \cite {gromov-filling}).



 
%%%%%%%%%%%%%%%%%%%%%%%%%%%%%%%%%%%%%%%%%%%%%%%%%%
\parbf{\ref{Normal exponential map}.} 
\textit{Normal exponential map.}
Assume contrary; that is, there is a point $p\in M$ 
such that the image of normal exponential map to $N$
 does not touch $\eps$-neighborhood of $p$.

Show that given $R>0$ there is $\delta>0$ such that 
if $x\in N$ and $|p-x|_M<R$ 
then there is a unit speed curve in $N$
which moves to $p$ with velocity at least $\delta$.
(In fact, the value $\delta$ depends on $\eps$, $R$ and the curvature bounds in $B(p,R)$.)

\begin{wrapfigure}{o}{25mm}
\begin{lpic}[t(-0mm),b(0mm),r(0mm),l(0mm)]{pics/spiral}
\end{lpic}
\end{wrapfigure}

Following this curve for sufficient time brings us to $p$;
that is, $p\in N$, a contradiction.

\parit{Comments.} 
The problem was suggested by Alexander Lytchak.

From the picture, you should guess an example of immersion 
$\iota\:\RR\looparrowright\RR^2$ 
such that one point does not lie in the image of the corresponding normal exponential.
It might be interesting to see in more details 
which sets can be avoided by such images.



%%%%%%%%%%%%%%%%%%%%%%%%%%%%%%%%%%%%%%%%%%%%%%%%%%
\parbf{\ref{Symplectic squeezing in the torus}.} 
\textit{Symplectic squeezing in the torus.}
Equip $\RR^4$ with $(x_1,y_1,x_2,y_2)$-coordinates
so that 
\[\omega=dx_1\wedge dy_1+dx_2\wedge dy_2\]
is the symplectic form. 

The embedding will be given as a composition of a linear symplectomorphism $\lambda$ 
with the quotient map $\phi\:\RR^4\to \TT^2\times\RR^2$ by the integer $(x_1,y_1)$-lattice.
Clearly $\phi\circ\lambda$ preserves the symplectic structure,
it remains to find $\lambda$ such that the restriction $\phi\circ\lambda|_\Omega$
is injective.

Without loss of generality,
we can assume that $\Omega$ is a ball centered at the origin.
Choose an oriented 2-dimensional subspace $V$ subspace of $\RR^4$ 
such that the integral of $\omega$ over 
$\Omega\cap V$ is small positive number, say $\tfrac\pi4$. 

Note that there is a linear symplectomorphism $\lambda$
 which maps planes parallel to $V$ to planes
parallel to the $(x_1,y_1)$-plane, 
and that maps the disk $V\cap\Omega$ to a disk.
It follows that the the intersection of $\lambda(\Omega)$ 
with any plane parallel to the $(x_1,y_1)$-plane is a disk of radius at most $\tfrac 12$.
In particular $\phi\circ\lambda|_\Omega$
is injective.

\parit{Comments.}
This construction is given 
by Larry Guth in \cite{guth-symplectic}
and attributed to Leonid Polterovich.

Note that according to the Gromov's non-squeezing theorem in \cite{gromov-pseudoholomorphic}, 
an analogous statement with $\CC\times \DD$ as the target does not hold, here $\DD\subset \CC$ is the open disc with incuced symplectic structure.
In particular, it shows that
the projection of $\lambda(\Omega)$ as above 
to $(x_1,y_1)$-plane
can not be made arbitrary small.

%%%%%%%%%%%%%%%%%%%%%%%%%%%%%%%%%%%%%%%%%%%%%%%%%%
\parbf{\ref{Diffeomorphism test}.} 
\textit{Diffeomorphism test.}
Since $N$ is simply connected, 
it is sufficient to show that $f\:M\to N$ is a covering map.

Note that $f$ is an open immersion.
Let $h$ be the pullback metric on $M$ for $f\:M\to N$.
Clearly $h\ge g$.
In particular $(M,h)$ is complete and the map $f\:(M,h)\to N$ is a local isometry. 

It remains to prove that any local isometry between complete connected Riemannian manifolls of the same dimension if a covering map. 

%???\parit{Comments.} 

%%%%%%%%%%%%%%%%%%%%%%%%%%%%%%%%%%%%%%%%%%%%%%%%%%
\parbf{\ref{Volume of tubular neighborhoods}.} 
\textit{Volume of tubular neighborhoods.}
Let us denote by $\mathrm{N} M$ and $\mathrm{T} M$ the normal and tangent bundle of $M$ in $\RR^m$.

Consider the the normal exponential map $\exp_M\:\mathrm{N} M\to\RR^m$
and denote by $J_V$ its Jacobian at $V\in \mathrm{N}_pM$.
Note that for all small $\eps>0$, we have
\[\vol B_\eps(M)=\int\limits_M d_p\vol_m\cdot\int\limits_{B(0,r)_{\mathrm{N}_pM}}J_V\cdot d_V\vol_{n-m}.\leqno{({*})}\]

Set $m=\dim M$.
Given $p\in M$, 
denote by $s_p\:\mathrm{T}_p\times \mathrm{T}_p\to \mathrm{N}_p$
the \hyperref[Second fundamental form]{second fundamental form} of $M$.
Recall that the curvature tensor of $M$ at $p$ can be expressed the following way
\[R_p(X\wedge Y), V\wedge W\rangle 
=\langle s_p(X,W), s_p(Y,V)\rangle-\langle s_p(X,V), s_p(Y,W)\rangle.\]

Given $V\in \mathrm{N}_p M$,
express $J_V$ in terms of $\<s_p(X,Y),V\>$.
Show that for small $r$ the integral
\[v(r)=\int\limits_{B(0,r)_{\mathrm{N}_pM}}J_V\cdot d_V\vol_{n-m}\]
is a polynomial 
of $r$ and its coefficients can be expressed in terms of the curvature tensor $R_p$.

It follows that the right hand side in $({*})$ can be expressed in terms of curvature tensor of $M$.
The problem follows since the curvature tensor can be expressed in terms of metric tensor of $M$.



\parit{Comments.} The formula for volume of tubular neighborhood 
was given by Hermann Weyl in \cite{weyl}.


\parbf{\ref{Disc}.} 
\textit{Disc.}
Show that given a positive integer $n$ one can construct a tree $T$ embedded into the disc such that any homotopy of the boundary of the disc to a point pass through a curve which intersects $n$ different edges.
(For the tree on the diagram $n=3$.)

%%%%%%%%%%%%%%%%%%%%%%%%%%%%%%%%%%%%%%%%%%%%%%%%%%
{
\begin{wrapfigure}{o}{42mm}
%\begin{center}
\begin{lpic}[t(-5mm),b(-3mm),r(0mm),l(0mm)]{pics/tree(1)}
%\lbl[b]{24,41;x}
\end{lpic}
%\end{center}
\end{wrapfigure}

Fix small $\eps>0$, say $\eps=\tfrac1{10}$.
Consider the disc with embedded tree $T$ as above.
We will construct a metric on the disc 
with diameter and length of its boundary below $1$
such that 
the distance between any two edges of $T$ of without common vertex 
is at least $\eps$ .

To construct such a metric, first fix a metric on the cylinder $\mathbb S^1\times [0,1]$ such that

}

\begin{itemize}
\item The $\eps$-neighborhoods of the boundary components are product metrics.
\item Any vertical sigment $x\times[0,1]$ has length $\tfrac 12$.
\item One of the boundary component has length $\eps$.
\item The other boundary component has length $2\cdot m\cdot \eps$, 
where $m$ is the number of edges in $T$.
\end{itemize}
Equip $T$ with a length-metric so that each edge has length $\eps$
and glue the long boundary component of the cylinder to $T$ by piecewise isometry so that the resulting space is homeomorphic to disc and the tree corresponds to it-self.



According to the first construction,
for any null-homotopy of the boundary 
the least length is at least $n\cdot\tfrac{\eps}{10}$.
The obtained metric is not Riemannian, but is is easy to smooth.
Since $n$ is arbitrary the result follows.

\parit{Comments.} 
This example was constructed by Sidney Frankel and Mikhail Katz in \cite{frankel-katz}.
 

%%%%%%%%%%%%%%%%%%%%%%%%%%%%%%%%%%%%%%%%%%%%%%%%%%
\parbf{\ref{short-homotopy}.} 
\textit{Shortening homotopy.}
Set 
\[p=\gamma_0(0)\ \ \text{and}\ \  \ell_0=\length\gamma_0.\]

By compactness argument,
there exists $\delta>0$ 
such that no geodesic loops based at $p$ with has length in the interval $(L-D, L+D+\delta]$. 

Assume $\ell_0\ge L+\delta$.
Choose $t_0\in [0,1]$ such that
\[\length\left(\gamma_0|_{[0,t_0]}\right)=L+\delta\]
Let $\sigma$ be a the minimizing geodesic from $\gamma(t_0)$
to $p$.
Note that $\gamma_0$ is homotopic to the joint 
\[\gamma_0'=\gamma_0|_{[0,t_0]}*\sigma*\bar\sigma*\gamma|_{[t_0,1]},\]
where $\bar\sigma$ denotes the backward parametrization of $\sigma$.

Consider the loop $\lambda_0$ at $p$
formed by joint of $\gamma|_{[0,t_0]}$ and $\sigma$.
Applying a curve shortening process to $\lambda_0$, 
we get a curve shortening homotopy $\lambda_t$
rel. its ends 
from the loop $\lambda_0$ to a geodesic loop $\lambda_1$ at $p$.
From above, 
\[\length\lambda_1\le L-D.\]

The joint $\gamma_t=\lambda_t*\bar\sigma*\gamma|_{[t_0,1]}$
is a homotopy
from $\gamma_0'$ to an other curve $\gamma_1$.
From the construction it is clear that 
\begin{align*}
 \length \gamma_t&\le \length \gamma_0+2\cdot \length\sigma\le
 \\
 &\le \length \gamma_0+2\cdot D
\end{align*}
for any $t\in[0,1]$
and 
\begin{align*}
 \length \gamma_1&=\length\lambda_1+\length\sigma+\length\gamma|_{[t_0,1]}\le
\\ &\le L-D+D+\length\gamma-(L+\delta)=
\\ &=\ell_0 -\delta.
\end{align*}
Repeating the procedure few times we get we get curves $\gamma_2$, $\gamma_3,\dots,\gamma_n$
joint by the needed homotopies so that 
$\ell_{i+1}\le\ell_i-\delta$ and $\ell_n< L+\delta$,
where $\ell_i=\length\gamma_i$.

If $\ell_n\le L$, we are done.
Otherwise repeat the argument once more for $\delta'=\ell_n-L$.

\parit{Comments.}
The problem is due to 
Alexander Nabutovsky 
and Regina Rotman,
see \cite{nabutovsky-rotman}.

It is not at all easy to find an example of a manifold  which satisfy the above condition for some $L$;
they are found among the Zoll spheres
by Florent Balachev, Christopher Croke and Mikhail Katz, 
see \cite{balacheff-croke-katz}.

%%%%%%%%%%%%%%%%%%%%%%%%%%%%%%%%%%%%%%%%%%%%%%%%%%
\parbf{\ref{Convex hypersurface}.} 
\textit{Convex hypersurface.}
Let $h$ be the maximal distance from points in $W$ to $M$.

Fix a fine triangulation of $W$ 
so that $M$ becomes a subcomplex.
Say, let us assume that the diameter of each simplex in $\tau$ is less than 
$\eps$.
We can assume that $\tau$ is a barycentric subdivision of an other triangulation, so all the vertices of $\tau$ can be colored into colors $(0,\dots, m+1)$
in such a way that the vertices of each simplex 
get different colors.
Denote by $\tau_i$ the maximal $i$-dimensional subcomplex of $\tau$ 
with all the vertices colored by $0,\dots, i$.

For each vertex $v$ in $\tau$ 
choose a point $v'\in M$ on the distance $\le h$.
Note that if $v$ and $w$ are the vertices of one simplex then
\[|v'-w'|_M<2\cdot h+\eps.\]

If $\tfrac{r}{2\cdot(m+1)}>h$, take $\eps<\tfrac{r}{2\cdot(m+1)}-h$.
Let us extend the map $v\mapsto v'$ 
to a continuous 
map $W\to M$.
The map is already defined on $\tau_0$.
Using the cone construction we can extend it to $\tau_1$;
we can do this since the distance between vertices in one simplex are below injectivity radius of $M$.
Repeat the cone construction recursively, to extend the map to $\tau_2,\dots,\tau_{m+1}=\tau$;
some distance estimates are needed here.

It follows that fundamental class of $M$ vanish in the homology ring of $M$, 
a contradiction.  

\parit{Comment.}
This problem is a stripped version of the bound on filling radius given by Mikhael Gromov in \cite{gromov-filling}.  



\section*{Metric geometry}



%%%%%%%%%%%%%%%%%%%%%%%%%%%%%%%%%%%%%%%%%%%%%%%%%%
\parbf{\ref{Noncontracting map}.} 
\textit{Noncontracting map.}
Given any pair of point $x_0,y_0\in K$, 
consider two sequences $x_0,x_1,\dots$ and $y_0,y_1,\dots$
such that 
and $x_{n+1}=f(x_n)$ and $y_{n+1}=f(y_n)$ for each $n$.

Since $K$ is compact, 
we can choose an increasing sequence of integers $n_k$
such that both sequences $(x_{n_i})_{i=1}^\infty$ and $(y_{n_i})_{i=1}^\infty$
converge.
In particular, both of these sequences  are Cauchy;
that is,
\[
|x_{n_i}-x_{n_j}|_K, |y_{n_i}-y_{n_j}|_K\to 0
\ \ 
\text{as}
\ \ \min\{i,j\}\to\infty.
\]


Since $f$ is noncontracting, we get
\[
|x_0-x_{|n_i-n_j|}|
\le 
|x_{n_i}-x_{n_j}|.
\]

It follows that  
there is a sequence $m_i\to\infty$ such that
\[
x_{m_i}\to x\ \ \text{and}\ \ y_{m_i}\to y\ \ \text{as}\ \ i\to\infty.
\leqno({*})\]

Set \[\ell_n=|x_n-y_n|_K,\]
where $|{*}-{*}|_K$ denotes the distance between points in $K$.
Since $f$ is noncontracting, $(\ell_n)$ is a nondecreasing sequence.

By $({*})$, it follows that $\ell_{m_i}\to\ell_0$ as $m_i\to\infty$.
It follows that $(\ell_n)$ is a constant sequence.

In particular 
\[|x_0-y_0|_K=\ell_0=\ell_1=|f(x_0)-f(y_0)|_K\]
for any pair of points $(x_0,y_0)$ in $K$.
I.e., $f$ is distance preserving, in particular injective.

From $({*})$, we also get that $f(K)$ is everywhere dense.
Since $K$ is compact $f\:K\to K$ is surjective. Hence the result follows.

\parit{Comment.}
This is a basic lemma in the introduction to Gromov--Hausdorff distance;
see for example \cite{bbi}*{7.3.30}.
The proof presented here was given by Travis Morrison, when he was a students at MASS program at Penn State (Fall 2011).



%%%%%%%%%%%%%%%%%%%%%%%%%%%%%%%%%%%%%%%%%%%%%%%%%%
\parbf{\ref{compact}.} 
\textit{Embedding of a compact.}
Let $K$ be a compact metric space.
Denote by $B(K)$ the space of bounded functions on $K$
equipped with sup norm; 
that is, 
\[|f|=\sup_{x\in K}|f(x)|.\]

Note that the map $\phi\:K\to B(K)$, defied by $x\mapsto \dist_x$
is a distance preserving embedding.

Denote by $W$ the linear convex hull of the image $\phi(K)\subset B(K)$ with the metric induced from $B(K)$.
It remains to show that $W$ forms a compact length-metric space.

\parit{Comment.}
The map $\phi$ is called \emph{Kuratowski embedding},
it was constructed in \cite{kuratowski},
although essentially the same map 
was described by Maurice Fr\'echet 
in the same paper he introduced metric spaces, see \cite{frechet}.

%%%%%%%%%%%%%%%%%%%%%%%%%%%%%%%%%%%%%%%%%%%%%%%%%%
\parbf{\ref{Metric compactification}.}
\textit{Metric compactification.}
Take $X$ to be the set of nonnegative integers with the metic $\rho$ defined as 
$\rho(m,n)=m+n$ for $m\ne n$.

Assume $F_X$ is not an embedding.
Then there is a sequence of points $z_1,z_2,\dots$ 
and a point $z_\infty$,
such that $f_{z_n}\to f_{z_\infty}$ in $C(X,\RR)$
as $n\to \infty$, 
while $|z_n-z_\infty|_X>\eps$ 
for some fixed $\eps>0$ and all $n$.

If $X$ is geodesic we can choose $\bar z_n$ on the a geodesic $[z_\infty z_n]$ such that $|z_\infty-z_n'|=\eps$.
Note that 
\begin{align*}
f_{z_n}(z_\infty)-f_{z_n}(\bar z_n)&=\eps
\intertext{and}
f_{z_\infty}(z_\infty)-f_{z_n}(\bar z_n)&=-\eps
\end{align*}
for any $n$.
In particular $f_{z_n}\not\to f_{z_\infty}$ in $C(X,\RR)$,
a contradiction.

\parit{Comments.}
Denote by $\bar X$ 
the closure of $F_X(X)$ in $C(X,\RR)$;
note that $\bar X$ is compact.
That is, $F_X$ is an embedding 
then $\bar X$ forms a compactification of $X$,
which is called \emph{metric compactification}.
The complement 
$\partial_\infty X=\bar X\backslash F_X(X)$ 
is called \emph{metric absolute} of $X$.

The compactifiacation
and absolute
was inroduced by Mikhael Gromov in \cite{gromov-hyperbolic}.
The problem above was sugested by Linus Kramer.



%%%%%%%%%%%%%%%%%%%%%%%%%%%%%%%%%%%%%%%%%%%%%%%%%%
\parbf{\ref{2-sphere is far from a ball}.} 
\textit{Disc and 2-sphere.}
Assume contrary, let $(\mathbb{S}^2,g)$ is sufficiently close to $B^2$.

Choose a closed simple curve $\gamma$ in $\mathbb{S}^2$ which is close to the boundary of $B^2$.
Choose two points $p_1$ and $p_2$ in $\mathbb{S}^2$ 
on the opposite sides of $\gamma$ which are sufficiently close to the center of $B^2$.

On one had $p_1$ and $p_2$ have to be close in $\mathbb{S}^2$.
On the other hang, to get from $p_1$ to $p_2$ in $\mathbb{S}^2$,
one has to cross $\gamma$.
Hence the distance from $p_1$ to $p_2$ in $\mathbb{S}^2$ has to be about $2$,
a contradiction.

\parit{Comment.}
In fact if $X$ is a Gromov--Hausdorff limit of $(\mathbb{S}^2,g_n)$
then any point $x_0\in X$ either admits a neighborhood homeomorphic to $\RR^2$ or it is a cut point;
that is $X\backslash\{x_0\}$ is disconnected; see \cite{gromov-MetStr}*{3.32}.

%%%%%%%%%%%%%%%%%%%%%%%%%%%%%%%%%%%%%%%%%%%%%%%%%%
\parbf{\ref{3-sphere is close to a ball}.} 
\textit{Ball and 3-sphere.}
Make fine burrows in the standard 3-ball which do not change its topology,
but at the same time a come sufficiently close to any point in the ball.

Consider the doubling of obtained ball in its boundary.
Clearly the obtained space is homeomorphic to $\mathbb{S}^3$.
Prove that the burrows can be made 
so that it is sufficiently close to the original ball 
in the Gromov--Hausdorff metric.

It remains to smooth the obtained space slightly 
to get a genuine Riemannian metric with needed property.

\parit{Comment.}
This construction is a stripped version of theorem proved by Steven Ferry and Boris Okunin in \cite{ferry-okun}.
The theorem states that Riemannian metrics on a smooth closed manifold $M$ with $\dim M\ge 3$ 
can approximate given compact length-metric space $X$ 
if and only if 
there is a continuous map $M\to X$
which is surjective on the fundamental groups. 

%%%%%%%%%%%%%%%%%%%%%%%%%%%%%%%%%%%%%%%%%%%%%%%%%%
\parbf{\ref{macrodimension}.} 
\textit{Macrodimension.}
Choose a point $p\in M$,
denote by $f$ the distance function from $p$.

Let us cover $M$ by the connected components of the preimages 
$f^{-1}((n-1,n+1))$.
Clearly any point in $M$ is covered by at most two such components.
It remains to show that each of these components has diameter less than $100$.

Assume contrary; let $x$ and $y$ be two points in such connected component 
and $|x-y|_M\ge 100$.
Connect $x$ to $y$ by a curve $\tau$ in the component.
Consider the closed curve $\sigma$ formed by two geodesics $[px]$, $[py]$ and $\tau$.

Prove that $\sigma$ can be divided into 4 arcs $\alpha$, $\beta$, $\gamma$ and $\delta$
in such a way that the minimal distance from $\alpha$ to $\gamma$ as well as the minimal distance from $\beta$ to $\delta$ is at least $10$.

Use the last statement to show that $\sigma$ 
can not be shrank 
by a disc it its $1$-neighborhood;
the later contradicts the assumption.

\parit{Comment.}
The problem was discussed it a talk by Nikita Zinoviev around 2004.

%%%%%%%%%%%%%%%%%%%%%%%%%%%%%%%%%%%%%%%%%%%%%%%%%%
\parbf{\wrenches\ref{anti-collaps}.} 
\textit{Anti-collapse.}
Fix a decreasing sequence $\eps_0,\eps_1,\dots$ of positive numbers converging to $0$ as $n\to \infty$.

Let $T$ be the infinite binary tree
and $T_n\subset T$ be the subtree up to level $n$.
Let us equip $T$ with the length-metric such that the edges coming from $(n-1)$-th level to $n$-th level have length $\eps_{n-1}-\eps_n$.

Denote by $\bar T$ the completion of $T$.
Note that $C=\bar T\backslash T$ is a Cantor set;
The set $C$ can be identified with the set of $\{0,1\}$-sequences 
with the distance between two sequences $\bm{x}=(x_0,x_1,\dots)$ and $\bm{y}=(y_0,y_1,\dots)$ defined as $\eps_n$, where $n$ is the least number such that $x_n\ne y_n$.

Choosing $\eps_n$ one can make $C$ to have arbitrary large Hausdorff dimension.

Now choose $\delta_n\ll\eps_n$ and prepare for each edge of $T$ a cilinder with hight ... and radius of the base $\delta_n$.
 

Note that there is a natural embedding $T_{n}\to T_{n+1}$ for all $n$.


Assume $S$ be a surface with a flat disc $D_0\subset S$ of radius $r_0$.
Let us cut from $D_0$ two discs $D_1$ and $D_1'$ 
of radii $r_1=r_0/10$ and glue instead a cylinder with high $\eps_0$ 
with discs on the top.
Now repeat the operation for each $D_1$ and $D_1'$ cutting 
from each two discs of radius $r_2=r_1/10$
and glue instead cylinder with high $\eps_0$ 
with discs on the top.
Continue the process, we get an increasing sequence of Riemannian metrics on $S$.
%???PIC

\parit{Comments.}
The problem is due
Dmitri Burago, 
Sergei Ivanov 
and David Shoenthal,
see \cite{BIS}.

%%%%%%%%%%%%%%%%%%%%%%%%%%%%%%%%%%%%%%%%%%%%%%%%%%
\parbf{\ref{weird-metric}.} 
\textit{No short embedding.}
Consider a chain of disjoint circles $c_0,c_1,\dots,c_n$ in $\RR^3$;
that is, $c_i$ and $c_{i-1}$ are linked for each $i$. 

%\begin{wrapfigure}{o}{50mm}
\begin{center}
\begin{lpic}[t(-0mm),b(0mm),r(0mm),l(0mm)]{pics/chain(1)}
\lbl[t]{5,0;$c_0$}
\lbl[t]{11,0;$c_1$}
\lbl[t]{31,0;$\cdots$}
\lbl[t]{54,0;$c_n$}
\end{lpic}
\end{center}
%\end{wrapfigure}

Assume that $\RR^3$ is equipped with a length-metric $d$,
such that the total length of the circles is $\ell$
and $U$ is an open set containing all the circles $c_i$.
Note that for any short homeomorphism $f\:(U,d)\to\RR^3$ the distance from $f(c_0)$ to $f(c_n)$ is less than $\ell$.

Fix a line segment $[ab]$ in $\RR^3$.
Modify 
the length-metric on $\RR^3$ in arbitrary small neighbohood of $[ab]$
so that there is a chain $(c_i)$ of circles as above,
which goes from $a$ to $b$ 
such that
(1) the total length, say $\ell$, 
of $(c_i)$ is arbitrary small,
but 
(2) the obtained metric $d$ 
is arbitrary close to the canonical, say
\[\bigl|d(x,y)-|x-y|\bigr|<\eps\]
for any two points $x,y\in\RR^3$
and fixed in advanced small $\eps>0$.
The construction of $d$ 
is done by shrinking the length of each circle
and expanding the length in the normal diections  
to the circles in their small neigbourhood.
The later is made in order to make impossible to use the circles $c_i$ as a shortcut;
that is, one spends more time to go from one circle to an other 
than saves on going along the circle.

Set $a_n=(0,\tfrac1n,0)$ and $b_n=(1,\tfrac1n,0)$.
Note that the line segmments $[a_nb_n]$ are disjoint and converging
to $[a_\infty b_\infty]$
where $a_\infty=(0,0,0)$ and $b_\infty=(1,0,0)$.

Apply the above construction in nonoverlaping convex neighborhoods of $[a_nb_n]$ 
and for a sequences 
$\eps_n$ and $\ell_n$ 
which converge to zero very fast.

The obtainded length-metric $d$ is still close to the canonical,
but for any open set $U$ containing $[a_\infty b_\infty]$
the space $(U,d)$ does not admit 
a short homeomorphism to $\RR^3$.
Indeed, if such homeomorphism $h$
exists then 
from the above construction,
we get 
\begin{align*}
|h(a_\infty)-f(b_\infty)|
&\le 
|h(a_n)-f(b_n)|
+
\\
&\ \ \ \ \ +
|h(a_\infty)-f(a_n)|
+
|h(b_n)-f(b_\infty)|
\le
\\
&\le
\ell_n+\tfrac2n+100\cdot\eps_n.
\end{align*}
The right hand side converges to $0$ as $n\to\infty$.
Therefore 
\[h(a_\infty)=f(b_\infty),\] 
a contradiction.

It remains to performs similar construction countably many times so a bad segment as $[a_\infty b_\infty]$ above
appears in any open set of $\RR^3$.


\parit{Comments.}
The problem is due
Dmitri Burago, 
Sergei Ivanov 
and David Shoenthal,
see \cite{BIS}.

%%%%%%%%%%%%%%%%%%%%%%%%%%%%%%%%%%%%%%%%%%%%%%%%%%
\parbf{\ref{sub-Riemannian}.} 
\textit{Sub-Riemannian sphere.}
Prove that there is a nondecreasing sequence of Riemannian metric tensors
$g_0\le g_1\le ...$ such that the induced metrics converge to the given sub-Riemannian metrics.
The metric $g_0$ can be assumed to be a metric on round sphere.

Applying the construction as in Nash--Kuiper theorem,
one can produce a sequence of smooth embedings $h_n\:\mathbb{S}^m\to \RR^{m+1}$ with the induced metrics $g_n'$
such that $|g_n'-g_n|\to 0$.

Moreover, assume we assign a positive real number $\eps(h)$ for any smooth embedding $h\:\mathbb{S}^m\to\RR^{m+1}$.
Then we can assume that 
\[|h_{n+1}(x)-h_n(x)|<\eps(h_n)\] for any $x\in \mathbb{S}^m$ and $n$.

Show that for a right choice of function $\eps(h_n)$,
the sequence $h_n$ converges, say to $h_\infty$, 
and the metric induced by $h_\infty$ coincides with the given sub-Riemannian metric.

\parit{Comments.} 
The original papers of John Nash \cite{nash} 
and Nicolaas Kuiper \cite{kuiper} are very readable.

The problem appeared 
on this list first rediscovered later by Enrico Le Donne in \cite{le-donne}.
Similar construction described in the the lecture notes by Allan Yashinski and me \cite{petrunin-yashinsky} 
which is aimed for undergraduate students. 
Yet my paper \cite{petrunin-paths} is closely relevant.

%%%%%%%%%%%%%%%%%%%%%%%%%%%%%%%%%%%%%%%%%%%%%%%%%%
\parbf{\ref{two2one}.} 
\textit{Length-preserving map.}
Assume there is a length-preserving map $f\:\RR^2\to \RR$.

Note that $f$ is Lipschitz.
Therefore by Rademacher's theorem, $f$ is differentiable almost everywhere.

Fix a unit vector $u$.
Prove that, for almost all $x$, the length of curve 
$\alpha\:t\mapsto x+t\cdot u$, $t\in[0,1]$ can be expressed as the integral
\[\int\limits_0^1 (d_{\alpha(t)}f)(u) \cdot dt.\]

It follows that $|d_xf(v)|=|v|$ for almost all $x,v\in\RR^2$;
in particular $d_xf$ is defined and has rank 2 at some point $x$, a contradiction.  

\parit{Comment.} The Rademacher's theorem appears in \cite{rademacher}.
The idea above can be also used to solve the following problem.

{\it Assume $d$ is a metric on $\RR^2$ 
which is induced by a norm.
Show that $(\RR^2,d)$ admits 
a \hyperref[Length-preserving map]{\emph{length-preserving map}} 
to $\RR^3$ 
if and only if 
$(\RR^2,d)$ is isometric to the Euclidean plane.}



%%%%%%%%%%%%%%%%%%%%%%%%%%%%%%%%%%%%%%%%%%%%%%%%%%
\parbf{\ref{Hyperbolic space}.} 
\textit{Hyperbolic space.}
Note that $2$-dimensional hyperbolic space 
can be viewed as $(\RR^2,g)$, where 
\[g(x,y)=\left(\begin{matrix}
     1&0
     \\
     0&e^{x}
    \end{matrix}\right).\]
The same way $3$-dimensional hyperbolic space 
can be viewed as $(\RR^3,h)$, where 
where 
\[h(x,y,z)=\left(\begin{matrix}
     1&0&0
     \\
     0&e^{x}&0
     \\
     0&0&e^{x}
    \end{matrix}\right).\]

Prove that the map $\RR^3\to \RR^4$ defined as
$$(x,y,z)\mapsto (x,y,x,z)$$
is a quasi-isometry from $(\RR^3,h)$ to its image in $(\RR^2,g)\times (\RR^2,g)$.

\parit{Comments.}
In the proof we used that horosphere in the hyperbolic space is isometric to the Euclidean plane.
This observation appears already in the book of Nikolai Lobachevsky, see \cite{lobachevsky}*{34}.



%%%%%%%%%%%%%%%%%%%%%%%%%%%%%%%%%%%%%%%%%%%%%%%%%%
\parbf{\ref{Fixed segment}.} 
\textit{Fixed segment.}
Note that it is sufficient to show that if 
\[f(a)=a\ \ \text{and}\ \ f(b)=b\]
for some $a,b\in\RR^m$
then 
\[f(\tfrac{a+b}2)=\tfrac12\cdot(f(a)+f(b)).\]

(This statement is not trivial since in general
metric midpoint of $a$ and $b$ in $(\RR^m,d)$ 
are not defined uniquely.)

Without loss of generality, we can assume that $b+a=0$.

Set $f_0=f$.
Consider the recursively defined sequence of isometries $f_0$, $f_1,\dots$ defined recursively
\[f_{n+1}(x)= -f_n^{-1}(-f_n(x)).\]

Note that $f_n(a)=a$ and $f_n(b)=b$ for any $n$ and 
$$|f_{n+1}(0)|=2\cdot|f_n(0)|.$$
The later condition implies that if $f(0)\ne 0$
then $|f_n(0)|\to\infty$ as $n\to\infty$.
On the other hand, since $f_n$ is isometry and $f(a)=a$,
we get $|f_n(0)|\le 2\cdot |a|$, a contradiction.

\parit{Comment.}
The problem is a stripped version of Mazur--Ulam theorem proved in  \cite{mazur-ulam};
it states that any isometry of $(\RR^m,d)$ to itself 
forms an affine map. 

The solution above
is the main step in the proof of this theorem 
given by Jussi V\"ais\"al\"a's in \cite{vaisala}.

%%%%%%%%%%%%%%%%%%%%%%%%%%%%%%%%%%%%%%%%%%%%%%%%%%

\parbf{\ref{Pogorelov's construction}.} 
\textit{Pogorelov's construction.}
Positivity and symmetry of $\rho$ is evident.

The triangle inequality follows since
\[[B(x,\tfrac \pi2)\backslash B(y,\tfrac\pi2)]
\cup 
[B(y,\tfrac\pi2)\backslash B(z,\tfrac\pi2)]
\supset 
B(x,\tfrac \pi2) \backslash B(z,\tfrac\pi2).
\leqno(*)\]

Note that we get equality in $(*)$ if and only if $y$ lies on the great circle arc from $x$ to $z$.
Therefore the second statement follows.

\parit{Comments.} 
This construction was given by 
Aleksei Pogorelov in \cite{pogorelov}.
It is closely related to the construction given 
by David Hilbert in \cite{hilbert}
which was the motivating example of his 4-th problem,
see \cite{hilbert-problems}.


%%%%%%%%%%%%%%%%%%%%%%%%%%%%%%%%%%%%%%%%%%%%%%%%%%
\parbf{\ref{Straight geodesics}.} 
\textit{Straight geodesics.}
From uniqueness of straight segment between given points in $\RR^m$,
it follows that any straight line in $\RR^m$ forms a geodesic in $(\RR^m,\rho)$.

Set 
\[\|\bm{v}\|_{\bm{x}}=\rho(\bm{x},(\bm{x}+\bm{v})).\]
Note that 
\[ \|\lambda\cdot\bm{v}\|_{\bm{x}}
=
|\lambda|\cdot\|\bm{v}\|_{\bm{x}}\]
for any $\bm{x},\bm{v}\in\RR^m$ and $\lambda\in\RR$.

Prove that 
\[
\|\lambda\cdot\bm{v}\|_{\bm{x}}
-
\|\lambda\cdot\bm{v}\|_{\bm{x}'}
\le 
\Const\cdot |\bm{x}-\bm{x'}|\]
for any $\bm{x},\bm{x'},\bm{v}\in\RR^m$, 
$\lambda\in\RR$
and some fixed $\Const\in\RR$.

Passing to the limit as $\lambda\to\infty$, 
we get
$\|\bm{v}\|_{\bm{x}}$ does not depend on $\bm{x}$;
hence the result follows.



%%%%%%%%%%%%%%%%%%%%%%%%%%%%%%%%%%%%%%%%%%%%%%%%%%
\parbf{\ref{hom-near-QI}.} 
\textit{A homeomorphism near quasi-isometry.}
Let $M\ge 1$ and $A\ge 0$.
Define $(M,A)$-quasi-isometry
as a map $f\:X\to Y$ between metric spaces $X$ and $Y$ such that for any $x,y\in X$,
 we have
\[\tfrac1M\cdot |x-y|-A\le |f(x)-f(y)|\le M\cdot |x-y|+A\]
and any point in $Y$ lies on the distance at most $A$ from a point in the immage $f(X)$.

Note that $(M,0)$-quasi-isometry is a $[\tfrac1M,M]$-bi-Lipschitz map.
Moreover,
if $f_n\:\RR^m\to\RR^m$ is a  $(M,\frac1n)$-quasi-isometry 
for each $n$ then any partial limit of $f_n$ as $n\to\infty$
is a $[\tfrac1M,M]$-bi-Lipschitz map.

It follows that given $M\ge 1$ and $\eps>0$ there is $\delta>0$ such that 
for any $(M,\delta)$-quasi-isometry $f\:\RR^m\to\RR^m$ and any $p\in \RR^m$
there is an $[\tfrac1M,M]$-bi-Lipschitz map $h\:B(p,1)\to \RR^m$
such that
\[|f(x)-h(x)|<\eps\]
for any $x\in B(p,1)$.

Applying recaling, we can get the following equivalent formulation. 
Given $M\ge 1$, $A\ge 0$ and $\eps>0$
there is big enough $R>0$ such that for any $(M,A)$-quasi-isometry 
$f\:\RR^m\to\RR^m$ and any $p\in\RR^m$ there is a $[\tfrac1M,M]$-bi-Lipschitz map $h\:B(p,R)\to \RR^m$
such that 
\[|f(x)-h(x)|<\eps\cdot R\]
for any $x\in B(p,R)$.

Now cover $\RR^m$ by balls
$B(p_n,R)$, construct a $[\tfrac1M,M]$-bi-Lipschitz map $h_n\:B(p_n,R)\to \RR^m$ for each $n$.

The maps $h_n$ are $2\cdot \eps\cdot R$ close to each other on the overlaps of their domains of definition.
This makes possible to deform slightly each $h_n$ so that they agree on the overlaps.
This can be done by Siebenmann' Theorem, see \cite{siebenmann}.
If instead you apply Sullivan's theorem, you get a bi-Lipschitz homeomorphism $h\:\RR^m\to\RR^m$,
see \cite{sullivan}.


\parit{Comments.}
The problem was suggested by Dmiti Burago.





%%%%%%%%%%%%%%%%%%%%%%%%%%%%%%%%%%%%%%%%%%%%%%%%%%
\parbf{\ref{hausdorff-section}.} 
\textit{A family of sets with no section.}
Identify $\mathbb{S}^1$ with $[0,1]/(0\sim 1)$.
Consider one parameter family of Cantor sets $K_t$
formed by all possible sums $\sum_{n=1}^\infty a_n\cdot t^n$,
where $a_i$ is $0$ or $1$ and $t\in[0,\tfrac12]$.

Note that $K_{\frac12}=\mathbb{S}^1$.

Denote by $\rho_\alpha\:\mathbb{S}^1\to\mathbb{S}^1$ 
the rotation by angle $\alpha$.
Set $Z_t=\rho_{\frac1{1-2\cdot t}}(K_t)$ for $t\in[0,\tfrac12)$ and $Z_{\frac12}=\mathbb{S}^1$.

Prove that the family of sets $Z_t$
is a continuous in Hausdorff topology and it does not have a section.

\parit{Comments.} The problem is suggested by Stephan Stadler.

It is instructive to check that any Hausdorff continuous family of closed sets in $\RR$ admits a continuous section.



%%%%%%%%%%%%%%%%%%%%%%%%%%%%%%%%%%%%%%%%%%%%%%%%%%
\parbf{\ref{pr:Sasaki metric}.} 
\textit{Sasaki metric.}
Show that there is a constant $\ell$
such that for any two unit tangent vectors $v\in\mathrm{T}_p\mathbb{S}^2$ 
and $w\in T_q\mathbb{S}^2$
there is a path 
$\gamma\:[0,1]\to\mathbb{S}^2$ from $p$ to $q$
such that 
\[\length \gamma\le \ell\] 
and
$w$ is the parallel transformation of $v$ along $\gamma$.

Note that once it is proved, 
it follows that diameter of the set of all vectors of fixed length in $\mathrm{T} \mathbb{S}^2$ has diameter at most $\ell$;
in particular the map $\mathrm{T}\mathbb{S}^2\to[0,\infty)$ defined as $v\mapsto |v|$ 
preserves the distance with the maximal error $\ell$.
Hence the result follows.

\section*{Actions and coverings}




%%%%%%%%%%%%%%%%%%%%%%%%%%%%%%%%%%%%%%%%%%%%%%%%%%
\parbf{\ref{Bounded orbit}.} 
\textit{Bounded orbit.}
Note that we can assume that the orbit $x_n=\iota^n(x)$ is dense in $X$;
otherwise pass to the closure of this orbit.
In particular, we can choose a finite number of positive integers values $n_1$, $n_2,\dots,n_k$
such that the points $x_{n_1}$, $x_{n_2},\dots,x_{n_k}$ form a $\tfrac1{10}$-net in $B(x_0,10)$.

Prove that that if $x_m\in B(x_0,1)$ then $x_{m+n_i}\in B(x_0,1)$ for some $i\in\{1,\dots,k\}$.

Set $N=\max_i\{n_i\}$.
It follows 
that among any $N$ elements in a row $x_{i+1},\dots x_{i+N}$
there is at least one in $B(x_0,1)$.
In particular, $N$ isometric copies of $B(x_0,1)$ cover whole $X$.
Hence the result follows.

\parit{Comments.}
The problem is due to Aleksander Ca{\l}ka's, see \cite{calka}.

%%%%%%%%%%%%%%%%%%%%%%%%%%%%%%%%%%%%%%%%%%%%%%%%%%
\parbf{\ref{Finite action}.}
\textit{Finite action.}
Without loss of generality, we may assume that the action is generated by a nontrivial homeomorphism 
\[a\:\mathbb{S}^m\to\mathbb{S}^m\] 
and $a^p=\id_{\mathbb{S}^m}$ for some prime $p$.

Assume that any $a$-orbit lies in an open hemisphere.
Then 
\[h(x)=\sum_{n=1}^p a^n\cdot x\ne0\]
for any $x\in\mathbb{S}^m$.

Consider the map $f\:\mathbb{S}^m\to\mathbb{S}^m$ 
defined as  $f(x)=\tfrac{h(x)}{|h(x)|}$.
Note that 
\begin{enumerate}[(i)]
\item if $a(x)=x$ then $f(x)=x$;
\item\label{f(x)=f(a(x))} $f(x)=f\circ a(x)$ for any $x\in\mathbb{S}^m$.
\end{enumerate}

Prove that $f$ is homotopic to the identity; 
in particular 
\[\deg f=1.\leqno({*})\]

Fix $x\in \mathbb{S}^m$ such that $a(x)\ne x$.
Note that $a$ acts without fixed points 
on the preimage $W=f^{-1}(V)$ 
of a small open neighborhood $V\ni x$.
Therefore the quotient map $\theta\:W\to W'=W/\ZZ_p$ is a $p$-fold covering.
From  (\ref{f(x)=f(a(x))}),
there is $f'\:W'\to V$ such that
$f|_W=f'\circ\theta$.

Assume $p\ne 2$.
Show that $f'$ and $\theta$ have well defined degrees and 
\[\deg f\equiv\deg \theta\cdot\deg f'\pmod p\]
Since $\theta$ is a $p$-fold covering, we have $\deg \theta\equiv0\pmod p$.
Therefore
\[\deg f\equiv 0\pmod p.\leqno({*}{*})\]

Finally observe that $({*})$ contradicts $({*}{*})$.

In the case $p=2$ the same proof works, 
but the degree is defined only modulo $2$.

\parit{Comments.}
Along the same lines one can get a lower bound for the maximal diameter of orbit for any nontrivial actions of finite groups on a Riemannian manifold.

Applying the problem to a conjugate action, one gets that if a fixed point set of a finite group action $F\acts \mathbb{S}^m$
has nonempty interior then the action is trivial.
The same holds for any connected manifold;
it was proved by Max Newman in \cite{newman}.

The Newman's theorem was used by Deane Montgomery in \cite{montgomery} 
to show that 
\emph{if $h$ is a homeomorphism of a connected manifold $M$ 
such that each $h$-orbit is finite 
then $h^n=\id_M$ for some positive integer $n$}.


%%%%%%%%%%%%%%%%%%%%%%%%%%%%%%%%%%%%%%%%%%%%%%%%%%
\parbf{\ref{figure-eight-1}.} 
\textit{Covers of figure eight.}
First show that any compact metric space can be presented as a limit of a sequence of finite metric graphs $\Gamma_n$.
Moreover, show that one can assume  each vertex of $\Gamma_n$ has degree 3 
and the length of each edge in $\Gamma_n$ is multiple of $\tfrac 1n$.

\begin{wrapfigure}{o}{27mm}
%\begin{center}
\begin{lpic}[t(-2mm),b(-5mm),r(0mm),l(0mm)]{pics/fig8(1)}
%\lbl[b]{24,41;x}
\end{lpic}
%\end{center}
\end{wrapfigure}

It remains to approximate $\Gamma_n$ by finite coverings of $(\Phi,d/n)$.
Guess this part  
from the following picture; it shows the needed approximation of the doted graph.

\parit{Comments.} 
The same idea works if instead of figure eight, we have any compact length-metric space $X$ wich admits a map $X\to\Phi$
which is surjective on fundamental groups.
Such spaces $X$ can be found among compact hyperbolic manifolds of any dimension $\ge 2$.
For more details see the thesis of Vedrin Sahovic \cite{sahovic}.

A similar idea was used later to show that any group can appear as a fundamental group of underlying space of 3-dimensional hyperbolic orbifold, see \cite{panov-petrunin-telescopic}.





%%%%%%%%%%%%%%%%%%%%%%%%%%%%%%%%%%%%%%%%%%%%%%%%%%
\parbf{\ref{m-fold-cover}.} 
\textit{Diameter of \textit{m}-fold cover.}
Fix points $\tilde p,\tilde q\in\tilde M$.
Let  
$\tilde\gamma\:[0,1]\to \tilde M$ be a minimizing geodesic from $\tilde p$ to $\tilde q$. 

We need to show that 
\[\length \tilde\gamma\le m\cdot diam(M).\]
Suppose the contrary.

Denote by $p,q$ and $\gamma$ the projections of $\tilde p,\tilde q$ and $\tilde \gamma$ in $M$. 
Represent $\gamma$
as joint of $m$ paths of equal length,
\[\gamma=\gamma_1{*}\dots{*}\gamma_m,\] 
so
\[\length(\gamma_i)=\length(\gamma)/m>\diam(M).\] 

Let $\sigma_i$ be a minimizing geodesic in $M$ connecting the endpoints of $\gamma_i$. 
Note that 
\[\length\sigma_i\le \diam M< \length\gamma_i.\] 

Consider $m+1$ paths $\alpha_0,\dots,\alpha_m$ defined as 
\[\alpha_i=\sigma_1{*}\dots{*}\sigma_i{*}\gamma_{i+1}{*}\dots{*}\gamma_m.\]

Consider their liftings $\tilde\alpha_0,\dots,\tilde\alpha_m$ 
with $\tilde q$ as the endpoint.
Note that two curves, say $\alpha_i$ and $\alpha_j$ for $i<j$, 
have the same starting point in $\tilde M$.

Consider the path
\[\beta=\gamma_1{*}\dots{*}\gamma_i{*}\sigma_{i+1}{*}\dots{*}\sigma_j{*}\gamma_{j+1}{*}\dots{*}\gamma_m.\]
Prove that there is lift $\tilde\beta$ of $\beta$ 
which connects $\tilde p$ to $\tilde q$ in $\tilde M$.
Clearly $\length \beta<\length \gamma$, a contradiction.

\parit{Comments.}
The question was asked by Alexander  Nabutovsky
and answered by Sergei Ivanov, 
see \cite{ivanov}.



%%%%%%%%%%%%%%%%%%%%%%%%%%%%%%%%%%%%%%%%%%%%%%%%%%
\parbf{\ref{Symmetric square}.} 
\textit{Symmetric square.}
Let $\Gamma=\pi_1 X$ and $\Delta=\pi_1((X\times X)/\ZZ_2)$.
Consider the homomorphism $\phi\:\Gamma\times \Gamma\to \Delta$
induced by the projection $X\times X\to (X\times X)/\ZZ_2$.

Prove that the restrictions $\phi|_{\Gamma\times \{1\}}$ and $\phi|_{\{1\}\times\Gamma}$
are onto.

It remains to note that 
$$\phi(\alpha,1)\phi(1,\beta)=\phi(1,\beta)\phi(\alpha,1)$$
for any $\alpha$ and $\beta$ in $\Gamma$.

\parit{Comments.} The problem was suggested by Rostislav Matveyev.



%%%%%%%%%%%%%%%%%%%%%%%%%%%%%%%%%%%%%%%%%%%%%%%%%%
\parbf{\ref{Sierpinski triangle}.} 
\textit{Sierpinski triangle.}
Denote the Sierpinski triangle by $\triangle$.

Let us show that any homeomorphism of $\triangle$ is also its isometry.
Therefore the group homeomorphisms is the symmetric group $S_3$. 

Let $\{x,y,z\}$ be a 3-point set in $\triangle$ such that $\triangle \backslash\{x,y,z\}$ has 3 connected components.
Prove that there is unique choice for the set $\{x,y,z\}$ and 
it is formed by the midpoints of its big sides.

It follows that any homeomorphism of $\triangle$ permutes the set $\{x,y,z\}$.

A similar argument shows that this permutation  uniquely describes the homeomorphism.

\parit{Comments.}
The problem was suggested by Bruce Kliener.



%%%%%%%%%%%%%%%%%%%%%%%%%%%%%%%%%%%%%%%%%%%%%%%%%%
\parbf{\ref{Boys and girls in a Lie group}.} 
\textit{Latices in a Lie group.}
Denote by $V_\ell$ and $W_m$
the Voronoi domain of for each $\ell\in L$ and $m\in M$ correspondingly;
that is,
\[V_\ell=\set{g\in G}{|g-\ell|_G\le|g-\ell'|\ \text{for any}\ \ell'\in L}\]
\[W_m=\set{g\in G}{|g-m|_G\le|g-m'|\ \text{for any}\ m'\in M}\]

Note that for any $\ell\in L$ and $m \in M$ we have
\[\begin{aligned}
\vol V_\ell&=\vol(L\backslash (G,h))=
\\
&=\vol(M\backslash (G,h))=
\\
&=\vol W_m.
\end{aligned}
\leqno({*})
\]

Consider the bipartite graph $\Gamma$ with vertices formed by the elements of $L$ and $M$
such that $\ell\in L$ is adjacent  to $m \in M$ if and only if $V_\ell\cap W_m\ne\emptyset$.

By $({*})$ the graph $\Gamma$ satisfies the condition in the Hall's marriage theorem.
Therefore there is a bijection $f\: L\to M$ such that 
\[V_\ell\cap W_{f(\ell)}\ne\emptyset\] for any $\ell\in L$. 

It remains to notice that $f$ is bi-Lipschitz.

\parit{Comments.} The problem is due to 
Dmitri Burago 
and Bruce Kleiner,
see \cite{burago-kleiner}. 
For a finitely generated group $G$  
it is not known if $G$ and $G\times \ZZ_2$ can fail to be bi-Lipscitz.
(The groups are assumed to be equipped with word metric.)
 



%%%%%%%%%%%%%%%%%%%%%%%%%%%%%%%%%%%%%%%%%%%%%%%%%%
\parbf{\ref{Piecewise Euclidean quotient}.} 
\textit{Piecewise Euclidean quotient.}
Note that the group $\Gamma$ serves as holonomy group of the quotient space $P=\RR^m/\Gamma$ with the induced polyhedral metric.
More precisely, one can identify $\RR^m$ with the tangent space of a regular point $x_0$ of $P$ in such a way that
for any $\gamma\in\Gamma$ there is a loop $\ell$ in $P$ which pass only through regular points and has the holonomy $\gamma$.

Fix $\gamma\in\Gamma$. 
Let $\ell$ be the corresponding loop.
Since $P$ is simply connected, we can shrink $\ell$ by a disc.
By general position argument we can assume that the disc 
only pass through simplices of codimension $0$, $1$ and $2$
and intersect the simplices of codimension $2$ transversely.

In other words, $\ell$ can be presented as a product of 
loops such that each loop goes around a single simplex of codimension $2$ and comes back.
The holonomy for each of these loops is a rotation around a hyperplane.
Hence the result follows.

\parit{Comments.}
The converse to the problem also holds;
it was proved by Christian Lange in \cite{lange},
his proof based ealier results of 
Marina Mikhailova, see \cite{mikhailova}.

Note that the cone over spherical suspension over Poincar\'e sphere is homeomorphic to $\RR^5$ and it is quotient of $\RR^5$ by a finite subgroup of $\SO(5)$. 
Therefore, if one exchanges ``PL-homeomorphism'' to ``homeomorphism'' in the formulation then the answer is different; 
a complete classification of such actions was also obtained in \cite{lange}.

%%%%%%%%%%%%%%%%%%%%%%%%%%%%%%%%%%%%%%%%%%%%%%%%%%
\parbf{\ref{Subgroups of free group}.} 
\textit{Subgroups of free group.}
Let $G$ be a finitely generated subgroup of free group with $m$ generators, further denoted by $F_m$.

Let $W$ be the wedge sum of $n$ circles, 
so  $\pi_1(W,p)=F_m$.
Equip $W$ with the length-metric 
such that each circle has unit length.

Pass to the metric cover $\tilde W$ of $W$ 
such that  $\pi_1(\tilde W,\tilde p)=G$ 
for a lift $\tilde p$ of $p$.

Fix sufficiently large integer $n$ and consider doubling of the closed ball $\bar B(\tilde p,n+\frac12)$ in its boundary.
Let us denote the obtained doubling by $Z_n$ and set $G_n=\pi(Z_n,\tilde p)$.

Prove that $Z_n$ is a metric covering of $W$;
it makes possible to consider $G_n$ as a subgroup of $F_m$.
By construction, $Z_n$ is compact;
therefore $G_n$ has finite order in $F_m$.


It remains to show that that 
\[G=\bigcap_{n>k} G_n,\]
where $k$ is the maximal length of word in the generating set of $G$.

%???+PIC
 
\parit{Comments.} 
Originally the problem was solved by Marshall Hall in \cite{hall}.
The proof presented here is close to the solution of John Stalings in \cite{stallings};
see also \cite{wilton}.

The same idea can be used to solve the following problems and many others.
\begin{itemize}
\item Show that subgroups of free groups are free.
\item Show that two elements of the free groups $u$ and $v$ commute 
if and only if they are both powers of
the some element $w$.
\end{itemize}



%%%%%%%%%%%%%%%%%%%%%%%%%%%%%%%%%%%%%%%%%%%%%%%%%%
\parbf{\ref{Lengths of generators of the fundamental group}.}
\textit{Lengths of generators of the fundamental group.}
Choose a length minimizing loop $\gamma$ 
which represents a given element $a\in\pi_1M$.

Fix $\eps>0$.
Represent $\gamma$ 
as a joint 
\[\gamma=\gamma_1{*}\dots{*}\gamma_n\]
of paths with $\length\gamma_i<\eps$ for each $i$.
 
Denote by $p=p_0,p_1,\dots, p_n=p$ the endpoints of these arcs.
Connect $p$ to $p_i$ by a minimizing geodesic $\sigma_i$.
Note that $\gamma$ is homotopic to a product of loops
\[\alpha_i=\sigma_{i-1}{*}\gamma_i{*}\sigma_{i-1}\]
and $\length \alpha_i<2\cdot\diam M+\eps$ for each $i$.

It remains to show that for sufficiently small $\eps>0$
any loop with length less than $2\cdot\diam M+\eps$ 
is homotopic to a loop with length at most $2\cdot\diam M$.

\parit{Comments.} The statement is due to 
Mikhael Gromov, 
see \cite{gromov-MetStr}*{Prop. 3.22}.

%%%%%%%%%%%%%%%%%%%%%%%%%%%%%%%%%%%%%%%%%%%%%%%%%%
\parbf{\ref{Short basis}.}
\textit{Short basis.}
Consider universal Riemannian cover $\tilde M$ of $M$.
Note that $\tilde M$ is nonnegatively curved and
$\pi_1M$ acts by isometries on $\tilde M$.

Fix $p\in \tilde M$.
Given  $a\in \pi_1M$,
set 
\[|a|=|p- a\cdot p|_{\tilde M}.\]
Construct a sequence of elements $a_1,a_2,\dots\in \pi_1M$ the following way:
\begin{enumerate}[(i)]
\item Choose $a_1\in\pi_1M$ so that $|a_1|$ takes the minimal value.
\item Choose $a_2\in\pi_1M\backslash\langle a_1 \rangle$ so that $|a_2|$ takes the minimal value.
\item Choose $a_3\in\pi_1M\backslash\langle a_1,a_2 \rangle$ so that $|a_2|$ takes the minimal value.
\item and so on.
\end{enumerate}

Note that the sequence terminates at $n$-th step if the 
$(a_1,a_2,\dots,a_n)$ form a generating system.
By construction, we have
\begin{align*}
|a_j\cdot a_i^{-1}|&\ge |a_j|\ge |a_i|
\intertext{for any $j>i$. 
Set $p_i=a_i\cdot p$.
Note that}
|p_j-p_i|_{\tilde M}
&=|a_j\cdot a_i^{-1}|\ge
\\
&\ge |a_j|=
\\
&=|p_j-p|_{\tilde M}\ge
\\
&\ge|a_i|=
\\
&=|p_i-p|_{\tilde M}.
\intertext{By Toponogov comparison theorem we get}
\tilde\measuredangle \hinge p{p_i}{p_j}&\ge \tfrac\pi3
\end{align*}
for any $i\ne j$.
Hence the result follows.

\parit{Comments.} This construction introduced by Mikhael Gromov 
in his paper on almost flat manifolds, 
see \cite{gromov-almost-flat}.

\section*{Topology}

%%%%%%%%%%%%%%%%%%%%%%%%%%%%%%%%%%%%%%%%%%%%%%%%%%

\parbf{\ref{Immersed disks}.} 
\textit{Immersed disks.}
Both circles on the picture bound essentially different discs.

\begin{wrapfigure}[5]{o}{46mm}
%\begin{center}
\begin{lpic}[t(-8mm),b(0mm),r(0mm),l(0mm)]{pics/milnors-discs()}
%\lbl[b]{24,41;x}
\end{lpic}
%\end{center}
\end{wrapfigure}

It is a good exercise to count the essentially different discs in these examples. 
(The answers are 2 and  5 correspondingly.) 

\parit{Comments.}
The first example is generally attributed to John Milnor.
The second example was given by Daniel Bennequin in \cite{bennequin}.

\begin{wrapfigure}[4]{o}{19mm}
%\begin{center}
\begin{lpic}[t(-8mm),b(0mm),r(0mm),l(0mm)]{pics/annuli()}
%\lbl[b]{24,41;x}
\end{lpic}
%\end{center}
\end{wrapfigure}

An easier problem would be to construct two essentially different immersions of annuli with the same boundary curves; a solution is shown on the picture.

%???IS NOT IT OBVIOUS???
In \cite{shor-van wyk}, it was conjectured that 
if an immersed circle bounds an immersed disc with at most two layers at each point then this disc is essentially unique. (Milnor's example has 3 layers in the middle.)


%%%%%%%%%%%%%%%%%%%%%%%%%%%%%%%%%%%%%%%%%%%%%%%%%%
\parbf{\ref{Positive Dehn twist}.} 
\textit{Positive Dehn twist.}
Consider the universal covering 
$\tilde\Sigma\to\Sigma$.
The surface $\tilde \Sigma$ comes with the orientation induced from $\Sigma$.

Note that we may assume that $\tilde\Sigma$ has infinite number of boundary components.

Fix a point $x_0$ on the boundary of $\tilde \Sigma$.
Given other points $y$ and $z$ we will write
$y\prec z$ if $z$ lies on the left side from one (and therefore any) simple curve from $x_0$ to $y$ in $\tilde\Sigma$.
Note that  $\prec$ defines a linear order on $\partial\tilde\Sigma\backslash\{x_0\}$.
We will write $y\preceq z$ 
if $y\prec z$ or $y=z$.

Note that any homeomorphism $h\:\Sigma\to\Sigma$ which is identity on the boundary
lifts to unique homeomorphism $\tilde h\:\tilde \Sigma\to\tilde\Sigma$ 
is such a way that $\tilde h(x_0)=x_0$.

Assume $h$ is positive Dehn twist.
Show that 
$y\preceq \tilde h(y)$ for any  $y\in\partial\tilde\Sigma\backslash\{x_0\}$
and there is a point $y_0\in\partial\tilde\Sigma\backslash\{x_0\}$
such that $y_0\prec \tilde h(y_0)$.

Finally note that the later property is a homotopy invariant 
and it survives under compositions of maps.
Hence the statement follows.

\parit{Comments.} The problem was suggested by Rostislav Matveyev.

%%%%%%%%%%%%%%%%%%%%%%%%%%%%%%%%%%%%%%%%%%%%%%%%%%
\parbf{\ref{Function with no critical points}.} 
\textit{Function with no critical points.}
Construct an immersion $\psi\:B^m\z\to\RR^m$ such that 
\[\ell\circ\phi\ne\ell\circ\psi\]
for any embedding  $\phi\:B^m\to\RR^m$.

It remains to note that the composition $f=\ell\circ\psi$ has no critical points.

\parit{Comments.} 
The problem was suggested by Petya Pushkar.

%%%%%%%%%%%%%%%%%%%%%%%%%%%%%%%%%%%%%%%%%%%%%%%%%%
\parbf{\ref{Conic neighborhood}.} 
\textit{Conic neighborhood.}
Let $V$ and $W$ be two conic neighborhoods of $p$.
Without loss of generality, we may assume that $V\subset W$.

We will need to construct a sequence of embeddings $f_n\:V\to W$
such that 
\begin{enumerate}[(i)]
\item 
For any compact set $K\subset V$ 
there is a postive ineteger $n=n_K$ such that 
$f_n(k)=f_m(k)$ for any $k\in K$ and $m\ge n$.
\item For any point $w\in W$ there is a point $v\in V$ such that $f_n(v)=w$ for all large $n$.
\end{enumerate}

Note that once such sequence is constructed, $f\:V\to W$ defined as $f(v)=f_n(v)$ for all large values of $n$ gives the needed homeomorphism.

The sequence $f_n$ can be constructed recursively, setting
\[f_{n+1}=\Psi_n\circ f_n\circ \Phi_n,\]
where $\Phi_n\:V\to V$ 
and $\Psi_n\:W\to W$ 
are homeomorphisms
of the form 
\[\Phi_n(x)=\phi_n(x)\cdot x\quad \Phi_n(x)=\psi_n(x)\cdot x,\]
where $\phi_n\:V\to \RR_+$, $\psi_n\:W\to \RR_+$ are suitable continuous functions 
and 
``$\cdot$'' denotes the ``multiplication'' in the cone structures of $V$ and $W$ correspondingly.

\parit{Comments.}
The problem is due to Kyung Whan Kwun, see \cite{kwun}.

Note that for two cones $\mathop{\rm Cone}(\Sigma_1)$ and $\mathop{\rm Cone}(\Sigma_2)$ might be homeomorphic while $\Sigma_1$ and $\Sigma_2$ are not.



%%%%%%%%%%%%%%%%%%%%%%%%%%%%%%%%%%%%%%%%%%%%%%%%%%
\parbf{\wrenches\ref{No knots}.} 
\textit{No $C^0$-knots.}

%???+PIC

\parit{Comment.}
This problem was suggested by Greg Kuperberg, see \cite{One-step problems in geometry}.


%%%%%%%%%%%%%%%%%%%%%%%%%%%%%%%%%%%%%%%%%%%%%%%%%%
{
\begin{wrapfigure}{o}{37mm}
\begin{lpic}[t(-5mm),b(0mm),r(0mm),l(0mm)]{pics/Simple-stabilization(1)}
\lbl[]{8.7,8.3;{\color{white} $K_1$}}
\lbl[]{28.7,8.3;{\color{white} $K_2$}}
\end{lpic}
\end{wrapfigure}

\parbf{\ref{Simple stabilization}.} 
\textit{Simple stabilization.}
The example can be guessed from the diagram.

\parit{Caomments.} 
I learned this problem 
in my analysis class taught by 
Maria Goluzina.

}

%%%%%%%%%%%%%%%%%%%%%%%%%%%%%%%%%%%%%%%%%%%%%%%%%%
\parbf{\ref{Isotropy}.}
\textit{Isotropy.}
Fix a hoemeomrphism $\phi\:K_1\to K_2$.

By Tietze extension theorem,
the hoemeomrphisms $\phi\:K_1\to K_2$ and $\phi^{-1}\:K_2\to K_1$ can be extended to a continuous maps,
say $f\:\RR^m\to \RR^m$ and $g\:\RR^m\to \RR^m$ correspondingly.

Consider the homeomorphisms
$h_1, h_2, h_3\:\RR^m\times\RR^m\to\RR^m\times\RR^m$ defined the following way
\begin{align*}
h_1(x,y)&=(x,y+f(x)),
\\
h_2(x,y)&=(x-g(y),y),
\\ 
h_3(x,y)&=(y,-x).
\end{align*}

It remains to prove that each homeomorphism $h_i$ is isotopic to the indentity map and
\[h=h_3\circ h_2\circ h_1.\] 

\parit{Comments.}
The problem is due to Victor Klee, 
see \cite{klee}.
Problem \ref{mono-homotopy} is closely related.


%%%%%%%%%%%%%%%%%%%%%%%%%%%%%%%%%%%%%%%%%%%%%%%%%%
\parbf{\wrenches\ref{Knaster's circle}.} 
\textit{Knaster's circle.}
A map $f\:\mathbb S^1\to\mathbb S^1$ will be called $\eps$-crooked 
if for any arc $\II\subset \mathbb S^1$ with end points $a$ and $b$ there are points $x,y\in \II$ such that the points $a,y,x,b$ appear on $\II$ in the same order and
\[|f(x)-f(a)|_{\mathbb S^1},|f(y)-f(b)|_{\mathbb S^1}<\eps.\]

Show that for any $\eps>0$ there is an $\eps$-crooked map $f\:\mathbb S^1\to\mathbb S^1$ of degree $1$.

Take a sequence of $\eps_n$-crooked maps for a sequence $\eps_n$ which converge fast ot $0$
and use this map to construct a nested sequence of embedding of annuli in the plane.
Each annulus bounds a disc and the intersection 
of all these annuli bound a disc which is the union of all thse discs.

It remains to show that the boundary of the obtained disc does not contain a simple curve.
%???PIC
\parit{Comments.}
\cite{wayne}.

%%%%%%%%%%%%%%%%%%%%%%%%%%%%%%%%%%%%%%%%%%%%%%%%%%
\parbf{\ref{Boundary in R}.} 
\textit{Boundary in $\RR$.}
Prove that the Cantor's set forms a boundary of three disjoint open set in $\RR$.

\parit{Comments.} 
In $\RR^2$
one can assume in addition that each set is connected.
This examples are called \emph{lakes of Wada};
these are disjoint open discs in the plane which have identical boundary.   
This example described by Kuniz\^{o} Yoneyama in \cite{yoneyama}.
It is easy to see that the boundary of each lake contains no simple nontrivial curves
and it is related to so called pseudo-arc constructed by Bronis{\l}aw Knaster in \cite{knaster}. 

In $\RR^3$, a similar construction can be used to produce a Cantor's set with non simply connected complement.
This example was constructed by Louis Antoine in \cite{antoine}.
The construction can be guessed from the first and second itaration on the shown on the pictures%
\footnote{These are black-and-white versions of the pictures 
made by \href{http://en.wikipedia.org/wiki/User:Blacklemon67}{Blacklemon67} 
for the article on Antoine's Necklace in Wikipedia.}.
\begin{center}
\begin{lpic}[t(-0mm),b(0mm),r(0mm),l(0mm)]{pics/Antoine's_Necklace_Iteration_1(.1)}
\end{lpic}
\begin{lpic}[t(-0mm),b(0mm),r(0mm),l(0mm)]{pics/Antoine's_Necklace_Iteration_2(.1)}
\end{lpic}
%???https://en.wikipedia.org/wiki/User:Blacklemon67
\end{center}
In a similar fashion one can construct so called \emph{Whitehead manifold}, 
an example of open 3-dimensional manifold which is contractible, 
but not homeomorphic to $\RR^3$.
One have to start with standard sphere and remove the intersection of nested sequence of solid tori embedded in each other in a certain way,
see \cite{whitehead}.

%Yet an other related example is Casson handle, Dogbone space???

%%%%%%%%%%%%%%%%%%%%%%%%%%%%%%%%%%%%%%%%%%%%%%%%%%
\parbf{\ref{Homeomorphism of cube}.} 
\textit{Homeomorphism of cube.}
Without loss of generality, we may assume that the cube $\square^m$ is inscribed in the unit sphere centered at the origin of $\RR^m$.

Let us extend the homeomorphism $h$ to whole $\RR^m$ by reflecting the cube in its facets.
We get a homeomorphism say $\tilde h\:\RR^m\to\RR^m$ such that $\tilde h(x)=h(x)$ for any $x\in\square^m$ and 
\[\tilde h\circ\gamma=\gamma\circ \tilde h\]
for any motion $\gamma\:\RR^m\to\RR^m$ in the group generated by the reflections in the facets of the cube.

Notice that $\tilde h$ has \emph{displacement} at most $2$;
that is, 
\[|\tilde h(x)-x|\le 2\]
for any $x\in\RR^m$.

\begin{wrapfigure}{o}{46mm}
%\begin{center}
\begin{lpic}[t(-10mm),b(0mm),r(0mm),l(0mm)]{pics/Phi(1)}
\end{lpic}
%\end{center}
\end{wrapfigure}

Fix a smooth increasing concave function $\phi\:\RR\to\RR$ such that
$\phi(r)=r$ for any $r\le 1$ and $\sup_r\phi(r)=2$.

Consider $\RR^m$ with polar coordinates $(u,r)$, where $u\in\mathbb{S}^{m-1}$ and $r\ge 0$.
Let $\Phi\:\RR^m\to\RR^m$
is defined by $\Phi(u,r)=(u,\phi(r))$.

Set 
\[
f(x)=\left[
\begin{aligned}
&x&&\text{if}\ |x|\ge 2
\\
&\Phi\circ \tilde h \circ \Phi^{-1}(x)&&\text{if}\ |x|< 2
\end{aligned}
\right.
\]
Prove that $f\:\RR^m\to\RR^m$ is a solution.

\parit{Comments.} 
The problem is a stripped from the proof of Robion Kirby in \cite{kirby}.
The condition that face is mapped to face can be removed and 
instead of homeomorphism one can take an embedding which is close enough to the identity.

An interesting twist to this idea was given by Dennis Sullivan in \cite{sullivan}.
Instead of the discrete group of motions of Euclidean space,
he use a discrete group of motions of hyperbolic space in the Poincare model.
Say, assume we repeat the same argument if instead of cube we have a Coxeter polytope in the hyperbolic space.
Then the constructed map 
coinsides with the identity on the absolute and therefore the last ``shrinking'' step in the proof above is not needed.
Moreover, if the original homeomorphism is bi-Lipschitz then the construction also produce a bi-Lipschitz homeomorphism;
this is the main advantage of Sullivan's construction.
  

%%%%%%%%%%%%%%%%%%%%%%%%%%%%%%%%%%%%%%%%%%%%%%%%%%
\parbf{\wrenches\ref{Finite topological space}.} 
\textit{Finite topological space.}
Let $F$ be a finite topological space.
Given two points $p,q\in F$ we will write $p\preccurlyeq q$ if $p$ lies in any closed set containing $q$.

Prepare a cell for each point in $F$


Consider a finite CW complex $W$.

Denote by $S$ the set of all cells of $W$
and equip $S$ with the topology  such that ???


\section*{Piecewise linear geometry}


%%%%%%%%%%%%%%%%%%%%%%%%%%%%%%%%%%%%%%%%%%%%%%%%%%
\parbf{\ref{4-poly}.} 
\textit{Triangulation of 3-sphere.}
Choose 100 distinct points $x_1,x_2,\z\dots,x_{100}$
on the curve 
\[\gamma\:t\mapsto (t,t^2,t^3,t^4)\] 
in $\RR^4$.
Let $P$ be the convex hull of $\{x_1,x_2,\z\dots,x_{100}\}$.

Prove that for any two points $x_i$ and $x_j$ there is a hyperplane $H$ in $\RR^4$ which pass through $x_i$ and $x_i$ and leaves $\gamma$ on one side.
The later statement implies that any two vertices $x_i$ and $x_j$
of $P$ are connected by an edge.

The statement follows
since the surface of $P$ is homeomorphic to $\mathbb{S}^2$.

\parit{Comments.} 
The polyhedron $P$ above is an example 
of so called \emph{cyclic polytopes}.

%%%%%%%%%%%%%%%%%%%%%%%%%%%%%%%%%%%%%%%%%%%%%%%%%%
\parbf{\ref{Spherical arm lemma}.} 
\textit{Spherical arm lemma.}
Let us cut the polygon $A$ from the sphere and glue instead the polygon $B$.
Denote by $\Sigma$ the obtained spherical polyhedral space.
Note that 
\begin{itemize}
\item $\Sigma$ is homeomorphic $\mathbb S^2$.
\item $\Sigma$ has curvature $\ge 1$ in the sense of Alexandrov; that is, the total angle around each singular point is less than $2\cdot \pi$.
\item All the singular points of $\Sigma$ 
lie outside of an isometric copy of a hemisphere $\mathbb{S}^2_+\subset \Sigma$
\end{itemize}

It is sufficient to show that $\Sigma$ is isometric to the standard sphere.
Assume contrary.
If $n$ denotes be the number of singular points in $\Sigma$, 
it means that $n>0$.

We will arrive to a contradiction applying induction on $n$.
The base case $n=1$ is trivial; 
that is, $\Sigma$ can not have single singular point.

Now assume $\Sigma$ has $n>1$ singular points.
Choose two singular points $p, q$,
cut $\Sigma$ along a geodesic $[pq]$
and patch the hole so that the obtained new polyhedron $\Sigma'$ has curvature $\ge 1$.
The patch is obtained by doubling a
spherical triangle in two sides.
For the right choice of the triangle,
the points $p$ and $q$ become regular in $\Sigma'$
and exactly one new singular point appears in the patch.

This way, constructed a  spherical polyhedral space $\Sigma'$
with $n-1$ singular points which satisfy the same conditions as $\Sigma$ 

By induction hypothesis $\Sigma'$ does not exist. Hence the result follow.

\parit{Alternative end of proof.} 
By Alexandrov embedding theorem, $\Sigma$ is isometric to the surface of convex polyhedron $P$ in the unit 3-dimensional sphere $\mathbb S^3$. 
The center of hemisphere has to lie in a facet, say $F$ of $P$.
It remains to note that $F$ contains the equator and therefore $P$ has to be hemisphere in $\mathbb S^3$ or intersection of two hemispheres.
In both cases its surface is isometric to $\mathbb S^2$.

\parit{Comments.}
The problem is due to Victor Zalgaller, 
see \cite{zalgaller-shperical-polygon};
the result of Victor Toponogov in \cite{toponogov} is closely related.
The alternative end of proof appears in \cite{panov-petrunin}.

The patch construction above was introduced by 
Aleksandr Alexandrov
in his proof of convex embeddabilty of polyhedrons;
the earliest reference we have found is
\cite{alexandrov1948}*{VI, \S7}.




%%%%%%%%%%%%%%%%%%%%%%%%%%%%%%%%%%%%%%%%%%%%%%%%%%
\parbf{\ref{Folding problem}.} 
\textit{Folding problem.}
Given a triangulation of $P$
consider the Voronoi domain for each vertex.
Prove that the triangulation can be subdivided if necessary
so that Voronoi domain of each vertex is isometric to a convex subset in the cone with vertex corresponding to the tip.

Note that the boundaries of all the Voronoi domains form a graph with straight edges.
One can triangulate $P$ so that each triangle has such edge as the base 
and the opposite vertex is the center of an adjusted Voronoi domain; such a vertex will be called \emph{main} vertex of the triangle.

Fix a triangle $\triangle vab$ in the constructed triangulation; 
let $v$ be its main vertex.
Given a point 
$x\in  \triangle$, set 
\[\rho(x)=|x-v|\ \ \text{and}\ \  \theta(x)=\min \{\measuredangle \hinge vax,\measuredangle\hinge vbx\}.\]
Map $x$ to the plane the point with polar coordinates $(\rho(x),\theta(x))$.

It is easy to see that the constructed map $\triangle\to\RR^2$ is piecewise distance preserving.
It remains to check that the constructed maps on all triangles agree on common sides.


\parit{Comments.}
This construction was given by Victor Zalgaller in \cite{zalgaller-polyhedra}, see also \cite{petrunin-yashinsky}.
Svetlana Krat generalized the statement to the higher dimensions,
see \cite{krat}.



%%%%%%%%%%%%%%%%%%%%%%%%%%%%%%%%%%%%%%%%%%%%%%%%%%
\parbf{\ref{iso-kirzhbraun}.} 
\textit{Piecewise linear extension.}
Let $a_1,a_2,\dots,a_n$
and $b_1$, $b_2,\z\dots,b_n$
be two collections of points in $\RR^2$
such that $|a_i-a_j|\ge |b_i-b_j|$ for all pairs $i$, $j$.
We need to construct a piecewise liner length-preserving map $f\:\RR^2\to\RR^2$
such that $f(a_i)=b_i$ for each $i$.

Assume that the problem is already solved if $n<m$;
let us do the case $n=m$.
By assumption, 
there is a piecewise liner length-preserving map $f\:\RR^2\to\RR^2$
such that $f(a_i)=b_i$ for each $i>1$.
Consider the set 
\[\Omega=\set{x\in\RR^2}{|f(x)-b_1|>|x-a_1|}.\]
If $\Omega=\emptyset$ then $f(a_1)=b_1$; 
that is, the problem is solved.

Prove that $\Omega$ is the interior of a polygon
which is star-shaped with respect to $a_1$.
Redefine the map $f$ inside $\Omega$ so that it remains piecewise liner length-preserving and $f(a_1)=b_1$.

\parit{Comments.}
The same proof works in all dimensions;
it was given by Ulrich Brehm in \cite{brehm}.
The same proof was rediscovered by Arseniy Akopyan and Alexey Tarasov in \cite{akopyan-tarasov},
сee also \cite{petrunin-yashinsky}.

The problem is closely related to Kirszbraun's theorem \cite{kirszbraun},
which was reproved by Frederick Valentine in \cite{valentine};
the proof of Brehm uses the same idea.






%%%%%%%%%%%%%%%%%%%%%%%%%%%%%%%%%%%%%%%%%%%%%%%%%%
\parbf{\ref{Minimal polyhedron}.} 
\textit{Minimal polyhedron.}
Arguing by contradiction, assume $T$ is a minimal polyhedral surface which is not saddle.

Prove that 
one can move one of the vertices of $T$ in such a way that the lengths of all edges starting at this vertex decrease.

Prove that if, 
by this deformation, 
the area does not decease 
then there are two adjusted triangles in the triangulation, 
say $\triangle pxy$ and $\triangle qxy$
such that 
\[\measuredangle \hinge pxy+\measuredangle \hinge qxy> \pi.\]

\begin{wrapfigure}{o}{21mm}
\begin{lpic}[t(-12mm),b(-6mm),r(0mm),l(0mm)]{pics/tent}
\end{lpic}
\end{wrapfigure}

Finally show that in this case exchanging triangles $\triangle pxy$ and $\triangle qxy$
to the triangles $\triangle pxq$ and $\triangle pyq$
leads to a polyhedral surface with smaller area.
I.e., $T$ was is not minimal, a contradiction.


\parit{Comments.}
This problem is discussed in my paper \cite{petrunin-monthly}.

For general polyhedral surface, the deformation which decrease the legths of all edges may not decrease the area.
Moreover, the surface which minimize the area among all surfaces with fixed  triagulation may be not saddle. 
An example of such surface can be seen on the picture. %???+PIC


%%%%%%%%%%%%%%%%%%%%%%%%%%%%%%%%%%%%%%%%%%%%%%%%%%
{
\begin{wrapfigure}{o}{22mm}
%\begin{center}
\begin{lpic}[t(-1mm),b(0mm),r(0mm),l(0mm)]{pics/Convex-triangulation(1)}
\end{lpic}
%\end{center}
\end{wrapfigure}

\parbf{\ref{Coherent  triangulation}.}
\textit{Coherent triangulation.} 
Look at the diagram and think.

\parit{Comments.}
The problem was discussed by 
Israel Gelfand, 
Mikhail Kapranov 
and Andrei Zelevinsky in \cite{GKZ}*{7C}.

}

%%%%%%%%%%%%%%%%%%%%%%%%%%%%%%%%%%%%%%%%%%%%%%%%%%
\parbf{\ref{conic neighborhoods}.} 
\textit{Characterization of polytope.}
Arguing by contradiction, let us assume that $P\subset \RR^m$
is a counterexample and $m$ takes minimal possible value.

Choose a finite cover $B_1,B_2,\dots B_n$ of $K$,
where $B_i=B(z_i,\eps_i)$ 
and $B_i\cap P=B_i\cap K_i$, 
where $K_i$ is a cone with the tip at $z_i$.

For each $i$, consider function $f_i(x)=|z_i-x|^2-\eps_i^2$.
Note that
\[W_{i,j}=\set{x\in\RR^m}{f_i(x)=f_j(x)}\]
is a hyperplane for any pair $i\ne j$.

The subset $P_{i,j}=P\cap W_{i,j}$ satisfies the same asumtions as $P$, but lies in a hyperplane.
Since $m$ is minimal, we get that $P_{i,j}$ is a polytope for any pair $i,j$.

Consider Voronoi domains 
\[V_{i}=\set{x\in\RR^m}{f_i(x)\ge f_j(x) \ \text{for any}\ j}.\]
Note that $P\cap V_i$ is formed by the points which lie on the segments from $z_i$ to a point in  $P\cap \partial V_i$.

The statement follows since $\partial V_i$ is covered by the hyperplanes $W_{i,j}$.

\parit{Comments.}
The problem is mentioned by Nina Lebedeva and me in \cite{lebedeva-petrunin}.

%%%%%%%%%%%%%%%%%%%%%%%%%%%%%%%%%%%%%%%%%%%%%%%%%%
\parbf{\ref{panov-S^3}.} 
\textit{A sphere with one edge.}
Such example $P$ can be found among the spherical polyhedral spaces which admit
an isometric $\mathbb{S}^1$-action with geodesic orbits.

Fix large relatively prime integers $p>q$. 
Consider the triangle $\Delta$ with angles $\tfrac\pi p$, $\tfrac\pi q$ and say $\pi\cdot(1-\tfrac1 p)$ in the sphere of radius $\tfrac12$.
Denote by $\hat \Delta$ the  doubling of $\Delta$ in its boundary.
Note that $\hat \Delta$ is homeomorphic to $\mathbb S^2$,
it has 3 singular points with total angles $2\cdot\tfrac\pi p$,
$2\cdot\tfrac\pi q$ and $2\cdot\pi\cdot(1-\tfrac1 p)$.

Consider $\mathbb S^1$-action on $\mathbb S^3\subset\CC^2$ by the diagonal matrices $\left(\begin{smallmatrix}z^p&0\\0&z^q\end{smallmatrix}\right)$, $z\in\mathbb S^1\subset\CC$.
Construct a spherical polyhedral metric $d$ on  $\mathbb S^3$
such that the $\mathbb S^1$-orbits become geodesics 
and the quotient $(\mathbb S^3,d)/\mathbb S^1$
is isometric to $\hat \Delta$.

In the constructed example 
the singular points with total angles $2\cdot\tfrac\pi p$ and
$2\cdot\tfrac\pi q$
should correspond to the points with isotropy groups $\ZZ/p$ and $\ZZ/q$ of the action.
The points in $P=(\mathbb{S}^3,d)$ on the orbits over these points will be regular points of $P$.
The singular locus $P^\star$
of $P$ will be formed by the orbit corresponding to the remaining singular point of  $\hat \Delta$.
By construction,
\begin{itemize}
\item $P^\star$ is a closed geodesic with angle $2\cdot\pi\cdot(1-\tfrac1p)$ around it.
\item $P^\star$ forms a $(p,q)$-torus knot in the ambient $\mathbb{S}^3$.
\end{itemize}


\parit{Comments.}
It is expected that only the torus knots can appear this way.

The construction given by Dmitri Panov in \cite{panov-Kaeler}.
The cone $K$ over $P$ is a polyhedral space with natural complex structure;
that is, one can cut simplices from $\CC^2$ and the glue the cone from them in such a way that complex structures will agree along the gluings.
Moreover the cone $K$ can be holomorphically parametrized by $\CC^2$ in such a way that its singular set becomes an algebraic curve $z^p=w^q$ in some $(z,w)$-coordinates of $\CC^2$.



%%%%%%%%%%%%%%%%%%%%%%%%%%%%%%%%%%%%%%%%%%%%%%%%%%
\parbf{\ref{Triangulation of a torus}.} 
\textit{Triangulation of a torus.}
Let us equip the torus with the flat metric such that each triangle is equilateral.
The metric will have two singular cone points,
the first corresponds to the vertex $v_5$ with 5 triangles,
the total angle around this point is $\tfrac53\cdot\pi$
and the second corresponds to the vertex $v_7$ with 7 triangles,
the total angle around this point is $\tfrac73\cdot\pi$.

Prove the following.

\parbf{Observation} \textit{The holonomy group of this metric is generated by rotation by $\tfrac\pi3$.}

\medskip

Consider a closed geodesic $\gamma_1$ which minimize the length of all circles which are not null-homotopic.
Let $\gamma_2$ be an other closed geodesic which minimize the length and is not homotopic to any power of $\gamma_1$.

Show that $\gamma_1$ and $\gamma_2$ intersect at a single point.

Show that $\gamma_i$ can not pass $v_5$.

Apply the observation above 
to show that if $\gamma_i$ pass through $v_7$ then the measure  
of one of two angles which $\gamma_i$ cuts at $v_7$ equals to $\pi$.
Use the later statement to show that  
one can push $\gamma_i$ aside so it does not longer pass through $v_7$, but remains a closed geodesic.

Cut $\TT^2$ along $\gamma_1$ and $\gamma_2$.
In the obtained quadrilateral, connect $v_5$ to $v_7$ by a minimizing geodesic and cut along it.
This way we obtain an annulus with flat metric.
Look at the neighborhood of the boundary components and show that the anulus can and can not be isometrically immersed into the plane;
this is a contradiction.

\begin{wrapfigure}{o}{20mm}
%\begin{center}
\begin{lpic}[t(-7mm),b(-4mm),r(0mm),l(0mm)]{pics/57-triangulation(1)}
%\lbl[b]{24,41;x}
\end{lpic}
%\end{center}
\end{wrapfigure}

\parit{Comments.}
There are flat metrics on the torus with 
only two singular points 
which have the total angles $\tfrac53\cdot\pi$ and $\tfrac73\cdot\pi$.
Such example can be obtained by identifying the the hexagon on the picture  according to the arrows.
But the holonomy group of the obtained torus is generated by the rotation by angle $\tfrac\pi6$. 
In particular, the observation is necessary in the proof.

The same argument shows that 
holonomy group of flat torus with exactly two singular points with total angle $2\cdot(1\pm \tfrac1n)\cdot\pi$ has more than $n$ elements.
In the solution we did the case $n=6$.

The problem was originally discovered and solved by Stanislav Jendro{\soft{l}}
and Ernest Jucovi\v{c}, in \cite{jendrol-jucovich},
their proof is combinatorial.
The solution described above was given by Rostislav Matveyev
in his lectures \cite{matveyev}.
A complex-analytic proof was later found by Ivan Izmestiev, Robert Kusner, G{\"u}nter Rote, Boris Springborn and John Sullivan in \cite{izmestiev-rote-springborn-kusner}.

%%%%%%%%%%%%%%%%%%%%%%%%%%%%%%%%%%%%%%%%%%%%%%%%%%
\parbf{\ref{Unique geodesics imply CAT}.} 
\textit{Unique geodesics imply $\mathrm{CAT}(0)$.}
Uniqueness of geodesics implies that $P$ is contractable.
In particular, $P$ is simply connected.

It remains to prove that $P$ is locally $\mathrm{CAT}(0)$;
equivalently, the space of directions $\Sigma_p$
at any point $p\in P$ is  a $\mathrm{CAT}(1)$ space.

We can assume that the statement holds in all dimensions less than $\dim P$. 
In particular, $\Sigma_p$ is locally $\mathrm{CAT}(1)$.
If $\Sigma_p$ is not $\mathrm{CAT}(1)$ then it contains a periodic geodesic $\gamma$ of length $\ell<2\cdot\pi$,
such that any arc of $\gamma$ of length $\tfrac\ell2$ is length minimizing.

Consider two points $x$ and $y$
in the tangent cone of $p$
in directions $\gamma(0)$ and $\gamma(\tfrac\ell2)$.
Show that there are two distinct minimizing geodesics between $x$ and $y$.
The later leads to a contradiction.

\parit{Comments.}
The existence of geodesic $\gamma$ was proved by Brian Bowditch in \cite{bowditch};
a simpler proof can be found in the book 
by Stephanie Alexandr, Vitali Kapovitch and me, see \cite{akp}.



%%%%%%%%%%%%%%%%%%%%%%%%%%%%%%%%%%%%%%%%%%%%%%%%%%
\parbf{\ref{No simple geodesic}.} 
\textit{No simple geodesic.}
The curvature of a vertex on the surface of a convex polyhedron
is defined as the $2\cdot\pi-\theta$, where $\theta$ is the total angle around the vertex.

Notice that a simple closed geodesic cuts the surface into two discs with total curvature $2\cdot\pi$ each.
Therefore it is sufficient to construct a convex polyhedron with curvatures of the vertices $\omega_1,\omega_2,\dots,\omega_n$ such that
$2\cdot\pi$ can not be obtained as sum of some of $\omega_i$.
An example of that type can be found among 3-simplexes.
 
\parit{Comments.} The problem is due to Gregory Galperin, see \cite{galperin}.

\section*{Discrete geometry}
%%%%%%%%%%%%%%%%%%%%%%%%%%%%%%%%%%%%%%%%%%%%%%%%%%
\parbf{\ref{box-in-box}.} 
\textit{Box in a box.}
Let $\Pi$ be a parallelepiped
with dimensions $a$, $b$ and $c$.
Denote by $v(r)$ the volume of  $r$-neighborhoodsof $\Pi$,
 
Note that for all positive $r$ we have
\[v(r)=w_3+w_2\cdot r+w_1\cdot r^2+w_0\cdot r^3,\leqno({*})\]
where 
\begin{itemize}
\item $w_0=\tfrac43\cdot \pi$ is the volume of unit ball,
\item $w_1=\pi\cdot (a+b+c)$,
\item $w_2=2\cdot(a\cdot b+b\cdot c+c\cdot a)$ is the surface area of $\Pi$,
\item $w_3=a\cdot b\cdot c$ is the volume of $\Pi$,
\end{itemize}

Assume $\Pi'$ be an other parallelepiped
with dimensions $a'$, $b'$ and $c'$.
For the volume $v'(r)$ the volume of  $r$-neighborhoods of $\Pi'$ we have a formula similar $({*})$.

Note that if $\Pi\subset \Pi'$ then $v(r)\le v'(r)$ for any $r$.
Checking this inequality for $r\to\infty$,
we get 
\[a+b+c\le a'+b'+c'.\]

\parit{Comments.}
The problem was discussed by Alexander Shen in \cite{shen}.

A formula analogous to $({*})$
holds for arbitrary convex body $B$ in arbitrary dimension $m$.
The coefficient $w_i(B)$ in the polynomial with different normalization constants 
uppear under different names most commonly
\emph{intrinsic volume} and
\emph{quermassintegral}.
They also can be defined as the average 
of area of projections of $B$ to the $i$-dimensional planes.
In particular if $B'$ and $B$ are convex bodies such that $B'\subset B$
then $w_i(B')\le w_i(B)$ for any $i$.
This generalize our problem quite a bit.
Further generalizations lead to so called \emph{mixed volumes},
see \cite{burago-zalgaller} for more on the subject.



%%%%%%%%%%%%%%%%%%%%%%%%%%%%%%%%%%%%%%%%%%%%%%%%%%
\parbf{\ref{Round circles}.} 
\textit{Round circles in $\mathbb{S}^3$.}
For each circle consider the containing it plane in $\RR^4$.
Note that the circles are linked 
if and only if 
the corresponding planes intersect at a single point inside $\mathbb{S}^3$.

Take the intersection of the planes with the sphere of radius $R\ge 1$,
rescale and pass to the limit as $R\to\infty$.  
This way we get needed isotopy.

\parit{Comments.} 
The problem was discussed in the thesis of Genevieve Walsh, see \cite{walsh}.

%%%%%%%%%%%%%%%%%%%%%%%%%%%%%%%%%%%%%%%%%%%%%%%%%%
\parbf{\ref{Harnack}.} 
\textit{Harnack's circles.}
Let $\sigma\subset \RP^2$ be a smooth algebraic curve of degree $d$.
Consider the complexification $\Sigma\subset \CP^2$ of $\sigma$.
Without loss of generality, we may assume that $\Sigma$ is regular.

Prove that all regular complex algebraic curves of degree $d$ in $\RP^2$
are homeomorphic to each other.
Straightforward calculation show that $\Sigma$ has genus $n\z=\tfrac12\cdot(d^2-3\cdot d+4)$.

The real curve $\sigma$ forms the fixed point set of $\Sigma$ by complex conjugation. 
Prove that each connected component of $\sigma$ adds $1$ to the genus of $\Sigma$.
Hence the result follows.

\parit{Comment.}
This problem was suggested by Greg Kuperberg, see \cite{One-step problems in geometry}.



%%%%%%%%%%%%%%%%%%%%%%%%%%%%%%%%%%%%%%%%%%%%%%%%%%
\parbf{\ref{2pts-on-line}.} 
\textit{Two points on each line.}
Take any complete ordering of the set of all lines 
so that each beginning interval has cardinality less than continuum.

Assume we have a set of points $X$ such that each line intersects $X$ at at most $2$ points and cardinality of $X$ is less than continuum.

Choose the least line $\ell$ in the ordering which intersect $X$ 
by $0$ or $1$ point.
Note that the set of all lines intersecting $X$ at two points has cardinality less than continuum.
Therefore we can choose a point on $\ell$ and add it to $X$ so that the remaining lines are not overloaded.

It remains to apply well ordering principle.

\parit{Comments.}
The following problem look similar but far more involved;
a soulution follows from the proof that a square can not be cuted into triangles of equal area given by Paul Monsky in \cite{monsky}.

{\it Subdivide the plane into three everywhere dense sets $A$, $B$ and $C$ such that each line meets exactly two of these sets.
}



%%%%%%%%%%%%%%%%%%%%%%%%%%%%%%%%%%%%%%%%%%%%%%%%%%
\parbf{\ref{Bodies with the same of shadows}.} 
\textit{Bodies with the same of shadows.}
Let $B$ be the unit ball in $\RR^3$ centered at the origin.

Fix small $\eps>0$.
Consider two bodies 
\begin{align*}
B''&=\set{(x,y,z)\in B}{x\le 1-\eps,\  y\le 1-\eps},
\\ 
B'''&=\set{(x,y,z)\in B}{x\le 1-\eps,\  y\le 1-\eps,\  z\le 1-\eps}.
\end{align*}
Prove that $B''$ and $B'''$ have the same shadows.

\parit{Comments.} The question was asked by Joel Hamkins and answered by Sergei Ivanov, see \cite{hamkins}.



%%%%%%%%%%%%%%%%%%%%%%%%%%%%%%%%%%%%%%%%%%%%%%%%%%
\parbf{\ref{pr:Kissing number}.} 
\textit{Kissing number.}
Let $m=\mathop{\rm kiss} B$
and $B_1,B_2,\dots, B_m$ the the copies of $B$ 
which touch $B$ and have no common interior points.
For each $B_i$ consider the vector $v_i$ from the center of $B$ to the center of $B_i$.
Note that $\measuredangle(v_i,v_j)\ge \tfrac\pi3$ if $i\ne j$.

For each $i$,
consider supporting hyperplane $\Pi_i$
to $W$
with outer normal vector $v_i$.
Denote by $W_i$ the reflection of $W$ in $\Pi_i$.

Prove that $W_i$ and $W_j$ have no common interior points if $i\ne j$;
the later gives the needed inequality.

\parit{Comments.}
The proof is given by 
Charles Halberg, 
Eugene Levin 
and Ernst Straus 
in \cite{halberg-levin-straus}.

It is expected that the same inequality holds for the orientation-preserving version of kissing number.



%%%%%%%%%%%%%%%%%%%%%%%%%%%%%%%%%%%%%%%%%%%%%%%%%%
\parbf{\ref{mono-homotopy}.} 
\textit{Monotonic homotopy.}
Note that we can assume
that $h_0(F)$ and $h_1(F)$ both lie in the coordinate $m$-spaces of $\RR^{2\cdot m}=\RR^m\times \RR^m$;
that is,
$h_0(F)\z\subset\RR^m\times\{0\}$
and $h_1(F)\subset  \{0\}\times\RR^m$.

Show that the following homotopy is monotonic
\[h_t(x)=\bigl(h_0(x)\cdot \cos\tfrac{\pi\cdot t}2
\,,\,
 h_1(x)\cdot\sin\tfrac{\pi\cdot t}{2}\bigr).\] 


\parit{Comment.}
This homotopy was discovered by Ralph Alexander in \cite{ralexander}.
It has number of applications, 
one of the most beautiful is the given 
by K{\'a}roly Bezdek 
and Robert Connelly \cite{bezdek-connelly} 
in their proof of 
Kneser--Poulsen  
and Klee--Wagon conjectures in dimension $2$.



%%%%%%%%%%%%%%%%%%%%%%%%%%%%%%%%%%%%%%%%%%%%%%%%%%
\parbf{\ref{Cube}.} 
\textit{Cube.}
Consider the cube $[-1,1]^m\subset \RR^m$.
Any vertex this cube has the form $\bm{q}=(q_1,q_2,\dots,q_m)$,
where  $q_i=\pm1$.

For each vertex $\bm{q}$,
consider the intersection of the corresponding octant with the unit sphere;
that is, the set
\[V_{\bm{q}}=\set{(x_1,x_2,\dots,x_m)\in\mathbb{S}^{m-1}}{q_i\cdot x_i\ge 0\ \text{for each}\ i}.\]

Consider the set $\mathcal{A}\subset\mathbb{S}^{m-1}$
formed by the union of all the sets $V_{\bm{q}}$ for $\bm{q}\in A$.
Note that 
\[\vol_{m-1}\mathcal{A}=\tfrac12\cdot\vol_{m-1}\mathbb{S}^{m-1}\]
and 
\[\vol_{m-2}\partial\mathcal{A}
=
\tfrac k{2^{m-1}}\cdot\vol_{m-2}\mathbb{S}^{m-2},\]
where $k$ is the number of edges of the cube with one end in $A$ and the other in $B$.

It remains to  show that 
\[\vol_{m-2}\partial\mathcal{A}
\ge \vol_{m-2}\mathbb{S}^{m-2}.\]
The later follows from the isoperimetric inequality for $\mathbb{S}^m$. 

\parit{Comment.}
The problem was suggested by Greg Kuperberg, 
see \cite{One-step problems in geometry}.



%%%%%%%%%%%%%%%%%%%%%%%%%%%%%%%%%%%%%%%%%%%%%%%%%%
\parbf{\ref{Right-angled polyhedron}.} 
\textit{Right-angled polyhedron.}
Before coming into proof read 
about \hyperref[Dehn--Sommerville equations]{\emph{Dehn--Sommerville equations}}
on page \pageref{Dehn--Sommerville equations}.

Let $P$ be a right-angled hyperbolic polyhedron of dimension $m$.
Note that $P$ is simple; that is, exactly $m$ facets meet at each vertex of $P$.

From the projective model of hyperbolic plane, 
one can see that for any simple compact hyperbolic polyhedron there is a simple Euclidean polyhedron with the same combinatorics. 
In particular Dehn--Sommerville equations hold for $P$.

Denote by $(f_0,f_1,\dots f_m)$ and $(h_0,h_1,\dots h_m)$ the $f$- and $h$-vectors of $P$.
Recall that $h_i\ge 0$ for any $i$ and $h_0=h_m=1$.
By Dehn--Sommerville equations, we get
\[f_2> \tfrac{m-2}4\cdot f_1.
\leqno({*})\]

Since $P$ is hyperbolic, each 2-dimensional face of $P$ has at least 5 sides.
It follows that
\[f_2\le \tfrac{m-1}5\cdot f_1.\]
The later contradicts $({*})$ for $m\ge 6$.

\parit{Comments.} 
The proof above 
is the core of proof of nonexistance of compact hyperbolic Coxeter's polyhedra of large dimensions 
given by Ernest Vinberg in \cite{vinberg}, see also \cite{vinberg-strong}.

Playing a bit more with the same inequalities, 
one gets nonexistance of  right-angled hyperbolic polyhedra,
in all dimensions starting from $5$.
In 4-dimensional case, an example of a bonded right-angled hyperbolic polyhedron
can be found among regular \emph{120-cells}.






