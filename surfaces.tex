\csname @openrightfalse\endcsname
\chapter{Surfaces}

We assume that the reader is familiar with smooth surfaces and the related definitions
including intrinsic metric, 
geodesics,
convex and saddle surfaces
and different types of curvature.
An introductory course in differential geometry should cover all necessary background material, 
for example \cite[][\S28--29]{hilbert-cohn-vossen}
or  
\cite{toponogov-curves-and-surfaces}.



%%%%%%%%%%%%%%%%%%%%%%%%%%%%%%%%%%%%%%%%%%%%%%
{
\begin{wrapfigure}{r}{25 mm}
\begin{lpic}[t(-0 mm),b(-4 mm),r(0 mm),l(0 mm)]{pics/convex-hat(1)}
\lbl{15,8.6;$\Sigma$}
\lbl[w]{5,18.6;$\Delta$}
\end{lpic}
\end{wrapfigure}%change???


\subsection*{Convex hat}


\begin{pr}{\easy}{Problem}\label{Convex hat}
Let $\Sigma$ be a smooth closed convex surface 
in $\RR^3$ 
and $\Pi$ be a plane which cuts from $\Sigma$ a disc $\Delta$.
Assume that the reflection of $\Delta$ in $\Pi$ lies inside $\Sigma$.
Show that $\Delta$ is \index{convex set}\emph{convex} in the intrinsic metric  of $\Sigma$;
that is, 
if the ends of a minimizing geodesic in $\Sigma$ 
lie in $\Delta$,
then whole geodesic lies in $\Delta$.
\end{pr}

}


\parit{Solution.}
Let $\gamma$ be a minimizing geodesic with the ends in $\Delta$.

Assume $\gamma\backslash\Delta\ne\emptyset$.
Denote by $\hat\gamma$ the curve formed by $\gamma\cap \Delta$ 
and the reflection on $\gamma\backslash\Delta$ in $\Pi$.
Note 
\[\length\hat\gamma=\length\gamma\]
and $\hat\gamma$ runs partly along $\Sigma$ 
and partly outside of $\Sigma$, 
but does not get inside $\Sigma$.

Denote by $\bar\gamma$ the closest point projection of $\hat\gamma$ on $\Sigma$.
Since $\Sigma$ is convex, the closest point projection shrinks the length.
Therefore 
the curve $\bar\gamma$ lies in $\Sigma$, 
it has the same ends as $\gamma$
and
\[\length\bar\gamma<\length\gamma.\]
The latter leads to a contradiction.\qeds

%%%%%%%%%%%%%%%%%%%%%%%%%%%%%%%%%%%%%%%%%%%%%%

{

\begin{wrapfigure}[10]{r}{39 mm}
\begin{lpic}[t(-0 mm),b(-4 mm),r(0 mm),l(0 mm)]{pics/unbend(1)}
\lbl[r]{8.5,26;$p$}
\lbl[r]{9,31;$p_t$}
\lbl[l]{33.5,25;$q$}
\lbl[t]{25,26;$\gamma(t)$}
\lbl{20,10;$\Sigma$}
\end{lpic}
\end{wrapfigure}

\subsection*{Unbended geodesic}

\begin{pr}{}{Unbended geodesic}\label{Unbended geodesic}
Let $\Sigma$ be a smooth closed strictly convex surface 
in $\RR^3$ 
and $\gamma\:[0,\ell]\z\to \Sigma$ be a unit-speed minimizing geodesic in $\Sigma$.
Set $p\z=\gamma(0)$, $q=\gamma(\ell)$ and 
$$p_t=\gamma(t)-t\cdot\dot\gamma(t),$$ 
where $\dot\gamma(t)$ denotes the velocity vector of $\gamma$ at $t$.
\end{pr}

}
\vskip-\medskipamount
\textit{Show that for any $t\in (0,\ell)$,
one {}\emph{cannot see}  $q$ from $p_t$;
that is, the line segment $[p_tq]$ intersects $\Sigma$ at a point distinct from $q$.}



{

\begin{wrapfigure}{r}{21 mm}
\begin{lpic}[t(-0 mm),b(-4 mm),r(0 mm),l(0 mm)]{pics/long-geodesic(1)}
\end{lpic}
\end{wrapfigure}

%%%%%%%%%%%%%%%%%%%%%%%%%%%%%%%%%%%%%%%%%%%%%%
\subsection*{Long geodesic}

Recall that a closed curve 
 called \index{simple closed curve}\emph{simple} if it has no self-intersections.

\begin{pr}{}{Long geodesic}\label{Long geodesic}
Assume that the surface $\Sigma$ of a convex body $B$ in $\RR^3$
admits an arbitrary long simple closed geodesic.
Show that $B$ is a tetrahedron with equal opposite sides.
\end{pr}

}

Let us mention couple of theorems about intrinsic metric on convex surfaces which should help to solve this problem.
These theorems can be proved easily for smooth or polyhedral surfaces
and then the general case can be done by approximation.
A very short but comprehensive introduction to the subject was written by Alexander Alexandrov in \cite{alexandrov1941}.

On a convex surface (not necessary smooth) one could define 
so
called \index{curvature measure}\emph{curvature measure} which we denote further by $\kappa$.
It is the (necessary unique) non-negative measure such that for any triangle $\triangle$, we have
\[\kappa(\triangle)=\alpha+\beta+\gamma-\pi,\] 
where $\alpha$, $\beta$ and $\gamma$ are the angles of $\triangle$, measured in the intrinsic metric of the surface.

For curvature measure, an analog of Gauss--Bonnet formula holds;
in particular
\[\kappa(\Sigma)=4\cdot\pi\]
for any closed convex surface $\Sigma$ in $\RR^3$

%???+PIC

Further, given a triangle $\triangle$ in a metric space,
its model triangle $\tilde \triangle$ is defined as a triangle in the plane with the same side lengths.
The angles $\tilde\alpha$, $\tilde\beta$ and $\tilde\gamma$ of the model triangle are called \index{model angle}\emph{model angle}
of triangle.
The comparison theorem states that for any triangle in a surface with non-negative curvature measure its model angles do not exceed the actual angles; that is,
\[\tilde\alpha\le \alpha,\quad
\tilde\beta\le\beta,\quad
\tilde\gamma\le\gamma.\]
The same holds for the area; that is,
\[\area\tilde\triangle\le \area\triangle.\]


%%%%%%%%%%%%%%%%%%%%%%%%%%%%%%%%%%%%%%%%%%%%%%
\subsection*{Geodesics for birds}

Let $\gamma\:[a,b]\to\RR^3$ be a curve.
The \index{total curvature}\emph{total curvature} of $\gamma$ is defined as the least upper bound of sum of external angles for broken lines inscribed in $\gamma$.
Namely, it is
\[\sup\set{\sum_{i=1}^{n-1}\alpha_i}{a=t_0<t_1<\dots<t_n=b},\]
where $\alpha_i=\pi-\measuredangle \hinge{\gamma(t_{i})}{\gamma(t_{i-1})}{\gamma(t_{i+1})}$.

If $\gamma$ is smooth, and parametrized by the arc length, 
then its total curvature equals to
\[\int\limits_a^b|\ddot\gamma(t)|\cdot dt,\]
the value $|\ddot\gamma(t)|$ is the curvature of $\gamma$ at $\gamma(t)$.

The \index{geodesic}\emph{geodesics} in the following problem are defined as the curves locally minimizing the length;
that is, a sufficiently short arc of the curve containing the given value of parameter is length minimizing.

\begin{pr}{}{Geodesics for birds}\label{liberman}
Let $f\:\RR^2\to\RR$ be a $\ell$-Lipschitz function.
Let $W\subset \RR^3$ be the epigraph of $f$;
that is,
$$W=\set{(x,y,z)\in\RR^3}{z\ge f(x,y)}.$$
Equip $W$ with the induced intrinsic metric.

Show that any geodesic in $W$ 
 has  total curvature at most $2\cdot\ell$. 
\end{pr}

%%%%%%%%%%%%%%%%%%%%%%%%%%%%%%%%%%%%%%%%%%%%%%
\subsection*{Simple geodesic}


\begin{pr}{}{Simple geodesic}\label{Simple geodesic}
Let $\Sigma$ be a complete unbounded convex surface in $\mathbb R^3$.
Show that there is a two-sided infinite geodesic in $\Sigma$ with no self-intersections.
\end{pr}

Let us review couple of statements about Gauss curvature  which might help to solve the problem.

If $\Sigma$ is a convex surface in $\RR^3$ then its Gauss curvature is nonnegative.

Assume that a simply connected region $\Omega$ in a surface $\Sigma$ is bounded by a closed broken geodesic $\gamma$.
Denote by $\kappa(\Omega)$ the integral of Gauss curvaure along $\Omega$.

For any point $p\in\Sigma$ consider outer unit normal vector $n(p)\in\mathbb{S}^2$.
Then 
\[\kappa(\Omega)=\area[n(\Omega)]=2\cdot\pi-\sigma(\gamma),\]
where $\sigma(\gamma)$ denotes the sum of the signed exterior angles of $\gamma$.
In particular,  $\sigma\le2\cdot\pi$.


%%%%%%%%%%%%%%%%%%%%%%%%%%%%%%%%%%%%%%%%%%%%%%
\subsection*{Immersed surface}
\label{Immersed surface}

\begin{pr}{}{Immersed surface}
Let $\Sigma$ be an connected immersed surface in $\RR^3$ with strictly positive Gauss curvature and nonempty boundary $\partial\Sigma$.
Assume $\partial\Sigma$ lies in a plane $\Pi$
and whole $\Sigma$ lies on one side from $\Pi$.
Prove that $\Sigma$ is an embedded disc.
\end{pr}

%%%%%%%%%%%%%%%%%%%%%%%%%%%%%%%%%%%%%%%%%%%%%%
\subsection*{Periodic asymptote}

\begin{pr}{}{Periodic asymptote}\label{Asymptotic geodesic}
Let $\Sigma$ be a closed smooth surface with non-positive curvature
and $\gamma$ be a geodesic in $\Sigma$.
Assume that $\gamma$ is not periodic
and the curvature of $\Sigma$ vanish at every point of $\gamma$.
Show that $\gamma$ does not have a periodic asymptote;
that is, there is no periodic geodesic $\delta$ such that the distance from $\gamma(t)$ to $\delta$  converges to $0$ as $t\to\infty$. 
\end{pr}

%%%%%%%%%%%%%%%%%%%%%%%%%%%%%%%%%%%%%%%%%%%%%%
\subsection*{Saddle surface}
\label{Saddle surface}

Recall that a smooth surface $\Sigma$ in $\RR^3$
is called \index{saddle surface}\emph{saddle} at point $p$ if its principle curvatures at this point have opposite signs. 
We say that $\Sigma$ is {}\emph{saddle} if it saddle at all points.

\begin{pr}{}{Saddle surface}
Let $\Sigma$ be a saddle surface in $\RR^3$
homeomorphic to a disc.
Assume that orthogonal projection to $(x,y)$-plane
maps the boundary of $\Sigma$
injectively to convex closed curve.
Show that the orthogonal projection to $(x,y)$-plane is injective on whole $\Sigma$.

In particular, $\Sigma$ is a graph $z=f(x,y)$ for a function $f$ defined on a convex figure in the $(x,y)$-plane.
\end{pr}


%%%%%%%%%%%%%%%%%%%%%%%%%%%%%%%%%%%%%%%%%%%%%%
\subsection*{Asymptotic line}

The saddle surfaces are defined in the previous problem.

Recall that \index{asymptotic line}\emph{asymptotic line} on the smooth surface $\Sigma\subset \RR^3$
is a curve always tangent to an {}\emph{asymptotic direction} of $\Sigma$; 
that is, a direction with vanishing normal curvature.

\begin{pr}{}{Asymptotic line}\label{asymptotic-line}
Let $\Sigma\subset \RR^3$ be the graph $z\z=f(x,y)$
of smooth function $f$ 
and $\gamma$ be a closed smooth asymptotic line in $\Sigma$.
Assume $\Sigma$ is saddle in a neighborhood of $\gamma$.
Show that the projection of $\gamma$ to the $(x, y)$-plane cannot be star-shaped.
\end{pr}

%%%%%%%%%%%%%%%%%%%%%%%%%%%%%%%%%%%%%%%%%%%%%%
\subsection*{A minimal surface}

Recall that a smooth surface in $\RR^3$ is called \index{minimal surface}\emph{minimal} if its mean curvature vanish at all points.
The \index{mean curvature}\emph{mean curvature} is defined as the sum of the principle curvatures at the point.

\begin{pr}{}{A minimal surface}%
\label{min-surf}
Let $\Sigma$ be a minimal surface in $\RR^3$ which has boundary on a unit sphere.
Assume $\Sigma$ passes through the center of the sphere.
Show that the area of $\Sigma$ is at least $\pi$.
\end{pr}

{

\begin{wrapfigure}[3]{r}{33 mm}
\begin{lpic}[t(-0 mm),b(-0 mm),r(0 mm),l(0 mm)]{pics/zhelob(1)}
\lbl{28.5,11;$\Omega$}
\lbl[bl]{6.3,5.3;$\gamma$}
\end{lpic}
\end{wrapfigure}

%%%%%%%%%%%%%%%%%%%%%%%%%%%%%%%%%%%%%%%%%%%%%%
\subsection*{Round gutter\hard}

A round gutter is the surface shown on the picture.

Formally: consider torus $T$;
that is, a surface generated by revolving a circle in $\RR^3$ about an axis coplanar with the circle.
Let $\gamma\z\subset T$ be one of the circles in $T$ which locally separates positive and negative curvature on $T$;
a plane containing $\gamma$ it tangent to $T$ at all points of $\gamma$.
Let $\Omega$ be an neighborhood of $\gamma$ in $T$.
The surface $\Omega$ will be called 
\index{gutter}\emph{round gutter}
and the circle $\gamma$ will be called its {}\emph{main latitude}.

}

\begin{pr}{\hard}{Gutter}\label{half-torus}
Let $\Omega\subset \RR^3$ is a round gutter with main latitude $\gamma$. 
Assume $\iota\:\Omega\z\to\RR^3$ 
is a smooth length-preserving embedding which is sufficiently close to the identity.
Show that $\gamma$ and $\iota(\gamma)$ are congruent;
that is, there is a motion of $\RR^3$ which sends $\gamma$ to $\iota(\gamma)$
\end{pr}



%%%%%%%%%%%%%%%%%%%%%%%%%%%%%%%%%%%%%%%%%%%%%%
\subsection*{Non-contractible geodesics}


\begin{pr}{\easy}{Non-contractible geodesics}\label{torus}
Give an example of a non-flat metric 
on the $2$-torus such that it has no contractible geodesics.
\end{pr}

%%%%%%%%%%%%%%%%%%%%%%%%%%%%%%%%%%%%%%%%%%%%%%
\subsection*{The last problem of Poincar\'e\hard}

\begin{pr}{\hard}{The last problem of Poincar\'e}\label{The last problem of Poincare}
Let $f\:\CC\to \CC$ be an area preserving homeomorphism
such that 
\[f(z)
=
\left[
\begin{aligned}
&z-i&&\text{if}&&\Re(z)\le -1,
\\
&z+i&&\text{if}&&\Re(z)\ge 1.
\end{aligned}
\right.
\] 
and $f(z+i)=f(z)+i$ for any $z\in\CC$.

Show that $f$ has a fixed point.
\end{pr}

%%%%%%%%%%%%%%%%%%%%%%%%%%%%%%%%%%%%%%%%%%%%%%
\subsection*{Two discs}

\begin{pr}{}{Two discs}\label{Two discs}
Let $\Sigma_1$ and $\Sigma_2$ be two smoothly embedded open discs in $\mathbb R^3$ 
which have a common closed smooth curve $\gamma$.
Show that there is a pair of points  $p_1\in \Sigma_1$ and $p_2\in \Sigma_2$ with parallel tangent planes.
\end{pr}



\section*{Semisolutions}
%%%%%%%%%%%%%%%%%%%%%%%%%%%%%%%%%%%%%%%%%%%%%%%%%%
\parbf{Unbended geodesic.}
Denote by $W$ the closed unbounded set formed by $\Sigma$ and its exterior points.
Fix $t\in (0,\ell)$ and consider the concatenation $\gamma_t$ of the line segment $[p_t\gamma(t)]$ and the arc $\gamma|_{[t,\ell]}$.
The key step in the proof is to show the following:
\begin{itemize}
 \item[$({*})$] The curve $\gamma_t$ is a minimizing geodesic in the intrinsic metric induced on $W$
\end{itemize}
Try to prove it before reading further.

%???+PIC

\medskip

Let $\Pi_t$ be the tangent plane to $\Sigma$ at $\gamma(t)$.
Consider the curve $\alpha(t)=p_t$.
Note that $\alpha(t)\in\Pi_t$,
$\dot\alpha(t)\perp\Pi_t$
and $\dot\alpha(t)$ points to the side of $\Pi_t$ opposite from $\Sigma$ for any $t$.

It follows that for any $x\in\Sigma$ the function  
\[t\mapsto |x - p_t|
\quad\text{and, therefore,}\quad
t\mapsto |x - p_t|_W\] are non-decreasing;
here $|x - p_t|_W$ stays for the intrinsic distance from $x$ to $p_t$ in $W$.

On the other hand, by construction 
\[|q - p_t|_W\le |q - p|_\Sigma;\] 
therefore, from above 
\[|q - p_t|_W= |q - p|_\Sigma\]
for any $t$.
Whence $(*)$ follows.

Assume $q$ is visible from $p_t$ for some $t$;
that is, the line segment $[qp_t]$ intersects $\Sigma$ only at $q$.
From above, 
$\gamma_t$  coincides with the line segment $[qp_t]$.
On the other hand $\gamma(t)$ lies on $\gamma_t$, a contradiction.\qeds

This  observation was used by Anatoliy Milka
to prove a generalization of comparison theorem for convex surfaces \cite[see][]{milka-geod}.

%%%%%%%%%%%%%%%%%%%%%%%%%%%%%%%%%%%%%%%%%%%%%%%%%%
\parbf{Long geodesic.}
By cutting the surface $\Sigma$ along a sufficiently long closed simple geodesic,
we get two discs.
The key step is to show that each of these discs 
is long and thin.

\medskip

Choose one of the discs, say $D$;
equip it with the intrinsic metric further denoted by $|{*}-{*}|_D$.


Since $\Sigma$ has non-negative curvature in the sense of Alexandrov,
so is $D$.
Choose a pair of points $p,q\in\partial D$ which maximize the distance $|p-q|_D$.
Clearly,
\[|x-p|_D,|x-q|_D\le |p-q|_D\] 
for any other point $x\in\partial D$.
By comparison, 
\[\measuredangle[x\,^p_q]\ge \tfrac\pi3.\leqno({*})\]



\begin{wrapfigure}{r}{28 mm}
\begin{lpic}[t(-0 mm),b(-0 mm),r(0 mm),l(0 mm)]{pics/long-geodesic-diam(1)}
\lbl[r]{0,8;$p$}
\lbl[l]{28,8;$q$}
\lbl[b]{10,16;$\gamma_1(t)$}
\lbl[l]{12,9,-60;$\ge\tfrac\pi3$}
\end{lpic}
\end{wrapfigure}

The points $p$ and $q$ divide $\partial D$ into two arcs,
say $\gamma_1$ and $\gamma_2$;
let us parametrize them by arclength from $p$ to $q$. 
Then by $({*})$
\[\tfrac{d}{dt}\left(|x-\gamma_i(t)|_D-|x-\gamma_i(t)|_D\right)
\ge
\tfrac12.\]
In particular
\[|p-q|_D\ge \tfrac18{\cdot}\length[\partial D].\]
That is, if the geodesic was long 
then $D$ has large diameter.

Choose two points $x\in \gamma_1$ and $y\in\gamma_2$ sufficiently close to $p$ such that $|x-q|_D=|y-q|_D$.
By comparison 
\[\area \tilde \triangle qxy\le \area \triangle qxy\le \area \Sigma.\]
It follows that 
\[\begin{aligned}|x-y|_D&\le2\cdot\frac{ \area[\tilde\triangle(xyq)]}{|q-x|_D}
\le 
\\
&\le 
100\cdot\frac{ \area\Sigma}{\length[\partial D]}.
\end{aligned}
\leqno({*}{*})\]

Cut from $D$ along a minimizing geodesic $[xy]$
and consider part (a loon) $L_p$ with the point $p$ in it.
Note that the curvature of $L_p$ is $\alpha+\beta$, where $\alpha$ and $\beta$ the angles as on the diagram.
By comparison $\alpha\ge \tilde\measuredangle(x\,^p_y)$ 
and $\beta\ge \tilde\measuredangle(y\,^p_x)$.
Therefore curvature of $L_p$ is at least $\pi-\tilde\measuredangle(p\,^x_y)$.
In particular, if $|x-y|_D$ much less then $|p-x|_D+|p-y|_D$ then the curvature of $L_p$ is almost $\pi$.

Fix $\eps>0$.
If $\length[\partial D]$ is long enough,
by $({*})$, 
we can find a loon $L_p$ with diameter at most $\eps$,
such that curvature $L_p$ is at least $\pi-\eps$.

Using the same construction for $p$ and $q$ in the disc $D$,
and for the other disc,
we get four loons in $\Sigma$ each of diameter at most $\eps$ and each with curvature at least $\pi-\eps$.

\begin{center}
\begin{lpic}[t(-0 mm),b(-0 mm),r(0 mm),l(0 mm)]{pics/long-geodesic-D(1)}
\lbl{46,6;$D$}
\lbl{6,6;$L_p$}
\lbl[r]{0,6;$p$}
\lbl[l]{92.5,6;$q$}
\lbl[b]{19,11.5;$x$}
\lbl[t]{12.5,.5;$y$}
\lbl[b]{46,12;$\gamma_1$}
\lbl[t]{46,-.5;$\gamma_2$}
\lbl[tr]{16.5,9;{\small $\alpha$}}
\lbl[br]{12.5,4.5;{\small $\beta$}}
\end{lpic}
\end{center}

By Gauss--Bonnet formula, the total curvature of $\Sigma$ is $4\cdot\pi$.
Since $\eps>0$ is arbitrary, we get that there are four singular points in $\Sigma$, each with curvature $\pi$
and the remaining part of $\Sigma$ is flat.

\begin{wrapfigure}{o}{21 mm}
\begin{lpic}[t(-4 mm),b(-3 mm),r(0 mm),l(0 mm)]{pics/akopyan(1)}
\end{lpic}
\end{wrapfigure}

It remains to show that any surface with this property 
is isometric to the surface of a tetrahedron with equal opposite edges.
To do this cut $\Sigma$ along three geodesics which connect one singular point to the remaining three,
develop the obtained flat surface on the plane and think;
also look at the diagram.\qeds

I learned the problem from Arseniy Akopyan,
the case of polyhedra was solved by Vladimir Protasov in \cite{protasov}.

%%%%%%%%%%%%%%%%%%%%%%%%%%%%%%%%%%%%%%%%%%%%%%%%%%
\parbf{Geodesics for birds.}
Consider a geodesic 
\[\gamma\:t\mapsto(x(t),y(t),z(t))\] 
in $W$;
assume it is defined in the interval $\II\subset \RR$.
Let us denote by $\phi$ the total curvature;
it is a measure on $\II$.
We need to estimate $\phi(\II)$.

Denote by $s=s(t)$ the natural parameter of the plane curve \[t\mapsto (x(t),y(t)).\]

Note that the function $f\:s\mapsto z$ is concave.
Indeed, if this is not the case then 
one can shorten $\gamma$ by pushing it up in arbitrary small neighborhood of some interior value $t_0$ in $\II$.
In particular, $\gamma$ is not locally length minimizing in $W$, 
a contradiction.

Given a semi-open interval $\mathbb J=(a,b]\subset \II$,
set
\[\mu(\mathbb J)=f^+(a)\z-f^+(b),\]
where $f^+$ denotes right derivatives.
The function $\mu$ extends to a measure which could be also written as
\[\mu=\tfrac{d^2z}{ds^2}\cdot ds.\]
if $\tfrac{dz^2}{d^2s}$ understood in the sense of distribution.
 
Note that $|\tfrac{dz}{ds}|\le \ell$.
In particular, $\mu(\II)\le 2\cdot\ell$.


Further note that $\phi\le \sqrt{1+\ell^2}\cdot\mu$.
In particular, 
$$\phi(\II)\le 2\cdot\ell\cdot\sqrt{1+\ell^2}.$$

A straightforward improvement of these estimates gives 
$$\phi(\II)\le 2\cdot\ell.$$
\qedsf

This bound is optimal, check a both side infinite geodesic on the graph of  
\[f(x,y)=-\ell\cdot\sqrt{x^2+y^2}.\]

The problem is due to David Berg \cite[see][]{berg}
the same bound for convex $\ell$-Lipschitz surfaces was proved earlier by Vladimir Usov in \cite{usov}.
The main observation (the concavity of the function $s\mapsto z$)
is called \index{Liberman’s lemma}\emph{Liberman’s lemma}; 
it was used yet earlier 
to bound the total curvature
of a geodesic on a convex surface \cite[see][]{liberman}.
This lemma is often useful when it comes geodesics on convex surfaces are considered.

%%%%%%%%%%%%%%%%%%%%%%%%%%%%%%%%%%%%%%%%%%%%%%%%%%
\begin{wrapfigure}{o}{41 mm}
\begin{lpic}[t(-3 mm),b(-0 mm),r(0 mm),l(0 mm)]{pics/bangert(1)}
\end{lpic}
\end{wrapfigure}

\parbf{Simple geodesic.} 
Look at two combinatoric types of self intersections shown on the diagram.
One of them can and the other can not appear as self intersections of geodesic on an unbounded convex surface.
Try to determine which is which before reading further.

\medskip

Let $\gamma$ be a two-sided infinite geodesic in $\Sigma$.
The following is the key statement in the proof.




\parbf{Claim.}
\textit{The geodesic $\gamma$ contains at most one simple loop.}
\medskip

To prove the claim use the following observations.
\begin{itemize}
\item The integral curvature $\omega$ of $\Sigma$ cannot exceed $2\cdot\pi$.
\end{itemize}
Indeed, since $\Sigma$ is unbounded we can choose a sequence of points $x_n\in\Sigma$ which diverge to infinity.
Fix a point $p\in \Sigma$.
Consider the sequence of the line segments $[px_n]$.
Note that its partial limit is a half-line starting from $p$ which lies in the convex region bounde by $\Sigma$.

Consider a coordinate system with this this half-line as the positive half of $z$-axis. 
In these coordinates $\Sigma$ is described as a graph $z=f(x,y)$ for a cnvex function $f$.
In particular the outer normal vectors to $\Sigma$ point in the south hemisphere.
Therefore the area of spherical image of $\Sigma$ is at most $2\cdot\pi$.
As it was mentioned above this area counsides with the integral of Gauss curvature along $\Sigma$.


\begin{itemize}
\item If $\phi$ is the angle at the base of a simple geodesic loop then the integral curvature surrounded by the loop equals to $\pi+\phi$; in particular there are no obtuse loops.
\end{itemize}

Once the claim is proved, 
note that, 
if a geodesic $\gamma$ has a self-intersection,
then it contains a simple loop.
From above there is only one such loop;
it cuts a disc from $\Sigma$ 
and can go around it either clockwise or counterclockwise.
This way we divide all the self-intersecting geodesics 
into two sets which we will call {}\emph{clockwise} and {}\emph{counterclockwise}.

Note that the geodesic $t\mapsto \gamma(t)$ is clockwise 
if and only if 
$t\mapsto \gamma(-t)$
is counterclockwise.
The sets of clockwise and counterclockwise are open and the space of geodesics is connected. 
It follows that there are geodesics 
which are, neither clockwise, nor counterclockwise;
by the definition, these geodesics have no self-intersections.\qeds


The problem is due to Stephan Cohn-Vossen, \cite[Satz 9]{convossen};
generalizations were obtained  by 
Vladimir Streltsov and Alexandr Alexandrov 
\cite[in][]{streltsov-alexandrov} 
and 
by Victor Bangert \cite[in][]{bangert}.

%%%%%%%%%%%%%%%%%%%%%%%%%%%%%%%%%%%%%%%%%%%%%%%%%%
\parbf{Immersed surface.}
Let $\ell$ be a linear function which vanishes on $\Pi$ 
and is positive on $\Sigma$.

Let $z_0$ be a point of maximum of $\ell$ on $\Sigma$;
set $s_0=\ell(z_0)$.
Given $s<s_0$, denote by $\Sigma_s$ the connected component of $z_0$ in $\Sigma\cap\ell^{-1}([s,s_0])$.
Note that for all $s$ sufficiently close to $s_0$
we have
\begin{itemize}
\item $\Sigma_s$ is an embedded disc;
\item $\partial\Sigma_s$ is convex plane curve.
\end{itemize}

Applying open-closed argument, we get that the same holds for all $s\in[0,s_0)$.

Since $\Sigma$ is connected, $\Sigma_0=\Sigma$.
Hence the result follows.\qeds


This problem is discussed in the lectures of Mikhael Gromov \cite[see \S$\tfrac12$~in][]{gromov-SGMC}.

%%%%%%%%%%%%%%%%%%%%%%%%%%%%%%%%%%%%%%%%%%%%%%%%%%
\parbf{Periodic asymptote.}
Assume the contrary.

Passing to a finite cover, we can ensure that the asymptote has no self intersections.
In this case 
the restriction $\gamma|_{[a,\infty)}$  
has no self-intersections, 
if $a$ is large enough.

Cut $\Sigma$ along $\gamma([a,\infty))$ and then cut from the obtained surface an infinite triangle $\triangle$. 
The triangle $\triangle$ should have two sides formed by both sides of cuts along $\gamma$;
let us denote these sides of $\triangle$ by $\gamma_-$ and $\gamma_+$.
Note that 
\[\area\triangle<\area \Sigma<\infty\leqno(*)\]
and both sides $\gamma_\pm$ 
form infinite minimizing geodesics in $\triangle$.

Consider the Busemann function $f$ for $\gamma_+$;
denote by $\ell(t)$ the length of the level curve $f^{-1}(t)$.
Let $-\kappa(t)$  be the total curvature of the sup-level set $f^{-1}([t,\infty))$.  
From Gauss--Bonnet formula,
\[\ell'(t)=\kappa(t).\leqno({*}{*})\]

The level curve $f^{-1}(t)$ can be parametrized by a unit-speed curve, say $\theta_t\:[0,\ell(t)]\to \triangle$.
By coarea formula we have
\[\kappa'(t)
=
-\int\limits_0^{\ell(t)} K_{\theta_t(\tau)}\cdot d\tau,
\]
where $K_x$ denotes the Gauss curvature of $\Sigma$ at the point $x$.
Since $K_{\theta_t(0)}=K_{\theta_t(\ell_t)}=0$ and the surface is smooth,
there is a constant $C$ such that $|K_{\theta_t(\tau)}|\le C\cdot \ell(t)^2$ for all $t$, $\tau$.
Therefore
\[\kappa'(t)\le C\cdot \ell(t)^3 \leqno(\asterism)\]

Together, $({*}{*})$ and $(\asterism)$ imply that there is $\eps>0$ such that
\[\ell(t)\ge \frac\eps{t-a}\]
for any large $t$.
By the coarea formula we get 
\[\area\triangle=\int\limits_a^\infty\ell(t)=\infty;\]
the latter contradicts $(*)$.\qeds

I've learned the problem from 
Dmitri Burago 
and Sergei Ivanov, 
it is originated from a discussion with
Keith Burns, 
Michael Brin 
and Yakov Pesin.

Here is its motivation.
Assume $\Sigma$ be a closed surface with non-positive curvature which is not flat.
The space $\Gamma$ of all unit-speed geodesics $\gamma\:\RR\to\Sigma$ can be identified with the unit tangent bundle $\UU\Sigma$. 
In particular $\Gamma$ comes with a natural choice of measure.
Denote by $\Gamma_0\subset \Gamma$ the set of geodesics which run in the set of zero curvature all the time.
It is expected that $\Gamma_0$ has vanishing measure.
In all known examples $\Gamma_0$ contains only periodic geodesics in only finitely many homotopy classes \cite[read more in][]{hertz}.

%%%%%%%%%%%%%%%%%%%%%%%%%%%%%%%%%%%%%%%%%%%%%%%%%%
\parbf{Saddle surface.}
Denote by $\Sigma^\circ$ the interior of $\Sigma$.
Fix a plane $\Pi$. 
Note that the intersection $\Pi\cap \Sigma^\circ$ 
locally  looks like a curve or two curves intersecting transversally;
in the latter case $\Pi$ is tangent to $\Sigma^\circ$ at the cross-point.

Further note that $\Pi\cap \Sigma^\circ$ has no cycle.
Otherwise $\Sigma$ fails to be saddle at the point of the disc surrounded by the cycle which maximize the distance to $\Pi$.

Note that if $\Sigma$ is not a graph then there is a point $p\in\Sigma$ with vertical tangent plane;
denote it by $\Pi$.
Note that the intersection $\Pi\cap\Sigma$ has cross-point at $p$.

Since the boundary of $\Sigma$ projects injectively to a closed convex curve in $(x,y)$-plane,
the intersection of $\Pi\cap\partial \Sigma$ has at most 2 points --- these are the only endpoints of $\Pi\cap\Sigma$.

It follows that the connected component of $p$ in $\Pi\cap\Sigma$ is a tree 
with a vertex of degree 4 at $p$ and at most two end-points, a contradiction.\qeds

The proof above is based of the observation 
that for any plane $\Pi$,
each connected component of $\Pi\cap\Sigma$ is simply connected.
One can define saddle surfaces as arbitrary (non necessarily smooth) surface which satisfies this condition.
The geometry of these surfaces is far from being understood,
Samuil Shefel has number of beautiful results about them, 
see \cite{shefel} and references there in.


%%%%%%%%%%%%%%%%%%%%%%%%%%%%%%%%%%%%%%%%%%%%%%%%%%
\parbf{Asymptotic line.}
Arguing by contradiction, assume that the projection $\bar\gamma$
of $\gamma$ on $(x, y)$-plane is star shaped with respect to the origin.

Consider the function 
$$h(t)=(d_{\bar\gamma(t)}f)(\gamma(t)).$$
Prove that $h'(t)\ne 0$.
In particular $h(t)$ is a strictly monotonic function of $\mathbb{S}^1$, a contradiction.\qeds

The problem is discussed by Dmitri Panov in \cite{panov-curves}.

%%%%%%%%%%%%%%%%%%%%%%%%%%%%%%%%%%%%%%%%%%%%%%%%%%
\parbf{A minimal surface.}
Without loss of generality we may assume that the sphere is centered at the origin of $\RR^3$.

Consider the restriction $h$ of the function $x\mapsto |x|^2$ to the surface $\Sigma$.
Prove that $\Delta_\Sigma h\le 4$ and apply the divergence theorem for $\nabla_\Sigma h$.
It follows that the function
\[f\:r\mapsto \frac{\area[\Sigma\cap B(0,r)]}{r^2}
\]
is non-decreasing in the interval $(0,1)$.
Hence the result follows.\qeds

We described a partial case of so called \index{monotonicity formula}\emph{monotonicity formula}.

The same argument shows that if $0$ is a double point
of $\Sigma$ then $\area\Sigma\ge 2\cdot \pi$.
This observation was used in the proof 
that the minimal disc bounded by a simple closed curve with total curvature $\le 4\cdot\pi$ 
is necessarily embedded.
It was proved by 
;
an amusing simplification and generalization
was obtained by 
Stephan Stadler. %???REF
This result also implies that any embedded circle of total curvature at most $4\cdot\pi$ is unknot.
The latter was proved inependently by Istv{\'a}n F{\'a}ry \cite[in][]{fary-knot} and  John Milnor \cite[in][]{milnor}.

Note that if we assume in addition that the surface is a disc,
then the statement holds for any saddle surface. 
Indeed, denote by $S_r$ the sphere of radius $r$ concentrated with the unit sphere. 
Then according to the problem ``A curve in a sphere'' [page \pageref{A curve in a sphere}], 
\[\length(\Sigma\cap S_r)\ge 2\cdot\pi\cdot r.\]
Then the coarea formula leads to the solution.

On the other hand there are saddle surfaces homeomorphic to the cylinder
that may have arbitrary small area in the ball. 

If $\Sigma$ does not pass through the center 
and we only know the distance, say $r$, 
from the center to $\Sigma$,
then the optimal bound is $\pi\cdot(1-r^2)$.
It was conjectured for about 40 years and proved by Simon Brendle and Pei-Ken Hung in \cite{brende-hung};
their proof is based on a similar idea and quite elementary.
Earlier Herbert Alexander, 
David Hoffman
and Robert Osserman 
proved it in two cases (1) if $\Sigma$ is homeomorphic to a disc and (2) for arbitrary area minimizing surfaces, any dimension and codimension
 \cite[see][]{alexander-osserman,alexander-hoffman-osserman}.






%%%%%%%%%%%%%%%%%%%%%%%%%%%%%%%%%%%%%%%%%%%%%%%%%%
\parbf{Round gutter.}
Let $K$ be the convex hull of $\Omega'=\iota(\Omega)$.
Consider the boundary curve $\gamma'$ of $\partial K\cap \Omega'$ in $\Omega'$.

First note that the Gauss curvature of $\Omega'$ has to vanish at the points of $\gamma'$;
in other words, $\gamma'=\iota(\gamma)$.
Indeed since $\gamma'$ lies on convex part, 
the Gauss curvature at the points of $\gamma'$ has to be non-negative. 
On the other hand $\gamma'$ bounds a flat disc in $\partial K$;
therefore its integral intrinsic curvature has to be $2{\cdot}\pi$.
If the Gauss curvature is positive at some point of $\gamma'$, 
then total intrinsic curvature of $\gamma'$ has to be $<2{\cdot}\pi$, a contradiction.

Prove that $\gamma'$ is an asymptotic line.
(Hint: assume that the asymptotic direction goes transversely to $\gamma'(t)$ and conclude $\gamma(t)\notin\partial K$.)

Without loss of generality, we can assume that the length of $\gamma$ is $2{\cdot}\pi$ and its intrinsic curvature is $1$ at all points.
Therefore, as the space curve,
$\gamma'$ has to be a curve with constant curvature $1$ and it should be closed.
Any such curve is congruent to a unit circle.\qeds

It is not known if $\Omega'$ is congruent to $\Omega$.

The solution presented above is based on my answer 
to the question of Joseph O'Rourke \cite[see][]{rourke}.
Here are some related statements.
\begin{itemize}
\item A half-torus is second order rigid;
this was proved by Eduard Rembs in
\cite{rembs}, see also \cite[][135]{efimov}.
\item Any second order rigid surface does not admit analytic deformation 
\cite[proved by Nikolay Efimov, see][121]{efimov}
and for the surfaces of revolution, the assumption of analyticity can be removed 
\cite[proved by Idzhad Sabitov, see][]{sabitov}.
\end{itemize}









%%%%%%%%%%%%%%%%%%%%%%%%%%%%%%%%%%%%%%%%%%%%%%%%%%
\parbf{Non-contractible geodesics.}
Take a torus of revolution $T$.
It has a family {}\emph{meridians} --- the family of circles which form closed geodesics.

Note that a geodesic on $T$ is either a meridian
or it intersects meridians transversally.
No closed curve of these types can be contractible.\qeds 




I learned this problem 
from the book of Mikhael Gromov \cite[see][]{gromov-MetStr},
where it is attributed to Y. Colin de Verdi\`ere.
I am not not know any generic metric of that type.

%%%%%%%%%%%%%%%%%%%%%%%%%%%%%%%%%%%%%%%%%%%%%%%%%%


\begin{wrapfigure}{r}{44 mm}
\begin{lpic}[t(-6 mm),b(0 mm),r(0 mm),l(0 mm)]{pics/birkhoff(1)}
\lbl[r]{33,44.5;{\small $\hat\gamma(0)$}}
\lbl[r]{31,57.5;{\small $\hat\gamma(1)$}}
\lbl[rb]{6.5,63;{\small $\hat\gamma(n)$}}
\lbl[l]{10.3,28.3;{\small $\check\gamma(0)$}}
\lbl[l]{12.4,18.6;{\small $\check\gamma(1)$}}
\lbl[l]{37,19;{\small $\check\gamma(n)$}}
\lbl[b]{21.5,35.5;$0$}
\lbl[br]{33.5,35.5;$1$}
\lbl{40,3;$H_+$}
\lbl{3,3;$H_-$}
\end{lpic}
\end{wrapfigure}

\parbf{The last problem of Poincar\'e.}
Set 
\begin{align*}
H_+&=\set{z\in\CC}{\Re(z)\ge 1},
\\
H_-&=\set{z\in\CC}{\Re(z)\le -1}.
\end{align*}

Assume $f$ has no fixed points;
in other words the image of the map 
\[\phi\:z\mapsto f(z)-z\] 
lies in $\CC^*=\CC\backslash\{0\}$.


Fix $\eps>0$ such that $|f(z)-z|>\eps$ for any $z\in\CC$.
Note that the map 
\[\check f\:z\mapsto f(z)+\eps\]
is area preserving and has no fixed points.


Prove that for some positive integer $n$,
there is a curve 
\[\check \gamma\:[0,n]\to \CC\]
which starts in $H_-$, ends in $H_+$
and 
$\check f\circ\check\gamma(t)=\check\gamma(t+1)$
for any $t\z\in [0,n-1]$.

Repeat the same construction for the function 
\[\hat f(z)=f(z)-\eps\] 
and obtain a curve 
\[\hat \gamma\:[0,m]\to \CC\] starting in $H_+$ and ending in $H_-$.

Connect $\check\gamma(n)$ to $\hat \gamma(0)$ by a curve in $H_+$ 
and 
$\hat\gamma(m)$ to  $\check\gamma(0)$ by a curve in $H_-$.
Denote by $\sigma$ the obtained loop.

Prove that
\begin{itemize}
\item The loop $\phi\circ\sigma$ has to be null-homotopic in $\CC^*$.
\item The loop $\phi\circ\sigma$ is a generator of $\pi_1\CC^*$.
\end{itemize}

\noindent
These two statements contradict each other. \qeds


The question was asked by Henri Poincar\'e \cite[see][]{poincare}
and answered by George Birkhoff in \cite{birkhoff}.



 






%%%%%%%%%%%%%%%%%%%%%%%%%%%%%%%%%%%%%%%%%%%%%%%%%%
\parbf{Two discs.}
Choose a continuous map $h\:\Sigma_1\to \Sigma_2$
which is identical on $\gamma$.
Let us prove that for some $p_1\in \Sigma_1$ and $p_2=h(p_1)\in \Sigma_2$
the tangent plane $\T_{p_1} \Sigma_1$ is parallel to the tangent plane $\T_{p_2} \Sigma_2$;
this is stronger than required.

Arguing by contradiction,
assume that such point does not exist.
Then for each $p\in\Sigma_1$
there is unique line $\ell_p\ni p$ 
which is parallel to each of the tangent planes $\T_{p} \Sigma_1$ and $\T_{h(p)} \Sigma_2$.

Note that the lines $\ell_p$ form a tangent line distribution over $\Sigma_1$
and $\ell_p$ is tangent to $\gamma$ at any $p\in\gamma$.

Let $\Delta$ be the disc in $\Sigma_1$ bounded by $\gamma$.
Consider the doubling of $\Delta$ along  $\gamma$;
it is diffeomorphic to $\mathbb S^2$.
The line distribution $\ell$ lifts to a line distribution on the doubling;
the latter contradicts the hairy ball theorem.\qeds


This proof was suggested nearly simultaneously 
by Steven Sivek 
and Damiano Testa \cite[see][]{two-discs}.

Note that the same proof works in case $\Sigma_i$ are oriented open surfaces such that $\gamma$ cuts a compact domain in each $\Sigma_i$.

There are examples of three disks $\Sigma_1$, $\Sigma_2$ and $\Sigma_3$
with a common closed curve $\gamma$ such that there 
no triple of points $p_i\in\Sigma_i$ with parallel tangent planes.
Such examples can be found among ruled surfaces \cite[see][]{three-discs}.