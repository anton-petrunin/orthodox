\csname @openrightfalse\endcsname
\chapter{Topology}

In this chapter we consider geometrical problems with strong topological flavor.
A typical introductory course in topology, say \cite{kosniowski},
contains all the necessary material.


%%%%%%%%%%%%%%%%%%%%%%%%%%%%%%%%%%%%%%%%%
\subsection*{Isotropy}

Recall that an isotopy is a continuous one parameter family of embeddings.

\begin{pr}{}{Isotropy}\label{Isotropy}
Let $K_1$ and $K_2$ be homeomorphic compact subsets of the coordinate subspace $\RR^m$ in $\RR^{2\cdot m}$.
Show that there is a homeomorphism 
\[h\:\RR^{2\cdot m}\z\to \RR^{2\cdot m}\] 
such that $K_2=h(K_1)$.
Moreover, $h$ can be chosen to be isotopic to the identity map.
\end{pr}

%%%%%%%%%%%%%%%%%%%%%%%%%%%%%%%%%%%%%%%%%%%%%%%%%%
\parit{Semisolution.}
Fix a homeomorphism $\phi\:K_1\to K_2$.

By Tietze extension theorem,
the homeomorphisms $\phi\:K_1\to K_2$ and $\phi^{-1}\:K_2\to K_1$ can be extended to a continuous maps,
say $f\:\RR^m\to \RR^m$ and $g\:\RR^m\to \RR^m$ correspondingly.

Consider the homeomorphisms
$h_1, h_2, h_3\:\RR^m\times\RR^m\to\RR^m\times\RR^m$ defined the following way
\begin{align*}
h_1(x,y)&=(x,y+f(x)),
\\
h_2(x,y)&=(x-g(y),y),
\\ 
h_3(x,y)&=(y,-x).
\end{align*}

It remains to prove that each homeomorphism $h_i$ is isotopic to the identity map and we have $K_2=h(K_1)$ for
\[h=h_3\circ h_2\circ h_1.\]
\qedsf 

The problem is due to Victor Klee \cite[see][]{klee}.
The problem ``Monotonic homotopy'' on page \pageref{mono-homotopy} is closely related.

%%%%%%%%%%%%%%%%%%%%%%%%%%%%%%%%%%%%%%%%%
\subsection*{Immersed disks}

Two immersions $f_1,f_2\:D\looparrowright \RR^2$ are called \index{essentially different immersions}\emph{essentially different} 
if there is no diffeomorphism $h\:D\z\to D$ such that
$f_1=f_2\circ h$.

\begin{pr}{}{Immersed disks}\label{Immersed disks} 
Construct two essentially different smooth immersions of the disk 
into the plane which coincide near the boundary. 
\end{pr}

%%%%%%%%%%%%%%%%%%%%%%%%%%%%%%%%%%%%%%%%%
\subsection*{Positive Dehn twist}


Let $\Sigma$ be a surface and $\gamma\:\RR/\ZZ\to\Sigma$ be non-contractible closed simple curve.
Let $U_\gamma$ be a neighborhood of $\gamma$ which admits a homeomorphism $h\:U_\gamma\to \RR/\ZZ\times (0,1)$.
\index{Dehn twist}\emph{Dehn twist} along $\gamma$ is a homeomorphism $f\:\Sigma\z\to\Sigma$
which is identity outside of $U_\gamma$ and 
such that
$$h\circ f\circ h^{-1}\:(x,y)\mapsto(x+y,y).$$

If $\Sigma$ is oriented 
and $h$ is orientation preserving,
then the Dehn twist described above is called \index{positive Dehn twist}\emph{positive}.

\begin{pr}{\easy}{Positive Dehn twist}\label{Positive Dehn twist} Let $\Sigma$ be an oriented surface with non empty boundary.
Prove that any composition of positive Dehn twists of $\Sigma$ is not homotopic to identity rel. boundary.
\end{pr}

%%%%%%%%%%%%%%%%%%%%%%%%%%%%%%%%%%%%%%%%%
\subsection*{Function with no critical points}

\begin{pr}{}{Function with no critical points}\label{Function with no critical points}
Given an integer $m\ge 2$, 
construct a smooth function $f\:\RR^m\to \RR$ 
with no critical points in the unit ball $B^m$ 
such that the restriction $f|_{B^m}$ does not factor through a linear function;
that is, 
$f|_{B^m}$ cannot be presented as a composition
$\ell\circ\phi$,
where $\ell\:\RR^m\to\RR$ is a linear function 
and $\phi\:B^m\to\RR^m$ is a smooth embedding.
\end{pr}

%%%%%%%%%%%%%%%%%%%%%%%%%%%%%%%%%%%%%%%%%
\subsection*{Conic neighborhood}
\label{Conic neighborhood}

Let $p$ be a point in a topological space $X$.
We say that an open neighborhood $U\ni p$ is \index{conic neighborhood}\emph{conic}
if there is a homeomorphism from a cone
to $U$ which sends its vertex to $p$.

\begin{pr}{}{Conic neighborhood}  
Show that any two conic neighborhoods of one point are homeomorphic to each other.
\end{pr}

%%%%%%%%%%%%%%%%%%%%%%%%%%%%%%%%%%%%%%%%%
\subsection*{Unknots}

\begin{pr}{}{Unknots}\label{No knots}
Prove that the set of smooth embeddings $f\:\mathbb{S}^1\z\to\RR^3$ equipped with the $C^0$-topology 
forms a connected space.
\end{pr}

%%%%%%%%%%%%%%%%%%%%%%%%%%%%%%%%%%%%%%%%%
\subsection*{Stabilization}

\begin{pr}{}{Stabilization}\label{Simple stabilization}
Construct two compact subsets $K_1, K_2\subset\RR^2$ such that
$K_1$ is not homeomorphic to $K_2$, but $K_1\times[0,1]$ is homeomorphic to $K_2\z\times[0,1]$.
\end{pr}

%%%%%%%%%%%%%%%%%%%%%%%%%%%%%%%%%%%%%%%%%
\subsection*{Homeomorphism of cube}

\begin{pr}{}{Homeomorphism of cube}\label{Homeomorphism of cube}
Let $\square^m$ be a cube in $\RR^m$
and $h\:\square^m\z\to\square^m$ be
a homeomorphism which sends each face of $\square^m$ to itself.
Extend $h$ to a homeomorphism $f\:\RR^m\to\RR^m$ which coincides with the identity map outside of a bounded set.    
\end{pr}

%%%%%%%%%%%%%%%%%%%%%%%%%%%%%%%%%%%%%%%%%
\subsection*{Finite topological space\easy}


\begin{pr}{\easy}{Finite topological space}\label{Finite topological space}
Given a finite topological space $F$ 
construct a finite simplicial complex $K$
which admits a weak homotopy equivalence  $K\to F$. 
\end{pr}

%%%%%%%%%%%%%%%%%%%%%%%%%%%%%%%%%%%%%%%%%
\subsection*{Dense homeomorphism\easy}

\begin{pr}{\easy}{Dense homeomorphism}\label{Dense homeomorphism}
Let $\mathcal{H}$ be the set of all homeomorphisms $\mathbb {S}^2\to\mathbb {S}^2$ 
equipped with the $C^0$-metric.
Show that there is a homeomorphism $h\in \mathcal{H}$ such that its conjugations $a\circ h\circ a^{-1}$ for all $a\in\mathcal{H}$ form a dense set in $\mathcal{H}$.
 
\end{pr}

%%%%%%%%%%%%%%%%%%%%%%%%%%%%%%%%%%%%%%%%%
\subsection*{Simple path\easy}


\begin{pr}{\easy}{Simple path}\label{Simple path}
Let $p$ and $q$ be distinct points in Hausdorff topological space $X$.
Assume $p$ and $q$ are jointed by a path.
Show that they can be jointed by a simple path;
that is there is an injective continuous map $\beta\:[0,1]\to X$
such that $\beta(0)=p$ and $\beta(1)=q$.
\end{pr}









\section*{Semisolutions}
%%%%%%%%%%%%%%%%%%%%%%%%%%%%%%%%%%%%%%%%%%%%%%%%%%






%%%%%%%%%%%%%%%%%%%%%%%%%%%%%%%%%%%%%%%%%%%%%%%%%%
\parbf{Immersed disks.}
Both circles on the picture bound essentially different discs.

{

\begin{wrapfigure}[6]{r}{46 mm}
\begin{lpic}[t(-0 mm),b(0 mm),r(0 mm),l(0 mm)]{pics/milnors-discs()}
\end{lpic}
\end{wrapfigure}

It is a good exercise to count the discs in these examples. 
(The answers are 2 and  5 correspondingly.) 
\qeds

The first example is generally attributed to John Milnor.
The second example was given by Daniel Bennequin in \cite{bennequin}.

}

\begin{wrapfigure}[6]{r}{19 mm}
\begin{lpic}[t(-6 mm),b(0 mm),r(0 mm),l(0 mm)]{pics/annuli()}
\end{lpic}
\end{wrapfigure}

An easier problem would be to construct two essentially different immersions of annuli with the same boundary curves; a solution is shown on the picture.


%%%%%%%%%%%%%%%%%%%%%%%%%%%%%%%%%%%%%%%%%%%%%%%%%%
\parbf{Positive Dehn twist.}
Consider the universal covering 
$\tilde\Sigma\to\Sigma$.
The surface $\tilde \Sigma$ comes with the orientation induced from $\Sigma$.

Note that we may assume that $\tilde\Sigma$ has infinite number of boundary components.

Fix a point $x_0$ on the boundary of $\tilde \Sigma$.
Given other points $y$ and $z$ we will write
$y\prec z$ if $z$ lies on the left side from one (and therefore any) simple curve from $x_0$ to $y$ in $\tilde\Sigma$.
Note that  $\prec$ defines a linear order on $\partial\tilde\Sigma\backslash\{x_0\}$.
We will write $y\preceq z$ 
if $y\prec z$ or $y=z$.

\begin{center}
\begin{lpic}[t(1 mm),b(1 mm),r(0 mm),l(0 mm)]{pics/dehn()}
\lbl[t]{13,6;$x_0$}
\lbl[bl]{23,24;$y$}
\lbl[br]{4,20;$z$}
\lbl[]{33,12;$\tilde \Sigma$}
\end{lpic} 
\end{center}

Note that any homeomorphism $h\:\Sigma\to\Sigma$ which is identity on the boundary
lifts to unique homeomorphism $\tilde h\:\tilde \Sigma\to\tilde\Sigma$ 
is such a way that $\tilde h(x_0)=x_0$.

Assume $h$ is positive Dehn twist.
Show that 
$y\preceq \tilde h(y)$ for any  $y\in\partial\tilde\Sigma\backslash\{x_0\}$
and there is a point $y_0\in\partial\tilde\Sigma\backslash\{x_0\}$
such that $y_0\prec \tilde h(y_0)$.

Note that the latter property is a homotopy invariant 
and it survives under compositions of maps.
Hence the statement follows.

\parit{Comments.} The problem was suggested by Rostislav Matveyev.

%%%%%%%%%%%%%%%%%%%%%%%%%%%%%%%%%%%%%%%%%%%%%%%%%%
\begin{wrapfigure}{r}{24 mm}
\begin{lpic}[t(-0 mm),b(0 mm),r(0 mm),l(0 mm)]{pics/no-critical-points()}
\end{lpic}
\end{wrapfigure}


\parbf{Function with no critical points.}
Construct an immersion 
$\psi\:B^m\z\looparrowright\RR^m$ such that 
\[\ell\circ\phi\ne\ell\circ\psi\]
for any embedding  $\phi\:B^m\to\RR^m$. 
(The two-dimensional case can be quessed from the picture.)

It remains to note that the composition $f=\ell\circ\psi$ has no critical points.\qeds

The problem was suggested by Petr Pushkar.

%%%%%%%%%%%%%%%%%%%%%%%%%%%%%%%%%%%%%%%%%%%%%%%%%%
\parbf{Conic neighborhood.}
Let $V$ and $W$ be two conic neighborhoods of $p$.
Without loss of generality, we may assume that the closure of $V$ lies in $W$.

We will need to construct a sequence of embeddings $f_n\:V\to W$
such that 
\begin{enumerate}[(i)]
\item 
For any compact set $K\subset V$ 
there is a positive integer $n=n_K$ such that 
$f_n(k)=f_m(k)$ for any $k\in K$ and $m\ge n$.
\item For any point $w\in W$ there is a point $v\in V$ such that $f_n(v)=w$ for all large $n$.
\end{enumerate}

Note that once such sequence is constructed, $f\:V\to W$ defined as $f(v)=f_n(v)$ for all large values of $n$ gives the needed homeomorphism.

The sequence $f_n$ can be constructed recursively, setting
\[f_{n+1}=\Psi_n\circ f_n\circ \Phi_n,\]
where $\Phi_n\:V\to V$ 
and $\Psi_n\:W\to W$ 
are homeomorphisms
of the form 
\[\Phi_n(x)=\phi_n(x)\cdot x\quad \Phi_n(x)=\psi_n(x)\cdot x,\]
where $\phi_n\:V\to \RR_+$, $\psi_n\:W\to \RR_+$ are suitable continuous functions 
and 
``$\cdot$'' denotes the ``multiplication'' in the cone structures of $V$ and $W$ correspondingly.\qeds


The problem is due to Kyung Whan Kwun \cite[see][]{kwun}.

Note that for two cones $\mathop{\rm Cone}(\Sigma_1)$ and $\mathop{\rm Cone}(\Sigma_2)$ might be homeomorphic while $\Sigma_1$ and $\Sigma_2$ are not.



%%%%%%%%%%%%%%%%%%%%%%%%%%%%%%%%%%%%%%%%%%%%%%%%%%
\parbf{Unknots.}
Observe that it is possible to draw tight arbitrary knot 
while keeping it smoothly embedded all the time including the last moment.\qeds


\begin{center}
\begin{lpic}[t(-0 mm),b(0 mm),r(0 mm),l(0 mm)]{pics/knot(1)}
\end{lpic}
\end{center}


This problem was suggested by Greg Kuperberg \cite[see][]{One-step}.


%%%%%%%%%%%%%%%%%%%%%%%%%%%%%%%%%%%%%%%%%%%%%%%%%%
{
\begin{wrapfigure}{o}{37 mm}
\begin{lpic}[t(0 mm),b(0 mm),r(0 mm),l(0 mm)]{pics/Simple-stabilization(1)}
\lbl[]{8.7,8.3;{\color{white} $K_1$}}
\lbl[]{28.7,8.3;{\color{white} $K_2$}}
\end{lpic}
\end{wrapfigure}

\parbf{Simple stabilization.}
The example can be guessed from the diagram.\qeds


I learned this problem 
in my analysis class taught by 
Maria Goluzina.

}




%%%%%%%%%%%%%%%%%%%%%%%%%%%%%%%%%%%%%%%%%%%%%%%%%%
\parbf{Homeomorphism of cube.}
Let us extend the homeomorphism $h$ to whole $\RR^m$ by reflecting the cube in its facets.
We get a homeomorphism say $\tilde h\:\RR^m\to\RR^m$ such that $\tilde h(x)=h(x)$ for any $x\in\square^m$ and 
\[\tilde h\circ\gamma=\gamma\circ \tilde h\]
for any motion $\gamma\:\RR^m\to\RR^m$ in the group generated by the reflections in the facets of the cube.

Without loss of generality, we may assume that the cube $\square^m$ is inscribed in the unit sphere centered at the origin of $\RR^m$.
In this case $\tilde h$ has \index{displacement}\emph{displacement} at most $2$;
that is, 
\[|\tilde h(x)-x|\le 2\]
for any $x\in\RR^m$.

\begin{wrapfigure}[10]{o}{47 mm}
\begin{lpic}[t(-4 mm),b(0 mm),r(0 mm),l(0 mm)]{pics/Phi(1)}
\end{lpic}
\end{wrapfigure}

Fix a smooth increasing concave function $\phi\:\RR\to\RR$ such that
$\phi(r)\z=r$ for any $r\le 1$ and $\sup_r\phi(r)=2$.

Consider $\RR^m$ with polar coordinates $(u,r)$, where $u\in\mathbb{S}^{m-1}$ and $r\ge 0$.
Let $\Phi\:\RR^m\to\RR^m$
is defined by $\Phi(u,r)=(u,\phi(r))$.

Set 
\[
f(x)=\left[
\begin{aligned}
&x&&\text{if}\ |x|\ge 2
\\
&\Phi\circ \tilde h \circ \Phi^{-1}(x)&&\text{if}\ |x|< 2
\end{aligned}
\right.
\]
Prove that $f\:\RR^m\to\RR^m$ is a solution.
\qeds

The problem is a stripped from the proof of Robion Kirby in \cite{kirby}.
The condition that face is mapped to face can be removed and 
instead of homeomorphism one can take an embedding which is close enough to the identity.

An interesting twist to this idea was given by Dennis Sullivan in \cite{sullivan}.
Instead of the discrete group of motions of Euclidean space,
he use a discrete group of motions of hyperbolic space in the Poincare model.
Say, assume we repeat the same argument if instead of cube we have a Coxeter polytope in the hyperbolic space.
Then the constructed map 
coincides with the identity on the absolute and therefore the last ``shrinking'' step in the proof above is not needed.
Moreover, 
if the original homeomorphism is bi-Lipschitz,
then the construction also produce a bi-Lipschitz homeomorphism;
this is the advantage.
  

%%%%%%%%%%%%%%%%%%%%%%%%%%%%%%%%%%%%%%%%%%%%%%%%%%
\parbf{Finite topological space.}
Given a point $p\in F$,
denote by $O_p$ the minimal open set in $F$ containing $p$. 
Note that we can assume that $F$ connected and is $T_0$;
that is $O_p=O_q$ if and only if $p=q$.

Let us write $p\preccurlyeq q$ 
if $O_p\subset O_q$.
The relation $\preccurlyeq$ is a partial order on $F$.

Let us construct a simplicial complex $K$ 
by taking $F$ as the set of its vertices
and saying that a collection of vertices form a simplex 
if they can form a an increasing sequence with respect to $\preccurlyeq$.

Given $k\in K$,
consider the minimal simplex $(f_0,\dots,f_m)\ni k$;
we can assume that $f_0\preccurlyeq \dots\preccurlyeq f_m$.
Set $h\:k\mapsto f_0$;
it defines a map $K\to F$.

It remains to check that $h$ is continuous 
and induces an isomorphism of all the homotopy groups.
\qeds

In a similar fashion, one can construct a finite topological space $F$ for given simplicial complex $K$ 
such that 
there is a weak homotopy equivalence $K\to F$.
Both constructions are due to Pavel Alexandrov, 
\cite[see][]{alexandrov-finite,mccord}.

\parbf{Dense homeomorphism.}
Note that there is countable set of homeomorphisms $h_1,h_2,\dots$ which is dense in $\mathcal{H}$
such that
each $h_n$ fix all the points outside an open round disc, say $D_n$.

Choose a countable disjoint collection of round discs $D_n'$
and consider the homeomorphism $h\:\mathbb S^2\to \mathbb S^2$
which fix all the points outside of $\bigcup_nD'_n$ and
for each $n$,
the restriction $h|_{D_n'}$ is conjugate to $h_n|_{D_n}$. 

Show that $h$ solves the problem.
\qeds

The problem was mentioned by Frederic Le Rox \cite[see][]{rox}.

%%%%%%%%%%%%%%%%%%%%%%%%%%%%%%%%%%%%%%%%%%%%%%%%%%
\parbf{Simple path.}
Let $\alpha$ be a path joining $p$ to $q$.

Passing to a subinterval if necessary,
we can assume that $\alpha(t)\ne p,q$ for $t\ne0,1$.

An open set in $[0,1]$ will be called {}\emph{suitable}
if for any connected component $(a,b)$ of $\Omega$ we have $\alpha(a)=\alpha(b)$.
Show that there is a maximal suitable open set $\Omega$;
that $\Omega$ is suitable and it is not a subset of any other suitable set.

Define $\beta(t)=\alpha(a)$ for any $t$ in a connected component $(a,b)\subset\Omega$.

It remains to reparametrize $\beta$ to make it injective.
\qeds

The problem inspired by a Lemma 7.13 
proved by 
Alexander Lytchak
and Stefan Wenger in \cite{lytchak-wenger}

A more involved solution goes the following way:
Note that one can assume that $X$ coincides with the image of $\alpha$.
In particular it is connected, locally connected and compact.

Any such space admits a length metric.
This statement was conjectured by Karl Menger in \cite{menger}
and proved independently 
by R. H. Bing  \cite[see][]{bing-length-0, bing-length-1} 
and Edwin Moise \cite[see][]{moise}.

It remains to consider a geodesic path from $p$ to $q$.